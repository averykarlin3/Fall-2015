\documentclass[11 pt, twoside]{article}
\usepackage{textcomp}
\usepackage[margin=1in]{geometry}
\usepackage[utf8]{inputenc}
\usepackage{color}
\usepackage{indentfirst} %Comment out for no first paragraph indent
\usepackage[parfill]{parskip}
\usepackage{setspace}
\usepackage{tikz}
\usepackage{amsmath}
\usepackage{amsfonts}
\usepackage{amssymb}
\usepackage{enumitem}
\usepackage{outlines}

\usepackage{fancyhdr}
\pagestyle{fancy}
\cfoot{\hyperlink{content}{\thepage}}
\lhead{}
\chead{}
\rfoot{}
\lfoot{}
\rhead{}
\renewcommand{\headrulewidth}{0pt}
\renewcommand{\footrulewidth}{0pt}


\usepackage{hyperref}
\hypersetup {
	colorlinks,
	citecolor=black,
	filecolor=black,
	linkcolor=black,
	urlcolor=black
}

\newcommand{\sepitem}{0pt} %Added room between items on the list, not including a list and its sublist
\newcommand{\seppar}{1pt} %Between items and lists overall

\setenumerate[1]{itemsep=\sepitem, parsep=\seppar}
\setenumerate[2]{itemsep=\sepitem, parsep=\seppar}
\setenumerate[3]{itemsep=\sepitem, parsep=\seppar}
\setenumerate[4]{itemsep=\sepitem, parsep=\seppar}

\newenvironment{outline*}
{
	\begin{outline}[enumerate]
	}
	{\end{outline}
}

\newcommand{\foot}[1]{\hyperlink{#1}{$_#1$}}

\begin{document}

\title{Physics I: Classical Mechanics}
\author{Avery Karlin}
\date{Fall 2015}
\newcommand{\textbook}{Young's University Physics}
\newcommand{\teacher}{Ali}

\maketitle
\newpage
\hypertarget{content}{\tableofcontents}
\vspace{11pt}
\noindent
\underline{Primary Textbook}: \textbook\\
\underline{Teacher}: \teacher
\newpage

\section{Chapter 2 - One Dimensional Motion}
\subsection{Displacement, Time, and Average Velocity}
\begin{outline*}
\1 All objects are placed on a coordinate system, and viewed as a particle, or a single point of negligible size and shape
\1 The displacement is equal to the vector from the initial point to the final point, written $\delta x = x_f - x_i$, measured in meters (m)
\2 Displacement is typically drawn as a function of time on an x-t graph, such that the secant line from two points is the average velocity within
\2 Distance is defined as the scalar quantity of the movement of the particle in the time interval
\1 Velocity is the rate of change of position with respect to time, in m/s
\1 $v_{average} = \frac{\delta x}{\delta t}$, such that $\delta t$ is the scalar change in time during the movement
\2 Speed is defined as the scalar distance over change in time, or $s = \frac{d}{\delta t}$
\2 $\delta t$ is typically represented as going from 0 to t, such that $\delta t$ is often represented at t
\end{outline*}
\subsection{Instantaneous Velocity}
\begin{outline*}
\1 Instantaneous velocity is the velocity at a specific point in time or position
\2 v < 0 is moving backward, v > 0 forward, and v = 0 is not moving
\2 Instantaneous speed is the instantaneous velocity, as a scalar quantity
\1 $v_{instantaneous} = \frac{dx}{dt}$
\2 Can be drawn on a graph as the slope of the secant line at a specific time and point
\1 Velocity can be drawn on a motion diagram, drawing a line for the x-axis, then a specific point in time on the line, with an instantaneous velocity vector drawn on it at that time
\end{outline*}
\subsection{Average and Instantaneous Acceleration}
\begin{outline*}
\1 Acceleration is the rate of change of velocity with respect to time, in $m/s^2$
\2 On a v-t graph, it is interpreted the same as velocity is on an x-t graph
\2 On an x-t graph, it is viewed as the concavity of the graph, equal to 0 at a point of inflection, or where there is no concavity
\2 Can be drawn on a motion diagram showing the change in velocity from one point in time to another, similarly to how velocity is drawn
\2 On an a-t graph, $\delta v$ is the area under the curve to the x-axis
\1 $a_{average} = \frac{\delta v}{t}$
\1 $a_{instantaneous} = \frac{dv}{dt} = \frac{d^2x}{dt^2}$
\end{outline*}
\subsection{Motion with Constant Acceleration}
\begin{outline*}
\1 Drawn by a straight line on an a-t graph, or a parabola on an x-t graph such that there is a constant rate of change of velocity
\1 $v_f = v_i + at$
\1 $v_{average} = \frac{v_i + v_f}{2}$
\1 $\delta x = v_it + \frac{at^2}{2}$
\1 $v_f^2 - v_i^2 = 2a\delta x$
\end{outline*}
\subsection{Freely Falling Bodies}
\begin{outline*}
\1 Particles falling without any force other than gravity acting on them are said to be in free fall, and move with constant acceleration of g, which is 9.8 $m/s^2$
\end{outline*}
\subsection{Motion with Non-Constant Acceleration}
\begin{outline*}
\1 $\delta v = \int_0^t a(t)dt$
\1 $\delta x = \int_0^t v(t)dt$
\1 $\delta d = \int_0^t |v(t)|dt$
\1 To get proper functions and values, the boundary conditions must always be used on integrals, rather than simply used as indefinite integrals
\end{outline*}
\section{Chapter 4 - Newton's Laws of Motion}
\subsection{Force}
\begin{outline*}
\1 Force is a vector quantity that one body or the environment exerts on another, measured in Newtons (N), or $kg*m/s^2$
\2 Measured with a spring balance, which stretches a spring to measure the force with a meter, or an inverse, which compresses to measure
\1 Contact force is a force through direct contact between the bodies, including normal and tension force
\2 Normal force is exerted by any surface a body is in contact with, perpendicular to the surface, only existing through direct contact
\1 Tension force is the force through a body in tension (being pulled from both sides), such as a frictionless rope, acting on the body attached in the direction the attached body is being pulled
\2 Tension forces is constant throughout a string, going in each direction being pulled, so that a block connected to a pulley has double the force, though a string separating into other strings has different forces on each
\2 Tension force is equal to the force being exerted on the object through the rope, if there is no responsive force on that exerting
\1 Spring force is the force exerted by the spring being stretched by some mass, such that F = kx, where x is the displacement from equilibrium
\2 The spring constant of springs in series is the reciprocal of the reciprocal sum, while parallel springs are the sum of the forces
\1 Long range forces are forces which act on separate bodies, such as electromagnetic force
\2 Weight is the gravitational force, exerted by the planet
\1 Net force on an object is equal to the vector sum of individual forces, allowing individual forces to be replaced by their component vectors
\1 Free body diagrams use vectors coming out from a particle to show all external forces acting on a specific body, to help understand the net force on the body
\end{outline*}
\subsection{Newton’s First Law}
\begin{outline*}
\1 A body acted on by no net force moves with constant velocity, and zero acceleration, thus staying at rest or at motion, due to the inertia of matter
\2 A body is at equilibrium if the net force is 0
\2 Assumes each body can be represented as a point particle
\1 The law is valid with an inertial frame of reference, such that the reference itself doesn’t have a force acting on it
\2 If a non-inertial reference has force acting on it, the relative velocity to the reference would change, though it would not change relative to another inertial reference
\2 The Earth is approximately inertial, though not, due to rotation and revolution around the Sun
\end{outline*}
\subsection{Newton’s Second Law}
\begin{outline*}
\1 $\sum F = ma$, such that $\sum F$ is the sum of external forces (forces exerted by other bodies on the body in question)
\1 The law is only valid in inertial frames of reference and when mass is constant
\1 m is the inertial mass of body, defined as the constant by which force is directly proportional to acceleration, measured typically in Kilograms (kg)
\2 Mass of bodies added together is summed
\2 Mass is determined by the number of subatomic particles in the matter
\2 Within the CGS metric system, 1 dyne = 1 $g*cm/s^2$ = $10^{-5}$ N, and in the British Imperial system, 1 pound = 4.4 N
\1 Acceleration can thus be gained, use to calculate velocity, displacement, or time, but the equations should be put in terms of the constants given, not time
\end{outline*}
\subsection{Weight}
\begin{outline*}
\1 Weight is the gravitational force of the planet on a body
\1 $F_w = mg$, such that g is the gravitational acceleration of the planet
\2 Gravitational acceleration on the Earth is 9.8 $m/s^2$, though it varies throughout the Earth slightly due to rotation and revolution
\2 Mass measured through a scale, using weight to measure it is called gravitational mass, and ideally, is equal to inertial mass
\1 Apparent weight is the normal force of an object when the surface has additional upward or downward forces acting on it, changing the scale measurement, since scales measure by the normal force they exert
\2 Apparent weightlessness is when in free fall, such that apparent weight is zero
\end{outline*}
\subsection{Newton’s Third Law}
\begin{outline*}
\1 If one body exerts a force on another, then the other body exerts an equal but opposite force on the first body $(F_{a on b} = -F_{b on a})$
\2 This law applies to both long range, and contact forces
\2 The two forces are called an action-reaction pair
\end{outline*}
\section{Chapter 6 - Work and Kinetic Energy}
\subsection{Work}
\begin{outline*}
\1 Work is the scalar product of displacement and the force in the direction of displacement, such that W = Fx and $W_{tot} = F_{net}x$ when force is constant
\2 $W = \int^{x_2}_{x_1} F_{||}(x) \cdot dx$ for non-constant force, where $F_{||}$ is the force function tangential/parallel to the movement at that point, called a line integral
\2 Work is measured in Joules (J), or N*m
\2 Measured in the British Imperial System with ft*lb, since lb is the measurement for force
\1 Work is zero if the force is perpendicular to the displacement, and negative if the force is opposite the displacement
\1 Stable equilibrium is the point where a shift in any direction would cause movement back to equilibrium, unstable is where it would cause a shift from, and neutral equilibrium causes a lack of shift after slight movement
\1 $W_{net} = \int^{x_2}_{x_1} F_{net}(x) \cdot dx = \sum_n W_n$
\end{outline*}
\subsection{Kinetic Energy}
\begin{outline*}
\1 $K = \frac{1}{2}mv^2$
\1 Increase in speed means positive acceleration in the direction of displacement, work is positive, showing the relationship between work and velocity
\1 The Work Energy Theorem states that $W_{total} = \delta K$ for any single particle or composite system within any inertial frame of reference
\2 This is true for any forces, conservative or nonconservative
\end{outline*}
\subsection{Power}
\begin{outline*}
\1 Power is the rate at which work is done, measured in Watts (W), or J/s
\2 Can also be measured in ft*lb/s, or horsepower (hp), such that 1 hp = 500 ft*lb/s
\1 $P_{av} = \frac{\delta W}{\delta t}$
\1 $P = \frac{dW}{dt} = F \cdot v$ (if F is constant)
\end{outline*}
\end{document}
