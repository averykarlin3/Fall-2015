\documentclass[11 pt, twoside]{article}
\usepackage{textcomp}
\usepackage[margin=1in]{geometry}
\usepackage[utf8]{inputenc}
\usepackage{color}
\usepackage{indentfirst} %Comment out for no first paragraph indent
\usepackage[parfill]{parskip}
\usepackage{setspace}
\usepackage{tikz}
\usepackage{amsmath}
\usepackage{amsfonts}
\usepackage{amssymb}
\usepackage{enumitem}
\usepackage{outlines}

\usepackage{fancyhdr}
\pagestyle{fancy}
\cfoot{\hyperlink{content}{\thepage}}
\lhead{}
\chead{}
\rfoot{}
\lfoot{}
\rhead{}
\renewcommand{\headrulewidth}{0pt}
\renewcommand{\footrulewidth}{0pt}


\usepackage{hyperref}
\hypersetup {
	colorlinks,
	citecolor=black,
	filecolor=black,
	linkcolor=black,
	urlcolor=black
}

\newcommand{\sepitem}{0pt} %Added room between items on the list, not including a list and its sublist
\newcommand{\seppar}{1pt} %Between items and lists overall

\setenumerate[1]{itemsep=\sepitem, parsep=\seppar}
\setenumerate[2]{itemsep=\sepitem, parsep=\seppar}
\setenumerate[3]{itemsep=\sepitem, parsep=\seppar}
\setenumerate[4]{itemsep=\sepitem, parsep=\seppar}

\newenvironment{outline*}
{
	\begin{outline}[enumerate]
	}
	{\end{outline}
}

\newcommand{\foot}[1]{\hyperlink{#1}{$_#1$}}

\begin{document}

\title{Physics I: Classical Mechanics}
\author{Avery Karlin}
\date{Fall 2015}
\newcommand{\textbook}{Young's University Physics}
\newcommand{\teacher}{Ali}

\maketitle
\newpage
\hypertarget{content}{\tableofcontents}
\vspace{11pt}
\noindent
\underline{Primary Textbook}: \textbook\\
\underline{Teacher}: \teacher
\newpage

\section{Chapter 2 - One Dimensional Motion}
\subsection{Displacement, Time, and Average Velocity}
\begin{outline*}
\1 All objects are placed on a coordinate system, and viewed as a particle, or a single point of negligible size and shape
\1 The displacement is equal to the vector from the initial point to the final point, written $\delta x = x_f - x_i$, measured in meters (m)
\2 Displacement is typically drawn as a function of time on an x-t graph, such that the secant line from two points is the average velocity within
\2 Distance is defined as the scalar quantity of the movement of the particle in the time interval
\1 Velocity is the rate of change of position with respect to time, in m/s
\1 $v_{average} = \frac{\delta x}{\delta t}$, such that $\delta t$ is the scalar change in time during the movement
\2 Speed is defined as the scalar distance over change in time, or $s = \frac{d}{\delta t}$
\2 $\delta t$ is typically represented as going from 0 to t, such that $\delta t$ is often represented at t
\end{outline*}
\subsection{Instantaneous Velocity}
\begin{outline*}
\1 Instantaneous velocity is the velocity at a specific point in time or position
\2 v < 0 is moving backward, v > 0 forward, and v = 0 is not moving
\2 Instantaneous speed is the instantaneous velocity, as a scalar quantity
\1 $v_{instantaneous} = \frac{dx}{dt}$
\2 Can be drawn on a graph as the slope of the secant line at a specific time and point
\1 Velocity can be drawn on a motion diagram, drawing a line for the x-axis, then a specific point in time on the line, with an instantaneous velocity vector drawn on it at that time
\end{outline*}
\subsection{Average and Instantaneous Acceleration}
\begin{outline*}
\1 Acceleration is the rate of change of velocity with respect to time, in $m/s^2$
\2 On a v-t graph, it is interpreted the same as velocity is on an x-t graph
\2 On an x-t graph, it is viewed as the concavity of the graph, equal to 0 at a point of inflection, or where there is no concavity
\2 Can be drawn on a motion diagram showing the change in velocity from one point in time to another, similarly to how velocity is drawn
\2 On an a-t graph, $\delta v$ is the area under the curve to the x-axis
\1 $a_{average} = \frac{\delta v}{t}$
\1 $a_{instantaneous} = \frac{dv}{dt} = \frac{d^2x}{dt^2}$
\end{outline*}
\subsection{Motion with Constant Acceleration}
\begin{outline*}
\1 Drawn by a straight line on an a-t graph, or a parabola on an x-t graph such that there is a constant rate of change of velocity
\1 $v_f = v_i + at$
\1 $v_{average} = \frac{v_i + v_f}{2}$
\1 $\delta x = v_it + \frac{at^2}{2}$
\1 $v_f^2 - v_i^2 = 2a\delta x$
\end{outline*}
\subsection{Freely Falling Bodies}
\begin{outline*}
\1 Particles falling without any force other than gravity acting on them are said to be in free fall, and move with constant acceleration of g, which is 9.8 $m/s^2$
\end{outline*}
\subsection{Motion with Non-Constant Acceleration}
\begin{outline*}
\1 $\delta v = \int_0^t a(t)dt$
\1 $\delta x = \int_0^t v(t)dt$
\1 $\delta d = \int_0^t |v(t)|dt$
\1 To get proper functions and values, the boundary conditions must always be used on integrals, rather than simply used as indefinite integrals
\end{outline*}
\section{Chapter 6 - Work and Kinetic Energy}
\subsection{Work}
\begin{outline*}
\1 Work is the scalar product of displacement and the force in the direction of displacement, such that W = Fx and $W_{tot} = F_{net}x$ when force is constant
\2 $W = \int^{x_2}_{x_1} F_{||}(x) \cdot dx$ for non-constant force, where $F_{||}$ is the force function tangential/parallel to the movement at that point, called a line integral
\2 Work is measured in Joules (J), or N*m
\2 Measured in the British Imperial System with ft*lb, since lb is the measurement for force
\1 Work is zero if the force is perpendicular to the displacement, and negative if the force is opposite the displacement
\1 Stable equilibrium is the point where a shift in any direction would cause movement back to equilibrium, unstable is where it would cause a shift from, and neutral equilibrium causes a lack of shift after slight movement
\1 $W_{net} = \int^{x_2}_{x_1} F_{net}(x) \cdot dx = \sum_n W_n$
\end{outline*}
\subsection{Kinetic Energy}
\begin{outline*}
\1 $K = \frac{1}{2}mv^2$
\1 Increase in speed means positive acceleration in the direction of displacement, work is positive, showing the relationship between work and velocity
\1 The Work Energy Theorem states that $W_{total} = \delta K$ for any single particle or composite system within any inertial frame of reference
\2 This is true for any forces, conservative or nonconservative
\end{outline*}
\subsection{Power}
\begin{outline*}
\1 Power is the rate at which work is done, measured in Watts (W), or J/s
\2 Can also be measured in ft*lb/s, or horsepower (hp), such that 1 hp = 500 ft*lb/s
\1 $P_{av} = \frac{\delta W}{\delta t}$
\1 P = \fdW/dt = Fv (if F is constant)
\end{outline*}
\end{document}
