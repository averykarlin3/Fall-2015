\documentclass[11 pt, twoside]{article}
\usepackage{textcomp}
\usepackage[margin=1in]{geometry}
\usepackage[utf8]{inputenc}
\usepackage{color}
\usepackage{setspace}
\usepackage{tikz}
\usepackage{amsmath}
\usepackage{indentfirst}
\usepackage{amsfonts}

\begin{document}

\title{StuyMUNC 2016: The Manhattan Project}
\author{Avery Karlin and Jackson Morgan}
\date{April 9-10, 2016}

\maketitle
\newpage
\tableofcontents
\vspace{11pt}
\newpage

\section{Letter from the Chair}
Hello delegates, my name is Avery Karlin, and I will be your chair for the Manhattan Project committee at StuyMUNC 2016, as we work to create the largest breakthrough in weapons research in history, and one of the largest breakthroughs in nuclear physics research. I am a senior at Stuyvesant High School, and have been on the Model UN team since freshman year, going to many different conferences. International politics and political science as a result, have always been a passion of mine, and combined with my academic passion for physics, gave me the idea for this committee. In the Model UN team, I currently am Undersecretary of Technology and Innovation as well as the Director of the Interior for StuyMUNC, helping to plan this conference. I also previously served as Director of Logistics previously to plan last year's conference.

I follow American politics very closely, as well as foreign politics. I also follow modern developments in physics, and plan to major in physics next year in university. I have also done research at the Metropolitan Museum of Art Scientific Research Labs, and am in the process of publishing a paper in spectroscopy. I consider the Manhattan Project the collective efforts of many of the greatest minds in history, and an endeavor which changed the future of the human race, and has the potential to greatly improve that future. In particular, scientists such as Richard Feynman and J. Robert Oppenheimer are personal idols of mine, having read their biographies, papers, and writings extensively.

In my free time, I enjoy binge watching TV shows, including the West Wing, House of Cards, Doctor Who, Sherlock, Star Trek and Breaking Bad. I also read extensively on a variety of subjects from philosophy to quantum mechanics to real analysis during my free time. Finally, I enjoy coin collecting and programming.

I look forward to seeing you all in committee, and hope this year's StuyMUNC is the best we have ever had.

\section{Letter from the Co-Chair}
	Hello delegates! My name is Jackson Morgan and I will be your co-chair in the Manhattan Project committee. I am a junior at Stuyvesant High School, and have been doing Model U.N. since 7th grade. I have attended more conferences than I can count. My capacity in the Stuyvesant Model U.N. club is as deputy delegate trainor. Aside from Model U.N., I sometimes do parliamentary debate. In my free time I like to play videogames. I also enjoy binge watching television and watching movies. In addition, I have been doing some programming, but haven’t had the time to make anything substantial.
	I am very interested in politics and have taken opportunities to learn more about the political landscape in this country. I did this through two internships. One, at the Manhattan Borough President’s office, where I learnt a lot about local government. Another was at an immigration law firm and where I learnt about immigration law. 	
	I went to the same middle school as the other chair in this committee, although Avery is a grade higher than me. We will use our combined talents to do our best to make this conference as fun as possible for everyone who is coming. I hope you all have a great StuyMUNC 2016!

\section{Parliamentary Procedure}
	This committee will be a historical crisis committee. Delegates will take action in committee through the use of directives. These directives can either be group directives which will be voted on for a necessary simple majority by the group, or they can be individual directives which may be secret and which may utilize the portfolio powers of that individual. It will be the Chair’s discretion whether a delegate or the committee is able to take a certain action, but members of the committee are encouraged to push the boundaries and be as creative as possible throughout committee. There will, of course, be numerous crisis events throughout the committee. These crises will relate to the Manhattan Project itself and the larger global conflict. Delegates should not expect these crises to necessarily reflect what actually happened. The committee will be run with a combination of moderated and unmoderated caucuses, but no initial speakers list, working purely through caucuses. Other standard parlimentary procedure will apply in committee. The dias will also function as President Franklin D. Roosevelt for the duration of the committee, able to exercise his authority when necissary.

\section{Background Information}

\subsection{New Technology}
	In 1938 nuclear fission was discovered by Otto Hahn and Fritz Strassmann. As the news of the discovery spread, so did the implications of the new technologies that could be developed. At the forefront of these implications was the fact that nuclear technology could theoretically be used to create bombs much more powerful than any created beforehand. This lead Leo Szilard to write a letter, signed by Albert Einstein, that warned the United States President, Franklin D. Roosevelt, of the potential for atomic bombs to be created. The letter also warned about the possibility that Germany would attempt to develop them. The letter suggested that the United States also develop a nuclear program. It was this suggestion that would eventually lead to the commencement of the Manhattan Project. After receiving this information, President Roosevelt ordered an investigation into the potential of nuclear technology. After receiving word that uranium could lead to the development of bombs vastly more powerful than any created beforehand, the government created a research committee called the S-1 Committee of the OSRD. In october of 1941 Roosevelt approved of a United States Atomic program.
\subsection{The Global Crisis}
	The findings of this new technology came at a rather volatile time for the nations of the world. This was because of the global crisis known as World War II. The United States was not involved in the war, and consequently not as concerned about a nuclear program as they were not involved in the war until the Pearl Harbor attacks of December 7, 1941. After this attack, it was clear that the United States would be going to war. Alarmed by this new development, the S-1 committee had its first meeting on the eighteenth of December that year. Over the next few months they came up with a series of guidelines for what a United States nuclear program would entail. These guidelines and the budget necessary were approved in June of 1942. During this time the planning process of the project was commenced. This work began in Manhattan, originally under the project name Development of Substitute Materials. Leslie Groves, who was working in the military wing of the department, believed the name would attract too much attention. Because of this the project was renamed the Manhattan Project. Later that year, development sites were established in Oak Ridge, Los Alamos, Argonne, and Hanford. It is at this place in time that the committee will commence. 
\subsection{Global Race}
	The United States was not the only country with a nuclear program. Nuclear programs had been established in the United Kingdom, Germany, and the USSR. If the United States wanted to be the first to develop nuclear technologies, there were many things that they had to consider. One was which nations to work with and which nations to work against. Would working with other nations be helpful or harmful? They would also have to decide which method would be best in order to develop nuclear technologies. There were many different recommendations from many different scientists. Not all of their proposals could be followed. Should the committee commission many different development procedures in the hopes that one of them would work, or should they focus in on one strategy? These were just some of many questions to consider. 

\section{Questions to Consider}

The year is 1943, as the Los Alamos National Laboratory was just build. Now the scientists must begin their true challange, commencing work on the most powerful the weapon the world will ever see.

\subsection{Safety}
The development of nuclear technologies hold many safety risks, although not all of them are currently known. The committee must decide whether it wants to allocate time and money in order to ensure safety or if it wants to pursue speed above all else.

\subsection{Funding}
Funding is always a problem for any government program. The development of nuclear technologies is expensive. However, ensuring funds is not as easy as it may seem. Nuclear weapons are currently a nebulous concept. Government officials may question why they are allocating money towards the program when the soldiers currently fighting also need significant funding. How can government awareness be raised with raising the suspicions of the public? 

\subsection{National Security}
While the United States wished to develop nuclear weapons before its competitors, it does not wish for its discoveries to be utilized by them. The committee must try its best to put in place measures that most ensures the national security of the United States. This includes deciding how to deal with foreign spies, deciding  whether or not to send spies to infiltrate other nuclear programs, and figuring out how to hide the program from the public or mislead them about the program. In addition, thousands of workers will be involved in the project. How can these workers be trusted with not spreading the top secret information about the project they are working on? Should the employees be lied to about what they are involved with?

\subsection{Censorship}
There are sure to be articles revealing the nature of the project or making the public aware of the implications of nuclear technology. What can the committee do to stop these articles from getting out? Should they try to stop the articles from being published? Is this even constitutional? 

\subsection{Testing}
Once nuclear weapons are developed they will have to be tested. How can these tests be done in a safe manner that will not alarm the general public to the existence of the weapons?

\section{Delegate Positions}

\subsection{The Scientists}
The scientists’ role is to advise the committee about the different ways they can apply nuclear technology, and how to develop a bomb. Different scientists will have different perspectives in regards to how to accomplish the goals of the Manhattan Project. Scientists, it is well known, have big egos. They may want to see their ideas pursued even if they are not the most effective. Scientists will also be able to make recommendations to the committee on how best to sabotage the nuclear development programs of other nations. For the purpose of the making sure the committee runs smoothly, scientists will  have a say in all else happening in the committee.

\subsection{Military Personnel and Non-Scientists}
It will be up to these delegates to try to ensure funding, maintain secrecy, and deal with threats from other nations. Some of these committee members may be spies. While everyone here wants  a swift completion of the project, they also may be in search of a promotion or personal accolades. These desires could hamper the productivity of the Manhattan Project. 

\section{Individual Positions}
\begin{enumerate}
\item Leslie R. Groves: Leslie Groves is at the forefront of the Manhattan Project. He is leading the military effort in the project. Groves will have immense resources to work, major connections with key political figures, and huge sway over other military officials in the committee. 
\item J. Robert Oppenheimer: Oppenheimer is a genius theoretical physicists. He worked at both Caltech and Harvard, and is well respected in the scientific community. Oppenheimer has made important contributions to the study of hydrogen and x-rays. Oppenheimer will have access to labs all over the nation, the expertise of other scientists, and his own research. His opinion will definitely be held in high regard throughout the committee.
\item Richard C. Tolman: Tolman is another important physicist, having made contributions to theoretical cosmology. He is a professor of mathematical physics and physical chemistry at Caltech. As an important scientist, he will be able to utilize his research and his connections to assist in the creation of a nuclear weapon as quickly as possible. 
\item Klaus Fuchs: Bringing some foreign flare to the committee, this German born British theoretical physicist has a lot to bring to the table. However, his roots may threaten his commit to the cause. Can he really turn his back on his people? Fuchs will have unique German and Soviet connections which he can utilize in committee. 
\item James B. Conant: Conant, the President of Harvard University, and a respected chemist, is no stranger to using science to develop chemical weapons. During World War I he helped develop poison gases. As the president of Harvard, he will have all of Harvard’s resources available to him, as well as many connections in the scientific community.
\item Vannevar Bush: Bush is an electrical engineer from MIT. He is also the chairman of the NAtional Advisory Committee for Aeronautics. He has a close relationship with the government and oversees many scientists. These are two advantages he can certainly use to his advantage.
\item Arthur Compton: Compton, who won a nobel prize in 1927 for discovering the Compton effect, is one  the leaders in the study of electromagnetic radiation. His expertise on the subject will certainly come in handy for overseeing different nuclear facilities.
\item Thomas Farrell: Farrell works under Leslie Groves. He served in the Military in WW I and worked his way up the ranks. He also instructed at West Point. In effect, he is second in command to Groves. Farrell’s years of military experience gives him an understanding of the world and access to resources that the scientists involved with the project do not have. 
\item Ernest Lawrence: One of the first nuclear physicists, Lawrence is an authority on the science that is needed to create nuclear weapons. He won the nobel prize in 1939 for his invention of the cyclotron. His opinion in committee will be very important. The respect he has gained through winning the nobel prize makes it more likely that he will be able to get favors. He will also have access to his labs.
\item James Marshall: Marshall is a U.S. army brigadier general. He graduated from West Point early due to the start of World War I. His leadership abilities will prove vital in this committee.
\item Kenneth Nichols: Nichols is an engineer from the United States army. He graduated fifth in his class from West Point in 1929. He previously worked with Groves during an expedition to Nicaragua. His links with Groves and his engineering knowledge make him unique in this committee.  
\item Harold Urey: Urey won the Nobel Prize in 1934 for the discovery of deuterium. Seen by others as a world expert on isotope separation, he was recruited to the Manhattan Project. His ideas can be very influential to the committee.
\item Stafford Warren: Warren is a physician who studied radiology and medicine. He is interested in   finding out about the health effects of radiation. His unique knowledge and studies will assist him in helping the committee succeed and prevent the potential negative health effects of radiation.
\item Enrico Fermi: Fermi is a physicist from Italy. He recently received the Nobel Prize for his unique research on induced radioactivity. He also obtains patents that detail the potential for nuclear weapons. His research and knowledge will give him a huge advantage in committee.  
\item Harry Truman: A senator from Missouri, Truman is an important and influential politician. He also   served in the military earlier in his life. He is well respected and has the best interests of the United States in mind. He has the ability to work directly with the federal government to get more support for the project.
\item Edward Teller: Teller is a physicist from Hungary. He moved to the US when he became a professor at George Washington University. He also was one of the first to think of the potential for a fusion based weapon. His immense experience, knowledge, and foreign contacts will come in handy.
\item David Greenglass: Born in New York, Greenglass went to, but did not graduate from Brooklyn  Polytechnic Institute. A low level worker for the Manhattan project, Greenglass may seem harmless. However, his connections with the Soviet Union may prove problematic. 
\item John Lansdale: Lansdale attended both the Virginia Military institute and the Harvard Law School. With strong intelligence training, Lansdale has the potential to be very useful in maintaining the security of the project. He can contact and utilize the FBI.
\item Franklin Mathias: Mathias is a Army Corps engineer. His knowledge of construction projects will prove vital in committee. He can provide insight on how best to develop the sites needed for the project.
\item Dorothy McKibbin: McKibbin is a secretary for the Manhattan Project. As one of the first points of contacts for new arrivals to the Project, she can provide information about the general moral of the populous. She works closely with Oppenheimer.
\item William ``Deak'' Parsons: Parsons is a naval officer. He represents the navy at the project and can call for the navy to assist on his behalf. This makes him tremendously important to the committee.
\item William Purnell: Purnell is a naval officer who served in World War I. He also represents the navy at the Manhattan project. He will work closely with Parsons to determine the best way to utilize the navy in this project.
\item Frank Spedding: Spedding is a Canadian American chemist. He has the assistance of other other scientists who are willing to help him work on the Manhattan project. He also has access to the resources of Iowa State College.
\item Charles Thomas: Thomas is a talented chemist and MIT graduate. His research focused on hydrocarbons and polymers. Thomas’ resources and knowledge will be great contributions to the committee. 
\item Paul Tibbets: Tibbets is a United States Air force brigadier general. He provides the perspective of the US air force and how the air force can coordinate with the committee. He will be able to utilize the resources of the air force.
\item William “Bud” Uanna: Uanna is a security expert who works in the Counter Intelligence Corps. His knowledge in counterintelligence and and experience in the Corps will allow him to protect the security of the committee.
\item Roscoe Charles Wilson: Wilson is an Air Force pilot who graduated from West Point in 1928. He is the liaison officer to the Manhattan Project. He will work with Tibbets to provide oversight on how the Air Force involves itself with the Manhattan Project. 
\item Richard Feynman: Feynman is a theoretical physicist who attended MIT. He also studied at Princeton. Straight out of Princeton, he was recruited to the Manhattan Project. He is inexperienced, but brilliant, and well known for his outside the box way of thinking, quirkiness, problem solving abilities, and bongo-playing. While he is young, he commands a great deal of respect among the senior staff at the project. 
\item Leo Szilard: Szilard came up with the idea of a nuclear chain reaction in 1933, patented the idea of a nuclear reactor, and wrote the letter Einstein signed. At the forefront of the nuclear technology movement, Szilard may be the most qualified to be working on the project.
\end{enumerate}

\section{References}
\begin{itemize}
\item http://www.u-s-history.com/pages/h1644.html 
\item http://www.ushistory.org/us/51f.asp 
\item http://www.atomicarchive.com/History/mp/index.shtml 
\item http://www.britannica.com/event/Manhattan-Project 
\item http://www.energy.gov/management/office-management/operational-management/history/manhattan-project.html
\item http://nuclearweaponarchive.org/Usa/Med/Med.html 
\item http://www.pitt.edu/~sdb14/atombomb.html 
\end{itemize}

\end{document}
