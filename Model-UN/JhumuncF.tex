\documentclass[12pt]{article}
\usepackage[margin=1in]{geometry}
\usepackage[utf8]{inputenc}
\usepackage{setspace}
\usepackage{indentfirst}
%\singlespace
\onehalfspace
%\doublespace

\begin{document}
\begin{flushleft}
Frank Underwood's Cabinet \\
JHUMUNC - February 18-21, 2016 \\
Freddy Hayes - Foreign Relations
\end{flushleft}
%\begin{center}
%\end{center}

The Middle East must be a central aspect of any foreign policy, but previous attempts to rely on the aid of Russia and Petrov are weak at best, relying on repairing decade-old tensions to achieve peace. Assuming we need Petrov creates a choice of ignoring human rights abuses, hurting popularity at home and abroad, and fighting against his record actively, but losing the ability for peace in the Middle East. Clearly, a middle ground is needed, such that while we do not actively go to war metaphorically with Petrov, we must show to the world that we are not a supporter of human rights abuses, and find an alternate solution to peace in the Middle East, such as relying on NATO troops in the Jordan Valley, rather than a US-Russian task force, and allowing them to join if they want to be a force for power and good, but not letting them derail the mission. Instead of asking them to join, and letting them refuse for fear of American influence, if peace in the Middle East is a priority to President Underwood, allow them to join if they wish to preserve their authority.

The Jordan Valley is central to the Israel-Palestinean conflict, with the Israelis claiming it is necissary to create a defensible Eastern border, and fearing that Palestinean takeover would put an Arab land between Samaria and East Jerusalem, leading the way for an Arab takeover of the city. On the other hand, continual Israeli settlements in the Jordan Valley have led to extensive human rights abuses against Palestinean children that must be addressed, and thus, since neither government has been able to set aside their convictions, an international force is needed.

The most important step to a sustainable peace process, on the other hand, is creating the proper infrastructure in Palestine to be able to support a stable government, preventing regions such as the Jordan Valley from being as unstable. Until reforms are taken in that direction, every fix is merely temporary. Sending troops to two unwilling countries are doomed to fail, if they merely feel the other side is out to defeat them. To gain success, you must expect success. Forcing peace is a temporarily solution to preserve a treaty, but mutual trust must be created.

\newpage
\subsection*{Bibliography}
\begin{itemize}
\item http://besacenter.org/perspectives-papers/jordan-valley-israels-security-belt/
\item https://www.hrw.org/report/2015/04/13/ripe-abuse/palestinian-child-labor-israeli-agricultural-settlements-west-bank
\end{itemize}

\end{document}
