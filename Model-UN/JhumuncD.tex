\documentclass[12pt]{article}
\usepackage[margin=1in]{geometry}
\usepackage[utf8]{inputenc}
\usepackage{setspace}
\usepackage{indentfirst}
%\singlespace
\onehalfspace
%\doublespace

\begin{document}

\begin{flushleft}
Avery Karlin \\
Frank Underwood's Cabinet \\
JHUMUNC - February 18-21, 2016 \\
Freddy Hayes - Domestic Affairs
\end{flushleft}

%\begin{center}
%\end{center}

The idea of the drop in entitlements in exchange for increased jobs in infrastructure, government, and the private sector is the basis of America Works as a new New Deal, removing the entire safety net in exchange for the promise of a job. While the original method of implementation through FEMA funds was not the ideal, it was a necessity in a gridlocked Congress, unwilling to make any form of modification to the status quo. Using FEMA funds allowed America Works to be tested, and proved that it does work. In recent years, there has been an extreme raise in entitlement spending, especially on able-bodied adults, social security, medicare, and medicaid, creating the need for reforms, such as the removal of food stamp waivers. In addition, the need to eliminate bureaucratic red tape is clearly a major flaw in entitlements.

While it could be argued that forcing the decrease in unemployment would considerably tighten the labor market, the guaranteed income from the fderal government would preserve wages without inconveniencing companies, preventing drastic inflation. On the other hand, it is noted the possible costs that could arise, due to the continued need for medicaid, social security, and food stamp payments by those unable to work.

While a less drastic action might be in order, the idea that billions are being spent taking care of those who want to work is severely limiting American economic growth, and further spending on entitlement management, rather than cutting and simplifying entitlements to those who truly need them, are tearing apart the economy. People must be allowed to survive if they have no other option, but otherwise, they are entitled to nothing. If we keep social security and medicare as they currently are, it is not a safety net, but free income to those who don't need it for a lot of Americans, at the cost of the American fiscal future. To remove America Works, taking jobs from the thousands who recently gained it, and the potentially millions others, in favor of continuing the cycle of unwilling dependency and poverty, would be an abomination, created by those too afraid of change.

In addition, it has been clear for a long time that reform is needed. Congression stagnation, with occasional criteria changes to prevent the complete insolvency of the funds, but on the other hand, that is predicted to occur regardless within the next 30 years. Reform is needed, and removing large amounts of the safety net in favor of simplification and provided jobs is the obvious method of doing so.

\subsection*{Bibliography}
\begin{itemize}
\item https://www.washingtonpost.com/news/the-fix/wp/2015/03/03/could-the-house-of-cards-america-works-program-actually-work/
\item https://www.aei.org/publication/the-easy-way-to-entitlement-reform-simply-assume-massive-cuts
\end{itemize}

\end{document}
