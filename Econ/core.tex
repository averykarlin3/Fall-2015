\section{CORE Chapter 1}

\subsection{National Differences}
\begin{enumerate}
\item In the 1300s, most of the world was fairly equal in the general amount of wealth of the population, even if there
were large differences between the rich and poor, often depending on parental status
\begin{enumerate}
\item Gross domestic product per capita, or average living standards, has raised in the last 700 years, but caused
differences by country, due to having a sudden rise at different times, leading to different standards
\item Often, independence from colonial rule or European interference caused the sudden economic growth, but Latin
America did not have the growth
\end{enumerate}
\item The ratio scale has the GDP y{}-axis go up by some multiple, used to compare growth rate, or (${\Delta}$GDP)/(GDPstart), such that 100\% means it doubles if ratio of 2
\begin{enumerate}
\item The ratio scale thus has the slope of the graph as the growth rate
\item Thus, the GDP per capita appears as a hockey stick curve, remaining without much growth, before a kink, leading to
a sudden rise
\end{enumerate}
\item Adam Smith argued that coordination of all the aspects of an economic culture from different parts of the world,
would be created on its own based on self{}-interest, rather than made by the government
\begin{enumerate}
\item He did believe there was ethical beliefs guiding behavior, and feared monopolies, especially government protected
\item He approved of government investment in education and public works, as well as justice and foreign policy through
the government
\end{enumerate}
\end{enumerate}

\subsection{Average Living Standards}
\begin{enumerate}
\item GDP per capita is different from average disposable income, which is close to average living standards, but ignores some aspects of happiness
\begin{enumerate}
\item Disposable income is the total income from factor markets, tranfer payments from both government, and from others, minus any governemtn taxes
\item Quality of social and physical environment, government services, and goods produced within a household, are important to well-being, but ignored by disposable income
\item Average disposable income also ignores distribution, due to extra income affecting the rich less than lack of affects the poor
\item People are also less happy based on how their income is related to others within the population, even if they have enough income to not be in poverty
\end{enumerate}
\item GDP is better when evalutating government services, but can only be measured easily through cost to produce, rather than selling value
\end{enumerate}

\subsection{The Modern World}
\begin{enumerate}
\item The sudden spike GDP per capita leading to the hockey stick appearence was caused by the Industrial Revolution, or major technological advances in textiles, energy, and transportation, leading to old skills rapidly becoming outdated
\begin{enumerate}
\item Technology is the process of creating an output from inputs, shortening the time to make the product as technology improves
\item The Industrial Revolution began a permenant technological revolution, or the drastic increase in technology at an exponential rate, allowing a drastic increase in living standards
\end{enumerate}
\item Modern technology has allowed international financial transactions to take place almost instantaneously, such that the speed of news travel is increasing exponentially
\item Population has also exponentially grown in recent centuries, with the improvement of public health and water services, though it has begun to slow in recent years, called the demographic transition, due to the desire for less children and fewer-child policies
\begin{enumerate}
\item At the same time, due to the increase in agricultural technology, less land and farmers were needed for the same output, leading to the growth of cities, creating additional interaction, the need for police
\end{enumerate}
\end{enumerate}

\subsection{Environmental Impact}
\begin{enumerate}
\item As production increased, environemtnal damage, especially climate change due to burning of fossil fuels (gas, oil, or coal) increasing $CO_2$ emissions
\begin{enumerate}
\item While temperature has always fluctuated due to volcanic events, such as the 1815 Mount Tambora eruption in Indonesia, lowering temperature in 1816, but has drastically risen over the last century
\item This can lead to rising sea levels, climate changes destroying farming, and polar ice cap melting, as well as cause respiratory issues and illnesses in cities
\end{enumerate}
\item The economy is a portion of society, which is within the overall biosphere, such that it can affect the other aspects
\item Overuse of local resources is also a possible environmental issue, caused along with climate change by economic expansion and organization (resources valued and conserved)
\begin{enumerate}
\item While the permanent technological revolution gave fossil fuel dependence, but has also permitted vastly more electricity for fewer resources, permitting the development of alternate sources, which may change it
\end{enumerate}
\end{enumerate}

\subsection{Capitalism}
\begin{enumerate}
\item An economic system is the organization of the production and distribution of products in an economy
\item Capitalism is the economic system made up of institutions, or sets of laws and social customs regulating the economy, of private property, markets, and firms
\begin{enumerate}
\item In prior economies, families typically were the third major institution, or the economy was regulated by a centralized government
\item Private property is possessions purely controlled and owned by an entity, able to control use or give ownership
\item Markets facilitate the transfer of goods in a reciprocal, voluntary trade
\item Firms are the main organization of production, owning the capital goods (which diffrentiates it from other economic organizations and systems), paying wages, managing employees, such that the products are private property of the owner, to make a profit
\begin{enumerate}
\item Firms are the main organization, with others being families, unions, government agencies, and non-profits
\item Firms utilize the labor market, unlike other organizations, such that firms can be created, destroyed, expand, or contract extremely quickly
\end{enumerate} 
\end{enumerate}
\item Capitalism both centralizes power in the hands of firm owners, and decentralizes from the government and other outside influences, creating power and cooperation inside, but competition outside
\begin{enumerate}
\item Capitalism relies on being dynamic, such that private property is secure, markets are fully competititive, and firm leadership is meritocratic, rather than aristocratic
\item This makes it unique, such that being in the upper classes is purely based on performance, creating private incentives for constant innovation, even if well-connected individuals can avoid failure easier than others
\item Government policies are also necissary for capitalism, to preserve said economic conditions, by a proper legal system, anti-monopoly regulations, lack of bail-outs, and providing essential products to all citizens, such as infrastructure, education, and defense
\item Thus, the capitalist revoltion, starting in the UK, was the creation of the private incentive, meritocratic, and public policy conditions, needed for dynamic conditions
\end{enumerate}
\item Defintions are used, not to set a specific rule, but to set general catagories which something can fall within to different degrees, such as China's five year plan with capitalism
\begin{enumerate}
\item Low quality of institutions, especially in later capitalist countries, such as African nations, caused misdirection of government funds and corruption, hurting growth
\item Asian nations, such as South Korea, Japan, and China, grew drastically, due to government promotin of industry, high quality education, and foreign trade requirements, creating a developmental state
\item Post-USSR nations also had slow growth, due to the lack of competitive markets, meritocracy, or secure private property (due to weak government and legal bodies)
\end{enumerate}
\item Capitalism can be shown to mainly be the cause of the economic growth and other effects within the last several centuries through natural experiments, or circumstances where the cause changes, to allow the change in result to be measured
\begin{enumerate}
\item One major example is that of East and West Germany, and the vastly higher GDP increase of the capitalist West, rather than the centralized East
\item Natural experiments replace planned experiments, due to the difficulty in experimentation, because of the scope and stakes of economic changes
\end{enumerate}
\item Government plays a large role in capitalism, as a major consumer, and that which establishes the legal structure to make it possible, such that different political structures creates different forms of capitalism
\begin{enumerate}
\item Democracies, the most common government with capitalism, ranges in terms of the structure, influence of big money, and individual rights
\item Some governments take amounts of GDP as revenue, >50\%, like Scandanavia, playing a large role in the economy as a result, while others take relatively little, but have a large degree of control over the economy, such as Asia
\end{enumerate}
\end{enumerate}

\subsection{Economic Inequality}
\begin{enumerate}
\item Economic inequality is measured by the Gini coefficient, finding disparity in some measure across a population, where 0 means all are equal, 1 means all is owned by 1
\begin{enumerate}
\item The Lorenz curve, showing the population ordered by wealth on the horizontal axis, where further is a greater percentage of the population, and the height is the fraction of income owned by that percent of the population
\item The Gini coefficient is the area between the perfect equality line (y = x) and the Lorenz curve, divided by the total area under the perfect equality line
\item Other methods for measuring are the ratio of 90th percentile to the 10th percentile of income, or the fraction of income recieved by the top 1\%
\end{enumerate}
\item Countries which take large amounts of taxes, such as Scandanavia, have far lower levels of inequality after taxes (disposable) than Asian countries, even if the GDP growth and before-tax (market income) disparity is similar
\item Capitalist economies can become more or less unequal over time, and may be drastically different from that of other economies, mainly due to differences in taxation, rather than differences in market income, with the UK and US as unusually unequal
\begin{enumerate}
\item These differences can be measured by the distribution ratio, or the measure of the Gini disposable to the Gini market
\end{enumerate}
\end{enumerate}

\subsection{Economics}
\begin{enumerate}
\item Economics is the study of interactions of people with themselves and the environment to produce products, and the changes over time
\begin{enumerate}
\item Firms produce capital goods for themselves, while households produce caring labor for themselves
\item Households trade the labour force and other resources, for goods and services from firms
\item Both firms and households take raw materials, energy, and land from the environment, and return pollution and waste to the environment
\end{enumerate}
\item GDP estimates can be taken frm around the world, but price differences between times and countries must be taken into account, to only analyze the quantity of products
\begin{enumerate}
\item Basic GDP, ignoring price, is measured as nominal GDP, while real GDP takes into account changes in price in a single country over time
\item Prices can be converted by the exchange rate to a single currency, but that is not a measure of overall purchasing power in the country, instead using PPP (purchasing power parity) prices, attempting to find equal purchasing power conversion
\item Richer countries typically have higher prices, such that the currency conversion creates a far bigger difference than PPP difference
\end{enumerate}
\end{enumerate}

\section{CORE Chapter 14 - Monetary Policy}
\subsection{Inflation vs Unemployment}
\begin{enumerate}
\item While unemployment has a more direct effect, since inflation goes up as unemployment goes down, it is generally considered more important to force down
\begin{enumerate}
\item Inflation targeting through monetary policy from the central bank resulted from stagflation in the 1990s, working to control the supply economy as much as demand
\item Fixed nominal income, such as pensioners, are hurt by inflation, even when indexation increases it by the previous years inflation, such that if inflation goes up further, more is lost
\item High inflation creates volitile rates, making the economy work less effectively by creating uncertainty, by making it difficult to determine scarcity of resources relatively rather than price changes
\item Deflation prevents spending on the other hand, leading to a drop in aggregate demand and activity
\item As a result of the relationship, during an expansion, if not controlled properly, the interest rate can increase too high for stability
\end{enumerate}
\item The Phillips curve can either be drawn with less inflation as higher on the y-axis, or more inflation, such that it is a decreasing concave up, tolerating less additional inflation when high even as employment increases or vice versa
\begin{enumerate}
\item They can also be drawn with more inflation higher on the y-axis, adjusted similarly
\item Unemployment is always at the lower values of the x-axis
\item Inflation occurs in an attempt to divide up a specific amount of goods, until the bargaining power makes it proportional
\item As employment is high (tighter labor market), wages are forced up in an increasing concave up as bargaining power goes up, while as employment rises, prices are forced up to compensate similarly
\item In addition, as production capacity use rises, investment spending tends to increase to expand production, temporarily having capacity constrained until further machines are created
\item As a result, since there is less competition needed to fully sell goods, the price level is able to rise in a concave increasing curve related to it
\item This can be viewed as the bargaining power of both firms and workers simultaneously rising, such that they both try take a larger portion of output (split between real wage and firm profit), until price level increases, raising prices until both groups are content, as part of the percentage model
\end{enumerate}
\item Indifference curves are curves which give the same levels of satisfaction, with the opposite concavity to the Phillips, but going in the same direction, throughout the graph
\begin{enumerate}
\item Policymaker indifference curves are such that they ideally want the point on the curve with the lowest general unemployment and inflation, but which also must intersect with the Phillips curve
\item Policymakers who are unemployment-averse have indifference curves toward low unemployment, while inflation-averse is vice versa, based on personal and constituent interests
\end{enumerate}
\end{enumerate}

\subsection{Monetary Policy}
\begin{enumerate}
\item Central banks use changes in the nominal policy interest rate to stabilize the economy by a desired real interest rate, leading to a desired demand shock
\item Policy interest rate directly changes market interest rates, asset prices, expectations and confidence (consistant, transparent policy and good communications can lead to proper expectations in the economy and spending, encouraging the desired direction)
\begin{enumerate}
\item These all change investment spending, shifting up the aggregate expenditure function
\item Commercial banks set market rates for loans, though central bank determination results in a similar change in market rate
\item Central banks set the rate by making a goal for aggregate demand, finding the real interest rate that will produce the goal demand by the aggregate expenditure function, using it to find the market interest rate to produce that, then the policy rate
\item Lowering of the policy interest rate causes general interest rates to go down, such that the value of assets with fixed rates (such as bonds) go up (new assets produced provide less return from interest over time), making people feel wealthier
\end{enumerate}
\item Monetary policy is extremely flexible, more so than fiscal policy, but there is a zero lower bound on the policy rate, but in extreme recessions, it may not be low enough
\begin{enumerate}
\item Qualitative easing is the policy of the central bank buying bonds and financial assets, moving up the price, reducing the interest rate yield of the assets, such that while the market interest rate may not be low enough, other asset rates are low to compensate 
\item This serves to drive down consumption spending, by reducing the difficulty of borrowing and the return of saving
\end{enumerate}
\item Monetary policy is also limited in a common currency area, such as the Eurozone, where a single bank sets the policy for many countries, which may not be applicable for all
\item Policy interest rate also has the ability to influence the exchange rate, especially in smaller nations, due to currency acting similarly to bonds in other countries, since currency acts as the method of purchasing government bonds, reducing demand for the currency as well
\begin{enumerate}
\item On the foreign exchange market, demand decreases as rate decreases, such that supply and price/exchange rate decreases as a result, causing deprication, or vice versa
\item As a result, exports become cheaper for foreigners, and imports become more expensive as interest rate falls, increasing net exports, and thus aggregate expenditures and income
\end{enumerate}
\end{enumerate}

\subsection{Demand and Supply Shocks}
\begin{enumerate}
\item After the tech bubble in the early 2000s, as a result of the negative demand shock, the Fed had to drop nominal interest rate from 3.9\% to 1.1\% to help investment recover
\begin{enumerate}
\item Expansionary fiscal policy had to be used during while waiting for investment to recover until 2003, when it became positive again, by tax cuts and increased spending
\item Inflation and GDP growth was fixed rapidly, though unemployment was slower as usual, and never fully reached its old level, possibly due to above capacity economic status before
\item This shifted to a better indifference curve, with slightly higher inflation, but lower unemployment than after the shock
\end{enumerate}
\item The oil price jumps from 1973-74, 79-80, and 2002-08 caused large negative supply shocks and negative demand shock, the latter due to higher priced imports, forcing less to be spent on consumer products in the short term, more on imports, due to needing oil
\begin{enumerate}
\item Supply shocks in general also cause a shift of the phillips curve, such that unemployment and inflation either increase or decrease simultaneously, moving to different indifference curves
\item In oil shocks, the supply side far outweighs the demand side, creating higher inflation
\item This can be thought of as fewer domestic output due to more imports, such that both workers and firms want to remain the previous amount, fighting through bargaining power, leading to inflation 
\end{enumerate}
\end{enumerate}

\subsection{Phillips Instability and Inflation Targeting}
\begin{enumerate}
\item The Phillips curve is not stable, but rather to keep unemployment at a value below the equilibrium value, creates rising inflation in the long term
\begin{enumerate}
\item It is considered an unfeasible set, such that as the point on it changes, the curve itself shifts as well, unlike the aggregate expenditure curve
\item This is due to inflation expectations, measured based on the previous year, included in the wage claims, either to compensate for less real wages than expected (due to rising inflation), or to prevent less real wages in the coming year than desired
\item This causes the bargaining gap, or the overlap in the claims of workers and employers, plus the expected inflation, to result in the inflation outcome
\item Thus, bargaining gaps tend to result in unexpectedly high inflation, which continues to rise as the gap persists, creating higher future expected inflation at every unemployment level, pushing the phillips curve upward
\item On the other hand, as bargaining power of some group falls, such as unions in the 1980s, the bargaining gap can lower, and possibly shift the curve downward
\end{enumerate}
\item The stable inflation rate of unemployment, or the natural rate of unemployment, is when the bargaining gap is 0, such that the inflation rate remains stable
\item Before the global 2008 financial crisis, most nations had low inflation, unemployment, and stable growth, called the great moderation
\begin{enumerate}
\item This was due partially to inflation targeting, using a central bank, independent of the government, to keep inflation at an explicitly stated target rate, to keep it at the stable rate (the maximum central bank indifference curve accessible)
\item Central bank independence was increased to prevent low unemployment political stances from moving it from the target rate (2\%)
\item There was a direct correlation between low inflation rate and central bank independence
\end{enumerate}
\item During the early 2000s oil shock before the crisis, banks were willing to loan to prevent decreased aggregate demand, relying on rising housing prices as collateral
\begin{enumerate}
\item The central bank also created confidence that the rate would remain constant, preventing increased expectations
\item Finally, decreased union membership decreased worker bargaining power, preventing the bargaining gap from growing
\end{enumerate}
\end{enumerate}
