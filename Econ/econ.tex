\documentclass[11 pt, twoside]{article}
\usepackage{textcomp}
\usepackage[margin=1in]{geometry}
\usepackage[utf8]{inputenc}
\usepackage{color}
\usepackage{setspace}
\usepackage{tikz}
\usepackage{amsmath}
\usepackage{amsfonts}

\begin{document}

\title{Macroeconomics}
\author{Avery Karlin}
\date{Fall 2015}

\maketitle
\newpage
\tableofcontents
\vspace{11pt}
\noindent
\underline{Main Textbook}: Krugman's Economics for AP\\
\underline{Secondary Textbook}: CORE Project Economics\\
\underline{Teacher}: Schweitzer
\newpage

\section{CORE Chapter 1}

\subsection{National Differences}
\begin{enumerate}
\item In the 1300s, most of the world was fairly equal in the general amount of wealth of the population, even if there
were large differences between the rich and poor, often depending on parental status
\begin{enumerate}
\item Gross domestic product per capita, or average living standards, has raised in the last 700 years, but caused
differences by country, due to having a sudden rise at different times, leading to different standards
\item Often, independence from colonial rule or European interference caused the sudden economic growth, but Latin
America did not have the growth
\end{enumerate}
\item The ratio scale has the GDP y{}-axis go up by some multiple, used to compare growth rate, or (${\Delta}$GDP)/(GDPstart), such that 100\% means it doubles if ratio of 2
\begin{enumerate}
\item The ratio scale thus has the slope of the graph as the growth rate
\item Thus, the GDP per capita appears as a hockey stick curve, remaining without much growth, before a kink, leading to
a sudden rise
\end{enumerate}
\item Adam Smith argued that coordination of all the aspects of an economic culture from different parts of the world,
would be created on its own based on self{}-interest, rather than made by the government
\begin{enumerate}
\item He did believe there was ethical beliefs guiding behavior, and feared monopolies, especially government protected
\item He approved of government investment in education and public works, as well as justice and foreign policy through
the government
\end{enumerate}
\end{enumerate}

\subsection{Average Living Standards}
\begin{enumerate}
\item GDP per capita is different from average disposable income, which is close to average living standards, but ignores some aspects of happiness
\begin{enumerate}
\item Disposable income is the total income from factor markets, tranfer payments from both government, and from others, minus any governemtn taxes
\item Quality of social and physical environment, government services, and goods produced within a household, are important to well-being, but ignored by disposable income
\item Average disposable income also ignores distribution, due to extra income affecting the rich less than lack of affects the poor
\item People are also less happy based on how their income is related to others within the population, even if they have enough income to not be in poverty
\end{enumerate}
\item GDP is better when evalutating government services, but can only be measured easily through cost to produce, rather than selling value
\end{enumerate}

\subsection{The Modern World}
\begin{enumerate}
\item The sudden spike GDP per capita leading to the hockey stick appearence was caused by the Industrial Revolution, or major technological advances in textiles, energy, and transportation, leading to old skills rapidly becoming outdated
\begin{enumerate}
\item Technology is the process of creating an output from inputs, shortening the time to make the product as technology improves
\item The Industrial Revolution began a permenant technological revolution, or the drastic increase in technology at an exponential rate, allowing a drastic increase in living standards
\end{enumerate}
\item Modern technology has allowed international financial transactions to take place almost instantaneously, such that the speed of news travel is increasing exponentially
\item Population has also exponentially grown in recent centuries, with the improvement of public health and water services, though it has begun to slow in recent years, called the demographic transition, due to the desire for less children and fewer-child policies
\begin{enumerate}
\item At the same time, due to the increase in agricultural technology, less land and farmers were needed for the same output, leading to the growth of cities, creating additional interaction, the need for police
\end{enumerate}
\end{enumerate}

\subsection{Environmental Impact}
\begin{enumerate}
\item As production increased, environemtnal damage, especially climate change due to burning of fossil fuels (gas, oil, or coal) increasing $CO_2$ emissions
\begin{enumerate}
\item While temperature has always fluctuated due to volcanic events, such as the 1815 Mount Tambora eruption in Indonesia, lowering temperature in 1816, but has drastically risen over the last century
\item This can lead to rising sea levels, climate changes destroying farming, and polar ice cap melting, as well as cause respiratory issues and illnesses in cities
\end{enumerate}
\item The economy is a portion of society, which is within the overall biosphere, such that it can affect the other aspects
\item Overuse of local resources is also a possible environmental issue, caused along with climate change by economic expansion and organization (resources valued and conserved)
\begin{enumerate}
\item While the permanent technological revolution gave fossil fuel dependence, but has also permitted vastly more electricity for fewer resources, permitting the development of alternate sources, which may change it
\end{enumerate}
\end{enumerate}

\subsection{Capitalism}
\begin{enumerate}
\item An economic system is the organization of the production and distribution of products in an economy
\item Capitalism is the economic system made up of institutions, or sets of laws and social customs regulating the economy, of private property, markets, and firms
\begin{enumerate}
\item In prior economies, families typically were the third major institution, or the economy was regulated by a centralized government
\item Private property is possessions purely controlled and owned by an entity, able to control use or give ownership
\item Markets facilitate the transfer of goods in a reciprocal, voluntary trade
\item Firms are the main organization of production, owning the capital goods (which diffrentiates it from other economic organizations and systems), paying wages, managing employees, such that the products are private property of the owner, to make a profit
\begin{enumerate}
\item Firms are the main organization, with others being families, unions, government agencies, and non-profits
\item Firms utilize the labor market, unlike other organizations, such that firms can be created, destroyed, expand, or contract extremely quickly
\end{enumerate} 
\end{enumerate}
\item Capitalism both centralizes power in the hands of firm owners, and decentralizes from the government and other outside influences, creating power and cooperation inside, but competition outside
\begin{enumerate}
\item Capitalism relies on being dynamic, such that private property is secure, markets are fully competititive, and firm leadership is meritocratic, rather than aristocratic
\item This makes it unique, such that being in the upper classes is purely based on performance, creating private incentives for constant innovation, even if well-connected individuals can avoid failure easier than others
\item Government policies are also necissary for capitalism, to preserve said economic conditions, by a proper legal system, anti-monopoly regulations, lack of bail-outs, and providing essential products to all citizens, such as infrastructure, education, and defense
\item Thus, the capitalist revoltion, starting in the UK, was the creation of the private incentive, meritocratic, and public policy conditions, needed for dynamic conditions
\end{enumerate}
\item Defintions are used, not to set a specific rule, but to set general catagories which something can fall within to different degrees, such as China's five year plan with capitalism
\begin{enumerate}
\item Low quality of institutions, especially in later capitalist countries, such as African nations, caused misdirection of government funds and corruption, hurting growth
\item Asian nations, such as South Korea, Japan, and China, grew drastically, due to government promotin of industry, high quality education, and foreign trade requirements, creating a developmental state
\item Post-USSR nations also had slow growth, due to the lack of competitive markets, meritocracy, or secure private property (due to weak government and legal bodies)
\end{enumerate}
\item Capitalism can be shown to mainly be the cause of the economic growth and other effects within the last several centuries through natural experiments, or circumstances where the cause changes, to allow the change in result to be measured
\begin{enumerate}
\item One major example is that of East and West Germany, and the vastly higher GDP increase of the capitalist West, rather than the centralized East
\item Natural experiments replace planned experiments, due to the difficulty in experimentation, because of the scope and stakes of economic changes
\end{enumerate}
\item Government plays a large role in capitalism, as a major consumer, and that which establishes the legal structure to make it possible, such that different political structures creates different forms of capitalism
\begin{enumerate}
\item Democracies, the most common government with capitalism, ranges in terms of the structure, influence of big money, and individual rights
\item Some governments take amounts of GDP as revenue, >50\%, like Scandanavia, playing a large role in the economy as a result, while others take relatively little, but have a large degree of control over the economy, such as Asia
\end{enumerate}
\end{enumerate}

\subsection{Economic Inequality}
\begin{enumerate}
\item Economic inequality is measured by the Gini coefficient, finding disparity in some measure across a population, where 0 means all are equal, 1 means all is owned by 1
\begin{enumerate}
\item The Lorenz curve, showing the population ordered by wealth on the horizontal axis, where further is a greater percentage of the population, and the height is the fraction of income owned by that percent of the population
\item The Gini coefficient is the area between the perfect equality line (y = x) and the Lorenz curve, divided by the total area under the perfect equality line
\item Other methods for measuring are the ratio of 90th percentile to the 10th percentile of income, or the fraction of income recieved by the top 1\%
\end{enumerate}
\item Countries which take large amounts of taxes, such as Scandanavia, have far lower levels of inequality after taxes (disposable) than Asian countries, even if the GDP growth and before-tax (market income) disparity is similar
\item Capitalist economies can become more or less unequal over time, and may be drastically different from that of other economies, mainly due to differences in taxation, rather than differences in market income, with the UK and US as unusually unequal
\begin{enumerate}
\item These differences can be measured by the distribution ratio, or the measure of the Gini disposable to the Gini market
\end{enumerate}
\end{enumerate}

\subsection{Economics}
\begin{enumerate}
\item Economics is the study of interactions of people with themselves and the environment to produce products, and the changes over time
\begin{enumerate}
\item Firms produce capital goods for themselves, while households produce caring labor for themselves
\item Households trade the labour force and other resources, for goods and services from firms
\item Both firms and households take raw materials, energy, and land from the environment, and return pollution and waste to the environment
\end{enumerate}
\item 
\end{enumerate}

\section{CORE Chapter 14 - Monetary Policy}
\subsection{Inflation vs Unemployment}
\begin{enumerate}
\item While unemployment has a more direct effect, since inflation goes up as unemployment goes down, it is generally considered more important to force down
\begin{enumerate}
\item Inflation targeting through monetary policy from the central bank resulted from stagflation in the 1990s, working to control the supply economy as much as demand
\item Fixed nominal income, such as pensioners, are hurt by inflation, even when indexation increases it by the previous years inflation, such that if inflation goes up further, more is lost
\item High inflation creates volitile rates, making the economy work less effectively by creating uncertainty, by making it difficult to determine scarcity of resources relatively rather than price changes
\item Deflation prevents spending on the other hand, leading to a drop in aggregate demand and activity
\item As a result of the relationship, during an expansion, if not controlled properly, the interest rate can increase too high for stability
\end{enumerate}
\item The Phillips curve can either be drawn with less inflation as higher on the y-axis, or more inflation, such that it is a decreasing concave up, tolerating less additional inflation when high even as employment increases or vice versa
\begin{enumerate}
\item They can also be drawn with more inflation higher on the y-axis, adjusted similarly
\item Unemployment is always at the lower values of the x-axis
\item Inflation occurs in an attempt to divide up a specific amount of goods, until the bargaining power makes it proportional
\item As employment is high (tighter labor market), wages are forced up in an increasing concave up as bargaining power goes up, while as employment rises, prices are forced up to compensate similarly
\item In addition, as production capacity use rises, investment spending tends to increase to expand production, temporarily having capacity constrained until further machines are created
\item As a result, since there is less competition needed to fully sell goods, the price level is able to rise in a concave increasing curve related to it
\item This can be viewed as the bargaining power of both firms and workers simultaneously rising, such that they both try take a larger portion of output (split between real wage and firm profit), until price level increases, raising prices until both groups are content, as part of the percentage model
\end{enumerate}
\item Indifference curves are curves which give the same levels of satisfaction, with the opposite concavity to the Phillips, but going in the same direction, throughout the graph
\begin{enumerate}
\item Policymaker indifference curves are such that they ideally want the point on the curve with the lowest general unemployment and inflation, but which also must intersect with the Phillips curve
\item Policymakers who are unemployment-averse have indifference curves toward low unemployment, while inflation-averse is vice versa, based on personal and constituent interests
\end{enumerate}
\end{enumerate}

\subsection{Monetary Policy}
\begin{enumerate}
\item Central banks use changes in the nominal policy interest rate to stabilize the economy by a desired real interest rate, leading to a desired demand shock
\item Policy interest rate directly changes market interest rates, asset prices, expectations and confidence (consistant, transparent policy and good communications can lead to proper expectations in the economy and spending, encouraging the desired direction)
\begin{enumerate}
\item These all change investment spending, shifting up the aggregate expenditure function
\item Commercial banks set market rates for loans, though central bank determination results in a similar change in market rate
\item Central banks set the rate by making a goal for aggregate demand, finding the real interest rate that will produce the goal demand by the aggregate expenditure function, using it to find the market interest rate to produce that, then the policy rate
\item Lowering of the policy interest rate causes general interest rates to go down, such that the price of assets with fixed rates (such as bonds) go up (new assets produced provide less return from interest over time), making people feel wealthier
\end{enumerate}
\item Monetary policy is extremely flexible, more so than fiscal policy, but there is a zero lower bound on the policy rate, but in extreme recessions, it may not be low enough
\begin{enumerate}
\item Qualitative easing is the policy of the central bank buying bonds and financial assets, moving up the price, reducing the interest rate yield of the assets, such that while the market interest rate may not be low enough, other asset rates are low to compensate 
\item This serves to drive down consumption spending, by reducing the difficulty of borrowing and the return of saving
\end{enumerate}
\item Monetary policy is also limited in a common currency area, such as the Eurozone, where a single bank sets the policy for many countries, which may not be applicable for all
\item Policy interest rate also has the ability to influence the exchange rate, especially in smaller nations, due to currency acting similarly to bonds in other countries, since currency acts as the method of purchasing government bonds, reducing demand for the currency as well
\begin{enumerate}
\item On the foreign exchange market, demand decreases as rate decreases, such that supply and price/exchange rate decreases as a result, causing deprication, or vice versa
\item As a result, exports become cheaper for foreigners, and imports become more expensive as interest rate falls, increasing net exports, and thus aggregate expenditures and income
\end{enumerate}
\end{enumerate}

\subsection{Demand and Supply Shocks}
\begin{enumerate}
\item After the tech bubble in the early 2000s, as a result of the negative demand shock, the Fed had to drop nominal interest rate from 3.9\% to 1.1\% to help investment recover
\begin{enumerate}
\item Expansionary fiscal policy had to be used during while waiting for investment to recover until 2003, when it became positive again, by tax cuts and increased spending
\item Inflation and GDP growth was fixed rapidly, though unemployment was slower as usual, and never fully reached its old level, possibly due to above capacity economic status before
\item This shifted to a better indifference curve, with slightly higher inflation, but lower unemployment than after the shock
\end{enumerate}
\item The oil price jumps from 1973-74, 79-80, and 2002-08 caused large negative supply shocks and negative demand shock, the latter due to higher priced imports, forcing less to be spent on consumer products in the short term, more on imports, due to needing oil
\begin{enumerate}
\item Supply shocks in general also cause a shift of the phillips curve, such that unemployment and inflation either increase or decrease simultaneously, moving to different indifference curves
\item In oil shocks, the supply side far outweighs the demand side, creating higher inflation
\item This can be thought of as fewer domestic output due to more imports, such that both workers and firms want to remain the previous amount, fighting through bargaining power, leading to inflation 
\end{enumerate}
\end{enumerate}

\subsection{Phillips Instability and Inflation Targeting}
\begin{enumerate}
\item The Phillips curve is not stable, but rather to keep unemployment at a value below the equilibrium value, creates rising inflation in the long term
\begin{enumerate}
\item It is considered an unfeasible set, such that as the point on it changes, the curve itself shifts as well, unlike the aggregate expenditure curve
\item This is due to inflation expectations, measured based on the previous year, included in the wage claims, either to compensate for less real wages than expected (due to rising inflation), or to prevent less real wages in the coming year than desired
\item This causes the bargaining gap, or the overlap in the claims of workers and employers, plus the expected inflation, to result in the inflation outcome
\item Thus, bargaining gaps tend to result in unexpectedly high inflation, which continues to rise as the gap persists, creating higher future expected inflation at every unemployment level, pushing the phillips curve upward
\item On the other hand, as bargaining power of some group falls, such as unions in the 1980s, the bargaining gap can lower, and possibly shift the curve downward
\end{enumerate}
\item The stable inflation rate of unemployment, or the natural rate of unemployment, is when the bargaining gap is 0, such that the inflation rate remains stable
\item Before the global 2008 financial crisis, most nations had low inflation, unemployment, and stable growth, called the great moderation
\begin{enumerate}
\item This was due partially to inflation targeting, using a central bank, independent of the government, to keep inflation at an explicitly stated target rate, to keep it at the stable rate (the maximum central bank indifference curve accessible)
\item Central bank independence was increased to prevent low unemployment political stances from moving it from the target rate (2\%)
\item There was a direct correlation between low inflation rate and central bank independence
\end{enumerate}
\item During the early 2000s oil shock before the crisis, banks were willing to loan to prevent decreased aggregate demand, relying on rising housing prices as collateral
\begin{enumerate}
\item The central bank also created confidence that the rate would remain constant, preventing increased expectations
\item Finally, decreased union membership decreased worker bargaining power, preventing the bargaining gap from growing
\end{enumerate}
\end{enumerate}

\section{Chapter 1}

\subsection{Study of Economics}
\begin{enumerate}
\item Economics is the study of scarcity and choice, mainly individual choice, as well as the economy, or the system
which coordinates choices about production and consumption, and distributes products
\begin{enumerate}
\item Market economies, like the US, is where productive and consumption are made by decentralized decisions of many
people
\item Command economies are those where industry is publically owned with a central authority for production and
consumption, typically failing due to lack of resources or being told to make unneeded products, not gathering
information as well, better for incentivizing needs, not complete control
\end{enumerate}
\item Economies rely on incentives, punishment or reward, for particular choices, such as higher prices for needed
products, causing more to be made
\begin{enumerate}
\item Property rights give ownership and allow trading, creating incentives to use resources for value
\item Marginal decisions balance cost{}-benefit, looked at by marginal analysis
\item Resources, which can be used to make something else, are scarce, or less than society desires, as incentives
\end{enumerate}
\item Factors of production, or resources, are divided into land, labor, capital (all manufactured goods to make other
goods, which are not used up in production), and entrepreneurship (firm ownership, not dependent on risk)
\begin{enumerate}
\item In a market economy, use of resources is based on the sum of individual decisions, though sometimes, when there is
no incentive, community decisions must interfere with the market for the general good
\item Opportunity costs are factors given up for a specific choice, such as time, money, or future prospects
\end{enumerate}
\item Macroeconomics are the study of the overall economy, mainly economic aggregates, or measures such as GDP,
unemployment, or inflation
\begin{enumerate}
\item Macroeconomics runs on the basis that the sum is greater than its parts, due to the overall dynamics, mattering
more than microfoundations
\item Microeconomics are the study of individual decisions of people, firms, or markets/industries
\end{enumerate}
\item Positive economics is definite factual questions about how the world actually works, rather than normative, or
uncertain questions about how it should work
\begin{enumerate}
\item The former deals with both economic forecasts, or predictions based on current conditions, and hypotheses of
predictions in different ones
\item Economic models are used to give simplified representations of reality, used for both types of positive analysis
\item Normative creates value judgements, up to opinion, unless there is a clear beneficial advantage of one, often
based on opportunity costs, not using models, but rather prior ideas and models for other measures
\end{enumerate}
\item Disagreements can be created by differences in values, or on the model of reality, exacerbated by political
interests
\end{enumerate}

\subsection{Intro to Macro}
\begin{enumerate}
\item The business cycle is the alternating cycle of down and upturns
\begin{enumerate}
\item Depressions are a very deep, long downturn with product output and employment falling, while shorter downturns are
called recessions
\item Expansions and recoveries are the opposite periods of upturn, typically lasting almost 5 years (57 months), rather
than 10 months of recessions
\end{enumerate}
\item Macroeconomic analysis is used to minimize the fluctuations of the economy
\begin{enumerate}
\item Unemployment is the number of people looking for work actively, who are not working, while the labor force is the
unemployed + employed, and the unemployment rate is the percentage of the force unemployed
\item Unemployment rate is a good economic indicator, though even during an expansion, there is a small unemployment
rate
\end{enumerate}
\item Aggregate output, or the total amount of goods and services produced in a given amount of time, is another
economic indicator
\item Inflation is a rise in the overall price level, while deflation is the opposite, the former discouraging saving,
and eventually making money worthless
\begin{enumerate}
\item Deflation encourages saving, instead of reinvesting to allow the economy to regrow, with price stability being the
most desirable
\end{enumerate}
\item Economic growth, or an increase in the maximum possible output, is an overall sustained rise over a long period of
time, outside the business cycle, allowing higher wages and standard of living
\begin{enumerate}
\item On the other hand, economic growth can be bad for stability of the business cycle, and vice versa
\end{enumerate}
\item Models are a simplified version of reality, studying economies in a smaller setting, such as a WWII prison for
cigarettes, or on a computer simulation
\begin{enumerate}
\item The other things equal (ceteris paribus) assumption is used to only study one change, by making all other factors
constant
\item Thought experiments, or simple, hypothetical scenarios, are another effective way of modeling, as well as graphing
\end{enumerate}
\end{enumerate}

\subsection{Production Possibility Curve Model}
\begin{enumerate}
\item Trade{}-offs are when something is giving up the opportunity costs of something for that of another option,
analyzed by the PPC
\begin{enumerate}
\item The PPC model assumes only two goods produced, such that points within are feasible, but not optimal/efficient,
while points on are both
\item The slope determines if the trade{}-off is constant, called a constant opportunity cost, often not true, due to
having to use less suited resources as the production increases, thus getting less and losing more
\item Input problems find the trade{}-offs to gain the same output of different products, while output find for the same
input for different products
\end{enumerate}
\item Efficiency in production is the lack of missed opportunities, or optimal improvement to one{}'s self, without
hurting others, exampled by unemployment of those who want work
\begin{enumerate}
\item Efficiency in allocation is the maximization of consumer happiness by the optimal production of the correct goods
\item Overall efficiency requires both in allocation and production
\end{enumerate}
\item Economic growth can also be defined as the expansion of production possibilities, shifting the curve outward,
since products made shift
\begin{enumerate}
\item This is typically caused by increase in resources or technology, the technical means of production of products
\item Since only one product on the curve may shift, there is a chance production may not rise, even as there is growth
\end{enumerate}
\end{enumerate}

\subsection{Comparative Advantage and Trade}
\begin{enumerate}
\item Trade is the division of tasks, such that people trade goods and services for those they want
\begin{enumerate}
\item Gains from trade are caused by specialization, due to engaging in a specific task allowing the production of more of the good
\item This is due to the time required for skill development in a field
\item This also results from comparative advantage, or the idea that some people are better at certain actions than others, resulting in a lower opportunity cost for production
\item People will only accept deals that cost less than their personal opportunity cost for production (terms of trade)
\end{enumerate}
\item Absolute advantage is the general ability to produce more, under any relative distribution of resources
\begin{enumerate}
\item Comparative advantage creates the mutual benefits of trade, not absolute advantage
\end{enumerate}
\end{enumerate}

\section{Chapter 2 - Supply and Demand}

\subsection{Intro to Demand}
\begin{enumerate}
\item Competitive markets are a market with many buyers and sellers of the same products, where a market is a group of consumers and producers exchanging products for payment
\begin{enumerate}
\item Thus, individual actions must not have a noticable effect on the price, such that in non-fully competitive markets, it doesn't apply completely
\item It is described by the demand and supply curves, sets of factors which cause each to shift, market equilibrium, and how market equilibrium changes when the curve shifts
\end{enumerate}
\item The demand for any good depends on the price, making a demand curve of the quantity demanded vs price, first making a demand schedule table of points
\begin{enumerate}
\item The quantity demanded is the amount consumers are willing to buy at a particular price
\item Demand curves typically have a downward slope, not always constant, such that the law of demand states that as price decreases, demand increases, and vice versa
\item Due to all other things equal, the curve does not account for changes in the world, such that changes in taste, income, related prices, number of consumers, or expectations (either in income or price) can shift the curve outward
\begin{enumerate}
\item Changes in demand are at the same price, while a movement on the curve are at a different price
\item Related good price changes are in goods which are substitutes, such that people are more willing to buy the other if price rises, or complements, which are goods that people are more willing if the price of the other falls
\item Normal goods are those where demand increases as income does, unlike inferior goods, typically those with better, more expensive alternatives
\item Number of consumers can change due to population, such that the individual demand curve (demand curve for a single person, such that the market curve is the horizontal sum) may not shift, but the market curve does 
\item One exception is conspicuous consumption, or goods which people purely buy due to high price to gain social status, or goods so cheap they can no longer be considering the same product
\end{enumerate}
\end{enumerate}
\end{enumerate}

\subsection{Supply and Equilibrium}
\begin{enumerate}
\item The quantity supplied when offered a specific price also varies with price, such that a schedule, and curve can be produced, forming a law of supply, where if price rises, supply will as well
\item Changes in supply can be caused by changes in input (items needed to produce the product) prices, related goods price, technology (methods used for production), expectations, and number of producers
\begin{enumerate}
\item Often, several related products are produced by the same producer, such that as the price of one good rises, the others (substitutes in production) are produced less
\item Biproducts of the same process are compliments in production, and will be made more
\end{enumerate}
\item The interaction of supply and demand creates equilibrium where supply is the same as demand, at the equilibrium/market-clearing price and quantity
\begin{enumerate}
\item On the same graph, the equilibrium point is the intersection of the two curves
\item In all established, ongoing markets, people converge toward a single market price, which is most beneficial to all parties involved, and the price moves to prevent surpluses or shortages
\end{enumerate}
\end{enumerate}

\subsection{Changes in Equilibrium}
\begin{enumerate}
\item Changes in equilibrium cause the shift of either the supply curve, the demand curve, or both simultaneously
\item When demand increases, the equilibrium price and quantity increase, and vice versa, such that the curve moves rightward and the intersection moves up-right
\item When supply increases, the equilibrium price decreases, but the quantity increases, such that the curve moves rightward and the intersection moves down-left
\item Simultaneous shifts depend on the relative shifts to determine in which way the equilibrium moves, such that one direction can be determined, but the other is ambiguous
\item Events in the short term can only change either supply or demand, not both, though in the long run, it moves toward equilibrium, causing a change
\end{enumerate}

\section{Chapter 3 - Economic Performance Measurement}
\subsection{Circular Flow and GDP}
\begin{enumerate}
\item The national income and product accounts, or national accounts, keep track of the flow of money from consumers to producers, as well as business investment, or government purchases
\begin{enumerate}
\item Accuracy of national accounts is an indication of how economically advanced a country is
\end{enumerate}
\item The circular flow diagram represents the flow of money in the economy, with money flowing from a household, or group of people sharing an income, for goods and services (spending)
\begin{enumerate}
\item The money can then flow from the market for goods and services to the firms (revenue), or organizations that employ households and make products, in exchange for products
\item Money then goes from the firms to the factor markets (costs), which give money in exchange for factors of production, especially labor from households (income)
\item Thus, the idea is that the money flowing into or out of each market or organization is equal to the money flowing out
\item All factors are assumed to be rented from the factor market from households, due to household ownership of firms, within the simplified model
\item This model assumes lack of saving, no surplus outputs, and no alternate sources of leakages (money flowing outside the simple circular flow model)
\end{enumerate}
\item This can be extended into a more complex model, adding the government, foreign nations, and the financial market, such that the flow from households to product market is consumer spending
\begin{enumerate}
\item In addition to selling labor, households use stocks, or firm shared ownership, and bonds, or loans to firms with interest, from the financial market to firms, which eventually goes back into the factor market to households in profits and interest
\item Rent is also given by households to the firms in exchange for land resources, through the factor market
\item While people spend most disposable income in the product market, some of total income is lost through taxes, while some can be gained through government tranfers, such as unemployment payment, given without a reciprocal service
\item Some disposable income is lost as private savings in the financial market, which, in addition to providing money to firms allows government borrowing
\item Government funds is also used for government purchases
\item Goods sold to other countries are exports, while those purchased are imports, as part of the product market, and foreign nations also participate in borrowing and investing in the financial market
\item Finally, firms also purchase products from the product market through investment spending, adding to their inventories, or raw materials and capital used for production
\end{enumerate}
\item Gross domestic product (GDP) is thus the sum of government purchases, investment, and consumer spending, minus imports, or the total of final goods and services produced during a period
\begin{enumerate}
\item Final products are those sold to the final user, while intermediate products are those that are inputs into the production of the final product, such that capital is final, while resources are intermediate
\end{enumerate}
\item GDP can be measured by adding the total value of production of final products (or sum of value added of all products), aggregate spending on domestically producted final products, or total factor income earned by households from domestic firms
\begin{enumerate}
\item Intermediate products are ignored for total value of production due to their value being added to that of the final product, such that they would be summed multiple times otherwise
\item Each product has value added of interest, rent, profit (both employee profits, and those paid to shareholders as dividends), and wages on that product, combined with the price of all intermediate products for the total value
\item Total factor income is found by the sum of each type of factor payment, such that it is the sum of total wages, interest, rent, and profits from each product, including intermediate
\item GDP by aggregate spending = consumer + investment + governemnt + exports - imports, where exports - imports can also be called net exports
\item Stocks and bonds are not counted due to not representing the sale of final goods or the direct production of final goods
\end{enumerate}
\item GDP can be calculated practically through the sum of aggregate spending or the value added by each sector of the economy (business, household labor, and government services)
\item Capital goods are eventually used up, such that it is accounted for by depreciation, such that net domestic product is GDP - depreciation
\begin{enumerate}
\item It can also be done by replacing investment spending with net investment, or investment - depreciation
\item Depreciation is the cost of the capital, divided by the number of years it was used for
\end{enumerate}
\item Products from the underground economy are not counted in the GDP, and those who work in the markets are counted as unemployed
\end{enumerate}

\subsection{Real GDP}
\begin{enumerate}
\item Stock variables are those measured at a specific point in time, such as nominal capital stock (value of all national assets at one time) while fluid variables are those measured over a period of time, such as nominal GDP
\item GDP describes the size of the economy, but measures the price of total products produced, rather than the amount of output, such that real GDP adjusts for price changes to describe aggregate output (the quantity of final products produced)
\begin{enumerate}
\item This prevents inflation or rise of prices from causing the increase in GDP, without increased aggregate output
\item Real GDP is calculated as the GDP if the price had remained constant from some base year, such that it is the total value of final products, assuming that price level
\item Nominal GDP is another term for non-real GDP
\item Since Real GDP can be calulated from a late or early base year, gaining different results, chain-linking measures the average of the two values, expressed in chain dollars
\end{enumerate}
\item Real GDP comparisons assume equal population, such that GDP (or real GDP) per capita (divided by the size of the population) can be used to account for that
\begin{enumerate}
\item Real GDP per capita compares labor productivity, but it is not a complete measure of living standards, but rather of potential living standards
\item It can also be difficult to determine who counts for population purposes
\item High living standards requires health, education, and good quality of life spending, rather than expenditures on negatives such as natural disasters or disease
\item Real GDP per capita also does not include other components to high standard of life, which are non-monetary, such as nature or leisure, distribution, and workplace health and safety
\end{enumerate}
\end{enumerate}

\subsection{Unemployment}
\begin{enumerate}
\item Unemployed are only those able to and actively (in the last 4 weeks) looking for work, such that retired or disabled people
\item Labor force participation rate is the percentage of the population $\geq$ 16 in the labor force (employed for pay or looking for work actively)
\item The US Census Bureau takes a Current Population Survey of 60k families, asking if they qualify as unemployed, to estimate the total unemployment rate
\item Unemployment rate can overstate the level of unemployment, such that those who are not working, but will easily be able to get a job, simply due to taking a few weeks to get a job, increase the rate and make sure it is never 0\%
\item It may also understate unemployment, due to marginally attached workers, who want a job and have looked for work in the past, but not currently due to some temporary, unexpected issue
\begin{enumerate}
\item Discouraged workers, who don't feel they will be able to find a job, and thus don't search, having been employed for a long time
\item Underemployed are those working part-time, due to lack of full-time work, not included in the unemployment rate, which may make it further understate it
\item The Bureau of Labor Statistics also makes measures of labor underutilization, the most broad of which is U6 including marginally attached and underemployed
\item U6, while far higher, typically mirrors movement of the standard unemployment rate
\end{enumerate}
\item Unemployment rate ignores demographic differences, due to it being far easier for experienced workers in the field, and prime working years workers (25-54 years old) to find jobs, while blacks have a more difficult time than most groups
\begin{enumerate}
\item Even when the unemployment rate is low, it may be higher than normal for certain groups
\end{enumerate}
\item During recessions, the unemployment rate always rises, while the opposite is often true during expansions, though not always, due to discouraged workers reentering the labor force
\begin{enumerate}
\item Generally though, above average increase in Real GDP resulted in decrease in unemployment rate
\item Periods when unemployment is rising during an expansion is during a growth recession
\item Thus, it can be used as an indicator for the quantity of those in the labor market for hiring, the likelihood of being fired, and the ease of finding a new job
\end{enumerate}
\end{enumerate}

\subsection{Causes and Catagories of Unemployment}
\begin{enumerate}
\item Even in low unemployment time periods, there are still amounts of job creation and seperations (quitting or firings), due to the structural change in the economy as change in supply due to taste or technology
\begin{enumerate}
\item Individual companies also create job loss due poor management or luck
\end{enumerate}
\item Those engaged in the job search typically don't take the first job, rather looking for one that suits their skills and pays properly, creating fractional unemployment
\begin{enumerate}
\item In addition, new people entering the labor market results in further fractional unemployment, including re-entrants, or those who entered after leaving
\item Small numbers ensure people are in jobs matched to their skills, and people tend to remain unemployed for a short period of time when fractional, though the time is increased by large benefits
\end{enumerate}
\item Structural unemployment is when the supply of workers is greater than the demand at the current wage price, due to the wage not decreasing as it should, creating a persistant surplus
\begin{enumerate}
\item This can be caused by minimum wage laws (though some argue that low-skill employers restrict highering to keep wages low, such that higher wages won't decrease jobs, explaining why many countries with higher minimum wages have low unemploymnet), or unemployment benefits (which are worth more than lower wages)
\item Labor unions, and threating strikes, can lead to higher wages (and benefits, which add to the wages)
\item Firms can also give efficiency wages for increased performance (labor extraction) above equlibrium, due to fear of firing and losing the higher wages, creating structural unemployment
\item Skills mismatch, or geographic immobility, also create rigidities which keep structural employment up
\item Structural also includes changes in an industry for any unknown reason
\end{enumerate}
\item Natural unemployment rate is fractional and structural unemployment, or the normal value, though it can be changed over time and effected by policy
\begin{enumerate}
\item Changes in the labor force, such as increased numbers of young workers (who are inexperienced and keep jobs for a shorter time, and who have less incentive to find jobs), can change the natural rate
\item Changes in labor market institutions, such as temp agencies (which help workers find jobs) or job-search websites, as well as technology changes (increasing demand for workers skilled in the new technology) can cause changes
\item Policy changes such as job-training programs to increase skills, or employment subsidies (payments to workers or employers for giving/taking jobs) can change it
\item Labor-saving tech can also allow the increase in production, with the increase in unemployment rate, which in turn reduces incentive for labor saving tehcnological change
\end{enumerate}
\item Cyclical unemployment is the deviation from the natural unemploymnet, resulting from the business cycle, and can overlap with frictional somewhat
\begin{enumerate}
\item Labor hoarding theory states it doesn't fluctuate as much, due to firms keeping people to save on hiring costs, especially high value workers
\end{enumerate}
\end{enumerate}

\subsection{Inflation}
\begin{enumerate}
\item Inflation does not make people richer or poorer, due to real wage/income, or the wage/income divided by price level, remains constant, due to the value of money changing overall
\item Inflation rate, or the overall increase in prices, is the change in price divided by the original price * 100\%, such that inflation rate may decrease, but price level will rise as long as it is positive
\begin{enumerate}
\item Higher inflation rate typically goes along with unstable inflation rate
\end{enumerate}
\item Shoe leather costs are the increased cost of financial transactions due to inflation
\begin{enumerate} 
\item This is due to inflation discouraging holding money in wallets or checking accounts, due to the purchasing power decreasing, leading to mass banking transactions, growing the banking industry rapidly, to try and protect assets
\end{enumerate}
\item Menu costs are the cost of changing the price of a good in the system, such that rapid price changes from inflation causes high menu costs
\item Unit of account costs are due to the use of money as a unit-of-account, signifying payment in contracts, such that if it is unstable, it cannot be used as easily
\begin{enumerate}
\item Thus, the gains over a year may be purely due to price changes, such that the value of assets didn't increase, called phantom gains, which are then taxed
\end{enumerate}
\item Nominal interest rate is the rate paid on a loan, while the real interest rate is nominal adjusted for inflation, subtracting the inflation rate from the nominal
\begin{enumerate}
\item If real interest rate is lower than expected, the money paid back will be worth less than expected, such that unstable currency results in unwillingness to enter contracts
\end{enumerate}
\item Disinflation, or the lowering of the inflation rate, is difficult once the rate is high, due to requiring a recession to depress the economy, increasing unemployment, such that government must stop inflation before the rate rises
\end{enumerate}

\subsection{Measuring Inflation}
\begin{enumerate}
\item Aggregate price level is the measurement of the overall price level of all products in an economy
\item The consumption bundle is the group of specific quantities of goods and services purchased in a time period, such that a market basket is the consumption bundle used to measure price changes
\begin{enumerate}
\item Price index is an aggregate price level, measured by some base year, such that it is the price of the market based divided by the price in a base year * 100\%
\end{enumerate}
\item Consumer Price Index is an aggregate of the typical market baset for a family of 4 in a city, taken by surveying market prices for those goods, with a 1982-1984 base period
\begin{enumerate}
\item It is typically put on a logarithmic scale to show the percent-increase over time
\item On the other hand, this tends to overstate the cost of living, as people change their personal goods purchased as prices rise, to cheaper goods, such that the market basket may not be the same as in the 1980s
\item In addition, there is more variety in goods, especially technology, due to innovation, increasing consumer choices, causing the drop in cost of living
\item Most countries calculate CPI, based on their own expected market basket
\end{enumerate}
\item Producer/Wholesale Price Index measured the cost of a market basket of raw non-capital commodities, purchased by producers
\begin{enumerate}
\item Commodity producers are quicker than other producers to raise prices as a result of increased demand, such that it often reacts faster than CPI, but fluctuates more than other measures
\end{enumerate}
\item GDP Deflator is the 100\% * (nominal GDP/real GDP) in terms of some given base year, not technically a price index, but used to measure approximate changes in price level
\item These aggregate price index measurements tend to move mostly in unison, such that they all result in similar changes
\end{enumerate}


\section{Chapter 4 - National Income and Price Determination}
\subsection{Income and Expenditure}
\begin{enumerate}
\item For aggregate spending change modeling, it must be assumed the interest rate is constant, and government spending, taxes, imports, and exports are zero
\begin{enumerate}
\item It is also assumed that producers will supply extra output at a fixed price, such that increase in demand will increase output without an increase in prices, true only in short-term
\end{enumerate}
\item Thus, if investment/consumer (antonomous change in aggregate spending) spending increases, both aggregate output and income would increase by that much, leading to increased consumer spending, and thus output
\begin{enumerate}
\item Maginal propensity to consume (MPC) is the rate of change of consumer spending as disposable income rises, while marginal propensity to save is the remaining amount of money (1 - MPC)
\item Thus, spending compounds the increase, such that total increase is the original amount, times the $(1 + MPC + MPC^2 + MPC^3 + ...) *$ original increase in spending, or (1/(1-MPC) * increase) = (increase/MPS) = (increase * multiplier)
\end{enumerate}
\item Consumer spending is a majority of spending, such the consumption function shows that as disposable income rises, consumption also does, or $c = a + MPC * y$, where c is single household spending, y is income, and a is the amount that would be spent if income was 0, or autonomous consumer spending, often through loans or savings
\begin{enumerate}
\item Aggregate consumption function is the same equation, and can be thought of as the horizontal sum of the individual functions
\item Consumption function can be shifted by change in expected future disposable income, such that autonomous spending changes, shifting the curve up/down
\item Thus, those with higher income often expect it to fall, and vice versa, saving a higher percentage, but during economic expansion, but during economic expansion, since future and current rise together, it is difficult to predict
\item Based on this, the permanent income hypothesis states spending is based on the long term expected income, rather than current
\item Aggregate wealth can also affect it, such that an increase in net worth can cause the curve to rise, and can also affect it by the life cycle hypothesis (specifically stating that people save more in assets until after peak working years)
\item Price levels are an exogenous variable on the other hand, such that it changes unaffected by disposable income
\end{enumerate}
\item The Keynesian cross states that in an economy, total income (or supply due to GDP being total paid through the factor market) and total expenditure (or demand) must be equal, such that the intersection of that line and the consumption curve is the equilibrium 
\begin{enumerate}
\item When not at equilibrium, unplanned investment spending compensates for the difference, quickly moving back to equilibirum
\end{enumerate}
\item Investment spending is a smaller amount, but often far more dramatic in changes, causing the business cycle, unaffected by disposable income changes
\begin{enumerate}
\item Planned investment spending, or the expected amount in a year, depends on interest rate (even if retained earnings, or past profits, are used, due to the same trade-off through lending the earnings to gain interest)
\item It also depends on expected future GDP (expected sale growth causes investment spending growth) and production capacity (excess capcity decreases investment spending, due to lack of need)
\item Actual investment spending is made up of planned, and unplanned inventory spending (due to trying to keep a proper inventory size, such that excess inventory remains as unplanned positive investment)
\end{enumerate}
\item Aggregate expenditures is the total spending on GDP, modeled by the aggregate expenditures function combined with additional spending on investment, government, and net export spending (aggregate autonomous spending)
\begin{enumerate}
\item Aggregate demand can be thought to be the aggregate expenditures, taking into account a range of prices
\end{enumerate}
\end{enumerate}

\subsection{Aggregate Demand and Determinants}
\begin{enumerate}
\item Positive or negative demand shock is the shifting of the aggregate demand curve of an economy, or the relationship between aggregate price level and aggregate output demanded (Real GDP), depicted similarly to a market demand curve
\begin{enumerate}
\item The downward slope is not due to the law of demand, since that assumes ceteris paribus, while aggregate demand assumes a simultaneous price change in all final products
\item The wealth/real balances/real assets effect of change in aggregate price level is the change in consumer spending, due to the decrease in value of assets during inflation
\item The interest rate effect in aggregate price level is the result of the attempt to borrow enough money to possess the same purchasing power, selling assets or borrowing, where the surge of loans drives interest rates up, preventing consumer and investment spending (especially the latter)
\item The foreign purchases effect in aggregate price level states that as national prices fall relative to other nations, the demand for nationally-produced goods will increase foreignly
\end{enumerate}
\item Shifting of the demand curve (demand shock), due to changing Real GDP, causes the multiplier process, caused by change in income expectations, wealth, fiscal and monetary politcy, and size of existing stock of physical capital
\begin{enumerate}
\item Increase in quantity of money in circulation by the central bank (monetary policy) causes an increase in consumer and investment spending, drives the interest rate down, leading to an increase in aggregate demand
\item Increase in wealth (real value of household assets, such as stocks or real estate) causes the increase in aggregate demand
\item Fiscal policy (use of government spending or taxation, responding to inflation by reducing spending or increasing taxes) can decrease aggregate demand (directly through less spending, indirectly by increased taxation, lowering disposable income and spending)
\item Planned investment spending decreases as the size of existing physical capital increases, such as residential or capital investment spending
\end{enumerate}
\end{enumerate}

\subsection{Aggregate Supply and Determinants}
\begin{enumerate}
\item In the short-run (short term economy), there is a positive relationship between aggregate output supplied and aggregate price
\begin{enumerate}
\item Profit per unit is the price minus the production costs per unit, where most production costs are fixed in the short-run, mainly wages (all workers compensation)
\item Nominal wages are fixed in the short-run by contract or informal aggreement, such that companies don't want to change it, to try and prevent resentment or constant wage increase demands, creating sticky wages, which hardly fall or rise due to the business cycle, but which change in the long-run
\end{enumerate}
\item In perfectly competitive markets, producers use the price given, while in imperfectly, they are able to somewhat choose the prices
\begin{enumerate}
\item Thus, in perfectly competitive, profits, and thus supply decreases as aggregate price decreases, while in an imperfectly, as demand increases, prices and output may both increase, and vice versa, in an attempt to maximize profit or limit losses 
\end{enumerate}
\item Wage production cost is generally fixed in the short-term, but change in commodity (standard input bought/sold in bulk) prices, nominal wages, or productivity can shift the curve (supply shock)
\begin{enumerate}
\item Commodity prices are not included in the curve, real GDP, or the aggregate price level, due to not being considered a final good
\item Cost of living allowances in contracts, resulting in higher nominal wages when price level rises, can cause this to occur in drastic aggregate price changes
\end{enumerate}
\item Due to flexible wages/costs in the long run, aggregate price level has no effect on aggregate supply, since as as prices change, wages eventually change to compensate
\begin{enumerate}
\item The long-run aggregate supply curve would thus have the prices have no effect, such that the value is the potential output, around which output fluctuates
\item The curve generally shifts constantly right due to increase in quantity or quality of resorches (such as better educated workforce), or technological progress
\item If the supply level is not on the long-run curve, eventually the shift in nominal wages (due to a labor shortage/surplus causing the different in potential and actual GDP) will move the curve such that they coincide
\end{enumerate}
\end{enumerate}

\subsection{Equilibrium in the AD-AS Model}
\begin{enumerate}
\item Short-run macroeconomic equilibrium is the intersection of the demand and short-run supply curve, giving the equilibrium aggregate price level and output
\begin{enumerate}
\item This functions similar to microeconomic to prevent shortages or surplus
\item Generally, there is an upward trend of aggregate output and price levels, such that changes in the variable mean in terms of the expected rise
\item Negative supply shock leads to decreased GDP with increased prices, or stagflation, creating national pessimism overall due to the dual issues, or vice versa, though they cannot be controlled by the government as much as demand shocks 
\end{enumerate}
\item Long-run macroeconomic equilibrium is the intersection of the three curves 
\begin{enumerate}
\item During a demand shock, a curve shifts, creating an inflationary or recessionary gap between the potential output and real output
\item Due to the general movement back to the long-run curve, the economy is considered self-correcting, restoring to a specific GDP in the long term, regardless of short-term events
\item $\text{Output gap} = \frac{\text{Actual - Potential}}{\text{Potential}}*100$, such that it is the percentage difference
\end{enumerate}
\item An alternate model has short-run aggregate supply hardly increasing, until intersecting long-run, at which point price level drastically rises, at potential production capacity
\end{enumerate}

\subsection{Economic Policy and the AD-AS Model}
\begin{enumerate}
\item While the economy is self-correcting, it can take over a decade, leading to the argument in favor of stabalization fiscal policy
\begin{enumerate}
\item Negative demand shocks can be shortened considerably by being anticipated and accounted for with policy, to create price stability and prevent unemployment
\item Positive demand shocks must also be prevented due to inflationary gaps increasing price level, corrected by a supply shock which causes further inflation
\item On the other hand, there is a risk of long term negative effects when offsetting demand shocks
\item Supply side negative shocks has no simple fiscal remedy, due to either demand shock hurting on measurement to aid the other
\end{enumerate}
\item The government is able to influence consumer spending (by taxes and transfers) and government spending (by purchases), and investment spending (by taxes and transfers), creating demand modulation
\begin{enumerate}
\item Taxes are payments to the government, federally mainly through personal and corporate income taxes, and social insurance taxes, state/locally through sales, property, income, and other taxes
\item Government purchases is mainly through defense and education, as well as state/local services, such as infrastructure, police, or firefighters
\item Government transfers mainly include medicare (for seniors), medicaid (for low income), and social security (income to elderly, disabled, and families of deceased recipients), paid for by the social insurance taxes
\end{enumerate}
\item Expansionary fiscal policy causes a positive demand shock to remove a recessionary gap, while contractionary policy does the opposite
\item There is a danger of overactive fiscal policy making the economy less stable, due to time lags between the time it takes to observe the gap, make a fiscal plan, and spend the money, especially since larger spending in projects is typically further on, making analysis difficult
\end{enumerate}

\subsection{Fiscal Policy and the Multiplier}
\begin{enumerate}
\item The multiplier is used to estimate the amount of shift due to fiscal policy, such that increases government spending directly causes the effect
\item Taxes and government transfers on the other hand, due to giving directly to the people, result in only the amount*MPC, leading to that being the initial increase in spending
\begin{enumerate}
\item On the other hand, taxes typically don't lower a specific amount (lump-sum taxes), but rather depend on income/real GDP
\item In addition, the specific group benefited changes the amount saved, due to different groups having different MPC (such as unemployed having higher MPC than shareholders)
\end{enumerate}
\item Income, sales, and corporate taxes take a some portion of the real GDP as each round of the multiplier effect occurs, lowering the effect
\begin{enumerate}
\item This works during a recessionary gap, lowering taxes as real GDP falls, reducing demand shocks automatically, and vice versa, called an automatic stabilizers
\end{enumerate}
\item Some transfers work as automatic stabilizers, such as unemployment benefits, Medicaid, or food stamps, reducing the change in disposable income, and thus the multiplier effect
\item Discretionary fiscal policy is due to deliberate action, but due to time lags, is used only in emergencies
\end{enumerate}

\section{Chapter 5 - The Financial Sector}

\subsection{Savings, Investment, and the Financial System}
\begin{enumerate}
\item Economic growth is caused by increase in human capital (increase in skills and knowledge to produce goods through public and private education) and physical capital (produced through government infrastructure and private investment spending)
\item Private investment spending is either through personal or corporate savings, but typically through borrowing from others, charged an interest rate (percentage of the amount borrowed for use of the money for one year)
\item The savings-investment spending identity states that savings and investment spending must always be equal by definition in an isolated economy without a government
\begin{enumerate}
\item In an economy with a government, the government can have a budget balance of surplus (saving) or deficit (dissaving), where national savings = private savings (disposable income - consumption) + budget balance
\item In a non-isolated economy, the capital inflow (inflow - outflow, where inflow is foreign investment in the country and outflow is domestic outside the country), added to national savings for total savings
\item Capital inflow has a higher national cost that national savings, due to the interest paid to foreigners rather than a domestic loaner
\item The thrift paradox, made by Keynes, states that as MPS increase, MPC decreases, resulting in lower GDP, income, and as a result, savings, such that total economy savings decreases as MPS increases, apparently counterintuitive
\end{enumerate}
\item Financial markets are the market for investment of current and accumulated savings, or wealth, in return for financial assets
\begin{enumerate}
\item Financial assets are paper clames to future income from the seller, while physical assets are claims on objects, giving the right to modify, sell, or destroy the object as they see fit
\item All assets are thus in a pair with a liability, given to the borrower, meaning the requirement to pay in the future
\end{enumerate}
\item The goals of financial systems are to reduce transaction costs (allowing selling bonds or getting bank loans, to avoid costs, such as contracts or negotiations), risk, and the desire for liquidity, to enhance efficiency
\begin{enumerate}
\item Liquid assets are those that can quickly be converted to cash without much loss of value, found within bonds, stocks, and banks, in case of some emergency
\item Financial risk is uncertainty about future outcomes of assets, allowing businesses to share the risk with a large group by stocks, weakening the risk to each by diversifying
\item Diversification is the invevstment of several assets with unrelated risks, such as bank deposits, businesses, and real estate, to reduce overall risk
\item As a result, people often desire to have diversifed portfolios of stocks from unrelated companies, as well as a series of other assets, such as cash or bonds
\end{enumerate}
\item The four main catagories of assets are loans, bonds, stocks, and bank deposits, with loan-backed securities as an additional
\item Loans are lending agreements between individuals, specific to the individual situation, leading to high transaction costs of investigating ability to pay, need, and history, or to negotiate
\item Bonds are issued by the borrower, selling a specific number worth a specific value, with a fixed yearly interest rate, and a specific maturity date (to repay the principle), to avoid transaction costs 
\begin{enumerate}
\item Bond rating agencies help to avoid costs by giving free bond issuer quality, to determine the risk of defaulting, or failing to make payments, such that higher risk causes higher rate
\end{enumerate}
\item Loan-backed securities are assets created by pooling loans, and securitizing, or selling shares to the loan pool, giving more liquidity and less risk than loans, but making it harder to evaluate the value of due to the pooling, such as in the 2008 housing bubble
\item Stocks are company ownership shares, assets to the owner, sold by most companies, though many are privately held, used primarily by companies instead of bonds to reduce risk
\begin{enumerate}
\item Stocks also generally provide higher yield than bonds, but are riskier to the asset-holder, due to bonds being paid first, especially during bankrupty,  while stocks pay only amounts of profits
\end{enumerate}
\item Financial intermediaries turn funds from individuals into financial assets, either mutal funds, pension funds, life insurance companies, or banks
\begin{enumerate}
\item Mutual funds are a diversified portfolio to reduce risks, but avoid high transaction costs with brokers for shares from many companies, selling shares of the portfolio, though there are fees to the funds, especially those who try to optimize the portfolio gains
\item Pension funds are nonprofits, which invest a portion of member savings into a diversified portfolio, giving income after retirement
\item Life insurance companies assure payment to family after the owner's death, to prevent the risk of financial difficulties after deaths
\item Banks provide high liquidity without the high transaction costs of sudden liquidation of stocks and bonds, and lowers the company costs of selling stocks and bonds, taking depositor funds in exchange for bank deposit assets, agreeing to immediately give cash needed
\item Banks keep a small amount as cash, most leant as illiquid loans able to be recalled if not paid, assuming most will not want cash at any given time, aided by the \$250k Federal Deposit Insurance Corp. agency guarantee to each depositor, to reduce risk and fears
\end{enumerate}
\end{enumerate}

\subsection{Definition of Money}
\begin{enumerate}
\item Money is an asset that can be easily used to buy products, including liquid assets (those which can easily be converted to cash, such as checkable/demand bank deposits, or bank accounts from which checks/debit can be written), and currency-in-circulation (cash)itself
\item Money solves the problem of relying on a double coincidence of wants, such that the skills are equally matched, acting as an indirect exchange (universal commodity)
\begin{enumerate}
\item Money thus acts as a medium of exchange, used in a trade for all transactions in a country generally, though in economic downturn, other currencies are often used (ie eggs and coal in post-WWI Germany)
\item Money acts as a store of value, or an asset that holds purchasing power over time
\item It also is a unit of account, or a measure used to set prices and make calculations, compare the value of different goods, and understand the terms of trasactions
\end{enumerate}
\item Money has traditionally been commodity money, or that which has value in other uses, such as gold for jewelry, later replaced by commodity-backed money, or items that could be converted into commodity money at any time
\begin{enumerate}
\item Commodity-backed money had the advantage of only requiring enough kept by banks for a portion of the money, relying that all will not request at once, reducing the amount of resources used up for money
\item Currently, money is fiat money, or that whose value is only by the ability to act as a means of exchange, requiring almost no resources, and able to have the quantity managed based on the economy, rather than on the production of a commodity
\item On the other hand, it opens the possibility for inflation by printing large amounts of money, causing inflation, and the possibility of counterfeiting
\end{enumerate}
\item Money supply is the total value of money in the economy, either defined as currency-in-circulation (interest-free), traveler's checks, and checkable bank deposits (low interest), or those and near-moneys (anything almost checkable, or easily converted, such as savings accounts)
\begin{enumerate}
\item The two monetary aggregates, M1 and M2, are calculated by the Fed, the former as the first definition, the latter as the second definition (M0 is also pure cash)
\item M2 thus contains savings deposits, time deposits (able to be withdrawn before maturity for a fine), and money market funds (mutual funds investing only in liquid assets), all which pay higher interest
\end{enumerate}
\end{enumerate}

\subsection{Time-Value of Money}
\begin{enumerate}
\item Money is worth more currently, due to interest earned/spent on it over the amount of time, or paid if borrowed until recieved later
\item The net effect of projects must be determined, not just by cost-benefit, but taking into account timing, converted to present values for analysis
\begin{enumerate}
\item Interest rate is a measure of the cost of delaying payment, such that it can be used to find the cost, such that the present value changes as the interest rate changes
\item Amoung Paid = Amount Borrowed * (1 + interest)^years, assuming yearly interest compounding
\item As a result, the net present value of a project can determine the project with the most ideal cost-benefit
\end{enumerate}
\end{enumerate}

\subsection{Banking and Money Creation}
\begin{enumerate}
\item M1 consists mainly of bills printed by the Treasury, but the remainder, and a majority of M2 is deposits from banks
\item T-accounts are a table to summarize financial position, with assets on the left, liabilities on the right
\begin{enumerate}
\item To account for those who wish to withdraw, a large amount of assets are held as bank reserves, either in the vault or as bank deposits in the Federal Reserve (able to be converted instantly)
\item Reserves and loans form bank assets, while deposits form bank liabilities, where the reserve ratio is the ratio of reserves to deposits
\end{enumerate}
\item Required reserve ratio is a minimum set by the Fed (10\% in the US), to protect against bank runs, or a large amount of depositors withdrawing at once due to fear of failure
\begin{enumerate}
\item Bank runs generally also lead to a loss of faith in banking, leading to other banks having runs
\item During bank runs, banks often have to sell the loans on short notice, generally for a large discount due to fears of loan issues, leading to bank failure, or being unable to pay back in full
\end{enumerate}
\item The Fed regulates banks after the 1930s, giving deposit insurance through the FDIC for up to \$250k per account
\begin{enumerate}
\item Reserve requirements and capital requirements (preventing risky behavior due to deposit insurance protection by requiring capital, or assets - liabilites, to be some percentage of asets, $\geq 7$ in US) are also used
\item Discount window is the final regulation, by the Fed able to loan money to a bank during a run, to avoid having to sell assets rapidly
\end{enumerate}
\item Banks allow the money supply to be greater than currency-in-circulation through checkable bank deposits, taking cash out of circulation and thus out of the money supply, but add deposits and loans into the money supply
\begin{enumerate}
\item Stage 1 is the currency-in-circulation, stage 2 is the currency deposited, and a portion loaned out, and stage 3 is the loan deposited, and a portion loaned out again, and so forth
\item While a certain percentage can leak by being kept as currency-in-circulation rather than being deposited, assuming this does not occur, the money multiplier is used to determine total currency as a result of deposits
\item The money multiplier also assumes all excess reserves, or assets above the minimum reserve ratio are leant out, such that Total Increase in Supply = (Initial Increase in Excess Reserves/Reserve Ratio)
\item It also assumes the minimum ratio is the amount of reserves kept by all banks, such that additional can leak from there
\end{enumerate}
\item The monetary base, or the currency in reserves and circulation, is determined by the Fed, such that it includes reserves, but not deposits, unlike money supply which is the opposite
\begin{enumerate}
\item Thus, in a more general system where some money is held rather than deposited, the money multiplier is the ratio of money supply to monetary base 
\end{enumerate}
\end{enumerate}

\subsection{Monetary Policy}
\begin{enumerate}
\item The Fed is controlled by the Board of Governors, appointed for 14 year terms by the president, controlling 12 district Fed banks
\item The Fed provides financial services to depository institutions, and acts as the bank of the US Treasury, controls monetary policy, regulates financial institutions, and provide financial stability (such as the discount window)
\item The reserve requirement allows them to use the multiplier to increase investment spending, but rarely uses it
\begin{enumerate}
\item If banks require additional reserves, they borrow from other banks trhough the federal funds market, at the federal funds rate, created organically
\item The reserve requirement is not the percentage, but the specific amount of money which needs to be held
\end{enumerate}
\item The discount rate is the rate of the window, typically 1\% higher than the federal fund rate to prevent it from being used to gain reserves, rarely used, though done in 2007 due to the financial crisis
\begin{enumerate}
\item Reducing the rate difference makes banks more willing to pay the higher rate, rather than paying the transaction costs of a proper sale, making it easier if they are low on reserves, used to increase lending (and the money supply)
\end{enumerate}
\item The Fed also had assets in the form of government debt, or treasury bills (typically 1 year maturity US government bonds, due to not being a part of the US government), and liabilities in the form of the monetary base
\begin{enumerate}
\item Open-market operations are buying and selling treasury bills, typically from commercial banks rather than the government, due to the latter being printing money to finance the government deficit, which causes rapid inflation
\item Purchasing in an open-market sale increases the monetary base and vice versa, able to create or remove the currency at will, triggering the money multiplier
\end{enumerate}
\end{enumerate}

\subsection{The Money Market}
\begin{enumerate}
\item The opportunity cost of holding money is based on the trade-off of ease of direct purchases for gaining no interest on it
\begin{enumerate}
\item This applies to a checking account rather than a savings account (or a certificate of deposit, with higher interest, but penelties for withdrawing before a certain amount of time)
\item This allows the federal government to adjust the federal funds interest rate (such as on 1/3 month Treasury bills), forcing down the rate of all short-term (< 1 year) interest rates by a similar amount, used in the financial crisis to incentivize currency
\item Thus, the Federal Funds rate is a measure of all short-term interest rates
\item Long-term rates are less applicable to holding money, and less related to the federal funds rate, generally ignored for the model, instead using short-term costs due to the main debate being over highly liquid assets or cash, due to the difficulties of conversion
\end{enumerate}
\item The money demand curve determines the quantity of money held based on the nominal interest rate, used rather than the real rate, due to the opportunity cost including the loss of purchasing power due to inflation
\begin{enumerate}
\item The curve is shifted by change in aggregate price level, real GDP, technology (making ease to get money), or institutions (changes in banking regulations, changing the opportunity costs)
\item Demand for money rises proportionally to price level, due to the need for more for the same purchasing power, such that the percent rise is the same for both
\item Real GDP increases the total amount of goods bought and sold, such that more goods are proportionally bought, and thus more money needs to be held
\end{enumerate}
\item The federal funds rate is the rate at which banks lend reserves to other banks to meet the required reserves ratio, targeted in terms of basis points (0.01\%) by the Federal Open Market Committee
\begin{enumerate}
\item The Fed Open Market Desk in NY, in charge of short-term debt, or treasury bills, then works to get the target
\item The liquidity preference model of the interest rate states that the money supply curve (a straight line, set by the Fed) intersects to find the equilibrium interest rate, due to the sellers adjusting the interest rate to get the correct number of buyers
\end{enumerate}
\end{enumerate}

\subsection{The Loanable Funds Market}
\begin{enumerate}
\item The loanable funds market is a hypothetical market of combined markets for all forms of financial assets, with the price as the yearly nominal interest rate assuming a constant inflation rate, due to the inability to know the future interest rate
\begin{enumerate}
\item Real interest rate is the actual price, but cannot be predicted if inflation rate changes
\item This assumes there is only one market, and thus interest rate, although there is a rate for each type of asset
\end{enumerate}
\item The rate of return on a project = 100\% * (revenue - cost)/cost
\begin{enumerate}
\item Thus, if the rate of return is greater than the interest rate, the loan will be taken, causing the demand curve to slope downward
\item Supply curves upward due to the higher gain from saving with higher interest rates
\item Thus, the intersection is the equilibrium interest rate
\item This allows projects with high return rates to be invested in, and causes those willing to lend funds for lower rates to generally loan, such that the groups are well-matched
\end{enumerate}
\item The demand curve is shifted by changes in percieved business opportunities (believed rate of return from specific investments) or government borrowing (deficit is the government borrowing, while surplus is supplying)
\begin{enumerate}
\item Thus, government deficit raises the interest rate, and decreases overall investment spending, called crowding out
\end{enumerate}
\item The supply curve is shifting by changes in capital inflows (perceptions of economic stability of a nation changes inflows) or private savings behavior (changes in consumer consumption)
\item Expected future inflation rate, and thus real interest rate, can also shift both curves, generally assuming previous trends
\begin{enumerate}
\item Higher expected interest rates cause the Fisher effect, or the upward shift to preserve the real interest rate
\end{enumerate}
\item Thus, as the money supply increases, the interest rate reduces, causing the increase in investment spending by the aggregate expenditure model, leading to an increase in GDP, increasing income by the multiplier, increasing savings
\begin{enumerate}
\item The increase in savings causes the shift of the supply curve, causing the shift of the loanable funds market model to the same interest rate point as the money supply model in the short run, due to long run output being the potential
\item In the long run, as money supply rises, such that the price level will rise proportionally by the neutrality of money, causing aggregate demand to shift it to the original rate
\item In addition, since GDP reverts to the original, causing the loanable funds supply to shift to the original position, the interest rate is not affected, such that only loanable funds can affect interest rate, not money demand
\item On the other hand, this assumes it begins are equilibrium, used ineffectively, such that rather, it can have a long-run effect
\end{enumerate}
\end{enumerate}

\section{Chapter 6 - Inflation, Unemployment, and Stabilization Policies}

\subsection{Deficits and Fiscal Policy Implications}
\begin{enumerate}
\item The change in the budget balance is used to monitor fiscal policy, but different forms of policy may have different effects on the economy (such as spending vs transfers)
\item In addition, budget balance changes are often the result, not the cause, of economic fluctuations due to stabilizers
\item Thus, the cyclically adjusted budget balance estimates the fiscal policy assuming potential output GDP
\item While long-term deficits are negative, but commonly used by politicians to gain support, an average budget balance is needed, but deficits are not negative due to allowing fiscal policy use
\begin{enumerate}
\item Long-term deficits lead to debt, keeping budgets by fiscal year from October 1st
\item While total government debt includes funds owed to certain government programs, public debt is to individuals and institutions, equal in 2009 53\% of GDP (69\% with state/local debt)
\item This creates the danger of crowding out, and pressure on future budgets due to interest, creating a danger of losing credit, preventing future loans, causing defaulting, such as Argentina in 2001 
\item The government can print money to avoid default, but that leads to inflation
\end{enumerate}
\item Long-term deficits do not show high levels of debt automatically, rather analyzed by the debt-GDP ratio, to determine the ease at which taxes can pay debts
\begin{enumerate}
\item Implicit liabilities are non-debts, which the government in committed to pay, such that it is effectively a debt
\item The US has high-entitlement implicit liabilities, such that it is a large percent of GDP, growing further over time
\end{enumerate}
\item Surplus in dedicated payroll taxes for entitlements, especially social security, were made since the 1980s, held by another part of the government as a Social Security trust fund
\begin{enumerate}
\item This is the remainder of the debt, not owed to the public, held in bonds, included due to being a committment for future payments (implicit liabilities to baby boomers)
\end{enumerate}
\end{enumerate}

\subsection{Monetary Policy}
\begin{enumerate}
\item The Federal Open Market Committee sets a target federal funds rate every 6 weeks, using open-market operations to change the money supply, and thus the rate through the money market curve
\begin{enumerate}
\item Other methods of interest rate regulation are not as commonly used by the Fed, except in major crisises
\end{enumerate}
\item The Fed tends to react to output gaps by adjustment, working to prevent recessions and inflation, only allowing inflationary gaps to remain in the 90s due to low inflation
\begin{enumerate}
\item The Taylor rule for monetary policy states that the Federal funds rate = 1 + (1.5 * inflation rate) + (0.5 * output gap), such that the direction has always been followed, and generally the amount
\item There are fewer lags in policy, due to lack of legal structure, but was limited in 2009 due to being unable to have a negative rate
\end{enumerate}
\item Many central banks (though not the Fed) set a target inflation rate (or range) instead of the Taylor rule, controlling monetary policy
\begin{enumerate}
\item Inflation targeting is based on future inflation, rather than past inflation (like the Taylor rule)
\item It provides transparency and accountability for the bank, but ignores other concerns, such as the stability of the financial system, which occasionally are the priorities (like 2007-08)
\end{enumerate}
\end{enumerate}

\subsection{Types of Inflation}
\begin{enumerate}
\item The classical model of price level assumes that the real quantity of money is always at the long-run level, instantaneously moving to the long-run, shown to be true by the aggregate demand-supply model
\begin{enumerate}
\item In the long run, changes in the nominal money supply, leads to the real quantity (M/P, where P is price level) to the same value
\item In low inflation, the short-run sticky wages are valid, but in high inflation, the adjustment period is shorter, such that sticky wages are non-valid, and the classical price level model is realistic
\item Thus, in high inflation, the increase in the money supply causes the immediate chnage in the price level
\end{enumerate}
\item In most modern countries, fiat money is created by the central bank, but the government can control it to some degree by the printing of T-bills (even with quantitative easing taking control away)
\begin{enumerate}
\item The US government pays interest to the Fed, though it is immediately returned, except for the money needed for Fed operations
\item Thus, the printing of money, or seignorage, is a form of revenue-production for the government, currently not used extensively to cover debts (<1\%), previously used during the Civil War, and in countries where lenders are unwilling to pay
\end{enumerate}
\item The printing of money causes government purchases to be paid for by those holding money, called an inflation tax, due to the reduction in value, imposing a tax equal to the rate
\begin{enumerate}
\item In hyperinflation, goods and interest-bearing assets are used for money, and try to destroy both nominal and real wealth, such that less money is taken by the inflation tax on real wealth
\item Seignorage is the change in money supply over some short period of time (\Delta M), such that real seignorage is the seignorage over the price level ($\frac{\Delta M}{P}$), or the rate of money supply growth * the real money supply ($\frac{M}{P}*\frac{\Delta M}{M}$)
\item Due to the destruction of wealth, the rate of money supply growth must be increased to gain the same real seignorage, causing higher inflation, creating a cycle
\end{enumerate}
\item Inflation is due to supply (cost-push inflation) from input prices changes, such as oil shocks, or demand (demand-pull inflation)
\begin{enumerate}
\item In the short run, policies causing rising inflation decrease unemployment (and cause an inflationary gap) and vice versa, creating political incentive against policies to lower inflation, even if seignorage is not used
\item This is due to the output gap determining if inflation is above or below the natural unemployment rate, allowing demand shocks to trigger inflation
\end{enumerate}
\end{enumerate}

\subsection{Phillips Curve}
\begin{enumerate}
\item The short-run Phillips curve is the trade-off between inflation and unemployment, reaching deflation as unemployment gets high enough, able to be shifted by supply shocks (up for negative shock) and changes in expectations of future inflation
\begin{enumerate}
\item Wage agreements take into account the expected inflation, agreed to by employers to avoid having to pay even higher wages for employees in the future, leading to the inflation rising by the same amount as the expected rise
\item People generally expect previous inflation to continue, such that high rates result in high expected rates of inflation
\end{enumerate}
\item In the long-run, there is no tradeoff of lower unemployment for inflation, due to temporary higher inflation for low unemployment causing the curve to shift upward by expectations when it moves back, such that the inflation at the original unemployment is the same as at the lower rate
\begin{enumerate}
\item Thus, there is an accelerating rate of inflation due to persistant attempts to lower unempoyment from the natural rate, or the nonaccelerating inflation rate of unemployment, where it remains constant
\item As a result of expectations, disinflation is difficult, keeping high unemployment, creating a trade-off of short-term high unemployment and recession for long-term rising inflation 
\item Clear policy transparency can also reduce the effects of disinflation policy by lowering expectations, requiring less action
\end{enumerate}
\item Deflation, common before WWII, existing in Japan in the 1990s, is a major issue due to debt deflation reducing aggregate demand by redistributing money to lenders, who are unlikely to increase consumption by as much as borrowers
\begin{enumerate}
\item The zero bound on the nominal interest rate is due to the lack of willingness to pay to loan money, such that the real interest rate
\end{enumerate}
\end{enumerate}

\end{document}
