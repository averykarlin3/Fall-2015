\documentclass[11 pt, twoside]{article}
\usepackage{textcomp}
\usepackage[margin=1in]{geometry}
\usepackage[utf8]{inputenc}
\usepackage{color}
\usepackage{setspace}
\usepackage{tikz}
\usepackage{amsmath}
\usepackage{amsfonts}

\begin{document}

\title{Macroeconomics}
\author{Avery Karlin}
\date{Fall 2015}

\maketitle
\newpage
\tableofcontents
\vspace{11pt}
\noindent
\underline{Main Textbook}: Krugman's Economics for AP\\
\underline{Secondary Textbook}: CORE Project Economics\\
\underline{Teacher}: Schweitzer
\newpage

\section{CORE Chapter 1}

\subsection{National Differences}
\begin{enumerate}
\item In the 1300s, most of the world was fairly equal in the general amount of wealth of the population, even if there
were large differences between the rich and poor, often depending on parental status
\begin{enumerate}
\item Gross domestic product per capita, or average living standards, has raised in the last 700 years, but caused
differences by country, due to having a sudden rise at different times, leading to different standards
\item Often, independence from colonial rule or European interference caused the sudden economic growth, but Latin
America did not have the growth
\end{enumerate}
\item The ratio scale has the GDP y{}-axis go up by some multiple, used to compare growth rate, or (${\Delta}$GDP)/(GDPstart), such that 100\% means it doubles if ratio of 2
\begin{enumerate}
\item The ratio scale thus has the slope of the graph as the growth rate
\item Thus, the GDP per capita appears as a hockey stick curve, remaining without much growth, before a kink, leading to
a sudden rise
\end{enumerate}
\item Adam Smith argued that coordination of all the aspects of an economic culture from different parts of the world,
would be created on its own based on self{}-interest, rather than made by the government
\begin{enumerate}
\item He did believe there was ethical beliefs guiding behavior, and feared monopolies, especially government protected
\item He approved of government investment in education and public works, as well as justice and foreign policy through
the government
\end{enumerate}
\end{enumerate}

\subsection{Average Living Standards}
\begin{enumerate}
\item GDP per capita is different from average disposable income, which is close to average living standards, but ignores some aspects of happiness
\begin{enumerate}
\item Disposable income is the total income from factor markets, tranfer payments from both government, and from others, minus any governemtn taxes
\item Quality of social and physical environment, government services, and goods produced within a household, are important to well-being, but ignored by disposable income
\item Average disposable income also ignores distribution, due to extra income affecting the rich less than lack of affects the poor
\item People are also less happy based on how their income is related to others within the population, even if they have enough income to not be in poverty
\end{enumerate}
\item GDP is better when evalutating government services, but can only be measured easily through cost to produce, rather than selling value
\end{enumerate}

\subsection{The Modern World}
\begin{enumerate}
\item %Include 1.3, 1.4, 1.5 here
\end{enumerate}

\subsection{Environmental Impact}
\begin{enumerate}
\item As production increased, environemtnal damage, especially climate change due to burning of fossil fuels (gas, oil, or coal) increasing $CO_2$ emissions
\begin{enumerate}
\item While temperature has always fluctuated due to volcanic events, such as the 1815 Mount Tambora eruption in Indonesia, lowering temperature in 1816, but has drastically risen over the last century
\item This can lead to rising sea levels, climate changes destroying farming, and polar ice cap melting, as well as cause respiratory issues and illnesses in cities
\end{enumerate}
\item The economy is a portion of society, which is within the overall biosphere, such that it can affect the other aspects
\item Overuse of local resources is also a possible environmental issue, caused along with climate change by economic expansion and organization (resources valued and conserved)
\begin{enumerate}
\item While the permanent technological revolution gave fossil fuel dependence, but has also permitted vastly more electricity for fewer resources, permitting the development of alternate sources, which may change it
\end{enumerate}
\end{enumerate}

\subsection{Capitalism}
\begin{enumerate}
\item An economic system is the organization of the production and distribution of products in an economy
\item Capitalism is the economic system made up of institutions, or sets of laws and social customs regulating the economy, of private property, markets, and firms
\begin{enumerate}
\item In prior economies, families typically were the third major institution, or the economy was regulated by a centralized government
\item Private property is possessions purely controlled and owned by an entity, able to control use or give ownership
\item Markets facilitate the transfer of goods in a reciprocal, voluntary trade
\item Firms are the main organization of production, owning the capital goods (which diffrentiates it from other economic organizations and systems), paying wages, managing employees, such that the products are private property of the owner, to make a profit
\begin{enumerate}
\item Firms are the main organization, with others being families, unions, government agencies, and non-profits
\item Firms utilize the labor market, unlike other organizations, such that firms can be created, destroyed, expand, or contract extremely quickly
\end{enumerate} 
\end{enumerate}
\item Capitalism both centralizes power in the hands of firm owners, and decentralizes from the government and other outside influences, creating power and cooperation inside, but competition outside
\item Defintions are used, not to set a specific rule, but to set general catagories which something can fall within to different degrees
\end{enumerate}

%Add 1.10-1.17 here

\section{CORE Chapter 14 - Monetary Policy}
\subsection{Inflation vs Unemployment}
\begin{enumerate}
\item While unemployment has a more direct effect, since inflation goes up as unemployment goes down, it is generally considered more important to force down
\begin{enumerate}
\item Inflation targeting through monetary policy from the central bank resulted from stagflation in the 1990s, working to control the supply economy as much as demand
\item Fixed nominal income, such as pensioners, are hurt by inflation, even when indexation increases it by the previous years inflation, such that if inflation goes up further, more is lost
\item High inflation creates volitile rates, making the economy work less effectively by creating uncertainty, by making it difficult to determine scarcity of resources relatively rather than price changes
\item Deflation prevents spending on the other hand, leading to a drop in aggregate demand and activity
\item As a result of the relationship, during an expansion, if not controlled properly, the interest rate can increase too high for stability
\end{enumerate}
\item The Phillips curve can either be drawn with less inflation as higher on the y-axis, or more inflation, such that it is a decreasing concave up, tolerating less additional inflation when high even as employment increases or vice versa
\begin{enumerate}
\item They can also be drawn with more inflation higher on the y-axis, adjusted similarly
\item Unemployment is always at the lower values of the x-axis
\item Inflation occurs in an attempt to divide up a specific amount of goods, until the bargaining power makes it proportional
\item As employment is high (tighter labor market), wages are forced up in an increasing concave up  as bargaining power goes up, while as employment rises, prices are forced up to compensate similarly
\item In addition, as production capacity use rises, investment spending tends to increase to expand production, temporarily having capacity constrained until further machines are created
\item As a result, since there is less competition to fully sell goods, the price level is able to rise in a concave increasing curve related to it
\item This can be viewed as the bargaining power of both firms and workers simultaneously rising, such that they both try take a larger portion of real wage, until price level increases
\end{enumerate}
\item Indifference curves are curves which give the same levels of satisfaction, with the opposite concavity to the Phillips, but going in the same direction, throughout the graph
\begin{enumerate}
\item Policymaker indifference curves are such that they ideally want the point on the curve with the lowest general unemployment and inflation, but which also must intersect with the Phillips curve
\item Policymakers who are unemployment-averse have indifference curves toward low unemployment, while inflation-averse is vice versa, based on personal and constituent interests
\end{enumerate}
\end{enumerate}

\subsection{Monetary Policy}
\begin{enumerate}
\item Central banks use changes in the nominal policy interest rate to stabilize the economy by a desired real interest rate, leading to a desired demand shock
\item Policy interest rate directly changes market interest rates, asset prices, expectations and confidence (consistant, transparent policy and good communications can lead to proper expectations in the economy and spending, encouraging the desired direction), and the exchange rate
\begin{enumerate}
These all change investment spending, shifting up the aggregate expenditure function
\item Commercial banks set market rates for loans, though central bank determination results in a similar change in market rate
\item Central banks set the rate by making a goal for aggregate demand, finding the real interest rate that will produce the goal demand by the aggregate expenditure function, using it to find the market interest rate to produce that, then the policy rate
\item Lowering of the policy interest rate causes general interest rates to go down, such that the price of assets with fixed rates go up (new assets produced provide less return from interest over time), making people feel wealthier
\end{enumerate}
\item Monetary policy is extremely flexible, more so than fiscal policy, but there is a zero lower bound on the policy rate, but in extreme recessions, it may not be low enough
\item Qualitative easing
\end{enumerate}

\input{unit123.tex}

\section{Chapter 4 - National Income and Price Determination}
\subsection{Income and Expenditure}
\begin{enumerate}
\item For aggregate spending change modeling, it must be assumed the interest rate is constant, and government spending, taxes, imports, and exports are zero
\begin{enumerate}
\item It is also assumed that producers will supply extra output at a fixed price, such that increase in demand will increase output without an increase in prices, true only in short-term
\end{enumerate}
\item Thus, if investment/consumer (antonomous change in aggregate spending) spending increases, both aggregate output and income would increase by that much, leading to increased consumer spending, and thus output
\begin{enumerate}
\item Maginal propensity to consume (MPC) is the rate of change of consumer spending as disposable income rises, while marginal propensity to save is the remaining amount of money (1 - MPC)
\item Thus, spending compounds the increase, such that total increase is the original amount, times the $(1 + MPC + MPC^2 + MPC^3 + ...) *$ original increase in spending, or (1/(1-MPC) * increase) = (increase/MPS) = (increase * multiplier)
\end{enumerate}
\item Consumer spending is a majority of spending, such the consumption function shows that as disposable income rises, consumption also does, or $c = a + MPC * y$, where c is single household spending, y is income, and a is the amount that would be spent if income was 0, or autonomous consumer spending, often through loans or savings
\begin{enumerate}
\item Aggregate consumption function is the same equation, and can be thought of as the horizontal sum of the individual functions
\item Consumption function can be shifted by change in expected future disposable income, such that autonomous spending changes, shifting the curve up/down
\item Thus, those with higher income often expect it to fall, and vice versa, saving a higher percentage, but during economic expansion, but during economic expansion, since future and current rise together, it is difficult to predict
\item Based on this, the permanent income hypothesis states spending is based on the long term expected income, rather than current
\item Aggregate wealth can also affect it, such that an increase in net worth can cause the curve to rise, and can also affect it by the life cycle hypothesis (specifically stating that people save more in assets until after peak working years)
\item Price levels are an exogenous variable on the other hand, such that it changes unaffected by disposable income
\end{enumerate}
\item The Keynesian cross states that in an economy, total income (or supply due to GDP being total paid through the factor market) and total expenditure (or demand) must be equal, such that the intersection of that line and the consumption curve is the equilibrium 
\begin{enumerate}
\item When not at equilibrium, unplanned investment spending compensates for the difference, quickly moving back to equilibirum
\end{enumerate}
\item Investment spending is a smaller amount, but often far more dramatic in changes, causing the business cycle, unaffected by disposable income changes
\begin{enumerate}
\item Planned investment spending, or the expected amount in a year, depends on interest rate (even if retained earnings, or past profits, are used, due to the same trade-off through lending the earnings to gain interest)
\item It also depends on expected future GDP (expected sale growth causes investment spending growth) and production capacity (excess capcity decreases investment spending, due to lack of need)
\item Actual investment spending is made up of planned, and unplanned inventory spending (due to trying to keep a proper inventory size, such that excess inventory remains as unplanned positive investment)
\end{enumerate}
\item Aggregate expenditures is the total spending on GDP, modeled by the aggregate expenditures function combined with additional spending on investment, government, and net export spending (aggregate autonomous spending)
\begin{enumerate}
\item Aggregate demand can be thought to be the aggregate expenditures, taking into account a range of prices
\end{enumerate}
\end{enumerate}

\subsection{Aggregate Demand and Determinants}
\begin{enumerate}
\item Positive or negative demand shock is the shifting of the aggregate demand curve of an economy, or the relationship between aggregate price level and aggregate output demanded (Real GDP), depicted similarly to a market demand curve
\begin{enumerate}
\item The downward slope is not due to the law of demand, since that assumes ceteris paribus, while aggregate demand assumes a simultaneous price change in all final products
\item The wealth/real balances/real assets effect of change in aggregate price level is the change in consumer spending, due to the decrease in value of assets during inflation
\item The interest rate effect in aggregate price level is the result of the attempt to borrow enough money to possess the same purchasing power, selling assets or borrowing, where the surge of loans drives interest rates up, preventing consumer and investment spending (especially the latter)
\item The foreign purchases effect in aggregate price level states that as national prices fall relative to other nations, the demand for nationally-produced goods will increase foreignly
\end{enumerate}
\item Shifting of the demand curve (demand shock), due to changing Real GDP, causes the multiplier process, caused by change in income expectations, wealth, fiscal and monetary politcy, and size of existing stock of physical capital
\begin{enumerate}
\item Increase in quantity of money in circulation by the central bank (monetary policy) causes an increase in consumer and investment spending, drives the interest rate down, leading to an increase in aggregate demand
\item Increase in wealth (real value of household assets, such as stocks or real estate) causes the increase in aggregate demand
\item Fiscal policy (use of government spending or taxation, responding to inflation by reducing spending or increasing taxes) can decrease aggregate demand (directly through less spending, indirectly by increased taxation, lowering disposable income and spending)
\item Planned investment spending decreases as the size of existing physical capital increases, such as residential or capital investment spending
\end{enumerate}
\end{enumerate}

\subsection{Aggregate Supply and Determinants}
\begin{enumerate}
\item In the short-run (short term economy), there is a positive relationship between aggregate output supplied and aggregate price
\begin{enumerate}
\item Profit per unit is the price minus the production costs per unit, where most production costs are fixed in the short-run, mainly wages (all workers compensation)
\item Nominal wages are fixed in the short-run by contract or informal aggreement, such that companies don't want to change it, to try and prevent resentment or constant wage increase demands, creating sticky wages, which hardly fall or rise due to the business cycle, but which change in the long-run
\end{enumerate}
\item In perfectly competitive markets, producers use the price given, while in imperfectly, they are able to somewhat choose the prices
\begin{enumerate}
\item Thus, in perfectly competitive, profits, and thus supply decreases as aggregate price decreases, while in an imperfectly, as demand increases, prices and output may both increase, and vice versa, in an attempt to maximize profit or limit losses 
\end{enumerate}
\item Wage production cost is generally fixed in the short-term, but change in commodity (standard input bought/sold in bulk) prices, nominal wages, or productivity can shift the curve (supply shock)
\begin{enumerate}
\item Commodity prices are not included in the curve, real GDP, or the aggregate price level, due to not being considered a final good
\item Cost of living allowances in contracts, resulting in higher nominal wages when price level rises, can cause this to occur in drastic aggregate price changes
\end{enumerate}
\item Due to flexible wages/costs in the long run, aggregate price level has no effect on aggregate supply, since as as prices change, wages eventually change to compensate
\begin{enumerate}
\item The long-run aggregate supply curve would thus have the prices have no effect, such that the value is the potential output, around which output fluctuates
\item The curve generally shifts constantly right due to increase in quantity or quality of resorches (such as better educated workforce), or technological progress
\item If the supply level is not on the long-run curve, eventually the shift in nominal wages (due to a labor shortage/surplus causing the different in potential and actual GDP) will move the curve such that they coincide
\end{enumerate}
\end{enumerate}

\subsection{Equilibrium in the AD-AS Model}
\begin{enumerate}
\item Short-run macroeconomic equilibrium is the intersection of the demand and short-run supply curve, giving the equilibrium aggregate price level and output
\begin{enumerate}
\item This functions similar to microeconomic to prevent shortages or surplus
\item Generally, there is an upward trend of aggregate output and price levels, such that changes in the variable mean in terms of the expected rise
\item Negative supply shock leads to decreased GDP with increased prices, or stagflation, creating national pessimism overall due to the dual issues, or vice versa, though they cannot be controlled by the government as much as demand shocks 
\end{enumerate}
\item Long-run macroeconomic equilibrium is the intersection of the three curves 
\begin{enumerate}
\item During a demand shock, a curve shifts, creating an inflationary or recessionary gap between the potential output and real output
\item Due to the general movement back to the long-run curve, the economy is considered self-correcting, restoring to a specific GDP in the long term, regardless of short-term events
\item $\text{Output gap} = \frac{\text{Actual - Potential}}{\text{Potential}}*100$, such that it is the percentage difference
\end{enumerate}
\end{enumerate}

\subsection{Economic Policy and the AD-AS Model}
\begin{enumerate}
\item While the economy is self-correcting, it can take over a decade, leading to the argument in favor of stabalization fiscal policy
\begin{enumerate}
\item Negative demand shocks can be shortened considerably by being anticipated and accounted for with policy, to create price stability and prevent unemployment
\item Positive demand shocks must also be prevented due to inflationary gaps typically leading to an eventual move in the other direction, leading to a recessionary gap
\item On the other hand, there is a risk of long term negative effects when offsetting demand shocks
\item Supply side negative shocks has no simple fiscal remedy, due to either demand shock hurting on measurement to aid the other
\end{enumerate}
\item The government is able to influence consumer spending (by taxes and transfers) and government spending (by purchases), and investment spending (by taxes and transfers), creating demand modulation
\begin{enumerate}
\item Taxes are payments to the government, federally mainly through personal and corporate income taxes, and social insurance taxes, state/locally through sales, property, income, and other taxes
\item Government purchases is mainly through defense and education, as well as state/local services, such as infrastructure, police, or firefighters
\item Government transfers mainly include medicare (for seniors), medicaid (for low income), and social security (income to elderly, disabled, and families of deceased recipients), paid for by the social insurance taxes
\end{enumerate}
\item Expansionary fiscal policy causes a positive demand shock to remove a recessionary gap, while contractionary policy does the opposite
\item There is a danger of overactive fiscal policy making the economy less stable, due to time lags between the time it takes to observe the gap, make a fiscal plan, and spend the money, especially since larger spending in projects is typically further on, making analysis difficult
\end{enumerate}

\subsection{Fiscal Policy and the Multiplier}
\begin{enumerate}
\item The multiplier is used to estimate the amount of shift due to fiscal policy, such that increases government spendingdirectly causes the effect
\item Taxes and government transfers on the other hand, due to giving directly to the people, result in only the amount*MPC, leading to that being the initial increase in spending
\begin{enumerate}
\item On the other hand, taxes typically don't lower a specific amount (lump-sum taxes), but rather depend on income/real GDP
\item In addition, the specific group benefited changes the amount saved, due to different groups having different MPC (such as unemployed having higher MPC than shareholders)
\end{enumerate}
\item Income, sales, and corporate taxes take a some portion of the real GDP as each round of the multiplier effect occurs, lowering the effect
\begin{enumerate}
\item This works during a recessionary gap, lowering taxes as real GDP falls, reducing demand shocks automatically, and vice versa, called an automatic stabilizers
\end{enumerate}
\item Some transfers work as automatic stabilizers, such as unemployment benefits, Medicaid, or food stamps, reducing the change in disposable income, and thus the multiplier effect
\item Discretionary fiscal policy is due to deliberate action, but due to time lags, is used only in emergencies
\end{enumerate}

\end{document}
