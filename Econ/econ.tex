\documentclass[11 pt, twoside]{article}
\usepackage{textcomp}
\usepackage[margin=1in]{geometry}
\usepackage[utf8]{inputenc}
\usepackage{color}
\usepackage{setspace}
\usepackage{tikz}

\begin{document}

\title{Macroeconomics}
\author{Avery Karlin}
\date{Fall 2015}

\maketitle
\newpage
\tableofcontents
\vspace{11pt}
\underline{Main Textbook}: Krugman's Economics for AP\\
\underline{Secondary Textbook}: CORE Project Economics\\
\underline{Teacher}: Schweitzer
\newpage

\section{CORE Chapter 1}

\subsection{National Differences}
\begin{enumerate}
\item In the 1300s, most of the world was fairly equal in the general amount of wealth of the population, even if there
were large differences between the rich and poor, often depending on parental status
\begin{enumerate}
\item Gross domestic product per capita, or average living standards, has raised in the last 700 years, but caused
differences by country, due to having a sudden rise at different times, leading to different standards
\item Often, independence from colonial rule or European interference caused the sudden economic growth, but Latin
America did not have the growth
\end{enumerate}
\item The ratio scale has the GDP y{}-axis go up by some multiple, used to compare growth rate, or (${\Delta}$GDP)/(GDPstart), such that 100\% means it doubles if ratio of 2
\begin{enumerate}
\item The ratio scale thus has the slope of the graph as the growth rate
\item Thus, the GDP per capita appears as a hockey stick curve, remaining without much growth, before a kink, leading to
a sudden rise
\end{enumerate}
\item Adam Smith argued that coordination of all the aspects of an economic culture from different parts of the world,
would be created on its own based on self{}-interest, rather than made by the government
\begin{enumerate}
\item He did believe there was ethical beliefs guiding behavior, and feared monopolies, especially government protected
\item He approved of government investment in education and public works, as well as justice and foreign policy through
the government
\end{enumerate}
\end{enumerate}

\section{Chapter 1}

\subsection{Study of Economics}
\begin{enumerate}
\item Economics is the study of scarcity and choice, mainly individual choice, as well as the economy, or the system
which coordinates choices about production and consumption, and distributes products
\begin{enumerate}
\item Market economies, like the US, is where productive and consumption are made by decentralized decisions of many
people
\item Command economies are those where industry is publically owned with a central authority for production and
consumption, typically failing due to lack of resources or being told to make unneeded products, not gathering
information as well, better for incentivizing needs, not complete control
\end{enumerate}
\item Economies rely on incentives, punishment or reward, for particular choices, such as higher prices for needed
products, causing more to be made
\begin{enumerate}
\item Property rights give ownership and allow trading, creating incentives to use resources for value
\item Marginal decisions balance cost{}-benefit, looked at by marginal analysis
\item Resources, which can be used to make something else, are scarce, or less than society desires, as incentives
\end{enumerate}
\item Factors of production, or resources, are divided into land, labor, capital (all manufactured goods to make other
goods, which are not used up in production), and entrepreneurship (firm ownership, not dependent on risk)
\begin{enumerate}
\item In a market economy, use of resources is based on the sum of individual decisions, though sometimes, when there is
no incentive, community decisions must interfere with the market for the general good
\item Opportunity costs are factors given up for a specific choice, such as time, money, or future prospects
\end{enumerate}
\item Macroeconomics are the study of the overall economy, mainly economic aggregates, or measures such as GDP,
unemployment, or inflation
\begin{enumerate}
\item Macroeconomics runs on the basis that the sum is greater than its parts, due to the overall dynamics, mattering
more than microfoundations
\item Microeconomics are the study of individual decisions of people, firms, or markets/industries
\end{enumerate}
\item Positive economics is definite factual questions about how the world actually works, rather than normative, or
uncertain questions about how it should work
\begin{enumerate}
\item The former deals with both economic forecasts, or predictions based on current conditions, and hypotheses of
predictions in different ones
\item Economic models are used to give simplified representations of reality, used for both types of positive analysis
\item Normative creates value judgements, up to opinion, unless there is a clear beneficial advantage of one, often
based on opportunity costs, not using models, but rather prior ideas and models for other measures
\end{enumerate}
\item Disagreements can be created by differences in values, or on the model of reality, exacerbated by political
interests
\end{enumerate}

\subsection{Intro to Macro}
\begin{enumerate}
\item The business cycle is the alternating cycle of down and upturns
\begin{enumerate}
\item Depressions are a very deep, long downturn with product output and employment falling, while shorter downturns are
called recessions
\item Expansions and recoveries are the opposite periods of upturn, typically lasting almost 5 years (57 months), rather
than 10 months of recessions
\end{enumerate}
\item Macroeconomic analysis is used to minimize the fluctuations of the economy
\begin{enumerate}
\item Unemployment is the number of people looking for work actively, who are not working, while the labor force is the
unemployed + employed, and the unemployment rate is the percentage of the force unemployed
\item Unemployment rate is a good economic indicator, though even during an expansion, there is a small unemployment
rate
\end{enumerate}
\item Aggregate output, or the total amount of goods and services produced in a given amount of time, is another
economic indicator
\item Inflation is a rise in the overall price level, while deflation is the opposite, the former discouraging saving,
and eventually making money worthless
\begin{enumerate}
\item Deflation encourages saving, instead of reinvesting to allow the economy to regrow, with price stability being the
most desirable
\end{enumerate}
\item Economic growth, or an increase in the maximum possible output, is an overall sustained rise over a long period of
time, outside the business cycle, allowing higher wages and standard of living
\begin{enumerate}
\item On the other hand, economic growth can be bad for stability of the business cycle, and vice versa
\end{enumerate}
\item Models are a simplified version of reality, studying economies in a smaller setting, such as a WWII prison for
cigarettes, or on a computer simulation
\begin{enumerate}
\item The other things equal (ceteris paribus) assumption is used to only study one change, by making all other factors
constant
\item Thought experiments, or simple, hypothetical scenarios, are another effective way of modeling, as well as graphing
\end{enumerate}
\end{enumerate}

\subsection{Production Possibility Curve Model}
\begin{enumerate}
\item Trade{}-offs are when something is giving up the opportunity costs of something for that of another option,
analyzed by the PPC
\begin{enumerate}
\item The PPC model assumes only two goods produced, such that points within are feasible, but not optimal/efficient,
while points on are both
\item The slope determines if the trade{}-off is constant, called a constant opportunity cost, often not true, due to
having to use less suited resources as the production increases, thus getting less and losing more
\item Input problems find the trade{}-offs to gain the same output of different products, while output find for the same
input for different products
\end{enumerate}
\item Efficiency in production is the lack of missed opportunities, or optimal improvement to one{}'s self, without
hurting others, exampled by unemployment of those who want work
\begin{enumerate}
\item Efficiency in allocation is the maximization of consumer happiness by the optimal production of the correct goods
\item Overall efficiency requires both in allocation and production
\end{enumerate}
\item Economic growth can also be defined as the expansion of production possibilities, shifting the curve outward,
since products made shift
\begin{enumerate}
\item This is typically caused by increase in resources or technology, the technical means of production of products
\item Since only one product on the curve may shift, there is a chance production may not rise, even as there is growth
\end{enumerate}
\end{enumerate}

\subsection{Comparative Advantage and Trade}
\begin{enumerate}
\item Trade is the division of tasks, such that people trade goods and services for those they want
\begin{enumerate}
\item Gains from trade are caused by specialization, due to engaging in a specific task allowing the production of more of the good
\item This is due to the time required for skill development in a field
\item This also results from comparative advantage, or the idea that some people are better at certain actions than others, resulting in a lower opportunity cost for production
\item People will only accept deals that cost less than their personal opportunity cost for production
\end{enumerate}
\item Absolute advantage is the general ability to produce more, under any relative distribution of resources
\begin{enumerate}
\item Comparative advantage creates the mutual benefits of trade, not absolute advantage
\end{enumerate}
\end{enumerate}

\end{document}
