\documentclass[11 pt, twoside]{article}
\usepackage{textcomp}
\usepackage[margin=1in]{geometry}
\usepackage[utf8]{inputenc}
\usepackage{color}
\usepackage{setspace}
\usepackage{tikz}
\usepackage{amsmath}
\usepackage{amsfonts}

\begin{document}

\title{Macroeconomics}
\author{Avery Karlin}
\date{Fall 2015}

\maketitle
\newpage
\tableofcontents
\vspace{11pt}
\noindent
\underline{Main Textbook}: Krugman's Economics for AP\\
\underline{Secondary Textbook}: CORE Project Economics\\
\underline{Teacher}: Schweitzer
\newpage

\section{CORE Chapter 1}

\subsection{National Differences}
\begin{enumerate}
\item In the 1300s, most of the world was fairly equal in the general amount of wealth of the population, even if there
were large differences between the rich and poor, often depending on parental status
\begin{enumerate}
\item Gross domestic product per capita, or average living standards, has raised in the last 700 years, but caused
differences by country, due to having a sudden rise at different times, leading to different standards
\item Often, independence from colonial rule or European interference caused the sudden economic growth, but Latin
America did not have the growth
\end{enumerate}
\item The ratio scale has the GDP y{}-axis go up by some multiple, used to compare growth rate, or (${\Delta}$GDP)/(GDPstart), such that 100\% means it doubles if ratio of 2
\begin{enumerate}
\item The ratio scale thus has the slope of the graph as the growth rate
\item Thus, the GDP per capita appears as a hockey stick curve, remaining without much growth, before a kink, leading to
a sudden rise
\end{enumerate}
\item Adam Smith argued that coordination of all the aspects of an economic culture from different parts of the world,
would be created on its own based on self{}-interest, rather than made by the government
\begin{enumerate}
\item He did believe there was ethical beliefs guiding behavior, and feared monopolies, especially government protected
\item He approved of government investment in education and public works, as well as justice and foreign policy through
the government
\end{enumerate}
\end{enumerate}

\subsection{Average Living Standards}
\begin{enumerate}
\item GDP per capita is different from average disposable income, which is close to average living standards, but ignores some aspects of happiness
\begin{enumerate}
\item Disposable income is the total income from factor markets, tranfer payments from both government, and from others, minus any governemtn taxes
\item Quality of social and physical environment, government services, and goods produced within a household, are important to well-being, but ignored by disposable income
\item Average disposable income also ignores distribution, due to extra income affecting the rich less than lack of affects the poor
\item People are also less happy based on how their income is related to others within the population, even if they have enough income to not be in poverty
\end{enumerate}
\item GDP is better when evalutating government services, but can only be measured easily through cost to produce, rather than selling value
\end{enumerate}

\subsection{Environmental Impact}
\begin{enumerate}
\item As production increased, environemtnal damage, especially climate change due to burning of fossil fuels (gas, oil, or coal) increasing $CO_2$ emissions
\begin{enumerate}
\item While temperature has always fluctuated due to volcanic events, such as the 1815 Mount Tambora eruption in Indonesia, lowering temperature in 1816, but has drastically risen over the last century
\item This can lead to rising sea levels, climate changes destroying farming, and polar ice cap melting, as well as cause respiratory issues and illnesses in cities
\end{enumerate}
\item The economy is a portion of society, which is within the overall biosphere, such that it can affect the other aspects
\item Overuse of local resources is also a possible environmental issue, caused along with climate change by economic expansion and organization (resources valued and conserved)
\begin{enumerate}
\item While the permanent technological revolution gave fossil fuel dependence, but has also permitted vastly more electricity for fewer resources, permitting the development of alternate sources, which may change it
\end{enumerate}
\end{enumerate}

\subsection{Capitalism}
\begin{enumerate}
\item An economic system is the organization of the production and distribution of products in an economy
\item Capitalism is the economic system made up of institutions, or sets of laws and social customs regulating the economy, of private property, markets, and firms
\begin{enumerate}
\item In prior economies, families typically were the third major institution, or the economy was regulated by a centralized government
\item Private property is possessions purely controlled and owned by an entity, able to control use or give ownership
\item Markets facilitate the transfer of goods in a reciprocal, voluntary trade
\item Firms are the main organization of production, owning the capital goods (which diffrentiates it from other economic organizations and systems), paying wages, managing employees, such that the products are private property of the owner, to make a profit
\begin{enumerate}
\item Firms are the main organization, with others being families, unions, government agencies, and non-profits
\item Firms utilize the labor market, unlike other organizations, such that firms can be created, destroyed, expand, or contract extremely quickly
\end{enumerate} 
\end{enumerate}
\item Capitalism both centralizes power in the hands of firm owners, and decentralizes from the government and other outside influences, creating power and cooperation inside, but competition outside
\item Defintions are used, not to set a specific rule, but to set general catagories which something can fall within to different degrees
\end{enumerate}

\section{Chapter 1}

\subsection{Study of Economics}
\begin{enumerate}
\item Economics is the study of scarcity and choice, mainly individual choice, as well as the economy, or the system
which coordinates choices about production and consumption, and distributes products
\begin{enumerate}
\item Market economies, like the US, is where productive and consumption are made by decentralized decisions of many
people
\item Command economies are those where industry is publically owned with a central authority for production and
consumption, typically failing due to lack of resources or being told to make unneeded products, not gathering
information as well, better for incentivizing needs, not complete control
\end{enumerate}
\item Economies rely on incentives, punishment or reward, for particular choices, such as higher prices for needed
products, causing more to be made
\begin{enumerate}
\item Property rights give ownership and allow trading, creating incentives to use resources for value
\item Marginal decisions balance cost{}-benefit, looked at by marginal analysis
\item Resources, which can be used to make something else, are scarce, or less than society desires, as incentives
\end{enumerate}
\item Factors of production, or resources, are divided into land, labor, capital (all manufactured goods to make other
goods, which are not used up in production), and entrepreneurship (firm ownership, not dependent on risk)
\begin{enumerate}
\item In a market economy, use of resources is based on the sum of individual decisions, though sometimes, when there is
no incentive, community decisions must interfere with the market for the general good
\item Opportunity costs are factors given up for a specific choice, such as time, money, or future prospects
\end{enumerate}
\item Macroeconomics are the study of the overall economy, mainly economic aggregates, or measures such as GDP,
unemployment, or inflation
\begin{enumerate}
\item Macroeconomics runs on the basis that the sum is greater than its parts, due to the overall dynamics, mattering
more than microfoundations
\item Microeconomics are the study of individual decisions of people, firms, or markets/industries
\end{enumerate}
\item Positive economics is definite factual questions about how the world actually works, rather than normative, or
uncertain questions about how it should work
\begin{enumerate}
\item The former deals with both economic forecasts, or predictions based on current conditions, and hypotheses of
predictions in different ones
\item Economic models are used to give simplified representations of reality, used for both types of positive analysis
\item Normative creates value judgements, up to opinion, unless there is a clear beneficial advantage of one, often
based on opportunity costs, not using models, but rather prior ideas and models for other measures
\end{enumerate}
\item Disagreements can be created by differences in values, or on the model of reality, exacerbated by political
interests
\end{enumerate}

\subsection{Intro to Macro}
\begin{enumerate}
\item The business cycle is the alternating cycle of down and upturns
\begin{enumerate}
\item Depressions are a very deep, long downturn with product output and employment falling, while shorter downturns are
called recessions
\item Expansions and recoveries are the opposite periods of upturn, typically lasting almost 5 years (57 months), rather
than 10 months of recessions
\end{enumerate}
\item Macroeconomic analysis is used to minimize the fluctuations of the economy
\begin{enumerate}
\item Unemployment is the number of people looking for work actively, who are not working, while the labor force is the
unemployed + employed, and the unemployment rate is the percentage of the force unemployed
\item Unemployment rate is a good economic indicator, though even during an expansion, there is a small unemployment
rate
\end{enumerate}
\item Aggregate output, or the total amount of goods and services produced in a given amount of time, is another
economic indicator
\item Inflation is a rise in the overall price level, while deflation is the opposite, the former discouraging saving,
and eventually making money worthless
\begin{enumerate}
\item Deflation encourages saving, instead of reinvesting to allow the economy to regrow, with price stability being the
most desirable
\end{enumerate}
\item Economic growth, or an increase in the maximum possible output, is an overall sustained rise over a long period of
time, outside the business cycle, allowing higher wages and standard of living
\begin{enumerate}
\item On the other hand, economic growth can be bad for stability of the business cycle, and vice versa
\end{enumerate}
\item Models are a simplified version of reality, studying economies in a smaller setting, such as a WWII prison for
cigarettes, or on a computer simulation
\begin{enumerate}
\item The other things equal (ceteris paribus) assumption is used to only study one change, by making all other factors
constant
\item Thought experiments, or simple, hypothetical scenarios, are another effective way of modeling, as well as graphing
\end{enumerate}
\end{enumerate}

\subsection{Production Possibility Curve Model}
\begin{enumerate}
\item Trade{}-offs are when something is giving up the opportunity costs of something for that of another option,
analyzed by the PPC
\begin{enumerate}
\item The PPC model assumes only two goods produced, such that points within are feasible, but not optimal/efficient,
while points on are both
\item The slope determines if the trade{}-off is constant, called a constant opportunity cost, often not true, due to
having to use less suited resources as the production increases, thus getting less and losing more
\item Input problems find the trade{}-offs to gain the same output of different products, while output find for the same
input for different products
\end{enumerate}
\item Efficiency in production is the lack of missed opportunities, or optimal improvement to one{}'s self, without
hurting others, exampled by unemployment of those who want work
\begin{enumerate}
\item Efficiency in allocation is the maximization of consumer happiness by the optimal production of the correct goods
\item Overall efficiency requires both in allocation and production
\end{enumerate}
\item Economic growth can also be defined as the expansion of production possibilities, shifting the curve outward,
since products made shift
\begin{enumerate}
\item This is typically caused by increase in resources or technology, the technical means of production of products
\item Since only one product on the curve may shift, there is a chance production may not rise, even as there is growth
\end{enumerate}
\end{enumerate}

\subsection{Comparative Advantage and Trade}
\begin{enumerate}
\item Trade is the division of tasks, such that people trade goods and services for those they want
\begin{enumerate}
\item Gains from trade are caused by specialization, due to engaging in a specific task allowing the production of more of the good
\item This is due to the time required for skill development in a field
\item This also results from comparative advantage, or the idea that some people are better at certain actions than others, resulting in a lower opportunity cost for production
\item People will only accept deals that cost less than their personal opportunity cost for production (terms of trade)
\end{enumerate}
\item Absolute advantage is the general ability to produce more, under any relative distribution of resources
\begin{enumerate}
\item Comparative advantage creates the mutual benefits of trade, not absolute advantage
\end{enumerate}
\end{enumerate}

\section{Chapter 2 - Supply and Demand}

\subsection{Intro to Demand}
\begin{enumerate}
\item Competitive markets are a market with many buyers and sellers of the same products, where a market is a group of consumers and producers exchanging products for payment
\begin{enumerate}
\item Thus, individual actions must not have a noticable effect on the price, such that in non-fully competitive markets, it doesn't apply completely
\item It is described by the demand and supply curves, sets of factors which cause each to shift, market equilibrium, and how market equilibrium changes when the curve shifts
\end{enumerate}
\item The demand for any good depends on the price, making a demand curve of the quantity demanded vs price, first making a demand schedule table of points
\begin{enumerate}
\item The quantity demanded is the amount consumers are willing to buy at a particular price
\item Demand curves typically have a downward slope, not always constant, such that the law of demand states that as price decreases, demand increases, and vice versa
\item Due to all other things equal, the curve does not account for changes in the world, such that changes in taste, income, related prices, number of consumers, or expectations (either in income or price) can shift the curve outward
\begin{enumerate}
\item Changes in demand are at the same price, while a movement on the curve are at a different price
\item Related good price changes are in goods which are substitutes, such that people are more willing to buy the other if price rises, or complements, which are goods that people are more willing if the price of the other falls
\item Normal goods are those where demand increases as income does, unlike inferior goods, typically those with better, more expensive alternatives
\item Number of consumers can change due to population, such that the individual demand curve (demand curve for a single person, such that the market curve is the horizontal sum) may not shift, but the market curve does 
\item One exception is conspicuous consumption, or goods which people purely buy due to high price to gain social status, or goods so cheap they can no longer be considering the same product
\end{enumerate}
\end{enumerate}
\end{enumerate}

\subsection{Supply and Equilibrium}
\begin{enumerate}
\item The quantity supplied when offered a specific price also varies with price, such that a schedule, and curve can be produced, forming a law of supply, where if price rises, supply will as well
\item Changes in supply can be caused by changes in input (items needed to produce the product) prices, related goods price, technology (methods used for production), expectations, and number of producers
\begin{enumerate}
\item Often, several related products are produced by the same producer, such that as the price of one good rises, the others (substitutes in production) are produced less
\item Biproducts of the same process are compliments in production, and will be made more
\end{enumerate}
\item The interaction of supply and demand creates equilibrium where supply is the same as demand, at the equilibrium/market-clearing price and quantity
\begin{enumerate}
\item On the same graph, the equilibrium point is the intersection of the two curves
\item In all established, ongoing markets, people converge toward a single market price, which is most beneficial to all parties involved, and the price moves to prevent surpluses or shortages
\end{enumerate}
\end{enumerate}

\subsection{Changes in Equilibrium}
\begin{enumerate}
\item Changes in equilibrium cause the shift of either the supply curve, the demand curve, or both simultaneously
\item When demand increases, the equilibrium price and quantity increase, and vice versa, such that the curve moves rightward and the intersection moves up-right
\item When supply increases, the equilibrium price decreases, but the quantity increases, such that the curve moves rightward and the intersection moves down-left
\item Simultaneous shifts depend on the relative shifts to determine in which way the equilibrium moves, such that one direction can be determined, but the other is ambiguous
\item Events in the short term can only change either supply or demand, not both, though in the long run, it moves toward equilibrium, causing a change
\end{enumerate}

\section{Chapter 3 - Economic Performance Measurement}
\subsection{Circular Flow and GDP}
\begin{enumerate}
\item The national income and product accounts, or national accounts, keep track of the flow of money from consumers to producers, as well as business investment, or government purchases
\begin{enumerate}
\item Accuracy of national accounts is an indication of how economically advanced a country is
\end{enumerate}
\item The circular flow diagram represents the flow of money in the economy, with money flowing from a household, or group of people sharing an income, for goods and services (spending)
\begin{enumerate}
\item The money can then flow from the market for goods and services to the firms (revenue), or organizations that employ households and make products, in exchange for products
\item Money then goes from the firms to the factor markets (costs), which give money in exchange for factors of production, especially labor from households (income)
\item Thus, the idea is that the money flowing into or out of each market or organization is equal to the money flowing out
\item All factors are assumed to be rented from the factor market from households, due to household ownership of firms, within the simplified model
\item This model assumes lack of saving, no surplus outputs, and no alternate sources of leakages (money flowing outside the simple circular flow model)
\end{enumerate}
\item This can be extended into a more complex model, adding the government, foreign nations, and the financial market, such that the flow from households to product market is consumer spending
\begin{enumerate}
\item In addition to selling labor, households use stocks, or firm shared ownership, and bonds, or loans to firms with interest, from the financial market to firms, which eventually goes back into the factor market to households in profits and interest
\item Rent is also given by households to the firms in exchange for land resources, through the factor market
\item While people spend most disposable income in the product market, some of total income is lost through taxes, while some can be gained through government tranfers, such as unemployment payment, given without a reciprocal service
\item Some disposable income is lost as private savings in the financial market, which, in addition to providing money to firms allows government borrowing
\item Government funds is also used for government purchases
\item Goods sold to other countries are exports, while those purchased are imports, as part of the product market, and foreign nations also participate in borrowing and investing in the financial market
\item Finally, firms also purchase products from the product market through investment spending, adding to their inventories, or raw materials and capital used for production
\end{enumerate}
\item Gross domestic product (GDP) is thus the sum of government purchases, investment, and consumer spending, minus imports, or the total of final goods and services produced during a period
\begin{enumerate}
\item Final products are those sold to the final user, while intermediate products are those that are inputs into the production of the final product, such that capital is final, while resources are intermediate
\end{enumerate}
\item GDP can be measured by adding the total value of production of final products (or sum of value added of all products), aggregate spending on domestically producted final products, or total factor income earned by households from domestic firms
\begin{enumerate}
\item Intermediate products are ignored for total value of production due to their value being added to that of the final product, such that they would be summed multiple times otherwise
\item Each product has value added of interest, rent, profit (both employee profits, and those paid to shareholders as dividends), and wages on that product, combined with the price of all intermediate products for the total value
\item Total factor income is found by the sum of each type of factor payment, such that it is the sum of total wages, interest, rent, and profits from each product, including intermediate
\item GDP by aggregate spending = consumer + investment + governemnt + exports - imports, where exports - imports can also be called net exports
\item Stocks and bonds are not counted due to not representing the sale of final goods or the direct production of final goods
\end{enumerate}
\item GDP can be calculated practically through the sum of aggregate spending or the value added by each sector of the economy (business, household labor, and government services)
\item Capital goods are eventually used up, such that it is accounted for by depreciation, such that net domestic product is GDP - depreciation
\begin{enumerate}
\item It can also be done by replacing investment spending with net investment, or investment - depreciation
\item Depreciation is the cost of the capital, divided by the number of years it was used for
\end{enumerate}
\end{enumerate}

\subsection{Real GDP}
\begin{enumerate}
\item Stock variables are those measured at a specific point in time, such as nominal capital stock (value of all national assets at one time) while fluid variables are those measured over a period of time, such as nominal GDP
\item GDP describes the size of the economy, but measures the price of total products produced, rather than the amount of output, such that real GDP adjusts for price changes to describe aggregate output (the quantity of final products produced)
\begin{enumerate}
\item This prevents inflation or rise of prices from causing the increase in GDP, without increased aggregate output
\item Real GDP is calculated as the GDP if the price had remained constant from some base year, such that it is the total value of final products, assuming that price level
\item Nominal GDP is another term for non-real GDP
\item Since Real GDP can be calulated from a late or early base year, gaining different results, chain-linking measures the average of the two values, expressed in chain dollars
\end{enumerate}
\item Real GDP comparisons assume equal population, such that GDP (or real GDP) per capita (divided by the size of the population) can be used to account for that
\begin{enumerate}
\item Real GDP per capita compares labor productivity, but it is not a complete measure of living standards, but rather of potential living standards
\item High living standards requires health, education, and good quality of life spending, rather than expenditures on negatives such as natural disasters or disease
\item Real GDP per capita also does not include other components to high standard of life, which are non-monetary, such as natural beauty or leisure
\end{enumerate}
\end{enumerate}

\subsection{Unemployment}
\begin{enumerate}
\item Unemployed are only those able to and actively (in the last 4 weeks) looking for work, such that retired or disabled people
\item Labor force participation rate is the percentage of the population $\geq$ 16 in the labor force
\item The US Census Bureau takes a Current Population Survey of 60k families, asking if they qualify as unemployed, to estimate the total unemployment rate
\item Unemployment rate can overstate the level of unemployment, such that those who are not working, but will easily be able to get a job, simply due to taking a few weeks to get a job, increase the rate and make sure it is never 0\%
\item It may also understate unemployment, due to marginally attached workers, who want a job and have looked for work in the past, but not currently
\begin{enumerate}
\item Discouraged workers, who don't feel they will be able to find a job, and thus don't search, are part of the marginally attached
\item Underemployed are those working part-time, due to lack of full-time work, not included in the unemployment rate, which may make it further understate it
\item The Bureau of Labor Statistics also makes measures of labor underutilization, the most broad of which is U6 including marginally attached and underemployed
\item U6, while far higher, typically mirrors movement of the standard unemployment rate
\end{enumerate}
\item Unemployment rate ignores demographic differences, due to it being far easier for experienced workers in the field, and prime working years workers (25-54 years old) to find jobs, while blacks have a more difficult time than most groups
\begin{enumerate}
\item Even when the unemployment rate is low, it may be higher than normal for certain groups
\end{enumerate}
\item During recessions, the unemployment rate always rises, while the opposite is often true during expansions, though not always, due to growth not happening fast enough to stop the increase of unemployment on occasion
\begin{enumerate}
\item Generally though, above average increase in Real GDP resulted in decrease in unemployment rate
\item Periods when unemployment is rising during an expansion is during a growth recession
\end{enumerate}
\end{enumerate}

\subsection{Causes and Catagories of Unemployment}
\begin{enumerate}
\item 
\end{enumerate}

\end{document}
