\documentclass[11 pt, twoside]{article}
\usepackage{textcomp}
\usepackage[margin=1in]{geometry}
\usepackage[utf8]{inputenc}
\usepackage{color}
\usepackage{setspace}
\usepackage{tikz}

\begin{document}

\title{Macroeconomics}
\author{Avery Karlin}
\date{Fall 2015}

\maketitle
\newpage
\tableofcontents
\newpage

\section{Module 1}
\subsection{Comparative Advantage and Trade}
\begin{enumerate}
\note Trade is the division of tasks, such that people trade goods and services for those they want
\begin{enumerate}
\note Gains from trade are caused by specialization, due to engaging in a specific task allowing the production of more of the good
\note This is due to the time required for skill development in a field
\note This also results from comparative advantage, or the idea that some people are better at certain actions than others, resulting in a lower opportunity cost for production
\note People will only accept deals that cost less than their personal opportunity cost for production
\end{enumerate}
\note Absolute advantage is the general ability to produce more, under any relative distribution of resources
\begin{enumerate}
\note Comparative advantage creates the mutual benefits of trade, not absolute advantage
\end{enumerate}
\end{enumerate}

\end{document}
