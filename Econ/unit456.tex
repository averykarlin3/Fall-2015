\section{Chapter 4 - National Income and Price Determination}
\subsection{Income and Expenditure}
\begin{enumerate}
\item For aggregate spending change modeling, it must be assumed the interest rate is constant, and government spending, taxes, imports, and exports are zero
\begin{enumerate}
\item It is also assumed that producers will supply extra output at a fixed price, such that increase in demand will increase output without an increase in prices, true only in short-term
\end{enumerate}
\item Thus, if investment/consumer (antonomous change in aggregate spending) spending increases, both aggregate output and income would increase by that much, leading to increased consumer spending, and thus output
\begin{enumerate}
\item Maginal propensity to consume (MPC) is the rate of change of consumer spending as disposable income rises, while marginal propensity to save is the remaining amount of money (1 - MPC)
\item Thus, spending compounds the increase, such that total increase is the original amount, times the $(1 + MPC + MPC^2 + MPC^3 + ...) *$ original increase in spending, or (1/(1-MPC) * increase) = (increase/MPS) = (increase * multiplier)
\end{enumerate}
\item Consumer spending is a majority of spending, such the consumption function shows that as disposable income rises, consumption also does, or $c = a + MPC * y$, where c is single household spending, y is income, and a is the amount that would be spent if income was 0, or autonomous consumer spending, often through loans or savings
\begin{enumerate}
\item Aggregate consumption function is the same equation, and can be thought of as the horizontal sum of the individual functions
\item Consumption function can be shifted by change in expected future disposable income, such that autonomous spending changes, shifting the curve up/down
\item Thus, those with higher income often expect it to fall, and vice versa, saving a higher percentage, but during economic expansion, but during economic expansion, since future and current rise together, it is difficult to predict
\item Based on this, the permanent income hypothesis states spending is based on the long term expected income, rather than current
\item Aggregate wealth can also affect it, such that an increase in net worth can cause the curve to rise, and can also affect it by the life cycle hypothesis (specifically stating that people save more in assets until after peak working years)
\item Price levels are an exogenous variable on the other hand, such that it changes unaffected by disposable income
\end{enumerate}
\item The Keynesian cross states that in an economy, total income (or supply due to GDP being total paid through the factor market) and total expenditure (or demand) must be equal, such that the intersection of that line and the consumption curve is the equilibrium 
\begin{enumerate}
\item When not at equilibrium, unplanned investment spending compensates for the difference, quickly moving back to equilibirum
\end{enumerate}
\item Investment spending is a smaller amount, but often far more dramatic in changes, causing the business cycle, unaffected by disposable income changes
\begin{enumerate}
\item Planned investment spending, or the expected amount in a year, depends on interest rate (even if retained earnings, or past profits, are used, due to the same trade-off through lending the earnings to gain interest)
\item It also depends on expected future GDP (expected sale growth causes investment spending growth) and production capacity (excess capcity decreases investment spending, due to lack of need)
\item Actual investment spending is made up of planned, and unplanned inventory spending (due to trying to keep a proper inventory size, such that excess inventory remains as unplanned positive investment)
\end{enumerate}
\item Aggregate expenditures is the total spending on GDP, modeled by the aggregate expenditures function combined with additional spending on investment, government, and net export spending (aggregate autonomous spending)
\begin{enumerate}
\item Aggregate demand can be thought to be the aggregate expenditures, taking into account a range of prices
\end{enumerate}
\end{enumerate}

\subsection{Aggregate Demand and Determinants}
\begin{enumerate}
\item Positive or negative demand shock is the shifting of the aggregate demand curve of an economy, or the relationship between aggregate price level and aggregate output demanded (Real GDP), depicted similarly to a market demand curve
\begin{enumerate}
\item The downward slope is not due to the law of demand, since that assumes ceteris paribus, while aggregate demand assumes a simultaneous price change in all final products
\item The wealth/real balances/real assets effect of change in aggregate price level is the change in consumer spending, due to the decrease in value of assets during inflation
\item The interest rate effect in aggregate price level is the result of the attempt to borrow enough money to possess the same purchasing power, selling assets or borrowing, where the surge of loans drives interest rates up, preventing consumer and investment spending (especially the latter)
\item The foreign purchases effect in aggregate price level states that as national prices fall relative to other nations, the demand for nationally-produced goods will increase foreignly
\end{enumerate}
\item Shifting of the demand curve (demand shock), due to changing Real GDP, causes the multiplier process, caused by change in income expectations, wealth, fiscal and monetary politcy, and size of existing stock of physical capital
\begin{enumerate}
\item Increase in quantity of money in circulation by the central bank (monetary policy) causes an increase in consumer and investment spending, drives the interest rate down, leading to an increase in aggregate demand
\item Increase in wealth (real value of household assets, such as stocks or real estate) causes the increase in aggregate demand
\item Fiscal policy (use of government spending or taxation, responding to inflation by reducing spending or increasing taxes) can decrease aggregate demand (directly through less spending, indirectly by increased taxation, lowering disposable income and spending)
\item Planned investment spending decreases as the size of existing physical capital increases, such as residential or capital investment spending
\end{enumerate}
\end{enumerate}

\subsection{Aggregate Supply and Determinants}
\begin{enumerate}
\item In the short-run (short term economy), there is a positive relationship between aggregate output supplied and aggregate price
\begin{enumerate}
\item Profit per unit is the price minus the production costs per unit, where most production costs are fixed in the short-run, mainly wages (all workers compensation)
\item Nominal wages are fixed in the short-run by contract or informal aggreement, such that companies don't want to change it, to try and prevent resentment or constant wage increase demands, creating sticky wages, which hardly fall or rise due to the business cycle, but which change in the long-run
\end{enumerate}
\item In perfectly competitive markets, producers use the price given, while in imperfectly, they are able to somewhat choose the prices
\begin{enumerate}
\item Thus, in perfectly competitive, profits, and thus supply decreases as aggregate price decreases, while in an imperfectly, as demand increases, prices and output may both increase, and vice versa, in an attempt to maximize profit or limit losses 
\end{enumerate}
\item Wage production cost is generally fixed in the short-term, but change in commodity (standard input bought/sold in bulk) prices, nominal wages, or productivity can shift the curve (supply shock)
\begin{enumerate}
\item Commodity prices are not included in the curve, real GDP, or the aggregate price level, due to not being considered a final good
\item Cost of living allowances in contracts, resulting in higher nominal wages when price level rises, can cause this to occur in drastic aggregate price changes
\end{enumerate}
\item Due to flexible wages/costs in the long run, aggregate price level has no effect on aggregate supply, since as as prices change, wages eventually change to compensate
\begin{enumerate}
\item The long-run aggregate supply curve would thus have the prices have no effect, such that the value is the potential output, around which output fluctuates
\item The curve generally shifts constantly right due to increase in quantity or quality of resorches (such as better educated workforce), or technological progress
\item If the supply level is not on the long-run curve, eventually the shift in nominal wages (due to a labor shortage/surplus causing the different in potential and actual GDP) will move the curve such that they coincide
\end{enumerate}
\end{enumerate}

\subsection{Equilibrium in the AD-AS Model}
\begin{enumerate}
\item Short-run macroeconomic equilibrium is the intersection of the demand and short-run supply curve, giving the equilibrium aggregate price level and output
\begin{enumerate}
\item This functions similar to microeconomic to prevent shortages or surplus
\item Generally, there is an upward trend of aggregate output and price levels, such that changes in the variable mean in terms of the expected rise
\item Negative supply shock leads to decreased GDP with increased prices, or stagflation, creating national pessimism overall due to the dual issues, or vice versa, though they cannot be controlled by the government as much as demand shocks 
\end{enumerate}
\item Long-run macroeconomic equilibrium is the intersection of the three curves 
\begin{enumerate}
\item During a demand shock, a curve shifts, creating an inflationary or recessionary gap between the potential output and real output
\item Due to the general movement back to the long-run curve, the economy is considered self-correcting, restoring to a specific GDP in the long term, regardless of short-term events
\item $\text{Output gap} = \frac{\text{Actual - Potential}}{\text{Potential}}*100$, such that it is the percentage difference
\end{enumerate}
\item An alternate model has short-run aggregate supply hardly increasing, until intersecting long-run, at which point price level drastically rises, at potential production capacity
\end{enumerate}

\subsection{Economic Policy and the AD-AS Model}
\begin{enumerate}
\item While the economy is self-correcting, it can take over a decade, leading to the argument in favor of stabalization fiscal policy
\begin{enumerate}
\item Negative demand shocks can be shortened considerably by being anticipated and accounted for with policy, to create price stability and prevent unemployment
\item Positive demand shocks must also be prevented due to inflationary gaps increasing price level, corrected by a supply shock which causes further inflation
\item On the other hand, there is a risk of long term negative effects when offsetting demand shocks
\item Supply side negative shocks has no simple fiscal remedy, due to either demand shock hurting on measurement to aid the other
\end{enumerate}
\item The government is able to influence consumer spending (by taxes and transfers) and government spending (by purchases), and investment spending (by taxes and transfers), creating demand modulation
\begin{enumerate}
\item Taxes are payments to the government, federally mainly through personal and corporate income taxes, and social insurance taxes, state/locally through sales, property, income, and other taxes
\item Government purchases is mainly through defense and education, as well as state/local services, such as infrastructure, police, or firefighters
\item Government transfers mainly include medicare (for seniors), medicaid (for low income), and social security (income to elderly, disabled, and families of deceased recipients), paid for by the social insurance taxes
\end{enumerate}
\item Expansionary fiscal policy causes a positive demand shock to remove a recessionary gap, while contractionary policy does the opposite
\item There is a danger of overactive fiscal policy making the economy less stable, due to time lags between the time it takes to observe the gap, make a fiscal plan, and spend the money, especially since larger spending in projects is typically further on, making analysis difficult
\end{enumerate}

\subsection{Fiscal Policy and the Multiplier}
\begin{enumerate}
\item The multiplier is used to estimate the amount of shift due to fiscal policy, such that increases government spending directly causes the effect
\item Taxes and government transfers on the other hand, due to giving directly to the people, result in only the amount*MPC, leading to that being the initial increase in spending
\begin{enumerate}
\item On the other hand, taxes typically don't lower a specific amount (lump-sum taxes), but rather depend on income/real GDP
\item In addition, the specific group benefited changes the amount saved, due to different groups having different MPC (such as unemployed having higher MPC than shareholders)
\end{enumerate}
\item Income, sales, and corporate taxes take a some portion of the real GDP as each round of the multiplier effect occurs, lowering the effect
\begin{enumerate}
\item This works during a recessionary gap, lowering taxes as real GDP falls, reducing demand shocks automatically, and vice versa, called an automatic stabilizers
\end{enumerate}
\item Some transfers work as automatic stabilizers, such as unemployment benefits, Medicaid, or food stamps, reducing the change in disposable income, and thus the multiplier effect
\item Discretionary fiscal policy is due to deliberate action, but due to time lags, is used only in emergencies
\end{enumerate}

\section{Chapter 5 - The Financial Sector}

\subsection{Savings, Investment, and the Financial System}
\begin{enumerate}
\item Economic growth is caused by increase in human capital (increase in skills and knowledge to produce goods through public and private education) and physical capital (produced through government infrastructure and private investment spending)
\item Private investment spending is either through personal or corporate savings, but typically through borrowing from others, charged an interest rate (percentage of the amount borrowed for use of the money for one year)
\item The savings-investment spending identity states that savings and investment spending must always be equal by definition in an isolated economy without a government
\begin{enumerate}
\item In an economy with a government, the government can have a budget balance of surplus (saving) or deficit (dissaving), where national savings = private savings (disposable income - consumption) + budget balance
\item In a non-isolated economy, the capital inflow (inflow - outflow, where inflow is foreign investment in the country and outflow is domestic outside the country), added to national savings for total savings
\item Capital inflow has a higher national cost that national savings, due to the interest paid to foreigners rather than a domestic loaner
\item The thrift paradox, made by Keynes, states that as MPS increase, MPC decreases, resulting in lower GDP, income, and as a result, savings, such that total economy savings decreases as MPS increases, apparently counterintuitive
\end{enumerate}
\item Financial markets are the market for investment of current and accumulated savings, or wealth, in return for financial assets
\begin{enumerate}
\item Financial assets are paper clames to future income from the seller, while physical assets are claims on objects, giving the right to modify, sell, or destroy the object as they see fit
\item All assets are thus in a pair with a liability, given to the borrower, meaning the requirement to pay in the future
\end{enumerate}
\item The goals of financial systems are to reduce transaction costs (allowing selling bonds or getting bank loans, to avoid costs, such as contracts or negotiations), risk, and the desire for liquidity, to enhance efficiency
\begin{enumerate}
\item Liquid assets are those that can quickly be converted to cash without much loss of value, found within bonds, stocks, and banks, in case of some emergency
\item Financial risk is uncertainty about future outcomes of assets, allowing businesses to share the risk with a large group by stocks, weakening the risk to each by diversifying
\item Diversification is the invevstment of several assets with unrelated risks, such as bank deposits, businesses, and real estate, to reduce overall risk
\item As a result, people often desire to have diversifed portfolios of stocks from unrelated companies, as well as a series of other assets, such as cash or bonds
\end{enumerate}
\item The four main catagories of assets are loans, bonds, stocks, and bank deposits, with loan-backed securities as an additional
\item Loans are lending agreements between individuals, specific to the individual situation, leading to high transaction costs of investigating ability to pay, need, and history, or to negotiate
\item Bonds are issued by the borrower, selling a specific number worth a specific value, with a fixed yearly interest rate, and a specific maturity date (to repay the principle), to avoid transaction costs 
\begin{enumerate}
\item Bond rating agencies help to avoid costs by giving free bond issuer quality, to determine the risk of defaulting, or failing to make payments, such that higher risk causes higher rate
\end{enumerate}
\item Loan-backed securities are assets created by pooling loans, and securitizing, or selling shares to the loan pool, giving more liquidity and less risk than loans, but making it harder to evaluate the value of due to the pooling, such as in the 2008 housing bubble
\item Stocks are company ownership shares, assets to the owner, sold by most companies, though many are privately held, used primarily by companies instead of bonds to reduce risk
\begin{enumerate}
\item Stocks also generally provide higher yield than bonds, but are riskier to the asset-holder, due to bonds being paid first, especially during bankrupty,  while stocks pay only amounts of profits
\end{enumerate}
\item Financial intermediaries turn funds from individuals into financial assets, either mutal funds, pension funds, life insurance companies, or banks
\begin{enumerate}
\item Mutual funds are a diversified portfolio to reduce risks, but avoid high transaction costs with brokers for shares from many companies, selling shares of the portfolio, though there are fees to the funds, especially those who try to optimize the portfolio gains
\item Pension funds are nonprofits, which invest a portion of member savings into a diversified portfolio, giving income after retirement
\item Life insurance companies assure payment to family after the owner's death, to prevent the risk of financial difficulties after deaths
\item Banks provide high liquidity without the high transaction costs of sudden liquidation of stocks and bonds, and lowers the company costs of selling stocks and bonds, taking depositor funds in exchange for bank deposit assets, agreeing to immediately give cash needed
\item Banks keep a small amount as cash, most leant as illiquid loans able to be recalled if not paid, assuming most will not want cash at any given time, aided by the \$250k Federal Deposit Insurance Corp. agency guarantee to each depositor, to reduce risk and fears
\end{enumerate}
\end{enumerate}

\subsection{Definition of Money}
\begin{enumerate}
\item Money is an asset that can be easily used to buy products, including liquid assets (those which can easily be converted to cash, such as checkable/demand bank deposits, or bank accounts from which checks/debit can be written), and currency-in-circulation (cash)itself
\item Money solves the problem of relying on a double coincidence of wants, such that the skills are equally matched, acting as an indirect exchange (universal commodity)
\begin{enumerate}
\item Money thus acts as a medium of exchange, used in a trade for all transactions in a country generally, though in economic downturn, other currencies are often used (ie eggs and coal in post-WWI Germany)
\item Money acts as a store of value, or an asset that holds purchasing power over time
\item It also is a unit of account, or a measure used to set prices and make calculations, compare the value of different goods, and understand the terms of trasactions
\end{enumerate}
\item Money has traditionally been commodity money, or that which has value in other uses, such as gold for jewelry, later replaced by commodity-backed money, or items that could be converted into commodity money at any time
\begin{enumerate}
\item Commodity-backed money had the advantage of only requiring enough kept by banks for a portion of the money, relying that all will not request at once, reducing the amount of resources used up for money
\item Currently, money is fiat money, or that whose value is only by the ability to act as a means of exchange, requiring almost no resources, and able to have the quantity managed based on the economy, rather than on the production of a commodity
\item On the other hand, it opens the possibility for inflation by printing large amounts of money, causing inflation, and the possibility of counterfeiting
\end{enumerate}
\item Money supply is the total value of money in the economy, either defined as currency-in-circulation (interest-free), traveler's checks, and checkable bank deposits (low interest), or those and near-moneys (anything almost checkable, or easily converted, such as savings accounts)
\begin{enumerate}
\item The two monetary aggregates, M1 and M2, are calculated by the Fed, the former as the first definition, the latter as the second definition (M0 is also pure cash)
\item M2 thus contains savings deposits, time deposits (able to be withdrawn before maturity for a fine), and money market funds (mutual funds investing only in liquid assets), all which pay higher interest
\end{enumerate}
\end{enumerate}

\subsection{Time-Value of Money}
\begin{enumerate}
\item Money is worth more currently, due to interest earned/spent on it over the amount of time, or paid if borrowed until recieved later
\item The net effect of projects must be determined, not just by cost-benefit, but taking into account timing, converted to present values for analysis
\begin{enumerate}
\item Interest rate is a measure of the cost of delaying payment, such that it can be used to find the cost, such that the present value changes as the interest rate changes
\item Amoung Paid = Amount Borrowed * (1 + interest)^years, assuming yearly interest compounding
\item As a result, the net present value of a project can determine the project with the most ideal cost-benefit
\end{enumerate}
\end{enumerate}

\subsection{Banking and Money Creation}
\begin{enumerate}
\item M1 consists mainly of bills printed by the Treasury, but the remainder, and a majority of M2 is deposits from banks
\item T-accounts are a table to summarize financial position, with assets on the left, liabilities on the right
\begin{enumerate}
\item To account for those who wish to withdraw, a large amount of assets are held as bank reserves, either in the vault or as bank deposits in the Federal Reserve (able to be converted instantly)
\item Reserves and loans form bank assets, while deposits form bank liabilities, where the reserve ratio is the ratio of reserves to deposits
\end{enumerate}
\item Required reserve ratio is a minimum set by the Fed (10\% in the US), to protect against bank runs, or a large amount of depositors withdrawing at once due to fear of failure
\begin{enumerate}
\item Bank runs generally also lead to a loss of faith in banking, leading to other banks having runs
\item During bank runs, banks often have to sell the loans on short notice, generally for a large discount due to fears of loan issues, leading to bank failure, or being unable to pay back in full
\end{enumerate}
\item The Fed regulates banks after the 1930s, giving deposit insurance through the FDIC for up to \$250k per account
\begin{enumerate}
\item Reserve requirements and capital requirements (preventing risky behavior due to deposit insurance protection by requiring capital, or assets - liabilites, to be some percentage of asets, $\geq 7$ in US) are also used
\item Discount window is the final regulation, by the Fed able to loan money to a bank during a run, to avoid having to sell assets rapidly
\end{enumerate}
\item Banks allow the money supply to be greater than currency-in-circulation through checkable bank deposits, taking cash out of circulation and thus out of the money supply, but add deposits and loans into the money supply
\begin{enumerate}
\item Stage 1 is the currency-in-circulation, stage 2 is the currency deposited, and a portion loaned out, and stage 3 is the loan deposited, and a portion loaned out again, and so forth
\item While a certain percentage can leak by being kept as currency-in-circulation rather than being deposited, assuming this does not occur, the money multiplier is used to determine total currency as a result of deposits
\item The money multiplier also assumes all excess reserves, or assets above the minimum reserve ratio are leant out, such that Total Increase in Supply = (Initial Increase in Excess Reserves/Reserve Ratio)
\item It also assumes the minimum ratio is the amount of reserves kept by all banks, such that additional can leak from there
\end{enumerate}
\item The monetary base, or the currency in reserves and circulation, is determined by the Fed, such that it includes reserves, but not deposits, unlike money supply which is the opposite
\begin{enumerate}
\item Thus, in a more general system where some money is held rather than deposited, the money multiplier is the ratio of money supply to monetary base 
\end{enumerate}
\end{enumerate}

\subsection{Monetary Policy}
\begin{enumerate}
\item The Fed is controlled by the Board of Governors, appointed for 14 year terms by the president, controlling 12 district Fed banks
\item The Fed provides financial services to depository institutions, and acts as the bank of the US Treasury, controls monetary policy, regulates financial institutions, and provide financial stability (such as the discount window)
\item The reserve requirement allows them to use the multiplier to increase investment spending, but rarely uses it
\begin{enumerate}
\item If banks require additional reserves, they borrow from other banks trhough the federal funds market, at the federal funds rate, created organically
\item The reserve requirement is not the percentage, but the specific amount of money which needs to be held
\end{enumerate}
\item The discount rate is the rate of the window, typically 1\% higher than the federal fund rate to prevent it from being used to gain reserves, rarely used, though done in 2007 due to the financial crisis
\begin{enumerate}
\item Reducing the rate difference makes banks more willing to pay the higher rate, rather than paying the transaction costs of a proper sale, making it easier if they are low on reserves, used to increase lending (and the money supply)
\end{enumerate}
\item The Fed also had assets in the form of government debt, or treasury bills (typically 1 year maturity US government bonds, due to not being a part of the US government), and liabilities in the form of the monetary base
\begin{enumerate}
\item Open-market operations are buying and selling treasury bills, typically from commercial banks rather than the government, due to the latter being printing money to finance the government deficit, which causes rapid inflation
\item Purchasing in an open-market sale increases the monetary base and vice versa, able to create or remove the currency at will, triggering the money multiplier
\end{enumerate}
\end{enumerate}

\subsection{The Money Market}
\begin{enumerate}
\item The opportunity cost of holding money is based on the trade-off of ease of direct purchases for gaining no interest on it
\begin{enumerate}
\item This applies to a checking account rather than a savings account (or a certificate of deposit, with higher interest, but penelties for withdrawing before a certain amount of time)
\item This allows the federal government to adjust the federal funds interest rate (such as on 1/3 month Treasury bills), forcing down the rate of all short-term (< 1 year) interest rates by a similar amount, used in the financial crisis to incentivize currency
\item Thus, the Federal Funds rate is a measure of all short-term interest rates
\item Long-term rates are less applicable to holding money, and less related to the federal funds rate, generally ignored for the model, instead using short-term costs due to the main debate being over highly liquid assets or cash, due to the difficulties of conversion
\end{enumerate}
\item The money demand curve determines the quantity of money held based on the nominal interest rate, used rather than the real rate, due to the opportunity cost including the loss of purchasing power due to inflation
\begin{enumerate}
\item The curve is shifted by change in aggregate price level, real GDP, technology (making ease to get money), or institutions (changes in banking regulations, changing the opportunity costs)
\item Demand for money rises proportionally to price level, due to the need for more for the same purchasing power, such that the percent rise is the same for both
\item Real GDP increases the total amount of goods bought and sold, such that more goods are proportionally bought, and thus more money needs to be held
\end{enumerate}
\item The federal funds rate is the rate at which banks lend reserves to other banks to meet the required reserves ratio, targeted in terms of basis points (0.01\%) by the Federal Open Market Committee
\begin{enumerate}
\item The Fed Open Market Desk in NY, in charge of short-term debt, or treasury bills, then works to get the target
\item The liquidity preference model of the interest rate states that the money supply curve (a straight line, set by the Fed) intersects to find the equilibrium interest rate, due to the sellers adjusting the interest rate to get the correct number of buyers
\end{enumerate}
\end{enumerate}

\subsection{The Loanable Funds Market}
\begin{enumerate}
\item The loanable funds market is a hypothetical market of combined markets for all forms of financial assets, with the price as the yearly nominal interest rate assuming a constant inflation rate, due to the inability to know the future interest rate
\begin{enumerate}
\item Real interest rate is the actual price, but cannot be predicted if inflation rate changes
\item This assumes there is only one market, and thus interest rate, although there is a rate for each type of asset
\end{enumerate}
\item The rate of return on a project = 100\% * (revenue - cost)/cost
\begin{enumerate}
\item Thus, if the rate of return is greater than the interest rate, the loan will be taken, causing the demand curve to slope downward
\item Supply curves upward due to the higher gain from saving with higher interest rates
\item Thus, the intersection is the equilibrium interest rate
\item This allows projects with high return rates to be invested in, and causes those willing to lend funds for lower rates to generally loan, such that the groups are well-matched
\end{enumerate}
\item The demand curve is shifted by changes in percieved business opportunities (believed rate of return from specific investments) or government borrowing (deficit is the government borrowing, while surplus is supplying)
\begin{enumerate}
\item Thus, government deficit raises the interest rate, and decreases overall investment spending, called crowding out
\end{enumerate}
\item The supply curve is shifting by changes in capital inflows (perceptions of economic stability of a nation changes inflows) or private savings behavior (changes in consumer consumption)
\item Expected future inflation rate, and thus real interest rate, can also shift both curves, generally assuming previous trends
\begin{enumerate}
\item Higher expected interest rates cause the Fisher effect, or the upward shift to preserve the real interest rate
\end{enumerate}
\item Thus, as the money supply increases, the interest rate reduces, causing the increase in investment spending by the aggregate expenditure model, leading to an increase in GDP, increasing income by the multiplier, increasing savings
\begin{enumerate}
\item The increase in savings causes the shift of the supply curve, causing the shift of the loanable funds market model to the same interest rate point as the money supply model in the short run, due to long run output being the potential
\item In the long run, as money supply rises, such that the price level will rise proportionally by the neutrality of money, causing aggregate demand to shift it to the original rate
\item In addition, since GDP reverts to the original, causing the loanable funds supply to shift to the original position, the interest rate is not affected, such that only loanable funds can affect interest rate, not money demand
\item On the other hand, this assumes it begins are equilibrium, used ineffectively, such that rather, it can have a long-run effect
\end{enumerate}
\end{enumerate}

\section{Chapter 6 - Inflation, Unemployment, and Stabilization Policies}

\subsection{Deficits and Fiscal Policy Implications}
\begin{enumerate}
\item The change in the budget balance is used to monitor fiscal policy, but different forms of policy may have different effects on the economy (such as spending vs transfers)
\item In addition, budget balance changes are often the result, not the cause, of economic fluctuations due to stabilizers
\item Thus, the cyclically adjusted budget balance estimates the fiscal policy assuming potential output GDP
\item While long-term deficits are negative, but commonly used by politicians to gain support, an average budget balance is needed, but deficits are not negative due to allowing fiscal policy use
\begin{enumerate}
\item Long-term deficits lead to debt, keeping budgets by fiscal year from October 1st
\item While total government debt includes funds owed to certain government programs, public debt is to individuals and institutions, equal in 2009 53\% of GDP (69\% with state/local debt)
\item This creates the danger of crowding out, and pressure on future budgets due to interest, creating a danger of losing credit, preventing future loans, causing defaulting, such as Argentina in 2001 
\item The government can print money to avoid default, but that leads to inflation
\end{enumerate}
\item Long-term deficits do not show high levels of debt automatically, rather analyzed by the debt-GDP ratio, to determine the ease at which taxes can pay debts
\begin{enumerate}
\item Implicit liabilities are non-debts, which the government in committed to pay, such that it is effectively a debt
\item The US has high-entitlement implicit liabilities, such that it is a large percent of GDP, growing further over time
\end{enumerate}
\item Surplus in dedicated payroll taxes for entitlements, especially social security, were made since the 1980s, held by another part of the government as a Social Security trust fund
\begin{enumerate}
\item This is the remainder of the debt, not owed to the public, held in bonds, included due to being a committment for future payments (implicit liabilities to baby boomers)
\end{enumerate}
\end{enumerate}

\subsection{Monetary Policy}
\begin{enumerate}
\item The Federal Open Market Committee sets a target federal funds rate every 6 weeks, using open-market operations to change the money supply, and thus the rate through the money market curve
\begin{enumerate}
\item Other methods of interest rate regulation are not as commonly used by the Fed, except in major crisises
\end{enumerate}
\item The Fed tends to react to output gaps by adjustment, working to prevent recessions and inflation, only allowing inflationary gaps to remain in the 90s due to low inflation
\begin{enumerate}
\item The Taylor rule for monetary policy states that the Federal funds rate = 1 + (1.5 * inflation rate) + (0.5 * output gap), such that the direction has always been followed, and generally the amount
\item There are fewer lags in policy, due to lack of legal structure, but was limited in 2009 due to being unable to have a negative rate
\end{enumerate}
\item Many central banks (though not the Fed) set a target inflation rate (or range) instead of the Taylor rule, controlling monetary policy
\begin{enumerate}
\item Inflation targeting is based on future inflation, rather than past inflation (like the Taylor rule)
\item It provides transparency and accountability for the bank, but ignores other concerns, such as the stability of the financial system, which occasionally are the priorities (like 2007-08)
\end{enumerate}
\end{enumerate}

\subsection{Types of Inflation}
\begin{enumerate}
\item The classical model of price level assumes that the real quantity of money is always at the long-run level, instantaneously moving to the long-run, shown to be true by the aggregate demand-supply model
\begin{enumerate}
\item In the long run, changes in the nominal money supply, leads to the real quantity (M/P, where P is price level) to the same value
\item In low inflation, the short-run sticky wages are valid, but in high inflation, the adjustment period is shorter, such that sticky wages are non-valid, and the classical price level model is realistic
\item Thus, in high inflation, the increase in the money supply causes the immediate chnage in the price level
\end{enumerate}
\item In most modern countries, fiat money is created by the central bank, but the government can control it to some degree by the printing of T-bills (even with quantitative easing taking control away)
\begin{enumerate}
\item The US government pays interest to the Fed, though it is immediately returned, except for the money needed for Fed operations
\item Thus, the printing of money, or seignorage, is a form of revenue-production for the government, currently not used extensively to cover debts (<1\%), previously used during the Civil War, and in countries where lenders are unwilling to pay
\end{enumerate}
\item The printing of money causes government purchases to be paid for by those holding money, called an inflation tax, due to the reduction in value, imposing a tax equal to the rate
\begin{enumerate}
\item In hyperinflation, goods and interest-bearing assets are used for money, and try to destroy both nominal and real wealth, such that less money is taken by the inflation tax on real wealth
\item Seignorage is the change in money supply over some short period of time (\Delta M), such that real seignorage is the seignorage over the price level ($\frac{\Delta M}{P}$), or the rate of money supply growth * the real money supply ($\frac{M}{P}*\frac{\Delta M}{M}$)
\item Due to the destruction of wealth, the rate of money supply growth must be increased to gain the same real seignorage, causing higher inflation, creating a cycle
\end{enumerate}
\item Inflation is due to supply (cost-push inflation) from input prices changes, such as oil shocks, or demand (demand-pull inflation)
\begin{enumerate}
\item In the short run, policies causing rising inflation decrease unemployment (and cause an inflationary gap) and vice versa, creating political incentive against policies to lower inflation, even if seignorage is not used
\item This is due to the output gap determining if inflation is above or below the natural unemployment rate, allowing demand shocks to trigger inflation
\end{enumerate}
\end{enumerate}

\subsection{Phillips Curve}
\begin{enumerate}
\item The short-run Phillips curve is the trade-off between inflation and unemployment, reaching deflation as unemployment gets high enough, able to be shifted by supply shocks (up for negative shock) and changes in expectations of future inflation
\begin{enumerate}
\item Wage agreements take into account the expected inflation, agreed to by employers to avoid having to pay even higher wages for employees in the future, leading to the inflation rising by the same amount as the expected rise
\item People generally expect previous inflation to continue, such that high rates result in high expected rates of inflation
\end{enumerate}
\item In the long-run, there is no tradeoff of lower unemployment for inflation, due to temporary higher inflation for low unemployment causing the curve to shift upward by expectations when it moves back, such that the inflation at the original unemployment is the same as at the lower rate
\begin{enumerate}
\item Thus, there is an accelerating rate of inflation due to persistant attempts to lower unempoyment from the natural rate, or the nonaccelerating inflation rate of unemployment, where it remains constant
\item As a result of expectations, disinflation is difficult, keeping high unemployment, creating a trade-off of short-term high unemployment and recession for long-term rising inflation 
\item Clear policy transparency can also reduce the effects of disinflation policy by lowering expectations, requiring less action
\end{enumerate}
\item Deflation, common before WWII, existing in Japan in the 1990s, is a major issue due to debt deflation reducing aggregate demand by redistributing money to lenders, who are unlikely to increase consumption by as much as borrowers
\begin{enumerate}
\item The zero bound on the nominal interest rate is due to the lack of willingness to pay to loan money, limiting the ability to lower the rate, since negative rate makes holding cash have a higher real rate, such that spending will stop
\item Liquidity traps are situations where the zero bound prevents monetary policy use, when there is a loanable funds negative demand shock, lowering the rate, but as inflation expectations went up, the fisher effect made liquidty traps unlikely, raising the nominal rate
\item This occured again in Japan after the stock and housing bubbles burst, leading to deflation and a zero bound, and became a risk in the 2009 financial crisis, close to 0
\end{enumerate}
\end{enumerate}

\subsection{Historical and Alternative Macroeconomics}
\begin{enumerate}
\item The classical model of price level assumed that increases in the money supply led to inflation only, but it is commonly thought they didn't view short-run as vertical, just unimportant
\begin{enumerate}
\item Classical economists were aware of business cycles, but did not have data until the 1920s to measure them, and no theory
\item By the Great Depression, there was debate whether monetary/fiscal policy would prevent a recession, or postpone and enlarge it, and the degree of intervention, up to government ownership of firms
\item The Great Depression proved that the short run could cause major social and political problems, forcing theory development
\end{enumerate}
\item In 1936, Keynes published his ``The General Theory of Employment, Interest, and Money'', analyzing the Great Depression, becoming the main economic school of modern day
\begin{enumerate}
\item He emphazied the short-run as an important factor, creating the slanted short-run aggregate supply curve, rather than just the long-run curve
\item He also emphasized business confidence (along with other factors) in causing the business cycle, rather than classic idea of money supply as the main factor in shifting aggregate demand
\item Classical ideas stated that business confidence had virtually no effect on aggregate demand
\end{enumerate}
\item Keynsian theory encouraged macroeconomic political activism (fiscal policy), which had only been used to a limited degree before due to fears of deficits, or greater recessions later
\item Keynes stated monetary policy was limited due to liquidity traps, but in the 60s, Friedman and Schwarz showed that business cycles were historically associated with changes in the monetary supply
\begin{enumerate}
\item They believed monetary policy could have avoided the depression and needed to be used, taking policy away from politics, making it more mathematical, due to not prioritizing groups in monetary policy
\item Friedman then began the monetarism movement based on monetary over fiscal, and argued that a constant rate of supply growth would cause GDP growth, targeting a monetary supply growth rate of 3\%
\item They also argued that lags did not apply as much to monetary policy, but should only be used in emergencies due to lags, and that not changing the money supply in response to the business cycle would remove the effectiveness of fiscal policy by crowding out
\item He believed in the monetary policy rule, from the quantity of theory of money which relies on the velocity of money (the measure of times per year the money moves from buyer to seller), stating $MV = PY$, where M is money supply, V is velocity, P is aggregate price, and Y is real GDP
\item The quantity of money theory stated that velocity only changed gradually in the long-run, such that if money supply was stable, the other factors would be as well, the premise of which was shown to be wrong
\end{enumerate}
\item Keynsians also believed that it could be used to get full-employment, but in the late 60s, Phelps and Friedman created the idea of the natural rate of unemployment, limiting macroeconomic policy
\item It is also critisized for causing the policial business cycle, due to elections being decided the business cycle, leading to extreme poisitve policy before an election creating instability
\item In the 70-80s, new classical macroeconomic theory was invented, arguing that aggregate demand shifts affect only price level
\begin{enumerate}
\item The rational expectations theory was made by Muth in 1961, arguing that all individuals/firms make decisions optimally based on all available information, including expected inflation, monetary, and fiscal policy
\item Rational expectations takes away the natural rate hypothesis effect of short-run low inflation and unemployment, with rising inflation in the long-run, but rather immediately rising as the people observe the desired effect
\item Thus, lowering unemployment beyond the natural rate only works if it surprises the public
\item The real business cycle theory argued that the causes of recession are a slower rate of long-run productivity growth (generally less technological growth) cause the business cycle
\item Thus, ignoring short-run, it causes the shifting of the aggregate supply vertical curve, such that aggregate demand had no effect on GDP
\end{enumerate}
\item In the 90s, New Keynesian economics responded to rational expectations, arguing market imperfections preserve sticky prices, such as imperfect competition resulting in prices slightly around optimal levels, leading to slow price changing by monopolies
\begin{enumerate}
\item Real data has shown the New Keynesian economics is more valid than new classical, with the latter expanding understanding of the economy, and serving as a caution against monetary/fiscal policy
\end{enumerate}
\end{enumerate}
