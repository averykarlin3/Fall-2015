\section{Heine-Borel Theorem and Domain Properties}

\subsection{Heine-Borel Theorem}
Let K be a compact set in $\mathbb{R^d}$, and let ${U_\lambda | \lambda \in \Lambda}$ be a family of open sets in $\mathbb{R^d}$, which covers K, such that $K \subseteq \cup_{x \in \Lambda}U_\lambda$. Then $\exists \lambda_1, \lambda_2, \lambda_3 \text{... such that} K \subseteq U_{\lambda_1} \cup U_{\lambda_2} \cup \text{...} \cup U_{\lambda_n}.$

\subsubsection{Proof}
Since K is bounded, it can be fully contained within some rectangle (R), which can then be split into 4 congruent parts, $R_{i_1} (R_1, R_2, R_3, R_4)$. Suppose one quadrant cannot be covered by any finite collection of $U_\lambda$. By extension, if that quadrant is divided further, one of the subquadrants ($R_{i_1i_2}$) cannot be covered. This can continue for countable infinity divisions (able to be counted with an infinite amount of integer). Since $R_{i_1i_2...i_n} \neq \emptyset$, let $P_n$ be any point in $K \cap R_{i_1i_2...i_n}$, such that there is a bounded sequence of points in K. Thus it must have a convergent subsequence such that $P_{n_k} \to P, P \in K$. Thus, $P \in U_{\lambda^*}$ for some $\lambda^* \in \Lambda$. Since $U_{\lambda^*}$ is open, $\exists r > 0, \text{such that} B_r(P) \subseteq U_{\lambda^*}$. Diameter/diagonal length (diam) of $ R_{i_1i_2...i_n} = \frac{diam(R)}{2^n} < r$ as $n \to \infty$. Thus, $R_{i_1i_2...i_n} \subseteq B_r(P)$ as $n_k \to \infty$. $dist(P, P_{n_k}) < \frac{r}{2}$ and $dist(P_{n_k}, x) \leq diam(R_{i_1i_2...i_n})$ and by the triangle inequality, $diam(R_{i_1i_2...i_n}) < \frac{r}{2}$. Thus, there is a contradiction, and it must be covered by a finite number of sets.

\subsection{Uniform Continuity}
Let $f: D \to \mathbb{R}$, where $D \subseteq \mathbb{R}^d$. We say $f$ is
uniformally continuous on $D$ if:
$$\forall \epsilon > 0, \exists+\delta > 0: \forall \vec{x}, \vec{y} \in D,
||+\vec{x} - \vec{y}+|| < \delta \to |f(\vec{x}) - f(\vec{y})| < \epsilon$$

The difference between this and regular continuity is that the $\delta$ in
regular continuity is defined by both $\epsilon$ and the specific point we
are considering. Uniform continuity, however, the value $\delta$ is independent
to the point you chose within the domain and is just dependent on $\epsilon$.

For example, consider $y = \tan{x}$ where $D = (-\frac{\pi}{2}, \frac{\pi}{2})$.
the value required for $\delta$ for a fixed $\epsilon$ gets smaller and
smaller as $x$ approaches both endpoints. This function is continuous but not
uniformally so. If it were uniformally continuous,
that $\delta$ value would NOT change.

\underline{Theorem}: If $f$ is uniformally continuous on $D$, then $f$ is continuous
for every point in $D$.

\subsection{Uniform Continuity Theorem}
\underline{Theorem:}
If $f: K \to \mathbb{R}$, where $K \subseteq \mathbb{R}^d$ is compact, and
$f$ is continuous at each $\vec{x} \in K$. Then $f$ is uniformally continuous on
$K$.

\underline{Proof:}
Fix $\epsilon > 0$. For each $\vec{x} \in K$, let $u_{\vec{x}}$ be an open ball, centered at
$\vec{x}$ such that for any $\vec{y} \in K \cap 2u_{\vec{x}}$ [$2u_{\vec{x}}$ is
an open ball centered at $\vec{x}$ with twice the radius of $u_{\vec{x}}$], $|f(\vec{x}) -
f(\vec{y})| < \frac{\epsilon}{2}$. Because the function is continuous at
$\vec{x}$, there is a radius $2\delta$ around $\vec{x}$ such that $|f(\vec{x}) -
f(\vec{y})| < \frac{\epsilon}{2}$. Now we see that every $\vec{x} \in K$ is
covered by at least one such open ball, namely $u_{\vec{x}}$. The collection
$\{u_{\vec{x}} +|+ \vec{x} \in K\}$ is an open covering of $K$. By
Heine-Borel, we can select a finite set of points $\vec{x}_1, \vec{x}_2, \dots,
\vec{x}_n$ such that $K$ is covered by $u_{\vec{x}_1} \cup u_{\vec{x}_2} \cup \dots
\cup u_{\vec{x}_n}$. Take $\delta = \min\{\delta_1, \delta_2, \dots, \delta_n\}
> 0$  where $\delta_j$ is the radius of $u_{\vec{x}_j}$ for $j = 1,2,3,\dots,n$.

Let $\vec{p}, \vec{q} \in K$ such that $||+\vec{p} - \vec{q}+|| < \delta$. We
need to prove that $|f(\vec{p}) - f(\vec{q})| < \epsilon$. $\vec{p} \in K \to
\exists + j : \vec{p} \in u_{\vec{x}_j} \to ||+\vec{p} - \vec{x}_j+|| <
\delta_j$. But we know that $||+\vec{p} - \vec{q}+|| < \delta_j$. Now by the
Triangle Inequality, we get:

$$||+\vec{q} - \vec{x}_j+|| \leq ||+\vec{q} - \vec{p}+|| + ||+\vec{p} - \vec{x}_j+||
\leq 2 \delta_{\vec{x}_j}$$

This means that $\vec{q} \in 2u_{\vec{x}_j}$. By the way we picked our $\delta$,
we know that $|f(\vec{q}) - f(\vec{x}_j)| < \frac{\epsilon}{2}$. Similarly,
we also have $|f(\vec{p}) - f(\vec{x}_j)| < \frac{\epsilon}{2}$. Now if we
apply the triangle inequality again, we get:

$$|f(\vec{p}) - f(\vec{q})| = |f(\vec{p}) - f(\vec{x}_j) + f(\vec{x}_j)- f(\vec{q})|\leq |f(\vec{p}) - f(\vec{x}_j)| + |f(\vec{q}) -
f(\vec{x}_j)| < \epsilon$$

Therefore, $f$ is uniformally continuous over $K$.

\subsection{Connected}

\subsubsection{Definition}
$D \subseteq \mathbb{R}^d$ is said to be connected if $\forall \vec{p},
\vec{q} \in D$, $\exists$ a continuous function $\vec{\gamma}:[a, b] \to
\mathbb{R}^d$ that $\vec{\gamma}(a) = \vec{p}$, $\vec{\gamma}(b) = \vec{q}$,
and $\forall t \in [a, b], \vec{\gamma}(t) \in D$

\subsubsection{Theorem}

Let $D \subseteq \mathbb{R}^d$ be connected, $D \neq \emptyset$. Suppose $D =
A \cup B$, where $A$ and $B$ are open. Such that $A \cap B = \emptyset$. Then $A
= \emptyset$ or $B = \emptyset$.

\underline{Proof}:

Assume that $D$ can be broken up into two open, non-empty sets $A$ and $B$ such
that $D = A \cup B$. Because $A \neq \emptyset$, we can pick $\vec{p} \in A$ and
similarly we can pick $\vec{q} \in B$. Clearly, $\vec{p}, \vec{q} \in D$. Since
$D$ is connected, $\exists$ continuous $\vec{\gamma}: [a, b] \to D$ such that
$\vec{\gamma}(a) = \vec{p}$, $\vec{\gamma}(b) = \vec{q}$. Let $S = \{t \in [a,b]
+|+ \vec{\gamma}(t) \in A\}$. We know that $a \in S$, so $S \neq \emptyset$.
Note that $S \leq b$, and since it has a bound, it has a supremum. Let $t_0 =
\sup S$, note that $a \leq t_0 \leq b$. Since $\vec{\gamma}(t_0) \in D$, it is
either in $A$ or in $B$ (since $A \cap B = \emptyset$). Suppose
$\vec{\gamma}(t_0) \in A$. Since $A$ is open, we can draw an open ball
($B_\epsilon$) around $\vec{\gamma}(t_0)$. Now consider $\vec{\gamma}(t_0 +
\delta)$ for some small positive $\delta$. If $\delta$ is small enough, $\delta
\in B_\epsilon \in A$. This is a contradiction, as then it means that $t_0 +
\delta \in S$. This is a contradiction, as then $t_0 \neq \sup S$. A similar
contradiction can be made for the case that $\vec{\gamma}(t_0) \in B$. Therefore
the original assumption was false.

\section{Intermediate Value Theorem}

\subsection{Theorem in $\mathbb{R}$}
If f: [a, b] $\to \mathbb{R}$ is continuous, and $f(a) \neq f(b)$, $\forall y \in (f(a), f(b))$, then $\exists c \in (a, b): f(c) = y.$

\subsubsection{Proof}
Let there exist y, without loss of generality, such that $f(a) < y < f(b)$. Take the set $S = \{x \in [a, b] | f(x) \leq
y\}$. Since $a \in S$, $S \neq \emptyset$, we also know that $S \leq b$. Now
take $c = \sup S$, $a \leq c \leq b$ ($c \in [a, b]$). There are three possible
cases, $f(c)$ is either $> y, < y$, or $= y$.

Consider the case in which $f(c) < y$. If this is the case, then we can chose an
arbitrarily small $\delta > 0$ such that $f(c + delta) < y$. However, then $c +
\delta \in S$. But this causes a contradiction because $c = \sup S$ and there
should not be any element of $S$ that's larger than $c$. Therefore, this case is
impossible.

Now consider the case in which $f(c) > y$. Take some arbitrarily small $u \in
[0, \delta], \delta > 0$ such that $f(c - u) > y$. However, then $c - \delta$ is
therefore an upper bound of the set $S$, we once again reach the same
contradiction.

Therefore, since $c$ exists, $f(c) = y$.

\subsection{Theorem in $\mathbb{R^n}$}

If $f : D \to \mathbb{R}$ where $D \subseteq \mathbb{R}^d$ is connected
and continuous. $\forall \vec{p}, \vec{q} \in D$ such that $f(\vec{p}) \neq
f(\vec{q})$ and if $y \in \mathbb{R}$ is between $f(\vec{p})$ and $\vec{q}$,
then $\exists \vec{r} \in D$ such that $f(\vec{r}) = y$.

\subsubsection{Proof}

Without loss of generality, let us assume $f(\vec{p}) < f(\vec{q})$.

Since $D$ is connected, there is some path $\vec{\gamma}(t) = (x(t), y(t)) \in
D$, $a \leq t \leq b$ such that $\vec{\gamma}(a) = \vec{p}$ and $\vec{\gamma}(b)
= \vec{q}$ and is continuous over $[a, b]$. Now we construct $g(t) =
f(\vec{r}(t)) = (f\cdot\vec{r})(t) \in mathbb{R}$. Because $g(t)$ is a
composition of continuous functions, $g(t)$ is also continuous. $g(a) =
f(\vec{\gamma}(a)) = f(\vec{p})$ and $g(b) = f(\vec{\gamma}(b)) = f(\vec{q})$.
Now we see that $g(a) < y < g(b)$. Therefore, by the IVT for single-variable
functions, $\exists+t_0$ such that $g(t_0) = y$. Now we plug $t_0$ into
$\vec{\gamma}$, $\vec{r} = \vec{\gamma}(t_0) \in D$. Then $f(\vec{r}) =
f(\vec{\gamma}(t_0)) = g(t_0) = y$.

\section{Vector Valued Functions}

\subsection{Definition}

$f: D \to \mathbb{R}$, $D \subseteq \mathbb{R}^d$ is known as
\underline{real-valued} functions.

$\vec{f}: D \to \mathbb{R}^e$, $D \subseteq \mathbb{R}^d$ is known as
\underline{vector-valued/point-valued} functions.

$\vec{f}(\vec{p}) = (f_1{\vec{p}}, f_2{\vec{p}}, \dots, f_e(\vec{p}))$ where
$f_1, f_2, \dots, f_e$ are real-valued and are called the \underline{component
functions} of $f$.

\subsection{Continuity}
$f: \mathbb{R}^d \to \mathbb{R}^e$ is continuous at $\vec{a} = (a_1, a_2,
a_3, \dots, a_d) \in D$ iff $\forall \epsilon, \exists+\delta > 0 : \forall \vec{x} \in D$:

$$||+\vec{x} - \vec{a}+||_d < \delta \to ||+f(\vec{x}) - f(\vec{a})+||_e <
\epsilon$$

\subsubsection{Component-wise Nature of Continuity}
$f: D \to \mathbb{R}^e$ is continuous at a point $\vec{a} \in D$ iff $f_1,
f_2, \dots, f_e$ are all continuous at $\vec{a}$.

\underline{Proof}:

First fix an $\epsilon$, then by the basic distances bound lemma, we get a
bunch of inequalities:

\begin{align*}
    |f_1(\vec{x}) - f_1(\vec{p})| < \frac{\epsilon}{\sqrt{e}}, &\forall \vec{x} \in D \cap B_{\delta_1}(\vec{p})\\
    |f_2(\vec{x}) - f_2(\vec{p})| < \frac{\epsilon}{\sqrt{e}}, &\forall \vec{x} \in D \cap B_{\delta_2}(\vec{p})\\
    ... \text{ } ... \text{ } & \text{ } ... \text{ }...\\
    |f_e(\vec{x}) - f_e(\vec{p})| < \frac{\epsilon}{\sqrt{e}}, &\forall
    \vec{x} \in D \cap B_{\delta_e}(\vec{p})
\end{align*}

Then let $\delta = \min\{\delta_1, \delta_2, \dots, \delta_e\} > 0$, which must
exist and satisfy all the distance inequalities, specifically $\max_{1 \leq j
\leq e}(|f_j(\vec{x}) - f_j(\vec{p})|)$. Then again, by the basic distance
bounds lemma, we know that $||+\vec{f}(\vec{x}) - \vec{f}(\vec{p})+||_e \leq \max_{1 \leq j
\leq e}(|f_j(\vec{x}) - f_j(\vec{p})|)$. Therefore we get that for any fixed
$\epsilon$, we can find a $\delta$ such that $||+\vec{x} - \vec{p}+||_d <
\delta \to ||+f(\vec{x}) - f(\vec{p})+||_e < \epsilon$

\subsubsection{Composition of Continuous Functions}
If $\vec{f}: D \to E$, where $D \subseteq \mathbb{R}^d$, $E
\subseteq \mathbb{R}^e$ and $\vec{g}: E \to \mathbb{R}^k$ are both
continuous on their respective domains. Then $\vec{h} = \vec{g} \circ \vec{f}$
is continuous on $D$

\underline{Proof}:

To prove this, fix $\epsilon > 0$. There's a $\eta > 0$ such that
$\forall \vec{y} \in E \cap B_\eta(\vec{f}(\vec{p}))$, because we know that
$\vec{g}$ is continuous at $\vec{f}(\vec{p})$, we get:

$$||+\vec{g}(\vec{y}) - \vec{g}(\vec{f}(\vec{p}))+|| < \epsilon$$

To guarantee that $\vec{y} = \vec{f}(\vec{x})$ lies within $\eta$ units of
$\vec{f}(\vec{p})$ i.e. $||+\vec{f}(\vec{x}) - \vec{f}(\vec{p})+|| < \eta$, we
can take $\vec{x} \in D \cap B_\delta(\vec{p})$ where $\delta > 0$ corresponding
to $\eta$ [using the continuity of $\vec{f}$ at $\vec{p} \in D$].

Now, as long as $\vec{x} \in D \cap B_\delta(\vec{p})$, we have
$\vec{f}(\vec{x}) \in E \cap B_\eta(\vec{f}(\vec{p}))$. Thus, if we take
$\vec{y} = \vec{f}(\vec{p})$, we get: $||+\vec{g}(\vec{f}(\vec{x})) -
\vec{g}(\vec{f}(\vec{p}))+|| < \epsilon$, or $||+\vec{h}(\vec{x}) -
\vec{h}(\vec{p})+|| < \epsilon$. So $\vec{h}$ is continuous at $\vec{p}$.

\subsection{Compactness Theorem}

\subsubsection{Theorem}

Let $\vec{f} : D \to \mathbb{R}^e$ be a contnuous function, where $D \subseteq
\mathbb{R}^d$ is compact. Then its \textit{range} $\vec{f}(D) := \{f(\vec{p}) +|+
\vec{p} \in D\}$ is also compact. In other words: compactness is preserved under
continuous mappings. Note that this is the generalization of the Extreme Value
Theorem.

\subsubsection{Proof}

To prove this, write $R := f(D)$. We need to show that $R$ is closed and bounded
in $\mathbb{R}^e$. Boundedness is easy. Since each component function $f_j$
of $f$ is real valued, by EVT each component function $f_j$ has an absolute
bound $M_j$, so that $|+f_j(\vec{p})+| \leq M_j$ for all $\vec{p} \in D$. 
Take $M := \max\{M_1, M_2, \dots, M_e\}$. Then for all $\vec{p} \in D$, we have

$$||+\vec{f}(\vec{p})+|| = \sqrt{f_1(\vec{p})^2 + \dots + f_e(\vec{p})^2} \leq
\sqrt{e \times M^2} = M \sqrt{e}$$

This says that the range $R$ lies within the closed ball of radius $M\sqrt{e}$
centered at $\vec{0}$ in $\mathbb{R}^e$. It therefore certainly lies within
some closed cube centered at $\vec{0}$, and hence is bounded.

To prove closedness, let $(\vec{y}_n)_{n = 1}^\infty$ be a convergent sequence in
$\mathbb{R}^e$ with limit $\vec{y}$, such that $\vec{y}_n \in R$ for each $n
\geq 1$. We need to prove that $\vec{y} \in R$. Since $R$ is the range of $f$,
we must have $\vec{y} = \vec{f}(\vec{x_n})$, where $\vec{x_n} \in D$. Because
$D$ is bounded, we can pick a convergent subsequence $\vec{x}_{n_k} \to
\vec{x}$. But because $D$ is closed, we know that $\vec{x} \in D$. Because
$\vec{f}$ is continuous, we get that $\vec{y}_{n_k} = \vec{f}(\vec{x}_{n_k}) \to
\vec{f}(\vec{x})$. However, we know that $\vec{y}_{n_k} \to \vec{y}$. But since
each sequence converges to one point, we know that $\vec{y} = \vec{f}(\vec{x})$,
where $\vec{x} \in D$. Therefore, $\vec{y} \in R$.

Therefore, $R$ is closed and bounded.

\subsection{Connectedness Theorem}

\subsubsection{Theorem}

$\vec{f}: D \to \mathbb{R}^e$, $D \in \mathbb{R}^d$ is continuous on $D$.
If $D$ is connected, $E = \vec{f}(D)$ (the range of the domain), is also
connected. Note that this is the generalization of the Intermediate Value
Theorem.

\subsubsection{Proof}

$\forall \vec{u}, \vec{v} \in E$, we can find two points $\vec{p}, \vec{q}$ such
that $\vec{f}(\vec{p}) = \vec{u}$ and $\vec{f}(\vec{q}) = \vec{v}$. Now because
$D$ is connected, $\exists+ \vec{\gamma}: [a, b] \to \mathbb{R}^d$,
$\vec{\gamma}([a, b]) \subseteq D$. Now we consider $\vec{\delta} =
\vec{f}(\vec{\gamma}(t))$, $\vec{\delta} : [a, b] \to \mathbb{R}^e$. Since
it's a composition of continuous functions, $\vec{\delta}$ is also continuous.
And $\forall t \in [a, b]$, $\vec{\delta}(t) = \vec{f}(\vec{\gamma}(t)) \in E$.
We also know that $\vec{\delta}(a) = \vec{f}(\vec{\gamma}(a)) = \vec{f}(\vec{p})
= \vec{u}$, and $\vec{\delta}(b) = \vec{f}(\vec{\gamma}(b)) = \vec{f}(\vec{q})
= \vec{v}$. Therefore, $\forall \vec{u}, \vec{v} \in E$, there is a
continuous path that connects $\vec{u}$ to $\vec{v}$ and stays within $E$.
Therefore, $E$ is connected.

\section{Sequences of Functions}
\subsection{Infinity Norm}
For $f: [a,b] \to \mathbb{R}$ that is bounded, we say:

$$||f||_\infty := ||f||_D := \sup_{x\in[a,b]} |f(x)|$$

\subsection{Pointwise Convergence:}

\underline{Definition:}
For $f_n : [a,b] \to \mathbb{R}$ for $n \geq 1$, and assume that they are all bounded on $[a,b]$. ($\exists+ M_n > 0 : |f_n(x)| \leq M_n$ for all $x \in [a,b]$), and $f:[a,b] \to \mathbb{R}$, $f$ is bounded ($\exists+ M > 0: |f(x)| \leq M$ for all $x \in [a,b]$).

If $\forall x \in [a,b]: \lim_{n\to\infty} f_n(x) = f(x)$, then we say that $f_n \overset{p}{\to} f(x)$ ($f_n$ converges ``pointwise'' to $f$).

\underline{Theorem:}
If $\vec{f_n} \overset{u \text{(uniform)}}{\to} \vec{f}: \mathbb{R^d} \to \mathbb{R^e}$, then $\vec{f_n} \overset{p \text{(pointwise)}}{\to} \vec{f}$ on D: $\vec{f_n}(\vec{x}) \to \vec{f}(\vec{x})$ for every point $\vec{x} \in D$ (for any converging sequences, any point will also converge similarly, such that uniform convergence implies pointwise convergence, though the converse is untrue)

\underline{Proof:}
$\vec{f_n} \overset{u}{\to} \vec{f}$ on D (uniform convergence): $||\vec{f_n} - \vec{f}||_D = sup_{\vec{x} \in D} ||\vec{f_n}(\vec{x}) - \vec{f}(\vec{x})|| \to 0$ as $n \to \infty.$ $Then 0 \leq ||\vec{f_n}(\vec{x}) - \vec{f}(\vec{x})|| \leq  ||\vec{f_n} - \vec{f}|| \to 0.$

Then, by the squeeze theorem, for a uniform sequences, $f_n(x) \to f(x)$ on D, giving pointwise convergence.

\subsection{Uniform Convergence}
\subsubsection{In $\mathbb{R}$}
$f_n : [a,b] \to \mathbb{R}$ for $n \geq 1$, and assume that they are all
bounded on $[a,b]$. ($\exists+ M_n > 0 : |f_n(x)| \leq M_n$ for all $x \in
[a,b]$).

$f:[a,b] \to \mathbb{R}$, $f$ is bounded ($\exists+ M > 0: |f(x)| \leq M$
for all $x \in [a,b]$).

We then claim that $f_n \overset{u}{\to} f$ as $n \to \infty$ if $\forall
\epsilon$, $\exists+ N_\epsilon$ such that $\forall n \geq
N_\epsilon : ||+f_n - f+||_\infty < \epsilon$. In other words, this
forces that the greatest vertical difference between the two functions will be
arbitrarily small after $N_\epsilon$. This forces the two functions to be
``close'' as a whole.

\subsubsection{In $\mathbb{R^n}$}

Let $D \in \mathbb{R}^d$ be a non-empty set, and let $\vec{f}:
D\to\mathbb{R}^e$ and $\vec{f}_n: D \to \mathbb{R}^e$ for $n \geq 1$. We
say that $\vec{f_n} - \vec{f}$ on $D$ if:

$$||+f_n - f+||_D \to 0 \text{ as } n \to \infty$$

\subsection{Uniform Convergence Theorem}
\subsubsection{Theorem}
If $f_n \overset{u}{\to} f$ on $[a, b]$, where each $f_n$ is continuous on
$[a,b]$. Then $f$ is also continuous on $[a,b]$

\subsubsection{Proof}
Pick any $\vec{p} \in D$, and fix $\epsilon > 0$. We need to find $\delta$
such that $\forall \vec{x} \in D$ with $||\vec{x} - \vec{p}|| < \delta$,
then $||\vec{f}(\vec{x}) - \vec{f}(\vec{p})|| < \epsilon$. 
We can apply the triangle inequality and we get:

$$||\vec{f}(\vec{p}) - \vec{f}(\vec{p})|| \leq ||\vec{f}(\vec{x}) 
\vec{f}_n(\vec{x})|| + ||\vec{f}_n(\vec{x}) - \vec{f}_n(\vec{p})|| +
||\vec{f}_n(\vec{p}) - \vec{f}(\vec{p})||$$

For large enough $n$, we know that $||\vec{f}_n - \vec{f}||_D < \frac{\epsilon}{3}$
because $\vec{f}_n - \vec{f}$. Now we know that the first and third ter
are bounded by $\frac{\epsilon}{3}$. The second term is bounded b
$\frac{\epsilon}{3}$ because $\vec{f}_n$ is uniformally continuous.

Therefore, we know that $||\vec{f}(\vec{p}) - \vec{f}(\vec{p})|| \le
\epsilon$ for any $\delta$ we pick. $\therefore \vec{f}$ is continuous on $D$

\section{Differentiation}

\subsection{Differentiable in $\mathbb{R}$}
Let $f: [a, b] \to \mathbb{R},$ let $p \in (a, b) = [a, b]^o (int [a, b]),$, then f is \underline{differentiable} at p if $ \exists a \in \mathbb{R}$ such that $a = \lim{h \to 0} \frac{f(p+h) - f(p)}{h}$, called the \underline{derivative} at point p $(a = f'(p) = \frac{df}{dx}(p)).$

\subsection{Differentiable in $\mathbb{R^n}$}
This is generalized to high dimensions, such that for $f: D \subseteq \mathbb{R^d} \to \mathbb{R}$, and $\vec{p} \in D^o (\exists r > 0: B_r(\vec{p} \subseteq D)$. Thus, it can be approached from any given direction, due to the ball existing in all directions.

$\vec{h} \to \vec{0}$ iff $h_d \to 0$, such that it can be substituted for the limit. Due to the lack of vector division though, the definition of the derivative has to be changed.

By the previous definition, $\lim{h \to 0} |\frac{f(p+h) - f(p)}{h} - a| = 0$, rewritten  $\lim{h \to 0} \gamma(p, h) = 0$. $$ |h|\gamma(p, h) = |f(p+h) - f(p) - ah| < \epsilon|h| \text{ as } h \to 0 (|h| < \delta).$$ $\exists a \in \mathbb{R}$, such that $\forall \epsilon > 0, \exists \delta > 0: |h| < \delta$, then $|f(p + h) - f(p) - ah| < \epsilon|h|$, defining a as the derivative at p.

This is due to the idea that for some function, g(x), with linear approximation l(x), $\frac{|g(x) - l(x)|}{|x|} \to 0$ as $x \to 0$, such that the y distance decreases far faster than the x decay, called super-linear (suplinear) decay. Then, there must at most be 1 line that can superlinearly approximate g at x = 0. The superlinear decay curve is then the derivative function, where f(p + h) - f(p) = g(x) and ah = l(x).

\underline{Proof:}
Assume there are two function, l(x) = ax and m(x) = bx, then $\frac{|g(x) - ax|}{|x|} \to 0$ as $x \to 0$ and $\frac{|g(x) - bx|}{|x|} \to 0$ as $x \to 0$. Then $0 \leq |a - b| \frac{|ax - bx|}{|x|} = \frac{|(g(x) - bx) - (g(x) - ax|)}{|x|} \leq \frac{|g(x) - bx|}{|x|} + \frac{|g(x) - ax|}{|x|} \to 0$ as $x \to 0$, such that a = b. This proof can be done in each component for higher dimensions, using the dot product to remove all other components, by the properties of the dot product.

The definition can now be easily moved to higher dimensions, by superlinear decay, such that there is only one object in $\mathbb{R^{n-1}}$, a hyperplane, or the set of all vectors orthoganal to the non-zero normal vector, all anchored to a specific point, $\vec{p_0}$, such that superlinear decay takes place.

\underline{Definition:}
A hyperplane in $\mathbb{R^{d+1}}$ is a set of the form, Q = \{$\vec{x} \in \mathbb{R^{d+1}} | (\vec{x} - \vec{p_0} \cdot \vec{n} = 0)$, where $\vec{p_0}, \vec{n} \in \mathbb{R^{d+1}}$ with $\vec{n} \neq \vec{0}$, where n is a normal of Q $(\vec{n} \perp Q)$.

Thus to summerize differentiability for real valued functions, $\exists !$ (exists exactly one) or $~\exists \vec{a} \in \mathbb{R}$ such that $\forall \epsilon > 0, \exists \delta > 0: ||\vec{h}|| < \delta$, then $|f(\vec{p} + \vec{h}) - f(\vec{p}) - \vec{a}\vec{h}| < \epsilon||\vec{h}||$. If the latter is true, f(x) is said to be \underline{differentiable} at $\vec{p}$, since it is a unique value, denoted f'(p). 

\subsection{Gradient}

This is then defined specifically such that $\vec{a} = (\vec{\nabla}f)(p)$, or the gradient of f at p.

\underline{Claim:}
The graph of $y = f(\vec{p}) + \vec{a} \cdot (\vec{x} - \vec{p})$ is a hyperplane in $\mathbb{R^{d+1}}$, such that the gradient is a hyperplane.

\underline{Proof:}
Let $\vec{N}$ (the normal vector to the hyperplane$ \in \mathbb{R^{d+1}} = (-a_1, -a_2, ..., -a_d, 1), and \vec{P_0} \in \mathbb{R^{d+1}} = (p_1, p_2, ..., p_d, f(\vec{p}))$and $\vec{X} \in \mathbb{R^{d+1}} = (x_1, x_2, ..., x_d, y)$. Thus, $\vec(X) - \vec{P_0} = (x_1 - p_1, x_2 - p_2, ... x_d - p_d, y - f(\vec{p}))$, giving the equation of a hyperplane $(\vec{X} - \vec{P_0}) \cdot \vec{N} = 0$.

If $\vec{p} + \vec{h} = \vec{X}$, then $|f(\vec{X}) - [f(\vec{p}) + \vec{a} \cdot (\vec{X} - \vec{p})|| < \epsilon||\vec{h}||$. This is equal to the difference between the point on the graph and the function approximation. This can also be seen to be easily equal to the equation of the hyperplane from above, such that the gradient is the hyperplane.

\subsubsection{Gradient Representation}

Since $\vec{N}$ is the normal vector to the plane at that point, equal to $(-\vec{\nabla}f(\vec{p}), 1)$, or the projection of the negation of the normal vector onto the domain, such that it is the vector of the direction and magnitude of the fastest increase of the function at $\vec{p}$.

\underline{Theorem:}
If $f:D \subseteq \mathbb{R^d} \to \mathbb{R}$ is differentiable at $\vec{p} \in D^o$, then $\vec{u_0} = \frac{\vec{\nabla}f(\vec{p})}{||\vec{\nabla}f(\vec{p})||}$ is the direction of steepest ascent for f at $\vec{p}$, and $||\vec{\nabla}f(\vec{p})|| = max_{||\vec{u}||=1} \partial_\vec{u} f(\vec{p})$, where $\vec{u}$ is any unit vector, such that as a result, $\partial_{\vec{u}} f(\vec{p}) = \vec{\nabla}f(\vec{p}) \cdot \vec{u}$.

This is by the Cauchy-Schwartz equality case $(-||\vec{u}||||\vec{v}|| \leq \vec{u} \cdot \vec{v} \leq ||\vec{u}||||\vec{v}||$, with equality with the lower bound if $\vec{u}$ and $\vec{v}$ are in opposite directions, the higher bound if in the same direction), such that $-\vec{u_0}$ is the direction of steepest descent for f at $\vec{p}$ and $-||\vec{\nabla}f(\vec{p})|| = min_{||\vec{u}||=1} \partial_{\vec{u}}f(\vec{p}).$ 


\subsubsection{Gradient Calculation}

Take $\vec{h} = h\vec{e}_j$ , where $h\to 0$ and the set of $\vec{e}_j$ is known
as the \textit{standard basis vectors} in $\mathbb{R}^d$:

\begin{equation*}
\vec{e_j}=
\begin{cases}
\vec{e}_1 &= (1,0,0,\dots,0)\\
\vec{e}_2 &= (0,1,0,\dots,0)\\
\vdots & \vdots \hfill \vdots \hfill \vdots \hfill \vdots\\
\vec{e}_d &= (0,0,0,\dots,1)\\
\end{cases}
\end{equation*}

Note that because $||\vec{e}_j|| = 1$, $||\vec{h}|| = |h|||\vec{e}_j|| = |h|$. Now if we fix a $j \in \{1,2,3,\dots,d\}$ and apply the definition of differentiability, the unique gradient ($\vec{a}$) must satisfy:

$$\forall \epsilon > 0, \exists \delta > 0: \forall h \text{ with } |h| < \delta \implies |f(\vec{p} + h\vec{e}_j) - f(\vec{p}) - \vec{a}\cdot h\vec{e}_j| < \epsilon|h|$$

However, note that when we dot $\vec{a}$ with $\vec{e}_j$, the result is the $j^{th}$ component of $\vec{a}$, or $a_j$. Now if we divide through by $|h|$, we get:

$$\left|\frac{f(\vec{p} + h\vec{e}_j) - f(\vec{p})}{h} - a_j\right| < \epsilon$$

This says is that $\left|\frac{f(\vec{p} + h\vec{e}_j) - f(\vec{p})}{h}\right|$ approaches $a_j$ indefinitely, therefore, we can rewrite the relationship as a limit statement:

$$\boxed{a_j = \lim_{h \to 0} \left|\frac{f(\vec{p} + h\vec{e}_j) - f(\vec{p})}{h}\right|}$$

We call this $a_j$ as a \textbf{partial derivative} of $f(\vec{x})$ at $j^{th}$ component, which can be written as $\partial_{x_j} f(\vec{p})$ or $\frac{\partial f}{\partial x_j} (\vec{p})$.

Note that: $$\partial_{x_j}f(\vec{p}) = \frac{d}{dx_j}\left|_{x_j = p_j} f(p_1, p_2, \dots, p_{j-1}, x_j, p_{j+1}, \dots, p_d)$$

In other words, we can hold all other components of $f$ constant and differentiate based on only one component, and plug in the value $p_j$ after the differentiation. Now we know how to compute the gradient of $f$, it is simply the vector of all the partial derivatives:

$$\vec{\nabla}f = (\partial_{x_1} f(\vec{p}), \partial_{x_2} f(\vec{p}), \dots, \partial_{x_d}f(\vec{p}))$$

\subsection{Directional Derivative}

Take $\vec{u} \in \mathbb{R}^d$, $\vec{u} \neq \vec{0}$ and take a point on the function $f$, $\vec{p}$, we define the \underline{directional derivative}
$\partial_{\vec{u}}f(\vec{p})$ as:
\begin{align*}
\partial_{\vec{u}}f(\vec{p}) &= \lim_{h\to0} \frac{f(\vec{p} + h\vec{u}) - f(\vec{p})}{h}\\
&= \frac{d}{dt}\big|_{t=0} f(\vec{p} + t\vec{u})\\
\end{align*}
This quality describes how fast the function $f$ is changing at $\vec{p}$ in the direction of $\vec{u}$.

The latter definition gives the single variable function, such that g(t) = $\vec{p} + t\vec{u}$, where $\vec{u} \neq \vec{0}$, is called the uniform rectilinear motion curve, due to being a line with constant curve speed.

\subsection{Vector Valued Directional Derivative}
Given a vectored valued function $\vec{\gamma}(t)$:

\begin{equation*}
    \vec{\gamma}(t) = \left[
    \begin{array}{c}
        \gamma_1 (t)\\
        \gamma_2 (t)\\
        \vdots\\
        \gamma_d (t)
    \end{array} \right]
\end{equation*}

We define the ``speed'' vector of $\vec{\gamma}$ as its derivative, which is
defined as:

$$\frac{d\vec{\gamma}}{dt}(t) := \lim_{h\to0} \frac{\vec{\gamma}(t + h) - \vec{\gamma}(t)}{h}$$

But because of the componentwise nature of limits, we can distribute the limit
into each component of $\vec{\gamma}$, so we can rewrite the speed vector as:

\begin{equation*}
    \frac{d\vec{\gamma}}{dt}(t) = \left[
    \begin{array}{c}
        \gamma_1' (t)\\
        \gamma_2' (t)\\
        \vdots\\
        \gamma_d' (t)
    \end{array} \right]
\end{equation*}

\subsection{Differentiability Determination}

The mere existance of the set of partial derivatives at $\vec{p}$ with respect to each variable is not enough to guarantee differentiability or continuity at $\vec{p}$, since there must exist some tangential hyperplane, but rather must have a partial derivative in all directions.

\subsubsection{Sufficient Condition}

\underline{Theorem:}
There is a sufficient condition (true implies, but false does not imply the opposite) for differeniability, where if $f: D \subseteq \mathbb{R^d} \to \mathbb{R}$ and $\vec{p} \in D^o$, and if $\vec{\nabla}f(\vec{x})$ exists for all points $\vec{x} \in B_\delta (\vec{p}) (\exists \delta > 0)$ and is continuous at $\vec{p}$, then f is differentiable at $\vec{p}$.

\underline{Proof:}
Let $\vec{h} = (h_1, h_2, \dots, h_d)$ and consider the point $\vec{p} +
\vec{h}$. We will construct a path from $\vec{p}$ to $\vec{p} + \vec{h}$ such
that in each ``step'' we move $h_d$ in the $d^{th}$ axial direction, and we call
each intermediate point $\vec{p}_d$ in the following manner where $\vec{e}_d$ is
the $d^{th}$ standard basis vector.
\begin{align*}
    \vec{p}_0 &= \vec{p}\\
    \vec{p}_1 &= \vec{p}_0 + h_1 \vec{e}_1\\
    \vec{p}_2 &= \vec{p}_1 + h_2 \vec{e}_2\\
    \vdots &= \text{     } \vdots \text{     } \vdots \text{     } \vdots\\
    \vec{p}_d &= \vec{p}_{d-1} + h_d \vec{e}_d
\end{align*}

Now consider the function $\Delta f = f(\vec{p} - \vec{h}) - f(\vec{p})$, we can
write this as a telescoping sum:

$$\Delta f = \sum_{j=1}^d \{f(\vec{p}_j) - f(\vec{p}_{j-1})\}$$

Within each term, we can use the mean value theorem from one dimensional
calculus. We can do so because $\vec{p}_j$ and $\vec{p}_{j - 1}$ only differ in
1 coordinate, like the following:

\begin{align*}
    \vec{p}_{j - 1} &= (p_1 + h_1, \dots p_{j-1} + h_{j-1}, \boxed{p_j}, \dots p_d)\\
    \vec{p}_{j} &= (p_1 + h_1, \dots p_{j-1} + h_{j-1}, \boxed{p_j + h_j}, \dots p_d)
\end{align*}

We also know that the partial derivatives of $f$ exists for all points within
$B_r(\vec{p})$, therefore we can indeed apply the Mean Value Theorem.

Now we apply the MVT:

$$\Delta f = \sum_{j=1}^d \partial_{x_j} f(\vec{q}_j) h_j$$

Where $\vec{q}_j = (p_1 + h_1, \dots, p_{j-1} + h_{j-1}, p_j + \theta h_j,
\dots, p_d)$ where $0 < \theta < 1$. In other words, $\vec{q}_j$ is somewhere
between $\vec{p}_{j-1}$ and $\vec{p}_{j}$ in the $j^{th}$ coordinate.

Now let us consider the definition of differentiability, we need to prove that 


$$|\Delta f - \vec{\nabla}f(\vec{p}) \cdot \vec{h}| = |\sum_{j = 1}^d
\partial_{x_j} f(\vec{q}) h_j - \sum_{j = 1}^d \partial_{x_j} f(\vec{p}) h_j|$$

Now we can apply the triangle inequality on the summations and get:

$$|\Delta f - \vec{\nabla}f(\vec{p}) \cdot \vec{h}| < \sum_{j=1}^d |\partial_{x_j} f(\vec{q}_j) - \partial_{x_j} f(\vec{p})|
\+ |h_j|$$

Now we divide both sides by the length of $\vec{h}$:

\begin{align*}
    \frac{|\Delta f - \gradient f(\vec{p}) \cdot \vec{h}|}{||\vec{h}||} &\leq
    \sum_{j=1}^d |\partial_{x_j} f(\vec{q}_j) - \partial_{x_j} f(\vec{p})| \+
    \frac{|h_j|}{||\vec{h}||}\\
    &\leq \sum_{j=1}^d |\partial_{x_j} (\vec{q}_j) - \partial_{x_j} f(\vec{p})|
\end{align*}

Now let $\vec{h} \to \vec{0}$. Then each $\vec{q}_j$ approaches $\vec{p}$. Then
we know:

$$\sum_{j=1}^d |\partial_{x_j} (\vec{q}_j) - \partial_{x_j} f(\vec{p})| \to 0$$

Since the number of terms in a sum is fixed and each term within the sum is
going to 0, the entire sum is going to 0, which means $||\vec{h}|| <
\delta(\epsilon)$, we can say that the sum is less than $\epsilon$, which
means that

$$\frac{|\Delta f - \gradient f(\vec{p}) \cdot \vec{h}|}{||\vec{h}||} \leq
\sum_{j=1}^d |\partial_{x_j} f(\vec{q}_j) - \partial_{x_j} f(\vec{p})| \+
\frac{|h_j|}{||\vec{h}||} \leq \epsilon$$

Which means that a superlinear decay is possible, which means that $f$ is
differentiable.

\subsubsection{Mean Value Theorem}

If $f : [a, b] \to \mathbb{R}$ is differentiable on $(a, b)$ and continuous
on $[a, b]$, then $\Delta f = f(b) - f(a) = f'(c)(b - a)$ where $c \in (a, b)$.

\subsection{Higher-Order Partial Derivatives}

\subsubsection{Notations}
$f \in C^2$ if $\partial_{x_i}\partial_{x_j} f$ is continuous for any $i, j \in
\{1, 2, \dots, d\}$.

Classically, we also define second order partial derivatives as follows:

\begin{align*}
    \partial_{x_i} \partial_{x_i} &= \partial_{x_i}^2 f = \frac{\partial^2 f}
    {\partial x_i^2}\\
    \partial_{x_i} \partial_{x_j} &= \partial_{x_i} \partial_{x_j} f =
    \frac{\partial^2 f}{\partial x_i \partial x_j}
\end{align*}

\subsubsection{Partial Equality Theorem}
If $f:D \subseteq \mathbb{R^d} \to \mathbb{R}$ is C^2$ (On $B_r(\vec{p}) \partial_{x_j}\partial{x_j}$ is continuous for any $i, j \in {1, 2, 3, ..., d}$)$ near $\vec{p} \in D^o$, then $\partial_{x_j}\partial_{x_i} f(\vec{p}) = \partial_{x_i}\partial_{x_j} f(\vec{p})$.

\underline{Proof:}
WLOG, assume $i \leq j$. Let $x = x_i$ and $y = x_j$.

Define $g(x, y) = f(p_1,
\dots, p_{i-1}, x, p_{i+1}, \dots, p_{j-1}, y, p_{j+1}, \dots, p_d)$.

In terms of $g$, we
can rewrite the partials:

\begin{align*}
    \partial_{x_j}\partial_{x_i} f(p_1, \dots, x, \dots, y, \dots p_d) &= \partial_y
    \partial_x g(x, y)\\
    \partial_{x_i}\partial_{x_j} f(p_1, \dots, x, \dots, y, \dots p_d) &= \partial_x
    \partial_y g(x, y)\\
\end{align*}

We know that both partials exist in some neighborhood of $p_i, p_j$ and both are
continuous at $p_i, p_j$.
Conside a
new function $\Delta(h) = g(x + h, y + h) + g(x, y) - g(x + h, y) - g(x, y +
h)$. Let us rearrange some terms, $\Delta(h) = \{g(x + h,
y + h) - g(x + h, y)\} - \{g(x, y + h) - g(x, y)\}$. If we define $G(x) = g(x,
y+h) - g(x, y)$, we can rewrite $\Delta(h) = G(x+h) - G(x)$. Now apply the MVT, and we get:

$$\Delta(h) = G'(x + \theta_h h) \times h \text{        where } 0 < \theta_h < 1$$

Note that $G'(x) = \partial_x g(x, y+h) - \partial_x g(x, y)$. The derivates
also must exist because $C^2 \implies C^1$. Now if we expand $\Delta(h)$, we
get:

$$\Delta(h) = [\partial_x g(x + \theta_h h, y + h) - \partial_x g(x + \theta_hh,
y)] h$$

Note that here only $y$ is changing, so let us apply the MVT again, and we get:

$$\Delta(h) = \partial_y (\partial_x g) (x + \theta_h h, y + \phi h)h^2 \text{
where } 0 < \phi < 1$$

Now take the limit of $\frac{\Delta(h)}{h^2}$ as $h \to 0$, we get:

$$\lim_{h\to0} \frac{\Delta(h)}{h^2} = \partial_y (\partial_x g)(x, y)$$

But note that we could have first applied MVT on the $y$ and then on the $x$ due
to the symmetry of the function $\Delta(h)$. Therefore, $\partial_y (\partial_x
g)(x, y) = \partial_x (\partial_y g)(x, y)$

\underline{Corollary:}
If f is $C^k$ on $B_r(\vec{p})$ for $k \geq 2 [\partial{x_{i_1}}\partial{x_{i_2}}...\partial{x_{i_k}}$ is continuous on $B_r(\vec{p})$ for all $i_1, i_2, ..., i_k \in {1, 2, ..., d}$, and if $(i_1, i_2, ..., i_k)$ and $(j_1, j_2, ..., j_k)$ are permutations of one another, then $\partial_{x_{i_1}}...\partial{x_{i_k}}f(\vec{p}) = \partial{x_{j_1}}\partial{x_{j_k}}f(\vec{p})$.

This is proven fairly trivially by grouping the initially done partials (since function composition is associative), such that there are only 2 partials either grouped, or not grouped, which can then be switched by the theorem.

\subsubsection{Partial Continuity-Differentiability Theorems}
\underline{Theorem:}
If f is $C^k (k \geq 1)$, then f is also $C^{k-1}$. Thus, a function can be said to be $C^0$ if it is continuous itself on that region. 

\underline{Theorem:}
If f is differentiable at $\vec{p} \in D^o$, then f is continuous at $\vec{p}$.

\underline{Proof:}
$\forall \epsilon > 0, \exists \delta > 0, \forall \vec{h}$ with $||\vec{h}|| < \delta, |f(\vec{p} + \vec{h}) - f(\vec{p}) - \vec{\nabla}f(\vec{p}) \cdot \vec{h}| < \epsilon||\vec{h}||$.

Next, by the triangle and Cauchy-Schwartz inequalities, $|f(\vec{p} + \vec{h}) - f(\vec{p})| = |f(\vec{p} + \vec{h}) - f(\vec{p}) - \vec{a} \cdot \vec{h} + \vec{a} \cdot \vec{h}| \leq |f(\vec{p} + \vec{h}) - f(\vec{p}) - \vec{a} \cdot \vec{h}| + |\vec{a} \cdot \vec{h}| < \epsilon||\vec{h}|| + ||\vec{a}||||\vec{h}|| = (1 + ||\vec{a}||)||\vec{h}|| = M||\vec{h}||$ (where M is some constant), such that it can be made less than any constant, showinng continuity.

This is due to the fact that if the partial derivatives exist and are continuous around the point, then it must be differentiable, and as a result, continuous.

Thus, the base case of $C^1$, then $C^0$ is true, and it can be done by induction, using the induction step, stating $\partial_{x_{i_1}}\partial_{x_{i_2}}...\partial{x_{i_k}}f = \partial{x_{i_1}}g$, which must be continuous for any $i_1 \in {1, ..., d}$, since the original function was continuous. Thus, g is $C^1$, such that g must be continuous.