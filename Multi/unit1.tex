\section{Multivariable Calculus Basics}

Multivariable calculus is the study of the analogs, if any, of the fudemental theorem of calculus to higher dimensions, and any restrictions that may exist on higher dimensions

Let R be a simple closed curve, where simple means that each point is crossed by the curve at most once and closed means that the curve does not have a unqiue starting point and
end point.

When integrating in 2 dimensions, we also need to pick an ``interval.'' In this
case, $R$, the bounded region, would be an interval (analogous to $(a,b)$ in
single variable calculus) and $C$ would be the boundary (analogous to $a, b$).

If we were to integrate a function $f(x,y)$ over $R$, it is denoted by:
$$\iint_R f(x,y) dA_{xy}$$

The $dA_{xy}$ is the area element, an infinitesimally small piece of area,  analogous to $dx$, the length element in single variable calculus.

Note that in single variable calculus, there is an implied orientation, going
left to right is the positive ``direction,'' In multivariable calculus, it is
accepted that the positive direction for the curve to go in is the counterclockwise direction, such that the opposite direction adds a negative.

\subsection{Green's Theorem}

This is one of the FTC's generalization to higher dimensions. The Green's
Theorem works with functions that take in 2 variables.

Suppose there exists $f(x, y)$ and $g(x, y)$, and a region $R$ bounded by a
positively oriented, simple closed curve $C$. Then Green's theorem states that:

$$\iint_R (\frac{\partial g}{\partial x} - \frac{\partial f}{\partial y}) dA_{xy} =
\int_C fdx + gdy = \int_C (fx'(t) + gy'(t))dt$$

\subsection{Generalized FTC}

The goal of this course in multivariable calculus is to reach the following
conclusion:

For some function $\omega$ evaluated over the region $M$, where $\partial M$ is the boundry of $M$:
$$\int_M d\omega = \int_{\partial M} \omega$$

\section{Real Number Set}

\subsection{Definitions of Number Structures}
\subsubsection{$\mathbb{N}$}
We can define the natural number system by sets, like the following:

$$0 = \emptyset$$

\noindent And from there we introduce a succession operation:

$$n + 1 = n \cup \{n\}$$

\noindent So for example, $1 = \{0\} = \{\emptyset\}$, $1 = \{0, 1\} = \{\emptyset,
\{\emptyset\}\}$, etc.

\noindent Set theory is the deepest concept in mathematics, such that it must be assumed as postulates.

\subsubsection{$\mathbb{Z}$}
Positive integers are defined as an ordered pair of natural numbers, for
example, $2 = (2, 0)$, and $-2 = (0, 2)$

\subsubsection{$\mathbb{Q}$}
Rationals are defined as an infinite set of ordered pairs of integers.
A rational number $q = \frac{n}{m}$, then $q = \{(n, m), (2n, 2m), (3n, 3m)
\dots (-2n, -2m), (-3, -3m) \dots\}$

\subsubsection{$\mathbb{R}$}

There are several definitions of the real number system
\begin{itemize}
\item Each real number can be thouhgt of as an infinite sequence in the
following format:

$$(s, N, d_1, d_2, d_3 \dots)$$

Where $s = \pm 1$, $N \in \mathbb{N}$, and $d_n \in {0, 1, 2, 3,
\dots 8, 9}$. It is also not the case that $d_n = d_{n+1} = \dots = 0$.
If there is a terminal decimal, we express it as 9 repeated.
\item Each real number forms a subdivision of $\mathbb{Q}$ into two
disjoint sets that cover the entirety of $\mathbb{Q}$, one of which
lies entirely to the left of the other.
\end{itemize}

\subsection{Basic Structure of $\mathbb{R}^1$}
$\mathbb{R}$ is an instance of many kinds of mathematical structures, such as:
\begin{itemize}
\item Field
\begin{itemize}
\item A set closed under addition and multiplication (results are within set)
\item Obeys all associated laws of addition and multiplication
\end{itemize}
\item Ordered Set
\begin{itemize}
\item The set has an ordering that reflect the operations in the
field (such that x, y > 0, then x + y and  xy > 0)
\item This structure is what allows for comparisons, like the $<$
function
\end{itemize}
\item Metric Space
\begin{itemize}
\item There exists a standard distance operation between numbers
\item In $\mathbb{R}$, $dist(x, y) := |x - y|$ ($:=$ means ``is
defined as'')
\end{itemize}
\item Vector Space
\begin{itemize}
\item Elements can be thought of as vectors from the origin
\end{itemize}
\item Geometric Space
\begin{itemize}
\item This structure means that you can measure angle in a meaningful way
\item In $\mathbb{R}^1$, the angle measure can be either $0$ or $\pi$
\item In higher dimensions there are more angles possible
\end{itemize}
\end{itemize}

\subsection{Properties of $\mathbb{R}^n$}
In higher dimensions, several of the properties of $\mathbb{R}$ no longer hold.
$\mathbb{R}^n, n > 1$ is not a field, and not an ordered set, but it is a vector, metric, and geometric space.

\subsection{Basic Axioms for $\mathbb{R}^1$}
$\mathbb{R}$ is a field under $+$ and $\cdot$ ($x, y \in \mathbb{R}$)
\begin{enumerate}
\item Additive closure: $x + y \in \mathbb{R}$
\item Associative Property of Addition: $x + (y + z) = (x + y) + z$
\item Communicative Property of Addition: $x + y = y + x$
\item 0 is the identity element of addition: $x + 0 = x$
\item Every element has an additive inverse: $x + (-x) = 0$
\item Multiplicative closure: $xy \in \mathbb{R}$
\item Associative Property of Multiplication: $x(yz) = (xy)z$
\item Communicative Property of Multiplication: $xy = yx$
\item 1 is the identity element of multiplication: $x(1) = x$
\item Every element (except 0) has a multiplicative inverse: $x \cdot \frac{1}{x} = 1$
\begin{itemize}
\item Theorem: $x(0) = 0$
\item Proof: $x(0) = x(0 + 0)$, then we apply the distributive law,
and get $x(0) = x(0) + x(0)$. Now we add $-x(0)$ to both side,
and we get: $0 = x(0)$
\end{itemize}
\item Distributive Law: $x(y + z) = xy + xz$
\end{enumerate}
\vspace{11pt}
$\mathbb{R}$ is an ordered field and has a proper subset (aka not the entire
set) $\mathbb{R}^+$ (the \underline{positives}) such that:
\begin{enumerate}
\item $\mathbb{R}^+$ is closed under $+$ and $\cdot$
\item $1 \in \mathbb{R}^+, 0 \notin \mathbb{R}^+$
\item \textbf{Trichotomy Property}: for any $x \in \mathbb{R}$, $x$ is
either $0$, $\in \mathbb{R}^+$ or $\notin \mathbb{R}^+$

\item Definition of $<$ and $>$:
\begin{itemize}
\item $x < y$ means $y - x \in \mathbb{R}^+$
\item $x > y$ means $y < x$
\end{itemize}
\end{enumerate}

\subsection{The Seperation Axiom}

The main difference between the $\mathbb{Q}$ and the $\mathbb{R}$ is the
separation axiom which the rationals do not have. If $\mathcal{A} \subseteq \mathbb{R}$ and $\mathcal{B} \subseteq \mathbb{R}$, and:

\begin{enumerate}
\item $\mathcal{A} \cap \mathcal{B} = \emptyset$
\item $\mathcal{A} \neq \emptyset$, $\mathcal{B} \neq \emptyset$
\item $\mathcal{A} < \mathcal{B}$ ``$\mathcal{A}$ is to the left of
$\mathcal{B}$''
\begin{itemize}
\item $\forall$ (for all) $a \in \mathcal{A}, \forall b \in \mathcal{B}, a < b$
\end{itemize}
\end{enumerate}

\subsubsection{Proof of Irrationals}

$$\mathcal{A} = \mathbb{Q}^- \cup \{0\} \cup \{q \in \mathbb{Q}^+ | q^2 < 2\}$$
$$\mathcal{B} = \{q \in \mathbb{Q}^+ | q^2 \geq 2\}$$

Then, $\exists$ (there exists at least 1) $c \in \mathbb{R}$ such that $\mathcal{A} \leq c \leq \mathcal{B}$

We know that $\mathcal{A} \neq \emptyset$ and $\mathcal{B} \neq \emptyset$
because 0 is in $\mathcal{A}$ and 2 is in $\mathcal{B}$. We also know that
$\mathcal{A} \cup \mathcal{B} = \mathbb{Q}$ and $\mathcal{A} \cap \mathcal{B}
= \emptyset$. As well as the fact that $\mathcal{A} < \mathcal{B}$ (all elements of A are less than all elements of B.

Now, if we want to find the boundary element, $q_0$, which separates
$\mathcal{A}$ and $\mathcal{B}$. We know that $\mathcal{A} \leq q_0 \leq
\mathcal{B}$. So that must mean $q_0 = \sqrt{2}$. However, $\sqrt{2} \notin
\mathbb{Q}$. Therefore, we know that the set of rational numbers do not
follow the separation axiom.

\section{Sequence Theorems}

\subsection{The Least Upper Bound Theorem}

If $\mathcal{A} \subseteq \mathbb{R}$
is non-empty, and is \underline{bounded above} (So $\exists b_1 \in \mathbb{R}$
such that $\mathcal{A} < b_1$), then $\mathcal{A}$ has a
\underline{least upper bound}, i.e. a number $b_0 \in \mathbb{R}$ such that
$\mathcal{A} \leq b_0$ and for any $b$ with $\mathcal{A} \leq b$, $b_0 \leq b$

$$\mathcal{A}\subseteq\mathbb{R}, \mathcal{A} \neq \emptyset, (\exists\text{
} b_1 \in \mathbb{R}: \mathcal{A} \leq b_1)\to[\exists\text{ } b_0 \in
\mathbb{R}: \mathcal{A} < b_0, \forall b \in \mathbb{R} (\mathcal{A} \leq
b \to b_0 \leq b)]$$

$b_0$ is known as the least possible upper bound, or the \textit{supremum} of
$\mathcal{A}$, we write $b_0 = \textrm{sup } \mathcal{A}$.


Similarly, for any non-empty set $\mathcal{A}$ bounded below, it has a
\underline{greatest lower bound}, $\textrm{inf } \mathcal{A}$, called the \textit{infimum}
of $\mathcal{A}$

\subsubsection{Proof}
Define $\mathcal{B}$ to be the set of all upper bounds of $\mathcal{A}$. Let
$\mathcal{C} = \mathbb{R} \setminus \mathcal{B}$. Clearly $\mathcal{B}$ is
nonempty; also $\mathcal{C}$ is non-empty because it contains $x_0 - 1$, where
$x_0 \in \mathcal{A}$. By the way in which we defined $\mathcal{C}$,
$\mathcal{B} \cap \mathcal{C} = \emptyset$. Pick any $c \in \mathcal{C}$ and $b
\in \mathcal{B}$. By the definition of $\mathcal{C}$, $\exists$ $x_1 \in
\mathcal{A}: c < x_1$. But $x_1 \leq b$ by the definition of $\mathcal{B}$.
Therefore $\mathcal{C} < \mathcal{B}$. By
the separation postulate, $\exists$ $b_0 \in \mathbb{R}: \mathcal{C} \leq
b_0 \leq \mathcal{B}$. Note that $\mathcal{A}\setminus\{b_0\} \subseteq
\mathcal{C}$. Thus, $b_0$ is an upper bound for $\mathcal{A}$. Morever, it is
the least upper bound because $b_0 \leq \mathcal{B}$.

\subsection{Bounded Monotone Sequence Theorem}

For any sequence $\{a_n\}$
\begin{enumerate}
\item If $a_n \leq a_{n+1}$ for all $n \geq 1$, and $\exists$ $b \in
\mathbb{R}$ such that $a_n \leq b$ for all $n \geq 1$,  then
$\lim_{n\to\infty} a_n$ exists and is less than or equal to $b$
\item If $a_n \geq a_{n+1}$ for all $n \geq 1$, and $\exists$ $b \in
\mathbb{R}$ such that $a_n \geq b$ for all $n \geq 1$, then
$\lim_{n\to\infty} a_n$ exists and is greater than or equal to $b$
\end{enumerate}

\subsubsection{Proof}

We first convert the sequence $\{a_n\}$, which is bounded by $b$ into the
set $\mathcal{A} = \{a_n | n \geq 1\}$. We know that $\mathcal{A} \neq
\emptyset$ because the sequence has some terms. We also know that $\mathcal{A}$
is bounded above by $b$; $\mathcal{A} < b$
By the Least Upper Bound Theorem, $\exists$ $b_0 = sup \mathcal{A}$. We now show that $b_0 = \lim_{n\to\infty}a_n$. By the definition of limits, to say $b_0 =
\lim_{n\to\infty} a_n$ means to say $\forall \epsilon > 0, \exists$ $N > 0,
\forall n \geq N$, $|a_n - b_0| < \epsilon$

If we look at the number $b_0 - \epsilon$, it is not an upper bound on
$\mathcal{A}$ because $b_0$ is the least upper bound and $\epsilon > 0$.
Therefore, $\exists$ $a_N > b_0 - \epsilon$. Since $\{a_n\}$ is increasing,
$\forall n > N$, $a_n > b_0 - \epsilon$. If we rearrange the terms, we get $b_0
- a_n < \epsilon$. Therefore, $b_0$ (which exists by the least upper bound
theorem) is the limit of $a_n$ as $n \to \infty$.

\subsection{Archimedean Property}

For any positive numbers $x$ and $y$, it is possible to find some $n \in
\mathbb{N}$ such that $nx > y$.
$$\forall x, y > 0, \exists\text{ } n \in \mathbb{N}: nx > y$$

\subsubsection{Proof}
Assume that $\neg \exists\text{ } n:nx > y$, this is logically equivalent to
$\forall n: nx \leq y$. Let $\mathcal{C} = \{nx | n \in \mathbb{N}\}$. Then
$\mathcal{C} \leq y$, let $c = \text{sup } \mathcal{C}$.
We claim that $\exists N: c-\frac{1}{2}x < Nx \leq c$. This is true because if
such $N$ does not exist, then $c - \frac{1}{2}x$ would be an upper bound, but
$c$ is the least upper bound, so such $N$ must exist.
Now we've established the existence of $N$, let us consider $(N + 1)x$. $(N +
1)x = Nx + x > (c - \frac{1}{2}x) + x = c + \frac{1}{2}x > c$. But $(N + 1)x \in
\mathcal{C}$, so it should be $< c$. We have a contradiction. This shows that
the original assumption is false, so $\forall x, y > 0, \exists n \in
\mathbb{N}: nx > y$

\subsubsection{Consequences}
This property can be used to show that $\lim_{n\to\infty} \frac{1}{n} = 0$.
If we consider the definition of limits, the statement is equivalent to saying
that $\forall \epsilon \in \mathbb{R} > 0, \exists N \in \mathbb{N},
\forall n > N, n \in \mathbb{R}, \frac{1}{n} < \epsilon$.
If we rearrange the term, we get that we need to show $1 < \epsilon N$ for any
$\epsilon$. This is true because of the Archimedean Property. $\forall n > N$,
since $\epsilon > 0$, $1 < \epsilon N < \epsilon n$. Therefore we know the limit
is truely 0.

\subsection{Sunrise Lemma}

The Sunrise Lemma states for any sequence $(a_n)^\infty_n=1$ in $\mathbb{R}$, $\exists$ monotone subsequence $(a_{n_k})^\infty_{k=1}$, where $(n_k)^\infty_{k=1}$ is a strictly increasing sequence in $\mathbb{N}$ and $n_k \geq k$ for all $k \in \mathbb{N}$

Vistas are points in a sequence, $a_n$, where  $N \in \mathbb{N}$, such that $a_N > a_n$ for all $n > N$

This means that for any sequence, there exists a subset of points within, such that within that sequence, the sequence is monotone

\subsubsection{Well-Ordering Property}
For any set $A \subseteq \mathbb{N}, A \neq \emptyset, min(A)$ exists

\subsubsection{Proof}
\underline{Case I:} The set V of vistas, is infinite, such that $n_1 = min(v)$ and $n_k = min(V \cap n_{k-1}^\infty$, where $k \geq 2$, then $a_{n_k}$ is strictly decreasing\\
\underline{Case II:}
\[ n_1 =
\begin{cases}
1 & if v = \emptyset \\
1 + max(v) & if v \neq \emptyset
\end{cases} \]
$n_k = choice\{n > n_{k-1} | a_n \geq a_{n_{k-1}}$, thus $n_k \neq \emptyset$ because V is finite, thus $a_{n_k}$ is increasing

\subsection{Bolzano-Weierstrass Theorem}
Every bounded sequence in $\mathbb{R}$ has at least one convergent subsequence

\subsubsection{Proof}
Let $(a_n)^\infty_{n=1}$ be a bounded sequence. For any monotone sequence $(a_{n_k})^\infty_{k=1}$, then $(a_{n_k})^\infty_{k=1}$ is both bounded and monotone, so it converges.

\section{Extreme Value Theorem}
For some function $f:[a, b] \to \mathbb{R}$ (for the codomain, not the range) that is continuous ($f(x_0) = \lim_{x \to x_o}f(x)$ for any x $\in (a, b)$, $f(a) = \lim_{x \to a^+}f(x)$, $f(b) = \lim_{x \to b^-}f(x)$), then $\exists c, d \in [a, b]$ such that $f(c) \leq f(x) \leq f(d)$.

\subsection{Proof}
\subsubsection{$\exists M>0$ such that $f(x) \leq < M \forall x \in [a,b]$ (f(x) is bounded above)}
Lets assume that f is not bounded above, such that for any $n \in \mathbb{N}, \exists x_n \in [a, b]$ such that $f(x_n) > n$.
The sequence $(x_n)^\infty_{n=1}$ is bounded between a and b, such that it has a convergent subsequence $(x_{n_k})^\infty_{k=1}$ converging to some point t, when $ \lim_{k \to \infty}(x_{n_k})$, by the below claim.\par
\underline{Claim:} $t \in [a, b]$. 
Let $t>b$, then $\epsilon > 0$ such that $[a, b] \cap (t-\epsilon, t+\epsilon) = \emptyset$. But $\exists N$ such that $x_N \in (t-\epsilon, t+\epsilon)$ and $x_N \in [a, b]$, which is a contradiction.

By the previous claim, $lim_{k \to \infty}f(x_{n_k}) = f(t)$ by the assumed continuity of f. Thus, $\exists$ K such that for all $k \geq K, f(x_{n_k}) < f(t) + 1 \in \mathbb{R}$. On the other hand, $f(x_{n_k}) > n_k > f(t) + 1$ when k is sufficiently large. Thus, there is a contradiction, and f(x) is bounded from above. This can be reversed to show it is bounded from below, as well.

\subsubsection{The function reaches the supremum and infimum}
We now know $R := f([a, b]) = \{f(x) | x \in [a, b]\}$ is bounded, such that S := sup(R) and I := inf(R). By the definition of infimum and supremum, $\exists (y_n)^\infty_{n = 1}$ such that $y_n \in R  \forall n$, and $\lim_{n \to \infty}y_n = S$. Since $y_n \in R, \exists x_n [a, b]$ such that $f(x_n) = y_n$. Now $(x_n)^\infty_{n=1}$ is bounded between a and b so it has a convergent subsequence, $(x_{n_k})^\infty_{k=1}$, converging to $t \in [a, b]$. Also, by continuity of f, $\lim_{k \to \infty}f(x_{n_k}) = f(t)$. Thus, f(t) = S. This can be reversed to apply to the infimum.
