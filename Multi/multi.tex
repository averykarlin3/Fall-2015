\documentclass[11 pt, twoside]{article}
\usepackage{textcomp}
\usepackage[margin=1in]{geometry}
\usepackage[utf8]{inputenc}
\usepackage{color}
\usepackage{setspace}
\usepackage{tikz}
\usepackage{amsmath}
\usepackage{amsfonts}

\begin{document}

\title{Multivariable Calculus}
\author{Avery Karlin}
\date{Fall 2015}

\maketitle
\newpage
\tableofcontents
\vspace{11pt}
\noindent
\underline{Teacher}: Stern
\newpage

\section{Multivariable Calculus Basics}

Multivariable calculus is the study of the analogs, if any, of the fudemental theorem of calculus to higher dimensions, and any restrictions that may exist on higher dimensions

Let R be a simple closed curve, where simple means that each point is crossed by the curve at most once and closed means that the curve does not have a unqiue starting point and
end point.

When integrating in 2 dimensions, we also need to pick an ``interval.'' In this
case, $R$, the bounded region, would be an interval (analogous to $(a,b)$ in
single variable calculus) and $C$ would be the boundary (analogous to $a, b$).

If we were to integrate a function $f(x,y)$ over $R$, it is denoted by:
$$\iint_R f(x,y) dA_{xy}$$

The $dA_{xy}$ is the area element, an infinitesimally small piece of area,  analogous to $dx$, the length element in single variable calculus.

Note that in single variable calculus, there is an implied orientation, going
left to right is the positive ``direction,'' In multivariable calculus, it is
accepted that the positive direction for the curve to go in is the counterclockwise direction, such that the opposite direction adds a negative.

\subsection{Green's Theorem}

This is one of the FTC's generalization to higher dimensions. The Green's
Theorem works with functions that take in 2 variables.

Suppose there exists $f(x, y)$ and $g(x, y)$, and a region $R$ bounded by a
positively oriented, simple closed curve $C$. Then Green's theorem states that:

$$\iint_R (\frac{\partial g}{\partial x} - \frac{\partial f}{\partial y}) dA_{xy} =
\int_C fdx + gdy = \int_C (fx'(t) + gy'(t))dt$$

\subsection{Generalized FTC}

The goal of this course in multivariable calculus is to reach the following
conclusion:

For some function $\omega$ evaluated over the region $M$, where $\partial M$ is the boundry of $M$:
$$\int_M d\omega = \int_{\partial M} \omega$$

\section{Real Number Set}

\subsection{Definitions of Number Structures}
\subsubsection{$\mathbb{N}$}
We can define the natural number system by sets, like the following:

$$0 = \emptyset$$

\noindent And from there we introduce a succession operation:

$$n + 1 = n \cup \{n\}$$

\noindent So for example, $1 = \{0\} = \{\emptyset\}$, $1 = \{0, 1\} = \{\emptyset,
\{\emptyset\}\}$, etc.

\noindent Set theory is the deepest concept in mathematics, such that it must be assumed as postulates.

\subsubsection{$\mathbb{Z}$}
Positive integers are defined as an ordered pair of natural numbers, for
example, $2 = (2, 0)$, and $-2 = (0, 2)$

\subsubsection{$\mathbb{Q}$}
Rationals are defined as an infinite set of ordered pairs of integers.
A rational number $q = \frac{n}{m}$, then $q = \{(n, m), (2n, 2m), (3n, 3m)
\dots (-2n, -2m), (-3, -3m) \dots\}$

\subsubsection{$\mathbb{R}$}

There are several definitions of the real number system
\begin{itemize}
    \item Each real number can be thouhgt of as an infinite sequence in the
        following format:

        $$(s, N, d_1, d_2, d_3 \dots)$$

        Where $s = \pm 1$, $N \in \mathbb{N}$, and $d_n \in {0, 1, 2, 3,
        \dots 8, 9}$. It is also not the case that $d_n = d_{n+1} = \dots = 0$.
        If there is a terminal decimal, we express it as 9 repeated.
    \item Each real number forms a subdivision of $\mathbb{Q}$ into two
        disjoint sets that cover the entirety of $\mathbb{Q}$, one of which
        lies entirely to the left of the other.
\end{itemize}

\subsection{Basic Structure of $\mathbb{R}^1$}
$\mathbb{R}$ is an instance of many kinds of mathematical structures, such as:
\begin{itemize}
    \item Field
	\begin{itemize}
	\item A set closed under addition and multiplication (results are within set)
	\item Obeys all associated laws of addition and multiplication
	\end{itemize}
    \item Ordered Set
        \begin{itemize}
            \item The set has an ordering that reflect the operations in the
                field (such that x, y > 0, then x + y and  xy > 0)
            \item This structure is what allows for comparisons, like the $<$
                function
        \end{itemize}
    \item Metric Space
        \begin{itemize}
            \item There exists a standard distance operation between numbers
            \item In $\mathbb{R}$, $dist(x, y) := |x - y|$ ($:=$ means ``is
                defined as'')
        \end{itemize}
    \item Vector Space
        \begin{itemize}
            \item Elements can be thought of as vectors from the origin
        \end{itemize}
    \item Geometric Space
        \begin{itemize}
            \item This structure means that you can measure angle in a meaningful way
            \item In $\mathbb{R}^1$, the angle measure can be either $0$ or $\pi$
            \item In higher dimensions there are more angles possible
        \end{itemize}
\end{itemize}

\subsection{Properties of $\mathbb{R}^n$}
In higher dimensions, several of the properties of $\mathbb{R}$ no longer hold.
$\mathbb{R}^n, n > 1$ is not a field, and not an ordered set, but it is a vector, metric, and geometric space.

\subsection{Basic Axioms for $\mathbb{R}^1$}
$\mathbb{R}$ is a field under $+$ and $\cdot$ ($x, y \in \mathbb{R}$)
\begin{enumerate}
    \item Additive closure: $x + y \in \mathbb{R}$
    \item Associative Property of Addition: $x + (y + z) = (x + y) + z$
    \item Communicative Property of Addition: $x + y = y + x$
    \item 0 is the identity element of addition: $x + 0 = x$
    \item Every element has an additive inverse: $x + (-x) = 0$
    \item Multiplicative closure: $xy \in \mathbb{R}$
    \item Associative Property of Multiplication: $x(yz) = (xy)z$
    \item Communicative Property of Multiplication: $xy = yx$
    \item 1 is the identity element of multiplication: $x(1) = x$
    \item Every element (except 0) has a multiplicative inverse: $x \cdot \frac{1}{x} = 1$
        \begin{itemize}
            \item Theorem: $x(0) = 0$
            \item Proof: $x(0) = x(0 + 0)$, then we apply the distributive law,
                and get $x(0) = x(0) + x(0)$. Now we add $-x(0)$ to both side,
                and we get: $0 = x(0)$
        \end{itemize}
    \item Distributive Law: $x(y + z) = xy + xz$
\end{enumerate}
\vspace{11pt}
$\mathbb{R}$ is an ordered field and has a proper subset (aka not the entire
set) $\mathbb{R}^+$ (the \underline{positives}) such that:
\begin{enumerate}
    \item $\mathbb{R}^+$ is closed under $+$ and $\cdot$
    \item $1 \in \mathbb{R}^+, 0 \notin \mathbb{R}^+$
    \item \textbf{Trichotomy Property}: for any $x \in \mathbb{R}$, $x$ is
        either $0$, $\in \mathbb{R}^+$ or $\notin \mathbb{R}^+$

\item Definition of $<$ and $>$:
\begin{itemize}
    \item $x < y$ means $y - x \in \mathbb{R}^+$
    \item $x > y$ means $y < x$
\end{itemize}
\end{enumerate}

\subsection{The Seperation Axiom}

The main difference between the $\mathbb{Q}$ and the $\mathbb{R}$ is the
separation axiom which the rationals do not have. If $\mathcal{A} \subseteq \mathbb{R}$ and $\mathcal{B} \subseteq \mathbb{R}$, and:

\begin{enumerate}
    \item $\mathcal{A} \cap \mathcal{B} = \emptyset$
    \item $\mathcal{A} \neq \emptyset$, $\mathcal{B} \neq \emptyset$
    \item $\mathcal{A} < \mathcal{B}$ ``$\mathcal{A}$ is to the left of
        $\mathcal{B}$''
    \begin{itemize}
        \item $\forall$ (for all) $a \in \mathcal{A}, \forall b \in \mathcal{B}, a < b$
    \end{itemize}
\end{enumerate}

\subsubsection{Proof of Irrationals}

$$\mathcal{A} = \mathbb{Q}^- \cup \{0\} \cup \{q \in \mathbb{Q}^+ | q^2 < 2\}$$
$$\mathcal{B} = \{q \in \mathbb{Q}^+ | q^2 \geq 2\}$$

Then, $\exists$ (there exists at least 1) $c \in \mathbb{R}$ such that $\mathcal{A} \leq c \leq \mathcal{B}$

We know that $\mathcal{A} \neq \emptyset$ and $\mathcal{B} \neq \emptyset$
because 0 is in $\mathcal{A}$ and 2 is in $\mathcal{B}$. We also know that
$\mathcal{A} \cup \mathcal{B} = \mathbb{Q}$ and $\mathcal{A} \cap \mathcal{B}
= \emptyset$. As well as the fact that $\mathcal{A} < \mathcal{B}$ (all elements of A are less than all elements of B.

Now, if we want to find the boundary element, $q_0$, which separates
$\mathcal{A}$ and $\mathcal{B}$. We know that $\mathcal{A} \leq q_0 \leq
\mathcal{B}$. So that must mean $q_0 = \sqrt{2}$. However, $\sqrt{2} \notin
\mathbb{Q}$. Therefore, we know that the set of rational numbers do not
follow the separation axiom.

\section{Sequence Theorems}

\subsection{The Least Upper Bound Theorem}

If $\mathcal{A} \subseteq \mathbb{R}$
is non-empty, and is \underline{bounded above} (So $\exists b_1 \in \mathbb{R}$
such that $\mathcal{A} < b_1$), then $\mathcal{A}$ has a
\underline{least upper bound}, i.e. a number $b_0 \in \mathbb{R}$ such that
$\mathcal{A} \leq b_0$ and for any $b$ with $\mathcal{A} \leq b$, $b_0 \leq b$

$$\mathcal{A}\subseteq\mathbb{R}, \mathcal{A} \neq \emptyset, (\exists\text{
} b_1 \in \mathbb{R}: \mathcal{A} \leq b_1)\to[\exists\text{ } b_0 \in
\mathbb{R}: \mathcal{A} < b_0, \forall b \in \mathbb{R} (\mathcal{A} \leq
b \to b_0 \leq b)]$$

$b_0$ is known as the least possible upper bound, or the \textit{supremum} of
$\mathcal{A}$, we write $b_0 = \textrm{sup } \mathcal{A}$.


Similarly, for any non-empty set $\mathcal{A}$ bounded below, it has a
\underline{greatest lower bound}, $\textrm{inf } \mathcal{A}$, called the \textit{infimum}
of $\mathcal{A}$

\subsubsection{Proof}
Define $\mathcal{B}$ to be the set of all upper bounds of $\mathcal{A}$. Let
$\mathcal{C} = \mathbb{R} \setminus \mathcal{B}$. Clearly $\mathcal{B}$ is
nonempty; also $\mathcal{C}$ is non-empty because it contains $x_0 - 1$, where
$x_0 \in \mathcal{A}$. By the way in which we defined $\mathcal{C}$,
$\mathcal{B} \cap \mathcal{C} = \emptyset$. Pick any $c \in \mathcal{C}$ and $b
\in \mathcal{B}$. By thedefinition of $\mathcal{C}$, $\exists$ $x_1 \in
\mathcal{A}: c < x_1$. But $x_1 \leq b$ by the definition of $\mathcal{B}$.
Therefore $\mathcal{C} < \mathcal{B}$. By
the separation postulate, $\exists$ $b_0 \in \mathbb{R}: \mathcal{C} \leq
b_0 \leq \mathcal{B}$. Note that $\mathcal{A}\setminus\{b_0\} \subseteq
\mathcal{C}$. Thus, $b_0$ is an upper bound for $\mathcal{A}$. Morever, it is
the least upper bound because $b_0 \leq \mathcal{B}$.

\subsection{Bounded Monotone Sequence Theorem}

For any sequence $\{a_n\}$
\begin{enumerate}
    \item If $a_n \leq a_{n+1}$ for all $n \geq 1$, and $\exists$ $b \in
        \mathbb{R}$ such that $a_n \leq b$ for all $n \geq 1$,  then
        $\lim_{n\to\infty} a_n$ exists and is less than or equal to $b$
    \item If $a_n \geq a_{n+1}$ for all $n \geq 1$, and $\exists$ $b \in
        \mathbb{R}$ such that $a_n \geq b$ for all $n \geq 1$, then
        $\lim_{n\to\infty} a_n$ exists and is greater than or equal to $b$
\end{enumerate}

\subsubsection{Proof}

We first convert the sequence $\{a_n\}$, which is bounded by $b$ into the
set $\mathcal{A} = \{a_n | n \geq 1\}$. We know that $\mathcal{A} \neq
\emptyset$ because the sequence has some terms. We also know that $\mathcal{A}$
is bounded above by $b$; $\mathcal{A} < b$

By the Least Upper Bound Theorem, $\exists$ $b_0 = sup \mathcal{A}$. We now show
that $b_0 = \lim_{n\to\infty}a_n$. By the definition of limits, to say $b_0 =
\lim_{n\to\infty} a_n$ means to say $\forall \epsilon > 0, \exists$ $N > 0,
\forall n \geq N$, $|a_n - b_0| < \epsilon$

If we look at the number $b_0 - \epsilon$, it is not an upper bound on
$\mathcal{A}$ because $b_0$ is the least upper bound and $\epsilon > 0$.
Therefore, $\exists$ $a_N > b_0 - \epsilon$. Since $\{a_n\}$ is increasing,
$\forall n > N$, $a_n > b_0 - \epsilon$. If we rearrange the terms, we get $b_0
- a_n < \epsilon$. Therefore, $b_0$ (which exists by the least upper bound
theorem) is the limit of $a_n$ as $n \to \infty$.

\subsection{Archimedean Property}

For any positive numbers $x$ and $y$, it is possible to find some $n \in
\mathbb{N}$ such that $nx > y$.
$$\forall x, y > 0, \exists\text{ } n \in \mathbb{N}: nx > y$$

\subsubsection{Proof}
Assume that $\neg \exists\text{ } n:nx > y$, this is logically equivalent to
$\forall n: nx \leq y$. Let $\mathcal{C} = \{nx | n \in \mathbb{N}\}$. Then
$\mathcal{C} \leq y$, let $c = \text{sup } \mathcal{C}$.
We claim that $\exists N: c-\frac{1}{2}x < Nx \leq c$. This is true because if
such $N$ does not exist, then $c - \frac{1}{2}x$ would be an upper bound, but
$c$ is the least upper bound, so such $N$ must exist.
Now we've established the existence of $N$, let us consider $(N + 1)x$. $(N +
1)x = Nx + x > (c - \frac{1}{2}x) + x = c + \frac{1}{2}x > c$. But $(N + 1)x \in
\mathcal{C}$, so it should be $< c$. We have a contradiction. This shows that
the original assumption is false, so $\forall x, y > 0, \exists n \in
\mathbb{N}: nx > y$

\subsubsection{Consequences}
This property can be used to show that $\lim_{n\to\infty} \frac{1}{n} = 0$.
If we consider the definition of limits, the statement is equivalent to saying
that $\forall \epsilon \in \mathbb{R} > 0, \exists N \in \mathbb{N},
\forall n > N, n \in \mathbb{R}, \frac{1}{n} < \epsilon$.
If we rearrange the term, we get that we need to show $1 < \epsilon N$ for any
$\epsilon$. This is true because of the Archimedean Property. $\forall n > N$,
since $\epsilon > 0$, $1 < \epsilon N < \epsilon n$. Therefore we know the limit
is truely 0.

\subsection{Sunrise Lemma}

The Sunrise Lemma states for any sequence $(a_n)^\infty_n=1$ in $\mathbb{R}$, $\exists$ monotone subsequence $(a_{n_k})^\infty_{k=1}$, where $(n_k)^\infty_{k=1}$ is a strictly increasing sequence in $\mathbb{N}$ and $n_k \geq k$ for all $k \in \mathbb{N}$

Vistas are points in a sequence, $a_n$, where  $N \in \mathbb{N}$, such that $a_N > a_n$ for all $n > N$

This means that for any sequence, there exists a subset of points within, such that within that sequence, the sequence is monotone

\subsubsection{Well-Ordering Property}
For any set $A \subseteq \mathbb{N}, A \neq \emptyset, min(A)$ exists

\subsubsection{Proof}
\underline{Case I:} The set V of vistas, is infinite, such that $n_1 = min(v)$ and $n_k = min(V \cap n_{k-1}^\infty$, where $k \geq 2$, then $a_{n_k}$ is strictly decreasing\\
\underline{Case II:}
\[ n_1 =
\begin{cases}
1 & if v = \emptyset \\
1 + max(v) & if v \neq \emptyset
\end{cases} \]
$n_k = choice\{n > n_{k-1} | a_n \geq a_{n_{k-1}}$, thus $n_k \neq \emptyset$ because V is finite, thus $a_{n_k}$ is increasing

\subsection{Bolzano-Weierstrass Theorem}
Every bounded sequence in $\mathbb{R}$ has at least one convergent subsequence

\subsubsection{Proof}
Let $(a_n)^\infty_{n=1}$ be a bounded sequence. For any monotone sequence $(a_{n_k})^\infty_{k=1}$, then $(a_{n_k})^\infty_{k=1}$ is both bounded and mono0tone, so it converges.

\section{Extreme Value Theorem}
For some function $f:[a, b] \to \mathbb{R}$ (for the codomain, not the range) that is continuous ($f(x_0) = \lim_{x \to x_o}f(x)$ for any x $\in (a, b)$, $f(a) = \lim_{x \to a^+}f(x)$, $f(b) = \lim_{x \to b^-}f(x)$), then $\exists c, d \in [a, b]$ such that $f(c) \leq f(x) \leq f(d)$.

\subsection{Proof}
\subsubsection{$\exists M>0$ such that $f(x) \leq < M \forall x \in [a,b]$ (f(x) is bounded above)}
Lets assume that f is not bounded above, such that for any $n \in mathbb{N}, \exists x_n \in [a, b]$ such that $f(x_n) > n$.
The sequence $(x_n)^\infty_{n=1}$ is bounded between a and b, such that it has a convergent subsequence $(x_{n_k})^\infty_{k=1}$ converging to some point $k \lim_{k \to \infty} x_{n_k}$, by the below claim.
\underline{Claim:} $t \in [a, b]$. 
Let $t>b$, then $\epsilon > 0$ such that $[a, b] \cap (t-\epsilon, t+\epsilon) = \emptyset$. But $\exists N$ such that $x_N \in (t-\epsilon, t+\epsilon)$ and $x_N \in [a, b]$, which is a contradiction.

By the previous claim, $lim_{k \to \infty}f(x_{n_k}) = f(t)$ by the assumed continuity of f. Thus, $\exists$ K such that for all $k \geq K, f(x_{n_k}) < f(t) + 1 \in \mathbb{R}$. On the other hand, $f(x_{n_k}) > n_k > f(t) + 1$ when k is sufficiently large. Thus, there is a contradiction, and f(x) is bounded from above. This can be reversed to show it is bounded from below, as well.

\subsubsection{The function reaches the supremum and infimum}
We now know $R := f([a, b]) = \{f(x) | x \in [a, b]\}$ is bounded, such that S := sup(R) and I := inf(R). By the definition of infimum and supremum, $\exists (y_n)^\infty_{n = 1}$ such that $y_n \in R \forall n$, and $\lim_{n \to \infty}y_n = S$. Since $y_n \in R, \exists x_n [a, b]$ such that $f(x_n) = y_n$. Now $(x_n)^\infty_{n=1}$ is bounded between a and b so it has a convergent subsequence, $(x_{n_k})^\infty_{k=1}$, converging to $t \in [a, b]$. Also, by continuity of f, $\lim_{k \to \infty}f(x_{n_k}) = f(t)$. Thus, f(t) = S. This can be reversed to apply to the infimum. 

\section{Higher Dimensional Mathematics}

\subsection{Higher Dimensions}
\subsubsection{Cartesian Product}
The Cartesian Plane represents the set $\mathbb{R}^2 := \{(x, y)+|+x, y\in \mathbb(R)\}$. This is known as the \textbf{Cartesian Product} of $\mathbb{R}$ with
itself. The Cartesian Product of two sets $\mathcal{S}$ and $\mathcal{T}$,
$\mathcal{S} \times \mathcal{T} :=  \{(s, t)+|+s\in\mathcal{S},
t\in\mathcal{T}\}$.

\subsubsection{Shapes}
\underline{Open-ball}:
$B_r(P) := \{X+|+dist(X, P) < r\}$
\underline{Closed-ball}:
$\bar{B}_r := \{X+|+ dist(X, P) \leq r\}$
\underline{Sphere}:
$S_r := \{X+|+dist(X, P) = r\}$

\subsubsection{Boundary}
Given $D \subseteq \mathbb{R}^2, D \neq \emptyset$. We say $(a, b) \in
\partial D$, i.e. $(a, b)$ is on a \underline{boundry point} of $D$ if $\forall
\epsilon > 0$, there are points $(x, y)\in D$ and $(u, v) \in D^c$ ($D^c :=
\mathbb{R}^2 \setminus D$) such that $dist(x, y+;+ a, b) < \epsilon$ and
$dist(u, v+;+ a, b) < \epsilon$. Since the definition is symmetrical, $\partial D^c = \partial D$.

\subsubsection{Interior}
Let $D \subseteq \mathbb{R}^2$. We say $(a, b)$ is an
\underline{interior point} of $D$ if $\exists+ r > 0:B_r(a, b)\subseteq D$.
The set of all interior points is called the \underline{interior} of $D$ and is
written as $\text{int } D$.

\subsubsection{Exterior}
Let $D \subseteq \mathbb{R}^2$. We say $(a, b)$ is an
\underline{exterior point} for $D$ if it is an interior point of $D^c$.
$\exists+ r > 0: B_r(a, b) \subseteq D^c$.
The set of all exterior points for $D$ is the \underline{exterior} of $D$,
written as $\text{ext } D$.
\textbf{Thm}: For any $D \subseteq \mathbb{R}^2$, $\mathbb{R}^2 =
\text{int } D \cup \partial D \cup \text{ext } D$. And $\text{int } D \cap
\partial D = \emptyset$, $\text{int } D \cap \text{ext } D = \emptyset$,
$\partial D \cap \text{ext } D = \emptyset$.
\textbf{Thm}: $\text{int } D = \text{ext } D^c$ and $\text{ext } D =
\text{int } D^c$
\textbf{Thm}: $\text{ext } D \subseteq D^c$

\subsubsection{Closure}
The \textbf{closure} of $D \subseteq \mathbb{R}^2$:
$$\bar{D} := D \cup \partial D$$
\textbf{Thm}: $\partial \bar{D} = \partial D$, $\text{int } \bar{D} =
\text{int } D$, and $\text{ext } \bar{D} = \text{ext } D$
\textbf{Thm}: $\text{int } D \subseteq D \subseteq \bar{D}$.

\subsubsection{Ordered Pair}
The ordered pair $(a, b)$ can be thought of as a set, but a set is inheritly
unordered. To express the order, we can do the following: $(a, b) = \{\{a\},
\{a, b\}\}$. Now we know that $a$ is the first element because it appears in
both subsets.
We can then expand this into higher dimensions like the following:
$(a, b, c) = ((a, b), c)$. Note that this means that $((a, b), c) \neq (a, (b,
c))$. But this does not matter to us.

\textbf{Fundamental Postulate of Ordered Pairs}:
$(a_1, a_2, a_3, \dots, a_n) = (b_1, b_2, b_3, \dots, b_n)$ if and only if $a_1
= b_1 \wedge a_2 = b_2 \wedge \dots \wedge a_n = b_n$.

\subsubsection{Vector and Points}
Vectors are quantities of directionality and length, its location does not
matter. Points are just positions in space. In higher dimensions with no
ambiance space (flat space surrounding the surface, i.e. the shortest distance
in the ambiant space is the straight line),  we define a vector as all
the lines with the same direction at a certain point.

However, the nice thing about $\mathbb{R}^d$ is that there is always ambiance
space, so we will not make any notational distinction between a point and a
vector.

The length of a vector in $d$ space is defined as:
$$||\vec{a}|| := dist(\vec{0}, \vec{a}) = \sqrt{\sum_{i =
1}^d a_i^2}$$

\subsection{Dot Product}
\subsubsection{Definition}
The dot product arises naturally through the idea of geometric distance, such that if $a \neq \emptyset, b \neq \emptyset$, then $a \perp b iff dist(a; b)^2 = dist(\emptyset, a)^2 + dist(\emptyset, b)^2$, where $a = (a_1, a_2), b = (b_1, b_2)$. Thus, by expanding out, $a \perp b iff a_1b_1 + a_2b_2 = 0$ where $a \neq \emptyset, b \neq \emptyset$. In addition, orthoganal refers to both perpendicular vectors and where $a = \emptyset$ and/or $b = \emptyset$, so that no vector can be perpendicular to itself.

By extension, in $\mathbb{R^d}$, $a \dot b = a_1b_1 + a_2b_2 + a_3b_3 +...+ a_db_d$, such that it forms a scalar, rather than a vector.

\subsubsection{Properties}
The dot product is:
\begin{itemize}
\item Commutative
\item Distributive over Vector Sums
\end{itemize}

\subsubsection{Cauchy-Schwartz Inequality}
In $\mathbb{R^2}$, the p-norm of a vector, $||(a, b)||_p = (|a|^p + |b|^p)^(1\\p)$ for any p > 1, where p $\in \mathbb{Q}$.
The rational power of some number, m,  exists if there is some sequence, qn, where $n \to \infty, qn \to$ the rational number, only true if for any sequence which does this, the limit is equal.
This is proven by for any two sequences, qn and rn, $m^{qn}/m^{rn} = m^{qn-rn}$, such that as $n \to \infty$, it equals 1.

\subsection{Functions in Higher Dimensions}
The domain is a subset of $\mathbb{R^n}$.
Let the function of f(x, y) be an ordered pair within some curve, such that $(x, y) \in D$. Thus, the range of f, G = ${x, y, f(x,y) | (x, y) \in D} \subseteq{R^{n+1}}$.
Functions are defined as one-to-one if for f(x, y), $(x, y), (u, v) \in D, f(x, y) = f(u, v)$ iff (if and only if) x = u $\wedge$ (and) y = v (such that for every z value, there is only one point that will create it.

\subsection{Distance}
\subsection{Distance Functions}
In axiomatic geometry, certain axioms including the definition of euclidean distance are taken as assumed. In actuality, standard distance functions must qualify under several non-geometric requirements, of which only the Euclidean distance qualifies.

Distance functions must be \underline{translation-invariant}, or for any translation of two points, the distance must remain the same, such that $T_{h, k}: (x, y) \mapsto (maps to) (x+h, y+k), then dist(x+h, y+k; x'+h, y'+k) = dist(x, y; x', y')$. 

Thus, $dist(x,y; \tilde(x), \tilde(y)) = f(|x-\tilde(x)|, |y-\tilde(y)|)$, where f the distance function defined on $[0, \infty) x [0, \infty).$ As a result, it must be \underline{symmetrical}, such that $dist(x, y; \tilde(x), \tilde(y)) = dist(\tilde(x), \tilde(y); x, y)$.

In addition, it must have \underline{basic reflection symmetry (isotropy)}, such that $dist(x, y; 0, 0) = dist(y, x; 0, 0)$. Thus, f(u, v) = f(v, u) for any $u \geq 0, v \geq 0$. It must also have the \underline{self-distance of (0, 0)}, such that dist(0,0; 0,0) = 0.

It must \underline{recreate the standard distance function on each axis}, such that $dist(x,0; \tilde(x), 0) = |x - \tilde(x)|, dist(0, y; 0, \tilde(y)) = |y - \tilde(y)|. Therefore, f(u, 0) = u, f(0, v) = v \forall u \geq 0, v \geq 0$.

As a result, it must have \underline{asymptotic flatness}, where if a line is drawn to (x, y), where y is fixed, such that $dist(0,0; 0, y) = v_0, while dist(0, 0; x, 0) = u. Then, \lim_{u \to \infty} f(u, v_0)/u = 1$. This also applies in the opposite direction, where x is fixed.

It must be \underline{continuous} in its variables, such that with a minute movement of a point, the distance changes minutely as well.

\underline{The set of isometries} (distance preserving one-to-one functions) that fix the origin onto itself (f(0) = 0) is an infinite set.

Based on these requirements, an ansatz (educated guess, verified by later results) is made, such that $f(u, v) = F(G(u) + G(v)), where F: [0, \infty) \to \mathbb{R} and G: [0, \infty) \to \mathbb{R}$. The use of G(u) and G(v) is needed to assure symmetry. The use of addition is mandated by symmetry, using addition rather than another symmetrical operation simply due to ease of calculations.

\underline{Theorem:} $\exists only one suitable pair F, G; G(x) = x^2, F(x) = \sqrt{x}, that fits all requirements. If G(x) = x^n, F(x) = \sqrt[n]{x}$, it would have all required properties except infinite set of isometries.

\subsubsection{Euclidean Distance}
The distance function in one space between two points $a$ and $b$ is simply $|a
- b|$. However, we can also write it in the following way: $\sqrt{(a - b)^2}$

In $\mathbb{R}^2$, the distance function is:
$$dist(x, y++;+a,b) := \sqrt{(x - a)^2 + (y - b)^2}$$

And in $\mathbb{R}^3$, the distance function is:
$$dist(x, y, z+;+ a,b,c) := \sqrt{(x - a)^2 + (y - b)^2 + (c - z)^2}$$

The generalized form of Euclidean Distance in $N$ space is:
$$dist(\vec{p}, \vec{q}) = \sqrt{\sum_{j = 1}^N (p_j - q_j)^2}$$

This is known as the \underline{Euclidean Distance}. We use this specific
definition of distance because this is preserved under an infinite set of rigid
or isometric motions, such as rotation, reflection, translation, etc.

\subsubsection{Geometric Distance}
%Add

\textbf{Basic Transformations}:
\begin{itemize}
    \item $T_h : x \mapsto x+h$
    \item $R: x \mapsto -x$
\end{itemize}

\textbf{Properties}:
\begin{enumerate}
    \item $dist(\vec{p}, \vec{q}) = dist(\vec{q}, \vec{p})$
    \item $dist(\vec{p}, \vec{q}) \geq 0$
    \item $dist(\vec{p}, \vec{q}) = 0 \leftrightarrow \vec{p} = \vec{q}$
\end{enumerate}

\subsubsection{Basic Distance Bounds Lemma}
 $\forall \vec{p}, \vec{q} \in
\mathbb{R}^d$, and $\forall j \in \{1,2,3,\dots,d\}$:
$$|p_j - q_j| \leq dist(\vec{p}, \vec{q}) \leq \sqrt{d} \max_{1 \leq k \leq
d} |p_k - q_k|$$

\textbf{Proof}
Note that $(p_j - q_j)^2 \leq \sum_{k = 1}^d (p_k - q_k)^2$ is trivial, because
you can only add positive number when you add squares. Now let's take the square
root, and we get
$$\sqrt{(p_j - q_j)^2} = |p_j - q_j| \leq \sqrt{\sum_{k = 1}^d (p_k - q_k)^2} =
dist(\vec{p}, \vec{q})$$

To prove the other inequality, it is trivial as well. We can just factor out the
length of the vector $d$ and multiply that with the maximum value of the
distance vector. Then we get:

$$dist(\vec{p}, \vec{q}) = \sqrt{\sum_{k = 1}^d (p_k - q_k)^2} \leq \sqrt{d \max_{1 \leq k \leq d} (p_k -
q_k)^2} = \sqrt{d} \max_{1 \leq k \leq
d} |p_k - q_k|$$

\textbf{Cor}: Componentwise Nature of Convergence

Let $(\vec{p}_n)_{n = 1}^\infty$ be a sequence in $\mathbb{R}^d$, and let
$\vec{p} \in \mathbb{R}^d$. Then $\vec{p}_n \to \vec{p}$ if and only if
$p_{n|j} \to p_j$ ($\vec{p} = (p_1, p_2, p_3, \dots, p_d)$ and $\vec{p}_n =
(p_{n|1}, p_{n|2}, \dots, p_{n|d})$). Otherwise known as convergence of points
can be reduced to conversion of dimensions.

This follows directly from the inequality, because if the total distance goes to
0, then $|p_j - q_j|$ goes to 0. Therefore if the points converge, the
corresponding coordinates must converge.

To prove the converse, we prove using the other side of the distance bounds. If
all $d$ coordinates are going to 0, then if we take the maximum, that would be
going to 0. (the maximum of a sequence is less than the sum of the sequence, but
if every term of the sum is going to 0, then the sum is going to 0, then the
maximum is going to 0). Therefore the distance must also be going to 0. Thus the two points converges.

\subsubsection{Triangle Inequality}
$$dist(\vec{p}, \vec{q}) + dist(\vec{q}, \vec{r}) \geq dist(\vec{p}, \vec{r})$$

This can be generalized by mathematical induction to $dist(\vec{p_0}, \vec{q_n})
\leq \sum_{j = 1}^n dist(\vec{p}_{j - 1}, \vec{p}_{j})$ (Otherwise known that the
shortest distance between two points is the straight line, or the \textbf{Generalized
Triangle Inequality} or the ``Broken Line Inequality'')

\subsubsection{Other Distance}
Of course, there are other distance formulas, like the \underline{Minkowski Distance}
$$((x - a)^p + (y - b)^p)^{\frac{1}{p}} ++++ (p > 1)$$
This is another distance formula, but under this, only reflection preserves
distance.

\subsection{Metric Topology in $\mathbb{R^n}$}
\subsubsection{Continuity}
Let $f: D\to\mathbb{R}$, $D\subseteq\mathbb{R}^2$, $D\neq\emptyset$. Let $(a, b)\in D$. We say that $f$ is \underline{continuous} at $(a, b)$ if:
$$f(a, b) = \lim_{\substack{(x, y)\to(a,b)\\(x, y)\in D}} f(x, y)$$
Or in other terms:
$$\forall \epsilon > 0, \exists+ \delta>0, \forall (x, y)\in D: dist(x,
y+;+a,b) < \delta \to |f(x,y) - f(a,b)| < \epsilon$$

$f:D \to \mathbb{R}, D \subseteq \mathbb{R^2}, D \neq \emptyset is continuous if f is continuous at (a, b) for all (a, b) \in D.$

\subsubsection{Directional Limits}
Let $D \subseteq \mathbb{R} and a \in D \cup \partial D$ (the boundry, both already included in D, and not), $\lim_{x \to a^+} f(x) = L$ means $\forall \epsilon > 0, \exists \delta(\epsilon) > 0: \forall x \in D \cap (a, \infty), |x-a| < \delta(\epsilon) \Rightarrow |f(x) - L| < \epsilon$. The limit only exists if both directions equal the same value.

\subsubsection{Limits in Higher Dimensions}
The same theory can be applied to higher dimensions, such that if two arbitrary approaches are not the same, it doesn't exist, but if several approaches yield the same result, the definition of a limit is used. Polar coordinate substitutions can be used to give format to directions of approach.

Due to difficulty defining approaching through lines, it is said that $(x, y) \to (a, b) iff dist(x, y: a, b) \to 0$.

Let $f: D \to \mathbb{R}, D \subseteq \mathbb{R^2}, D \neq \emptyset, and (a, b) \in D \cup \delta D.$ Then, $L = \lim_{(x, y) \in D \to (a,b)} f(x) iff \forall \epsilon > 0, \exists \delta > 0, \forall (x, y) \in D: dist(x, y: a, b) < \delta \to |f(x, y) - L| < \epsilon$. As a corollary, when (a, b) is on the boundry, the approach can only be from the domain.

\subsection{Properties of Domain}
For the extreme value theorem to apply to a domain, the set must be compact, such that it must be bounded and closed over limits. On the $\mathbb{R}$ dimension, this applies to all closed intervals, as well as the empty set, though functions except the empty set cannot accept it as a domain.

\subsubsection{Bounded}
$If D \subseteq \mathbb(R^2)$ is bounded if $\exists M > 0: D \subseteq [-M, M] x [-M, M]$.

\subsubsection{Closed}
The term closed is used to apply to sets which are closed under limits. On $mathbb{R}$, if $x_n \in [a, b] for \forall n \in mathbb{N}, and x_n \to x \in \mathbb{R}, then x \in [a, b]$.

$D \subseteq \mathbb{R^2} is closed if for any points (x_n, y_n) \in D (for all n \in \mathbb{N}, if (x_n, y_n) \to (x, y) \in \mathbb{R^2}, then (x, y) \in D. (x_n, y_n) \to (a, b) as n \to \infty$ means $d_n = \sqrt[2]{(x_n - a)^2 + (y_n - b)^2} \to 0 as n \to \infty.$

This applies the definition of limits to sequences, such that $(x_n, y_n) \to (a, b) if dist(x_n, y_n: a, b) \to 0 as n \to \infty$.

Thus, $D \subseteq \mathbb{R^2}$ is closed if for any sequence $((x_n, y_n))^\infty_{n=1}$ in D that converges, the limit poiint (a, b) of the sequence also lies in D.

\subsubsection{Open Set Theorem}
$D \subseteq \mathbb{R^2}$ is open if $\forall (a, b) \in D, \exists r > 0:$ the disk of radius r, $B_r(a, b) \subseteq D$.

It follows that for any open set, the complement set within the space is a closed set.

\underline{Proof}:
Assume $D$ is open, we prove $D^c$ is closed. Choose any convergent sequence
$((x_n, y_n))_{n = 1}^\infty$, converging to $(a, b)$, where $(x_n, y_n) \in
D^c$ for all $n \geq 1$.

We prove this by contradiction. Assume that $(a, b) \in D$. Since $D$ is ipen,
$\exists+ r > 0:B_r(a, b)\subseteq D$. $\exists+ N, \forall n \geq
\mathbb{N}, (x_n, y_n) \in B_r(a, b)$ since $(x_n, y_n)\to(a,b)$. But we
assumed that $(x_n, y_n) \in D^c$, and $(x_n, y_n) \in D$. But $D \cap D^c =
\emptyset$. Therefore $D^c$ is closed under taking limits.

\end{document}
