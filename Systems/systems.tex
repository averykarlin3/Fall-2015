\documentclass[11 pt, twoside]{article}
\usepackage{textcomp}
\usepackage[margin=1in]{geometry}
\usepackage[utf8]{inputenc}
\usepackage{color}
\usepackage{setspace}
\usepackage{tikz}

\begin{document}

\title{System-level Programming}
\author{Avery Karlin}
\date{Fall 2015}

\maketitle
\newpage
\tableofcontents
\vspace{11pt}
\noindent
\underline{Teacher}: Dyrland-Weaver
\newpage

\section{Learning C}
\subsection{C Primitive Variable Types}
\begin{enumerate}
\item All C primitives are numeric, divided purely based on variable size, and integer or floating point
\begin{enumerate}
\item C variables have sizes based on the platform they were compiled by and for, such that sizeof(\textit{type}) can be used to determine the size in bytes
\item On a standard computer, int = 4 $(-2^31, 2^31 - 1)$, short = 2, long = 8, float = 4, double = 8, and char = 1 bytes (8 bits to a byte)
\item Types can also be specified as unsigned, such that it is not able to be given a negative value
\item Types can be placed within types of larger size and the same format without any form of conversion, but not of a different format
\item sizeof(\textit{type}) returns the size in bytes of the type
\end{enumerate}
\item Boolean values are numbers, such that 0 is false, and all nonzero numbers are considered true
\item Character literals can be represented inside single quotes rather than use a number, and Strings, though not an object, can use a double quotes literal
\begin{enumerate}
\item Strings are created by character arrays, using a null character (value 0), to show the end of the array, allowing it to be modified easier
\end{enumerate}
\item Variables are able to be initialized within a for loop, but are not able to be declared, such that it must be before the loop
\end{enumerate}

\subsection{C Programming}
\begin{enumerate}
\item All C programs are made up of a series of functions, run within the main function, which returns an integer (typically 0, or other values for errors)
\begin{enumerate}
\item They are compiled through ``gcc \textit{file.c} -o \textit{program\_name}'', then run through ``./\textit{program\_name}''
\end{enumerate}
\item Libraries are added, either .h files from the current directory through \#include ``\textit{file}.h'' or through premade libraries by \#include $<$file.h$>$
\begin{enumerate}
\item All files typically start with calling the C library with \#include$<$stdio.c$>$ (standard io) and $<$stdlib.h$>$ (standard library)
\end{enumerate}
\item The man pages, called by ``man \textit{command}'' or ``man \textit{section command}'', give information on both bash and C commands
\begin{enumerate}
\item (1) is user commands, (2) is system calls, (3) is library functions, such as the C libraries, (4) is devices, (5) is file formats, (6) is games and amusements, (7) is conventions and miscellany, and (8) is system admin and priveledged commands
\item (L) is used for local commands, installed by certain programs
\end{enumerate}
\item C functions are pass by value, such that they put the value into a new variable created by the function, though if pointers are passed, it is equivelent to pass by reference, due to being a ppointer to the same location
\begin{enumerate}
\item C functions are written similar to java, with the exception of the lack of the protection
\item Due to C being functional, the functions are created in the order written, such that it should already have created all functions and commands used within the function being compiled
\item Failure to declare first leads to an implicit declaration warning that it has not been formally declared yet, though it will still work if it is declared later
\item Headers can also be placed at the top of the function in addition to where they are defined, to avoid implicit declaration, or in a seperate header file
\end{enumerate}
\end{enumerate}

\subsection{C Structures}
\begin{enumerate}
\item ``printf(\textit{text, var1, var2})'' is used to print a String in terminal, where the text is a formatted string, with placeholders for variables following
\begin{enumerate}
\item \%f is a placeholder for a float, \%d for double, \%c for char, \%s for string, \%f for pointer, \%lf for double, \%ld for long, and \%d for int
\item println can be used instead of printf for non-formatted strings (without variables)
\item Print functions do not automatically add ``\\n'' at the end of a line, and must be written in the string
\end{enumerate}
\item Arrays in C are non-dynamic, such that they must have a fixed size, with no length function, and there are no errors for going outside boundries, rather going to a different point in memory
\begin{enumerate}
\item Arrays are declared by ``\textit{type}[\textit{size}];'' and must be initialized each part at a time
\end{enumerate}
\item String functions are held within the string.h library, always assuming the strings are null-terminated
\end{enumerate}

\section{Memory Management}
\subsection{Memory Allocation}
\begin{enumerate}
\item Memory allocation is either during compile time (static stack memory), or during runtime (dynamic heap memory)
\item Compiler allocated memory is packaged within the binary, unable to overwrite other programs memory due to protected memory, without a default value, where variables and arrays are allocated
\begin{enumerate}
\item Memory addresses of variables are fixed once they are placed, such that the data can be changed, but the location cannot be
\end{enumerate}
\item Runtime memory is temporary, used for values of variables
\item Systems have a bit limit which they can read at once, such that 32 bit systems are limited to 32 bit unsigned values, such that $[0, 2^32-1]$ is possible, or ~4 GB 
\item Pointers are variables designed to store memory addresses
\begin{enumerate}
\item \%\textit{variable} is used to get the address of a variable, such that the number returned can be the value of a pointer 
\item When a pointer is incremented, the location moves the number of bytes of the variable type which the pointer applies to
\item * is used before a variable name to declare a pointer, and is also used when calling a variable to get the value of the item at that location, preceeding before numeric operators except ++ and --
\item Thus, for some array a, with *a as the pointer, a[i] = *(a + i)
\end{enumerate}
\end{enumerate}

\subsection{Strings and Arrays}
\begin{enumerate}
\item Strings can be declared by several methods, "char \textit{str[byte\_num]}" to do basic allocation, or it can be set on the same line, with a null put in the byte after the last letter
\begin{enumerate}
\item It can also be declared with an empty byte number, but set such that it will be given the exact amount of space needed
\item It can also be declared as a pointer to the array by "char *\textit{str = data}", created the array the exact correct size, and a pointer to the array under that variable name
\item After declaration, each character must be set individually, instead of using the equal sign
\item On the other hand, if a pointer is used, the pointer can be changed to apply to a seperate array, using an equal sign, even after declaration
\end{enumerate}
\item The null character at the end is needed for string functions in string.h to work correctly, but is not a requirement
\item String/array variables are functionally immutable pointers to the first item in an array (such that the location cannot be changed)
\item Pointer-defined strings are literals, such that they are made in protected memory, where the pointer location can be redefined, but the string cannot be
\begin{enumerate}
\item Literals of the same string will point at the same location as previously made literals
\end{enumerate}
\end{enumerate}

\section{Structural Functions}
\subsection{String Functions}
\begin{enumerate}
\item String functions are found within <string.h>, assuming a null character at the end
\item int strlen(char *s) returns the length of s, ignoring the null character
\item int strcmp(char *s1, char *s2) returns 0 if equal, >0 if s1 > s2, and <0 otherwise
\item char* strcpy (char *destination, char *source) copies the string to destination, assuming the allocated destination space is the same size or larger
\item char* strcat (char *destination, char *source) adds source to the end of destination
\item strncat and strncpy has an integer as a final parameter, using only the first n characters of the source string, such that if it is longer than the string, it uses up to the null character 
\end{enumerate}

\end{document}
