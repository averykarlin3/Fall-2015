\documentclass[11 pt, twoside]{article}
\usepackage{textcomp}
\usepackage[margin=1in]{geometry}
\usepackage[utf8]{inputenc}
\usepackage{color}
\usepackage{setspace}
\usepackage{tikz}

\begin{document}

\title{System-level Programming}
\author{Avery Karlin}
\date{Fall 2015}

\maketitle
\newpage
\tableofcontents
\newpage

\section{Learning C}
\subsection{C Primitive Variable Types}
\begin{enumerate}
\item All C primitives are numeric, divided purely based on variable size, and integer or floating point
\begin{enumerate}
\item C variables have sizes based on the platform they were compiled by and for, such that sizeof(type) can be used to determine the size in bytes
\item On a standard computer, int = 4, short = 2, long = 8, float = 4, double = 8, and char = 1 bytes (8 bits to a byte)
\item Types can also be specified as unsigned, such that it is not able to be given a negative value
\end{enumerate}
\item Boolean values are numbers, such that 0 is false, and all nonzero numbers are considered true
\item Character literals can be represented inside single quotes rather than use a number, and Strings, though not an object, can use a double quotes literal
\begin{enumerate}
\item Strings are created by character arrays, using a null character (value 0), to show the end of the array, allowing it to be modified easier
\end{enumerate}
\item Variables are able to be initialized within a for loop, but are not able to be declared, such that it must be before the loop
\end{enumerate}

\subsection{C Programming}
\begin{enumerate}
\item All C programs are made up of a series of functions, run within the main function, which returns an integer (typically 0, or other values for errors)
\begin{enumerate}
\item They are compiled through ``gcc file.c -o program\_name'', then run through ``./program\_name''
\item All files typically start with calling the C library with \#include$<$stdio.c$>$
\end{enumerate}
\end{enumerate}
\end{document}
