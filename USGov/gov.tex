\documentclass[11 pt, twoside]{article}
\usepackage{textcomp}
\usepackage[margin=1in]{geometry}
\usepackage[utf8]{inputenc}
\usepackage{color}
\usepackage{indentfirst} %Comment out for no first paragraph indent
\usepackage[parfill]{parskip}
\usepackage{setspace}
\usepackage{tikz}
\usepackage{amsmath}
\usepackage{amsfonts}
\usepackage{amssymb}
\usepackage{enumitem}
\usepackage{outlines}

\usepackage{fancyhdr}
\pagestyle{fancy}
\cfoot{\hyperlink{content}{\thepage}}
\lhead{}
\chead{}
\rfoot{}
\lfoot{}
\rhead{}
\renewcommand{\headrulewidth}{0pt}
\renewcommand{\footrulewidth}{0pt}


\usepackage{hyperref}
\hypersetup {
	colorlinks,
	citecolor=black,
	filecolor=black,
	linkcolor=black,
	urlcolor=black
}

\newcommand{\sepitem}{0pt} %Added room between items on the list, not including a list and its sublist
\newcommand{\seppar}{1pt} %Between items and lists overall

\setenumerate[1]{itemsep=\sepitem, parsep=\seppar}
\setenumerate[2]{itemsep=\sepitem, parsep=\seppar}
\setenumerate[3]{itemsep=\sepitem, parsep=\seppar}
\setenumerate[4]{itemsep=\sepitem, parsep=\seppar}

\newenvironment{outline*}
{
	\begin{outline}[enumerate]
	}
	{\end{outline}
}

\newcommand{\foot}[1]{\hyperlink{#1}{$_#1$}}

\begin{document}

\title{US Government}
\author{Avery Karlin}
\date{Fall 2015}
\newcommand{\textbook}{Democracy in Action by Richard Remy}
\newcommand{\teacher}{Trainor}

\maketitle
\newpage
\hypertarget{content}{\tableofcontents}
\vspace{11pt}
\noindent
\underline{Primary Textbook}: \textbook\\
\underline{Teacher}: \teacher
\newpage

\section{Chapter 1 - People and Government}
\subsection{Principles of Government}
\begin{outline*}
\1 The state is a political community within a definite territory, with an organized government to make and enforce laws, typically without a higher authority
\2 The state in the US was used originally in this way, but was used as the political divisions in the US after the federal government formed as well
\2 Nations are a sizable group united by a common bond of race, custom, tradition, or religion, such that a nation-state is a state and nation sharing a boundary, even if not all citizens are within the nation
\2 Nations may include several states or vice versa, but are used the same
\1 States must have a population with a shared political and social consensus on basic beliefs to be stable, who are often mobile, causing political power shifting
\2 They must also have a specific territorial boundary from other states, government, which enforces decisions on the population, and sovereignty, or complete, legitimately accepted authority within
\2 Sovereignty theoretically makes all states equal, though economic and military power renders them not fully sovereign
\1 Evolutionary theory states that government began as an attempt to organize extended families as they grew larger and larger
\2 Force theory states that government formed when people were put under one authority by force, while divine right is based on the belief that rulers were chosen or descended from gods
\2 Diamond believed in evolution, saying that overtime, to sustain the need for more resources, force theory interjected, while religion gave unity
\2 Social contract theory was developed by Hobbes in England, stating that people gave power to the state to maintain order, in exchange for protection from natural state of violence
\2 Locke said that people had natural rights to life, liberty, and property, and could break the contract if the government broke them
\1 Governments maintain social order, such as taxes or drafts, and prevent conflict, because people don’t know how to do it by themselves, according to Locke
\2 Governments must provide services for general welfare, public service projects that people wouldn’t do alone, and create health and safety laws
\2 Federal government protects citizens from terrorism or wars, and handles foreign and trade relations, and can change states' foreign agreements
\2 Government can also make economic decisions to increase growth, reduce inequality and prevent rebellions and issues, in themselves and other nations, and can incentivize particular economic actions
\end{outline*}
\subsection{Formation of Governments}
\begin{outline*}
\1 Unitary systems of government are those where a single central government holds absolute power, while a federal system divides it between central and local
\2 Originally, the US was a confederacy, or loose union of independent states, but eventually became federal, when found ineffective
\1 Constitutions are government rules, creating ideals for the state (preamble), structure and powers of government, and supreme law
\2 It may be written, or unwritten (based on laws and decisions, such as in the UK), and is found in all governments
\2 Constitutional governments are those where the constitution is able to limit the powers of the government
\2 Constitutions are incomplete, and cannot include everything for the nation, and may not actually reflect the government, such as in China
\1 Politics are the effort to influence and control government policies, influencing services and benefits provided to their interests and values, allowing a peaceful clash of opinions in society
\2 Some, such as Madison in the Federalist, feared politics would allow special interest groups to do actions against the general welfare
\1 Most nations are unequal, some industrialized with large industry and advanced tech, while some are developing, and economic and political interdependence due to industry and tech growth takes away total sovereignty
\2 Many developing countries have also began to depend on foreign aid to cope with environmental or political issues
\2 Political motivated nonstate organizations, such as terrorists or national liberation organizations, also play a role in world politics, as well as multinational corporations and international organizations
\end{outline*}
\subsection{Types of Governments}
\begin{outline*}
\1 Autocracies are governments where power is in a single person, including totalitarian dictatorships, glorifying the ideas of a leader, and controlling all social and economic life, where the leader is not limited
\2 Monarchies are an inherited autocracy, where absolute monarchs have unlimited power over the people, and constitutional monarchies are limited by some alternate, elected source of power
\1 Oligarchies are systems where a group holds power due to economic, social, or military power, often attempting to stimulate a democratic system
\1 Democracies are those ruled by the people, either a direct democracy of public meetings to vote on issues, or a representative democracy, voting to give people power to vote for laws
\2 Republics are where the voters are fully the source of the government’s authority, with representatives held accountable
\2 Democracies promote freedom and equal opportunity to the greatest degree possible, and allow majority rule but give rights to minorities
\2 Korematsu v. US allowed the government to take liberties from Japanese Americans, but Endo v. US stated that as a natural born, it was racism
\1 Democracies rely on free and open elections, each vote with the same amount of power, minimum requirements to vote, allowed to voice opinions and get help or support from citizens, and have a secret ballot to prevent risk
\2 Political parties are groups with common interests, by simplifying choices, and serve as loyal opposition to critique policies of those in power
\1 Democracy requires active citizen participation in government, an educated public, an economy with a large, stable middle class and without extremes and control over their economic decisions, also called free enterprise
\2 It also requires a social consensus, or acceptance of democratic values and agreement about the goals and limits of government
\2 It requires a civil society, or groups independent of government, to make views known to the rest of society, and allow people to participate easier
\end{outline*}
\section{Chapter 2 - Origins of American Government}
\subsection{The Colonial Period}
\begin{outline*}
\1 Most of the original American colonists were from England, using English governmental ideas, also found in many Native cultures
\2 Magna Carta gave the idea of limited government, protecting against loss of natural rights, unjust punishment, and popular consent to some taxes
\2 It was made in 1215, starting on nobles, but eventually forming the basis of constitutional government
\2 The 1628 Petition of Right limited kings power from collecting taxes without Parliament, imprisoning without cause, house troops in homes without permission, and martial law without war
\1 The English Bill of Rights in 1688 by William and Mary gave limited power to monarchs, and required Parliament to stop laws, give taxes, and have an army, as well as prevented interference in Parliament elections and debates
\2 It also gave right to petition, trial by jury, and prevented cruel and unusual punishments
\1 Representative government was found in the Parliament, with the House of Lords and Commoners (merchants and property owners, mainly)
\1 Locke (in his Two Treatises of Government), Voltaire, and Rousseau believed government must protect natural rights by contract with the people
\1 Each colonial government had a governor, legislature, and court system, but had allegiance to the monarch, property qualifications for white males to vote
\2 Nine of the original colonies also persecuted religious dissenters
\2 They had a written constitution for limitations and liberties, elected legislature representatives, and Montesquieu’s separation of powers
\2 Written constitutions started with the 1620 Mayflower Compact for self-government, followed by the Puritan Great Fundamental in MA and Fundamental Orders in 1639 CT, giving representative and legal plans
\2 Fundamental Orders gave an judicial and executive of members from the legislative house, which can tax, spend, and was able to set their own calendar, all with term limits, the executive having a single 2 year term
\2 The VA House of Burgesses in 1619, elected church officials and government in Puritan MA, eventually changing in 1636 to two representatives to the MA General Court legislature
\2 Legislatures also had to adapt to unexpected civil, public works situations such as roads and schools as the colonies grew
\end{outline*}
\subsection{Unification for Independence}
\begin{outline*}
\1 The colonies were a raw materials source and market for British goods to aid England, but due to long distance, they self-governed
\2 During the French and Indian War from 1754 to 1763, the British took more control in exchange for protection from the French in Canada, fighting for land in Western PA and OH, won by the British
\2 The war took away the need for protection, but added war debts, putting the direct tax, Stamp Act of 1765, for all printed documents and dice
\2 They put additional taxes on tea, sugar, glass, paper, and other products, and began to control trade to benefit England further
\1 England’s revenue increased, but protests and boycotts led to the removal of the Stamp Act, replaced by other taxes, leading to the Tea Party in 1773
\2 Intolerable Acts then put MA under British rule and closed Boston Harbor
\1 In 1754, Franklin proposed the Albany Plan of Union to fight the French, but the colonies were too separate to give up power to a federal government
\2 The taxes led to a community, leading to the 1765 Stamp Act Congress to discuss protest, sending a petition that only colonies could do direct taxes
\2 The 1773 dozens of Committees of Correspondence communicated between colonies to organize protests
\1 After the Intolerable Acts, VA and MA called the First Continental Congress of the colonies except GA in September 1774 in Philadelphia, deciding on a trade embargo with England, and a meeting in 1775 if the King didn’t back down
\2 In April 19, 1775, the British attacked the minutemen at Lexington and Concord, and in May, they met for a 2nd Congress, organizing a government with Hancock as president, and a military under Washington
\1 Thomas Paine wrote Common Sense, saying King George was against liberty and corrupt, and Sam Adams called for independence, which Henry Lee proposed in June 1776, passed July 2nd, edited and approved July 4th
\2 The declaration justified the resolution and gave founding principals of the country, based on liberty, natural rights, and consent of governed, written by Jefferson, Adams, Franklin, Sherman, and Livingston, with 1/4 cut
\2 After, it lists specific complaints against King George, then finally states that attempts to resolve differences peacefully have failed
\1 States began to draft state constitutions as independent states, typically with a bill of rights and the idea of consent of the governed emphasized
\end{outline*}
\subsection{The Articles of Confederation}
\begin{outline*}
\1 In Lee’s resolution, he proposed the loose Articles of Confederation, to continue the government of the 2nd Congress, ratified in March 1781 by all states
\2 It was a unicameral legislature, with no federal court, a Committee of States of one delegate from each when Congress was not in session
\2 Executive positions were chosen from the Congress, which was only able to handle foreign affairs, maintain a military, fix weights and measures, Indian affairs, post offices, and decide state disputes
\1 The Articles were weak, due to not allowing taxes, only requesting from state taxes or borrowing money, could not regulate trade internally or externally, and needed 9 states approval for laws, when 3 or 4 were often absent from Congress
\2 Congress could not enforce laws, amendments needed to be unanimous, and without an executive, there were uncoordinated committees for management, and had no federal courts for interstate disputes
\2 The framers felt a Republic could only work in small, homogeneous communities, based on Montesquieu's teachings that the rich and small interests would control the government
\1 The Confederation did make the states cede claims to land West of the Appalachians, passing two land ordinances, 1785 for survey and division, and the Northwest of 1787 deciding they would be made into equal states
\2 It also signed the 1783 peace treaty with the UK, and gave all Eastern UK land, as well as setting up the executive Departments and secretaries (splitting military into War and Marine)
\2 It also had the full faith and credit clause, preventing discrimination against citizens of others, and respect legal decisions of another state
\1 States quickly fought over boundaries, tariffs, and trade with foreign nations, and the government owed \$40M to soldiers, and could not afford the military either
\2 Shay’s Rebellion due to a depression hurting farmers, led to an armed march on the MA Supreme Court, and then on Springfield federal arsenal
\2 The MA militia stopped it, but it showed the need for a strong federal
\1 Washington held the Mount Vernon convention for MD and VA to discuss import duties, navigation on the Potomac, and currency differences in 1785
\2 In 1786, the Annapolis Convention discussed commerce between the 5 states that came, where Hamilton and Madison called for the Constitutional Convention in 1787 due to Shay’s in Philadelphia
\2 This was intended to revise the Articles for a stronger government
\1 State level began to take private interests and became corrupt, with 1 year terms forcing them to push for reelection, such that laws shifted constantly
\2 As a result, they appealed to the majority, such that many believed it had to be further from the people to balance majority and minority interests
\end{outline*}
\subsection{The Constitutional Convention}
\begin{outline*}
\1 The 74 delegates (55 which attended) at the convention, many of which had helped on state constitutions, the Articles, the Declaration, and were leaders
\2 Washington gave it trustworthy appearance, while Franklin was a famous scientist and diplomat, as well as Wilson who read Franklin’s speeches and did detailed work, and Morris who wrote the final draft
\2 Madison also wrote the government plan, and notes on the Convention, called the Father of the Constitution
\1 Washington was chosen as leader, each state got 1 vote, a simple majority was needed for meetings and votes, and it was closed to press and public
\2 They quickly decided to start over, using a limited representative government, divided among the three branches
\2 They also agreed to strengthen the federal, and limit states ability to coin money or interfere with the rights of creditors
\1 Madison’s Virginia Plan stated that a national legislature, with a lower house elected, who pick the upper house, and a strong executive and judicial appointed by the legislature, setting the strong national tone for the convention
\2 The legislature would be able to remove unconstitutional state laws, but small states feared larger ones controlling a strong national government
\1 Paterson’s New Jersey Plan had one legislature with equal, single votes, that could set taxes and regulate interstate trade, a weak executive chosen by it, and a limited judiciary chosen by the executive, amending the articles
\1 The Hamilton Plan would set up an absolute executive for life, to make the Virginia Plan appear moderate
\1 Sherman’s Connecticut Compromise had a lower House based on population, where spending/taxing begins, and an upper house, chosen by state legislatures
\1 The 3/5ths Compromise stated that slaves would be counted as 3/5ths of a person for both tax and representation purposes
\2 The North also wanted complete federal control over international trade, while the South needed agricultural exports, fearing control over them
\2 They agreed to not abolish the slave trade until 1808, and let Congress regulate interstate and foreign commerce, but did not allow export taxes
\1 The Constitution also did not mention slavery, except allowing slaves to be returned to owners, ignoring slavery to allow it to be written without dispute
\2 They agreed to the voted electoral college (instead of people, state legislatures, and Congress), voted for the president, and a 4 year term
\1 9 states had to ratify it for it to apply to those, the rest for them, leading to a debate between the Federalists, supported by urban merchants, and the anti-federalists, supported by laborers and farmers
\2 Anti-Federalists said it was extralegal, not authorized since the convention was to revise the Articles, and said it took powers from states
\2 They also feared, without a written bill of rights, the government would violate rights, led by Henry, gaining the Bill of Rights as amendments (12, 2 rejected), even though most states had in their state constitution
\2 The Federalists thought strong government was needed to prevent anarchy, and protect the nation from internal and external problems
\1 Federalists feared state armies fighting or military leadership of smaller states out of fear, and thought the people would prevent lobbyists and that Senate age and citizen restrictions was enough regulation
\2 Anti-federalists felt not knowing representatives personally, not separated by different interests, would lead to tyranny, and that recall was needed to prevent Senate autocracy, as well as fearing drafts taking away rights
\2 They also feared a standing army in peace, not loyal to the government, and feared Senate influence, approving executive choices, regulating judicial actions, but lost due to no alternative
\2 Yates was a NY judge, and the anti-federalist leader called Brutus, while the Federalists called themselves Publius
\1 After NH, it took effect, but had to be approved in VA and NY, leading to Hamilton, Madison, and Jay writing the Federalist essays defending it, after which the government began on March 4, 1889 in NYC
\end{outline*}
\section{Chapter 3 - The Constitution}
\subsection{Structure and Principles}
\begin{outline*}
\1 The US Constitution is unusually simple and short, such that it could be understood vaguely by people in the future
\2 The Preamble is the explanation of why it was written, to serve the people, and protect liberties, stability and order
\2 There are 7 articles, Article I making the legislative, Section I making Congress, II and II making the House and the Senate, then elaborating on powers, restrictions, and procedures of Congress
\2 Article II makes the executive branch, describing qualifications, procedures, elections, and powers of the president and VP
\2 Article III makes the judicial, allowing them to make lower federal courts, followed by Section II, giving jurisdiction on specific types of cases, followed by Section III which defines treason
\2 Article IV gives citizens of other states, the rights of citizens of that state, discusses federal military protection to states, and admitting new states
\2 Article V discusses amendments, VI the supremacy clause of Federal law and treaties being the supreme law, and VII for ratification
\1 The Constitution is first based on popular sovereignty, or consent of governed, as well as separation of powers, to prevent any one branch from gaining too much
\2 It includes federalism, or the division of power between national and states, moving from the confederation, but not to a unitary due to fear
\1 Judicial review is the power to declare national, state, and local government actions unconstitutional, given to all federal courts, established over federal law by Marbury v. Madison in 1803, only to be changed by amendments or the court
\2 It is also based on checks and balances on other branches, such as presidential vetoes as well as 2/3 vote of both houses to overrule, as well as the president able to appoint judges, approved by the Senate
\2 Limited government was also given, specifically listing its powers in the Bill of Rights and Article IV
\end{outline*}
\subsection{Amending the Constitution}
\begin{outline*}
\1 Amendments must be proposed either by a 2/3 vote of both houses of Congress, or by a national convention, requested by 2/3 of states
\2 The latter was almost used in 1963, when 33/34 needed states petitioned for an amendment overturning state lawmaker elections
\2 By 1991, 32 had petitioned for a required balanced budget amendment, but after the 2000 balanced budget, it lost support
\1 Ratification must either be by 3/4 of state legislatures, or a ratifying convention in 3/4 states, able to try again for ratification later
\2 During the Equal Rights Amendment anti-gender discrimination debate, 5 tried to revoke it, which it was uncertain whether was legal
\2 Conventions have only been used once for the 21st amendment to repeal Prohibition (18th), determined over legislatures by Congress, planned and organized by states
\2 Congress can set the time limit for ratification, currently 7 years, though it can be specified in the amendment, influencing the results
\1 Informal changes are also used to subtly change the constitution, either through passing laws to clarify or expand the Constitutional provisions, such as tax law, cabinet departments and agencies, and the Judiciary Act of 1789
\1 Congress has also used changes through practice, like impeachment, accused in the House, tried in the Senate, to define high crimes and misdemeanors
\2 Presidents have also done this through presidential succession (interim or actual) by Tyler in 1841 after Harrison died, until 25th amendment in 1967
\2 Executive agreements directly between the president and foreign leaders, are often used in place of Senate approved treaties, and the president has began pushing for Congressional laws, rather than just enforce
\1 Federal court decisions by judicial review, either restraint, avoiding initiative unless it is in clear violation, or activism, shaping national policies such as the Warren Court from 1953-69, setting the bar for civil and accused rights
\2 In the 1930s, fiscal conservation activism was common, now typically liberal, while Court rulings have often changed as times did
\1 Changes can also be done by custom, such as political parties or term limits
\end{outline*}
\subsection{Amendments}
\begin{outline*}
\1 The first 10 amendments are the Bill of Rights, ratified in 1971 to limit the powers of government, the first of which gave freedom of religion, speech (written and verbal), press, assembly, and separation of church and state
\2 As a result, the press doesn’t have prior restraint, or government censorship, though it is unable to slander (false speech to damage a person’s reputation) or libel (false writing)
\2 People are also required to have responsible speech, and are not allowed to give state secrets or advocate violent revolution
\1 The 2nd gives the right to security, allowing an armed militia and citizen gun ownership, but does not prevent arm regulation
\1 The 3rd prevents forced quartering, though Congress has certain conditions in war, under which they can
\1 The 4th requires specific warrants to search and seizure, needing probable cause, as well as search or arrest warrants
\1 The 5th requires evidence before calling a trial, bans double jeopardy (tried for a crime found innocent for), self-incrimination, due process of law (proper trial procedures), and defines eminent domain, saying property can only be seized with fair compensation, for the good of the general public
\1 The 6th gives a speedy, public trial by an impartial jury, allowing a request to talk to the judge privately, and a change of venue if the jury cannot be impartial
\2 It also allows being informed of charges against, calling witnesses, hearing all questions against witnesses, and the right to an attorney
\1 The 7th gives right to a jury trial for civil disputes >\$20, able to be decided by a judge if both parties agree, while the 9th states all non-specified rights are not declines, and 10th gives all non-specified powers to the states
\2 The 8th prohibits excessive bail, fines, and cruel and unusual punishment
\1 The 11th limits federal court jurisdiction after Chisholm v. Georgia gave it full powers, while 12th creates separate ballots for president and VP
\1 The 13th outlaws slavery, the 14th protects all citizens from losing rights without due process, and gives equal protection under the law, and gives citizens to all born or naturalized, and the 15th prohibits loss of voting rights by race
\1 The 16th gives Congress the power to do individual income taxes, after it was ruled unconstitutional in 1895, the 17th gives popular election of senators after buying votes, and the 18th began the prohibition, ended by the 21st (though transport to states that ban alcohol was still illegal)
\2 The 19th gives women the right to vote, the 20th gives dates for inauguration and Congress start terms, ending terms earlier in January 3rd (Congress) and 20th, preventing lame ducks
\2 The 22nd gave presidential term limits, the 23rd gave presidential voting rights to DC, 24th banned poll taxes, 25th gave a process for VP succession, such that the VP and majority of the cabinet, or the president, can write the president pro tem of the Senate and Speaker of the House
\2 26th lowers voting age to 18, and the 27th makes constitutional pay raises effective only in the term after
\end{outline*}
\section{Chapter 4 - Federalism}
\subsection{National and State Powers}
\begin{outline*}
\1 The Constitution gives expressed (collect taxes, coin money, make war, raise a military, and regulate interstate commerce) and implied, or those required to carry out expressed powers by the elastic/necessary and proper clause
\2 Implied powers includes drafts, nuclear plants, or the space program
\2 Inherent powers are those automatic in a national government, such as diplomatic relations or immigration control
\1 Reserved powers are those not given to the federal or removed from the states, such as intrastate commerce, local governments, health, welfare, and morals
\2 The supreme clause prevents state law from conflicting with national law
\1 Concurrent powers are those given to both national and state government, like collecting taxes, spending money for the people, borrow money, or make courts
\1 Denied powers are those given to either states or national, such as export taxes for national, or foreign treaties, coin money, interfere with private contracts legally, or give nobility titles for states
\2 Interstate compacts/agreements or export/import taxes also require congressional permission, and neither can take away personal rights
\1 The national government guarantees protection from invasion and domestic violence to the states, and the president has authority to send troops when asked
\2 This occurred in summer 1967 to stop rioting in Detroit after MI national guard and police failed
\2 Federal law, property, or responsibility violations also allow troops to be sent by the president, even without permission such as the 1894 Chicago Railroad strike, interfering with mail and property
\2 It was also used for Little Rock and University of Mississippi and Alabama during integration, and has been extended to natural disasters
\1 The government can also not take territory from a state without permission, with the exception of WV in 1863, and must have a republican government
\2 States without a republican government are not permitted to be seated in Congress, used after the Civil War when blacks were not allowed to be citizens in states that didn’t ratify the civil war amendments
\2 States are required to pay for and plan all national elections
\1 Congress is able to admit new states, though the president can veto, starting with an enabling act to let the territory write a state constitution, approved by popular vote, then sent to Congress, who can pass an admission
\2 On the other hand, VT, ME, WV, KY, and TN were made from existing states, and WV and TX were admitted in usual ways
\2 Hawaii and Alaska both wrote a constitution without an enabling act, while WV broke away as the rest seceded, held to be the only legal part of the legislature at that time
\2 Texas declared independence from Mexico, and after several years, in 1845, the joint resolution of Congress had the US annex it, allowing it to become an immediate state, with the possibility of division into 5 states with Congress and legislature approval
\1 Congress is allowed to force certain conditions before accepting a constitution, but the Supreme Court only allows them to enforce it if the conditions don’t interfere with the state’s authority to manage its affairs, due to state equality
\2 In Arizona in 1911, Taft forced them to change recall of judge provisions, but after admission, they put the provisions back
\2 OK originally was required to keep Guthrie as the capital until 1913, but the Supreme Court let them move it to OK City
\1 The National Governors Association was called in 1908 by Roosevelt to discuss conservation, meeting regularly after, forming an organization in the 1960s in DC
\2 In the 70s, it made seminars and publications to help governors build the state power and solve problems
\2 In the 80s, it worked to produce national policy in welfare, education, and health-care reforms, and regional NGAs began to form as well
\1 The federal courts typically act to balance national and state power, first in McCulloch v. MD in 1819, deciding in a national vs state conflict, national wins
\2 Until the Depression, it typically ruled in favor of states, shifting in terms of national through the Depression and Civil Rights, until the 90s, then back
\2 It reviews state actions, to prevent violation of the Constitution, often the 14th amendment (deprivation of natural rights without due process)
\end{outline*}
\subsection{Interstate Relations}
\begin{outline*}
\1 The full faith and credit clause gives recognition to other states laws and legal decisions, to prevent states acting as foreign nations, or havens
\2 It applies to public acts (civil laws, or disputes between groups or the state, but not criminal), records (legal documents), and judicial rulings
\1 All citizens of other states gain the same privileges and immunities as citizens of that state, stopping discrimination, but doesn’t apply to reasonable discrimination
\2 This includes travel, property ownership, court use, contracts, and marriage, but not voting, jury duty, using certain public facilities (state schools or hospitals), or practicing certain professions (medicine or law)
\2 In addition, higher fees may be present for public facility use for non-residents, such as state colleges, and fishing or hunting licenses
\1 The extradition clause requires states to return fugitives to the state they are fleeing from, changed by the Supreme Court to simply apprehending in extreme cases, though it has recently been made a felony to flee to avoid prosecution
\1 Interstate compacts are used to write written agreements between multiple states, as well as the national government or foreign governments in some cases
\2 Congress must approve of them to avoid alliances threatening the Union
\2 States first used it for border disputes, but after began using it for pollution, transportation, pest control, or natural resource conservation
\1 Interstate lawsuits can be used in extreme cases, only able to be done in the Supreme Court, often for water rights in the West, pollution, or boundaries
\end{outline*}
\subsection{Federalism Development}
\begin{outline*}
\1 The states’ rights position states the Constitution is between the states, and that the power of the national government is created and limited by the states
\2 All debates about the power of the national are in favor of states
\2 Those, such as Chief Justice Taney in the mid-1800s, believed state government reflected the people better than national, which was viewed as a threat to liberty, ruling based on the 10th amendment
\2 From 1918 to 1936, decisions worked against child labor, industry, and agricultural regulation
\1 The nationalist position states that the people created the national and state government, such that the elastic clause allows expansion of government to carry out delegated powers, and that delegated powers should not limit it
\2 Those, such as Marshall’s Court in the early 1800s, believed national government stands for people, while states are only a part of the people
\2 In the late 1930s, as the Great Depression got worse, the court changed their opinions in favor of social welfare and public works
\1 Expansion of national government was based on the power to wage war, due to the economy, educational system, and national security being required for war
\2 Commerce regulation has expanded as well, to all buying, selling, and transporting of goods, such as the Civil Rights Act of 1964, ruling inn and restaurant discrimination restricted the transport of people and trade
\2 Taxing and spending ability, especially income tax, has been allowed to incentivize (tax breaks) or tax goods and social programs
\2 On the other hand, block grants work to allow the states more freedom over their decisions
\1 Congress has also began to influence state and local government through federal aid and preemption mandates (assume responsibility of state functions)
\2 Federal grants are the main type of aid, through large sums of money for specific purposes, working to aid unequally funded state governments
\2 Some governments fear federal grants, due to more national control
\2 In the 1960s, Congress began using preemption, such as the Nutritional Labeling and Education Act of 1990, fully power to set food labeling standards from the states, even when stricter than national
\2 Restraints are requirements set by Congress to prohibit an action, while mandates are orders to provide a service meeting national standards, such as disability, civil rights, or environmental standards
\2 Congress is not required to pay for mandates, forcing it on states
\2 The idea of federal aid coercion is the basis of coercive federalism, moving from cooperative, working together on the same issues, and further from the struggles over powers of dual
\end{outline*}
\section{Chapter 5 - Congressional Organization}
\subsection{Congressional Membership}
\begin{outline*}
\1 Congressional terms last for 2 years, starting on January 3rd of odd years, each term divided into 2 sessions with breaks for vacations and holidays
\2 They originally started in March until the Lame Duck (20th) amendment
\2 Each house requires a vote to adjourn, can be called back by the president for a special session, and may only adjourn 3 days without the other house’s approval
\1 Each state gets at least 1 member of the House, divided based on population, such that each member must be 25+ years old, citizens for 7+ years, and citizens of the state, and traditionally the district as well
\2 Special elections are held for first session vacancies, while procedures for second sessions range based on state
\2 The Census Bureau takes a national census every 10 years, such that reapportionment determines the number of representatives
\2 Representatives originally was 64 members, growing to 435 as population grew until the Reapportionment Act of 1929 limiting the increasing size
\2 DC, Guam, Samoa, Puerto Rico, and the Virgin Islands get 1 non-voting representative, but can attend, speak, vote in committees, and give bills
\1 State legislatures then redistrict, or draw, congressional districts, but have corrupted it by either unequal populations per district or gerrymandering
\2 Baker v. Carr in 1962 determined federal courts could decide redistricting disputes after an issue in TN, while Reynolds v. Sims in 1964 stated seats in state legislatures must be based on population, after an AL issue
\2 Wesberry v. Sanders in 1964 ruled that districts must be equal in population, leading to the modern 650k per district
\2 Gerrymandering is redistricting by parties to gain an advantage, started by Gerry to give win against Federalists, creating a salamander district
\2 Packing is the inclusion to include maximum opponents, while crowding is to divide opponents into other districts, the former used in 1992 in NC to add minority representatives but individually struck down for packing
\2 The Supreme Court stated districts must be compact and physically touching (contiguous), but is still an issue
\1 Senators must be 30+ years old, citizens for 9 years, and residents of the state, elected at-large, or statewide, with 6 year terms, typically winning reelection
\2 Vacancies are filled by the governor, who can also call a special election
\1 Those in Congress originally got low salaries, voting for pay increases to stop pay as a deterrent, in 1991 raising by \$23k to stop honoraria (paid speeches)
\2 In 1992, the 27th amendment was passed, and used in 1993 to try and ban cost of living allowances, but which was ruled constitutional
\2 They also get franking privilege (stationery and postage for official purposes), a clinic, a gym, allowances for staff and travel, income tax deduction to allow 2 residences, and \$150k+ pensions
\2 Members may not be arrested for non-treason, felonies, or breaches of peace to and from Congress, or get sued for things said at Congress
\2 In Hutchinson v. Proxmire in 1979 stated they could be sued for libel outside Congress, and a majority vote may refuse to seat a member through exclusion by Powell v. McCormack
\2 They may also punish members and censure (formally disapproved) by a majority for minor offenses, or expelled by a 2/3 vote, due to serious felonies or treason
\1 Congressmen are reelected 90\% for the time, often unchallenged, due to easier access to funds, gerrymandering, higher name recognition, and bias toward supporting those voted for
\2 Most use the internet to gain supporters, but only 10\% candidates use town hall meetings on their websites to determine important issues
\end{outline*}
\subsection{The House}
\begin{outline*}
\1 Lawmaking houses must have uniformity of proceedings to ensure fairness and order, based on past precedents, published every 2 years
\1 The House uses rules to speed through debates, due to more powerful leadership roles, using committees to organize and allow smaller debates
\2 They also allow specialization on important issues to constituents
\2 The House is organized by parties, sitting separately, where the majority appoints committee chairs and controls the flow
\2 In 1995, Republicans pushed reforms, centralizing power in the speaker, with fewer staff, committees, term limits for committee chairs and the Speaker, ending absentee voting, to make it more accountable
\1 House leadership works to unify and regulate party members, schedule work, get members to the floor for votes, transfer information, and contact the president
\2 A majority party caucus chooses the speaker, voted by the entire House, recognizing who gets to speak, appointing chairs, making schedules, giving bills to each committee, and gives favors for support
\2 The majority leader plans the party's legislative program, and rushes important bills through, elected by caucus only, as a party position
\2 The majority and deputy whips persuade people to follow party lines, and get members on the floor for voting, chosen by the party
\2 The minority party also gets similar positions, without scheduling power
\1 The House opens at noon, ringing buzzers throughout, not working much on Friday so they can visit home districts, and Monday is for routine work
\2 The House floor is also busy and distracted until the vote
\1 Laws begin as bills until passed by both houses, signed by the president, such that it starts by being introduced, dropped in the hopper box near the front
\2 It is sent by the Speaker to committee, 15\% surviving to the calendar for a vote, either the Union (monetary), Private (individual people or places), or the House Calendar (for other bills)
\2 Consent Calendar is used for bills unanimously voted to be read out of order, while Discharge is petitions to discharge a bill from a committee
\1 The Rules Committee directs flow of legislation from committees, determining when they move forward, leading to political battles
\2 In 1911, the House removed the Speaker, until replaced by the Democrats in 1975, letting him appoint all majority members by caucus
\2 Committee chairs are able to request the Rules Committee moves bills from the Calendars to the top of the calendar, due to the surplus of bills, and can limit debate time or amendments to the bill on the floor
\2 They can also settle disputes between committees, and protects people from unpopular positions by making the bill not reach the floor
\1 Quorums, or a majority (218) of House members, are necessary for a session, though a Committee of the Whole (100) can debate and amend, but not pass, reporting the changes to the bill to speed the process
\end{outline*}
\subsection{The Senate}
\begin{outline*}
\1 The Senate is more informal, to allow freedom for expression of ideals, allowing unlimited debate, with very few procedural rules, such that far fewer are present
\1 Party leadership is run the same in the Senate, with the VP preceding, but only voting in cases of a tie, though leadership has less power due to less structure
\2 The VP cannot take part in debates, but may recognize speakers, call votes, or influences votes outside of the Senate
\2 The president pro tempore is elected by the Senate, typically the most senior member of the majority party, acting in the VPs absence
\2 The majority leader plans the work schedule, agenda, and unifies the party to get bills passed, making party members go to votes
\2 Whips and assistants also work to aid party leadership, with at least one party leadership member remaining at all times in the Senate
\2 Seating is organized by parties, such that each party sits on its own side
\1 Any member may propose a bill, where flow is scheduled by the Calendar of General Orders for all bills, and the Executive for treaties and nominations
\2 Senate leaders control the flow of bills through informal negotiations, though a unanimous vote can take a bill immediately from the calendar
\1 The filibuster is to prevent a bill from getting a vote by stalling through talking, delaying issues, or procedure, though other matters can take place currently
\2 Filibusters can be stopped by a 3/5th majority of the Senate for cloture, allowing a maximum of 1 hour of debate, though a 3/5th majority is rare
\end{outline*}
\subsection{Congressional Committees}
\begin{outline*}
\1 Committees allow determination of which bills are worth further consideration, work out compromises, and hold public hearings to educate the public to issues
\2 Members by party are typically proportional to that of the chamber itself, though the majority has supermajority in the important committees
\1 Standing committees are permanent from session to session, such that the majority party is able to change committee rules
\1 Subcommittees specialize in facets of standing committees, typically continuing from session to session, with a maximum of 5 per, except Appropriations (13), Government reform and oversight (7), and Transportation and infrastructure (6)
\1 Select committees are temporary, to study a specific issue, then report findings, either major public issues, overlooked problems, or interest group issues
\2 They typically exist for 1 term, but can be renewed, sometimes reclassified as standing, such as the Select Intelligence Committee
\2 Select committees typically cannot submit bills directly to the chamber
\1 Joint committees are from both chambers, either select or standing, theoretically coordinating chamber efforts, realistically used for routine affairs, like printing
\2 They cannot directly propose bills to the chambers, or work with them
\2 Several have been made for volatile matters as well, such as nuclear energy, defense, or taxation
\1 Conference committees are for when different versions of the same bill are passed, made up of conferees to negotiate a final version to send to Congress
\2 When a majority of the committee agree, it is sent, voted on without possibility of amending in Congress, then sent to the president
\1 Membership can increase likelihood of reelection, especially those which directly impact their state, can allow influencing national policy and other legislators
\2 The best in the House are Rules, Ways and Means, and Appropriations, while in the Senate, Foreign Relations, Finance, and Appropriations
\1 Members are limited to a max number of committees, chosen by party leaders, while the seniority system determines who the chair is, modified in the 1970s
\2 The reforms had secret ballots of high ranking party members to determine chair
\1 The chairperson is able to control meeting schedules and bill discussions, as well as hearings, witnesses, budget, committee staff members, and run debate
\2 The Legislative Reorganization Act of 1970 allowed a majority of members to call a meeting without chair approval, and the minority must be allowed to present views and have reasonable notice for meetings
\2 1995 reforms prevented absent member votes from being cast by the chair, and required publishing votes
\end{outline*}
\subsection{Staff and Support}
\begin{outline*}
\1 Congressional staff help to manage the work of Congress, cut 15\% in the Senate, 50\% in the House in 1995, when the budget and GAO was also cut
\2 They communicate with voters, help run floor sessions and committee hearings, draft bills and reports, and attend committee meetings
\2 They also monitor publicity, issues, and popularity in the home district, write speeches, raise funds, and help them get reelected
\1 Until the early 1900s, Congressmen didn’t have aids, but privately paid assistants and until the Legislative Reorganization Act of 1946, had $\leq$ 400, shifting to 3000+
\2 The staff was used to get expert opinions on topics of which the legislator could not be knowledgeable in, as well as to communicate with home, and solve individual problems with their constituents (caseworkers)
\1 Personal staff work directly for legislators, with the number depending on budget, for Senators based on distance from the state’s capital city and population, the House getting a constant number, including caseworkers in a home office
\2 The administrative assistant runs the office, schedule, and gives political advice, as well as makes deals to aid reelection
\2 Legislative assistants research, draft, and studies bills, and writes speeches and articles, represents and assists them at committee meetings, and tracks the movement of bills
\2 Often legislative assistants write up speeches or questions for the end of committee, and the legislature comes in, and relies on their judgement, without knowing what the committee topic is
\2 They also tell the legislature how to vote during each vote during the day
\1 Committee staff are based on the size of the committee, controlled by the senior minority and the chairperson, working to draft bills, memos, and reports, collect and study information, and plan hearings
\1 Support organizations serve to make the legislative branch less reliant on the executive branch for information since the modern era
\1 The Library of Congress was made in 1800 to get books needed for Congress, currently containing 100M+ items, getting free copies of copyrighted works due to administering copyright law, with the Congressional Research Service
\2 100s of employees answer requests for information either to give to voters, or to use when writing bills
\1 The Congressional Budget Office was made in 1974 to study presidential budget proposals, and make cost projections, countering the executive Office of Management and Budget Making, watching Congress’s spending and the deficit
\2 They made a yearly report for how the deficit effects the economy, and studies trends
\1 The General Accounting Office was made in 1921, run by the Comptroller General, to settle and watch government finances, debts, and provide information on government programs funding to Congress
\1 The Government Printing Office prints all documents, including the daily Congressional Record (which can include revised or unmade speeches by legislators), or the Statistical Abstract of the US (of Bureau of Census data)
\end{outline*}
\section{Chapter 6 - Congressional Powers}
\subsection{Constitutional Powers}
\begin{outline*}
\1 The 2nd Bank of the US in 1816 created a debate about the power to charter a bank, and the power of Maryland to put a tax on federal bank notes
\2 McCulloch v. Maryland was based on a teller who issued notes without tax, leading to the Supreme Court ruling that it was illegal to tax
\1 The Bill of Rights denies certain legal powers, such as suspending writ of habeas corpus (court order to release the accused, to see if they were legally detained)
\2 Bills of attainder (laws establishing guilt without trial) are banned
\2 It also bans ex post facto (crimes of acts which were legal when done)
\1 Tax and spending powers include all revenue bills (for raising money) starting in the House in Article 1, Section 7, giving large states more power on taxation
\2 Appropriations bills (for spending) are requested by the executive, given by Article 1, Section 9, which requires laws to take from the Treasury
\2 Congress has used tax and spending for fiscal policy, gives grants to influence state and local policy, and incentivizes public choices
\1 Congress is also allowed to borrow money, through government security bonds or notes, gaining repayment to the public with interest after some length of time
\2 Since 1917, there has been a debt ceiling, constantly raised when reached to allow Congress to pay bills
\2 Congress is also allowed to coin money, punish counterfeiters, regulate value of currency, establish weights and measures
\2 It is also able to make bankruptcy law, to deal with assets of a business which cannot pay its debts, though it did not until 1898
\1 Article I, Section 8, Clause 3 is the commerce clause, first expanded in Gibbons v. Ogden when it ruled Gibbons could sail a ferry due to the service crossing state lines, rather than simply concerning products
\2 This has been expanded to include minimum wage, workers rights, broadcasting, banking, finance, and air and water pollution
\2 In Heart of Atlanta Motel v US in 1964, it was ruled that stores whose customers and food were from other states could not discriminate
\1 Congress also has the ability to approve treaties, declare war, and create, legislate, and maintain an army and navy, though the president has used force without a declaration of war $\geq$ 200 times, including Korea and Vietnam
\2 The War Powers Act of 1973 overrode Nixon’s veto, banning forces being sent more than 60 days without congressional notification within 2 days
\1 Section 8 gives the power to issue copyrights (lifetime + 50 years) on artistic work and patents (17 years + renewal) on inventions
\2 It also gives the power to make federal courts and post offices, the latter of which has been used to punish criminal mail usage
\1 Article I allows naturalization laws, IV.3 for admitting states and governing territories, and both for federal property regulation
\1 Congress also calls a joint session to count electoral college votes, the House chooses the president from the top three candidates if none have a majority with each state getting one vote, and the Senate picking the VP from the top two
\2 This allowed Jefferson in 1800 and Adams in 1824 to be from parties other than the president’s party in these cases
\1 The 20th and 25th amendments give the president the power to pick a new VP, but require joint approval by Congress, but also gives powers for vacancies
\2 The House can call for an impeachment, followed by a Senate trial requiring a 2/3 vote, presided by the Chief Justice if for the President, only able to in cases of high crimes/misdemeanors, treason, or bribery under Article II, Section 4
\2 The Senate also must approve presidential appointments, including Supreme Court justices, who are rejected 20\% (most often)
\1 Article II, Section 2 gives the Senate the power to ratify treaties with a 2/3 vote, though executive agreements have been used to bypass it
\2 The Strategic Arms Limitation Talks II Treaty was rejected in 1980, preventing a vote, though typically, most are ratified
\1 Congress can also draft amendments, passed by state legislatures except the 21st which was passed by conventions since legislatures were generally against
\end{outline*}
\subsection{Investigations and Oversight}
\begin{outline*}
\1 While not given or denied the power, from 1792 when they investigated the Army after a defeat by Natives, they can use committees for investigation, including having staff find witnesses to testify under oath and evidence
\2 The 1998 Senate Finance Committee investigation into the IRS led to a unanimous vote to reform the IRS, and 90s ethics panels rid corruption
\1 They are able to issue subpoenas, or a legal order to appear or produce documents, and can require oaths, facing perjury if found to be lying
\2 They can also hold those who refuse to aid in contempt, or willful obstruction, which can lead to jailing
\2 Until Watkins v. US in 1957, witnesses had no rights until it ruled that Congress must respect constitutional rights, similar to courts
\2 Self-incrimination rights are prevented by the ability to offer immunity, forcing them to answer, such that in 1987 to investigate Reagan for arms dealing to Iran and Nicaragua, Colonel North was freed due to information he gave with immunity leading to finding evidence against him
\1 The legislative oversight power of Congress is the ability to review how the effectively the executive branch carries out laws passed, as checks and balances
\2 The Legislative Reorganization Act of 1946 called for Congress to continuously watch executive agencies, while the Act of 1970 had each standing committee review that execution of laws under its jurisdiction
\2 This is used to assure that the executive is interpreting and following the laws as intended, a greater problem in recent years
\1 In actuality, supervision is infrequent, and typically only when the president is not from the majority party, mainly due to limited resources, vague laws, committee- agency friendship, and lack of voter support except in major scandals
\2 They require submission of reports to Congress often, such as the 1946 Employment Act requiring an annual economic report from the president
\2 The GAO is also able to monitor federal agency finances, and the appropriations bill requires an agency budget review
\2 The legislative veto, or provisions in laws to review or cancel executive agency actions to enforce those laws, was often used until 1983, when Immigration and Naturalization Service v. Chadha ruled it unconstitutional
\1 The 1978 Ethics in Government Act assured fair investigation for officials, letting Congress call independent counsel as special prosecutor, but in 1999, full power was given to the Attorney General after due to expensive, political investigations
\end{outline*}
\subsection{Legislative-Executive Relationship}
\begin{outline*}
\1 Presidents who have a less active role in legislation, like Eisenhower, typically have the best relationship with Congress, due to the checks and balances
\2 Since legislators are chosen by a smaller group, they may have different interests than national policy interests of the president
\2 Party politics has led to gridlock when they were oppositely controlled
\2 The difference is time in office, maximum of two terms, rather than unlimited, results in presidents rushing to get policy through. while House members have to avoid controversy, and Senators have more time
\2 Rules of procedure, such as the filibuster or committees, can also be used to block legislation proposed by the president
\1 The system of checks and balances has led to a conflict, with strong presidents questioning Congress’s monopoly on policy-making, such as Jackson, those in the Civil War, Depression, and the Cold War
\1 Until the 1970s, Congress had only reacted to the president’s national budget proposals, until the 1974 Congressional Budget and Impoundment Control Act, creating the CBO and a budget committee for each house, to take a greater role
\2 It also limited the president’s ability to impound, or refuse to spend money allocated for a specific program, unless both houses agreed
\1 Until declared unconstitutional in 1983, the legislative veto, or nullification of an executive action, was used by Congress extensively as a check on the executive
\1 In war, presidents have expanded their power, closing banks, controlling the economy, rationing goods during WWII, continuing the state of emergency until the 1976 National Emergency Act, ending it, forcing notification to Congress
\2 State of emergencies also last 1 year maximum, must be renewed otherwise, and Congress can pass laws ending it at any time
\end{outline*}
\section{Chapter 7 - Work of Congress}
\subsection{Bill Development}
\begin{outline*}
\1 Private bills are those involving specific people or places, originally a large number of bills, now only several hundred, changing the law for specific people
\2 Public bills are general matters, such that major bills get media coverage
\1 Simple resolutions are those involving only one house of Congress, not needing the force of law, while joint resolutions can be used to correct laws, requiring presidential approval, or to propose an amendment, which doesn’t
\2 Concurrent resolutions involve both houses, but are not public issues, not needing full legal authority, requiring being passed by both houses
\1 Riders are provisions on a subject other than the main bill subject, such that Christmas tree bills are essential bills with many riders, to force them through
\1 Only 16\% of bills become laws, due to the long process, most added as congress and bill quantity/complexity increased, giving opponents the advantage, and giving time for compromises with major interest groups for support
\2 Bills not passed by the end of session must be reintroduced in the next
\2 Many bills are also introduced purely to show support for a policy, or get media attention to themselves or the issue, not actually expected to pass
\1 Bills must first be introduced by a member of Congress, typically with cosponsors to show support, written or suggested by any citizen, called the first reading
\2 In the House, it is put in the hopper, designated H.R.n, in the Senate with the Senator recognized by the presiding officer, designated S.n
\1 They are then sent to a committee, put in a subcommittee by the chair, or pigeonholed, ignored to kill it, or killed by a majority vote
\2 If accepted, it can be rewritten, modified or recommended, then sent back
\2 To accept a bill, hearings are held, listening to expert, special interest, or government testimony, while committee staff does additional research
\2 Chairs can also use hearings to bring bills to public attention over the internet, influence opinion, and allow interest groups to lobby for changes
\2 After the hearing, the committee has a markup session to decide on changes, going section by section, requiring a majority committee vote
\2 There is then a vote to either kill, or report (give with a written report to the house floor) explaining the bill, major changes, and committee rationale
\2 Occasionally, bills are reported with a recommendation not to pass, if they want the House to vote fully, even if the committee disapproves
\1 On the floor of both houses, there is a second reading, debated by a few lawmakers, and amendments are proposed by any lawmaker, voted by a majority
\2 Often, opponents propose amendments to slow or kill the bill, adding so many objectionable amendments that it dies
\2 There is then a third reading and a vote, requiring a majority, either by voice vote, standing/division vote (calling each side to stand in turn), or a roll-call vote, while the House also has a recorded vote (electronically and displayed, used since 1973 to save time)
\1 If the versions passed by each house are different, a conference committee of conferees from each house committee is called to solve differences, theoretically only fixing differences, but often changing other sections
\2 Then a majority vote gets a committee report sent, for a new floor session
\1 It is then given to the president to sign, veto (sent back to the original house with a veto report, able to be overrided with a 2/3 majority from both houses), or wait 10 days (if Congress is in session at the end, it is law, otherwise a pocket veto)
\2 Presidents also want a line-item veto to reject specific sections, but which would require an amendment, solved in the Line Item Veto/Enhanced Recision Act of 1996 by giving line-item veto power on taxing and spending items, with a 2/3 majority to overrule
\2 Many feared it allowed Congress to push the duty for spending cuts to the president, and after reducing Medicaid for NY hospitals (and Idaho potato growers), Clinton v. City of New York in 1998 struck down the act
\1 Laws are then registered in the National Archives and Records Service and the US Code, given a code for the type (public/private), Congress, and law number
\2 THOMAS gives access to information about all legislation being considered, the full Congressional Record, committee reports, and bill summaries and history, but drafts and recommendations are not given
\2 This was made to allow citizens information, to lower lobbyist power
\end{outline*}
\subsection{Taxing and Spending Bills}
\begin{outline*}
\1 Taxes, or money citizens and businesses pay to support the government, are a majority of government funds, all bills of which start in the House
\2 The House Ways and Means Committee is the main tax work, deciding whether to follow presidential requests for tax cuts or increases, and which makes rules and regulations determining who pays the taxes
\1 Tax bills originally were under the closed rules, banning amendments on the floor to prevent non-committee members from directly writing
\2 The goal was to prevent special interests from interfering, and was due to the bills being too complicated for non-committee members to understand
\2 In 1973, House members forced the removal of the closed rule, and in 1974, forced Chairperson Mills to resign after a scandal, to fully remove the closed rule, leading to bills with many riders, like appropriation bills
\2 In the Senate, there was never a closed rule allowed as procedure
\1 The Senate is only able to propose amendments, thus allowing special interest groups the ability to influence tax bills, controlled by the Committee on Finance
\1 Appropriation of money to federal programs must start with an authorization bills, setting up the federal program under a specific executive department and determining the amount of money given
\2 Appropriations bills then are requested by the department, to be provided with money by Congress, as part of the executive budget
\2 The Appropriations committee writes the bills, and determines whether to accept the budget proposed by the executive branch
\1 Each house of Congress has appropriation committees with 13 subcommittees for different policy areas, where department, agency, and program heads testify each year about their budgets, testifying why they need the money
\2 This leads to heavy lobbying for special interest groups to the committees, and close relationships between committees and some departments, leading to easier hearings for some departments
\1 Uncontrollables, or expenditures promised to old federal programs, automatically committed, taking about 70\% of the budget, includes national debt, entitlements, called that due to automatically continuing, or federal contracts
\2 These programs are able to skip the committee hearings
\end{outline*}
\subsection{Congressional Influences}
\begin{outline*}
\1 Lawmakers are influenced most basically by their personality, risk taking, and their own beliefs on issues unimportant to their constituents
\2 Congressional staff can also control the decisions by the information given to the member, or by setting agendas for committees and members in favor of specific opinions
\1 The most major influence is that of voters, typically following what voters want in issues that affect voters’ lives, since while they don’t watch records, opponents and themselves use the voting record to give a case about reelection
\2 In close elections, voters’ opinions on following their wishes can decide it
\2 Legislators track opinions by regular visits to the district, pollsters, mailing surveys, internet polls, and emails/letters from constituents and special interest groups, screened by staff members
\2 They pay extra attention to campaign volunteers, major contributors, and their most loyal supporter base
\1 Political parties also play a large role in voting choices, especially in the House, following party policy 70\% of the time, especially on issues related to social, farm, and economic policy, rather than foreign policy, due to less fixed positions
\2 Parties generally share general beliefs on policy, and allow politicians without strong positions on specific issues to determine how to vote from those who know more about the issue
\2 Party leaders also often pressure members to vote to either support or fight the president’s program
\1 Presidents work to influence Congress, appearing on TV to influence opinion, putting pressure on Congress to support the position
\2 They also give or withhold support for individual legislators to force them to go along with the president's position
\2 Since the early 1900s, presidents tried to increase power, but recent reforms have worked to make Congress more autonomous from influence
\1 Special interest groups, represented by lobbyists, who work to persuade officials to support their point of view, either based around a specific group or issue
\2 Lobbyists visit legislators’ offices, encourage citizens to write legislators about issues, lobby specific congressional committees, and act as unpaid research staff to influence position since 1990s reforms
\2 In recent years, PACs have also allowed special interest groups to use funds to influence legislators, taking power from lobbyists
\end{outline*}
\subsection{Helping Constituents}
\begin{outline*}
\1 Casework, or helping constituents with specific problems and complaints, often unrealistic, about the federal government, is a major job of legislators
\2 Voters often claim to want less government, but demand large amounts of help from their Congressmen about executive agencies
\2 Caseworkers are staff for handling casework, getting the legislator only if it cannot be done directly with the agency
\2 Casework helps them get reelected, encouraging voters to send problems through mobile offices, emails, letters, and town meetings
\2 It also is a method of overseeing the executive branch, and provides a method of interacting with the national government, replacing local ward heelers from before the federal government controlled most services
\1 Congress also works to provide federal money to their districts, through pork-barrel laws, compete federal grants/contracts, and keeping federal projects
\2 Publics works bills spend on local projects, bringing jobs and money to the district, to improve facilities or services in the region
\2 Pork-barrel legislation are those which take federal money for local projects, such as moving federal organization headquarters to the state, using federal money purely for personal district gain, often disapproved of
\2 Logrolling is when multiple legislators support each other's pork-barrel laws, in an attempt to get reelected
\2 Executive branch agencies give federal contracts/grants, but legislators pressure them to aid their state, urge constituents to contact the agencies, and have staff specialists aid those petitioning for money
\end{outline*}
\section{Chapter 8 - Executive Branch}
\subsection{President and VP}
\begin{outline*}
\1 The president is the commander-in-chief of the military (wild card power), managing the defense budget and troop deployment, and appoint executive departments heads, federal judges, and ambassadors with Senate approval
\2 Presidents also make treaties (with advice and consent by the Senate), and meet heads of state, and foreign officials
\2 They also ensure that laws are executed properly, and are able to pardon, reduce sentences, and deliver the State of the Union to Congress to propose policy each year
\1 Washington held a precedent of two term maximum, and after Roosevelt broke it, the 22nd amendment in 1951 made it law, allowing vice presidents who take over and serve less than 2 years, to have 2 more terms
\1 Congress determines the compensation, or salary, raised from 200k to 400k in 2001, with travel of 100k and a \$148.4k pension, not able to change salary during the term, with all official expenses paid for, though not political expenses
\2 They also get Air Force One and other vehicles, free medical, dental, and health care, the White House, and a domestic staff
\1 The Constitution requires the president and VP a natural born citizen, 35+ years old, and a resident of the US for 14+ years, with being moderate as an unwritten
\2 Government experience, usually as a governor or senator, is an unwritten qualification, creating name recognition and political alliances to win
\2 Most presidents are northern European, middle class background, married, Protestant, wealthy men, with Ferraro as the female Democratic VP nominee in 1984, Jackson as the first black candidate in 1988
\2 People also generally like a story of personal growth, though flip-flopping is also generally looked down upon in modern day
\2 Large amounts of donations, and often personal wealth, are necessary to become president, though the Bipartisan Campaign Reform Act of 2000 limited individual donor, party, and total money, giving \$33M for primaries, \$67M for general, matched by the government unless it exceeds that
\1 8 presidents died in office, 4 assassinated, such that after Kennedy, the 25th amendment made the order of succession, putting the VP as president, getting confirmation by a majority of both houses for the new VP
\2 After Ford became president in 1974, appointed in 1973 after Agnew resigned, he appointed Rockefeller, the only time both were unelected
\2 The Succession Act of 1947 made the order of succession, with the Speaker then the President Pro Tempore, then Cabinet in order made
\2 The amendment also allows the Vice President to be acting president if the POTUS informs Congress of inability to perform duties, or the VP and majority of the cabinet informs them (the former which happened 5 times)
\2 In the former, the POTUS can inform them when he is ready to resume, in the latter, they either have to agree, or Congress has 21 days, where a 2/3 vote in both houses would allow the VP to keep office
\1 The VP’s main jobs depend on the responsibilities given by the president, almost nothing until Eisenhower, now aiding in policy discussions, defending the policies publicly, diplomacy overseas, and are on the National Security Council
\2 Cheney in particular represented Bush in cabinet meetings, with foreign officials, legislators, and developed policies in energy and defense
\end{outline*}
\subsection{Electoral College}
\begin{outline*}
\1 The idea of Congress selecting the president was rejected due to separation, and popular vote was rejected due to ignorant voters, and to prevent demagogues, instead creating the Electoral College in II.1 of the Constitution
\2 This gave each state electors of the number of legislators in Congress, meeting in each state to cast votes, called the electoral vote
\2 Originally, there was no popular vote cast, where the majority winner would be president, the second majority as the VP, with the House deciding by one vote per state in case of a tie or no majority
\1 After Washington, political parties formed, each nominating candidates for executive and for elector positions, each voting for their party candidates
\2 Jefferson and Burr, both DRs, were elected as a result of DR elector majority in 1800, but the Federalist House wanted to put Burr as President, instead of the elector prefered Jefferson
\2 The 12th amendment in 1804 then set up the modern split ballots, with the Senate choosing from the top two VP candidates in case of no winner
\2 In the 1820s, states put candidates on the ballot, such that electors were determined by parties after at state conventions or the committee
\1 Conventions are held in the late summer, ballots cast on the day after the first Monday in November, voting for the party electors, casting their votes in the monday after the 2nd wednesday in December, mailed to the pro tempore
\2 Electoral voters are winner-take-all in all states except ME and NE
\2 It is counted by both houses and the winner is declared on January 6
\2 Electors are generally not required to vote in favor, but typically do, except a WA elector for Reagan in 1976, and a MN elector for Edwards for both president and VP in 2004
\1 The winner-take-all system has been argued to be unfair to those voting for the losing candidate, allowing a president to lose the popular, usually winning large states by a small margin, with Quincy Adams, Hayes, Harrison, and W. Bush
\2 It also almost happened with Nixon in 1960, losing IL and TX by 1\%
\1 It also allows a third party candidate to win enough votes to prevent a majority, creating bargaining, used by Wallace in 1968 by the Independents, given to Nixon, rather than allowing the House to decide between him and Humphrey
\2 House elections also have the problem of ignoring population, the chance of a third party candidate preventing a majority, and if a state can not get a majority to vote for one candidate, the state loses the right to vote
\1 Reforms include electors based on districts, with the majority winning the two state electors, or electors being given by percent of popular vote, requiring them to vote for the candidate expected, or direct popular elections
\2 Some fear this would allow third parties to prevent a majority, and that popular elections would take power from states, hurting federalism
\1 On January 20th, the oath of office is taken outside the Capitol, riding with the old president from the White House, followed by the inaugural address
\end{outline*}
\subsection{The Cabinet}
\begin{outline*}
\1 The 15 secretaries, the VP, and several top officials are the members of the cabinet, acting as advisors and running bureaucracies
\2 The first act of the president is often appointing the Cabinet officials, often before they take the Oath of Office
\2 Secretaries earn \$161k per year, far less than the private sector, with less security and more politics, agreeing out of a sense of public service
\1 Cabinet members are selected based on their background (expertise, degrees, and respect within their field), relationship with special interest groups relevant to the department, and administrative skills and experience as the main factors
\2 Race, gender, and ethnicity has a factor in recent times, such as Johnson’s appointment of Weaver as Secretary of Housing and Urban Development, the first black secretary, in 1966
\2 Roosevelt elected Perkins, secretary of labor in 1933, then Ford put Hills as HUD secretary in 1975, both women
\2 Since then, most cabinets have women and blacks generally, Cavazos as secretary of education in 1988 as the first hispanic
\2 Clinton, in particular, used the Cabinet to increase representation in government by minorities groups
\1 The president-elect meets with advisors, Congressmen, and interest groups to find possible appointees, often leaking names to test public reaction
\2 The Senate holds hearings, going to the committee on the department, but are mainly routine, rarely rejected, though the publicity can bring bad publicity, forcing appointees to give up the position
\1 The cabinet meets when the president calls a meeting, generally closed to the press in the cabinet room, once a week at most, the role defined by the president
\2 Certain presidents, like Roosevelt’s brain trust professors or Jackson’s kitchen cabinet friends, have used non-cabinet bodies for advice
\2 Other presidents simply did not use advisors, deciding alone, like Lincoln
\2 The secretary of defense, state, treasury, and attorney general often act as the inner cabinet, due to running departments involving major, national issues, having the most contact and influence with the president
\1 Johnson wanted to get along with the Cabinet for a smooth transition after Kennedy’s assassination, but quickly stopped meeting regularly, only for orders
\2 Nixon’s cabinet later only met every several months at most
\2 Reagan wanted to use the cabinet as an inner circle of advisors, creating groups of the cabinet for large policy areas, but stopped within a year
\2 Clinton and Bush used the cabinet to test their ideas, rather than advising
\1 The president typically relies on the Executive Office assistants and personal staff, rather than the Cabinet, due to their loyalties to long standing programs or career officials in the departments, Congress, and special interest groups
\2 They may want control over inter-departmental programs, leading to bias
\2 There is also issues with keeping discussion secret due to the large number, and lack of trust due to the appointees often being strangers
\end{outline*}
\subsection{Executive Office}
\begin{outline*}
\1 The EO of the President (EOP) provides advice and information, and allows implementation and control over the executive, directly assisting the president
\2 In 1936, Roosevelt’s assistants were unable to coordinate the many New Deal agencies, making the President’s Comm. on Admin. Management to study the issue, suggesting a personal staff in the White House
\2 The Reorganization Act of 1939 made the office, moved the Bureau of Budget to the EOP, and made the advisory White House Office
\1 Currently, the EOP has 1.5k employees, many in the West Wing, expanded by each president in response to issues, taking experts for each problem, such as the Council of Economic Advisors, and coordinating between agencies, such as the Office of National Drug Control Policy in 1988 for the Drug War
\2 The size depends on the need of the president for each office
\1 The Office of Management and Budget is the largest agency, with a director as a major advisor, preparing the national budget proposal
\2 Reagan began the trend of influencing national policy through the budget, cutting billions from programs
\2 The OMB also reviews all agency budget proposals before they reach the president, and reviews legislative proposals through central clearance to make sure it aligns with the President’s policy goals
\1 The National Security Council was made after WWII to aid military and foreign policy, containing the State, Defense Secretary, and VP, as well as the National Security Advisor, used most under Nixon, Kissinger
\2 Kissinger acted as a second State department in Vietnam, China, and USSR arms control talks, with the NSC continuing in this role in Carter
\2 This created confused with other nations about represented the US
\1 The National Homeland Security Council was made after 9/11, headed by the Secretary after the department was made by the 2002 HS Act
\2 The anthrax scare was the first major issue, deciding to use antibiotics
\2 The FBI, CIA, FEMA, Defense, Treasury, Transportation, and Health and Human Services secretaries are also on the council
\1 The Council of Economic Advisors was made after WWII, making national economic policy, writing the annual Economic Report of the President
\1 Other agencies include the National economic Council for long-term policy, Office of Environmental Policy, Office of Science and Tech Policy, Domestic Policy Council, and United States Trade Representative Office for trade agreements
\1 Until the 1930s, the White House Staff contained only a couple secretaries or administrative assistants, and had been even smaller previously
\2 Staff is appointed generally by personal supporters, generally without constituencies, not requiring confirmation, including several hundreds
\2 The top assistants are the inner circle, including counsel, press secretary, chief of staff, and deputy chief of staff
\2 They have high levels of power, screening information, as specialists, political strategists (such as the chief assistant for legislative affairs to predict Congress’s reaction), Congress liaisons, and agency enforcers
\end{outline*}
\section{Chapter 9 - The President}
\subsection{Presidential Powers}
\begin{outline*}
\1 The purpose of the president is a strong executive, carrying out acts of Congress, and allowing them to enforce laws and respond to issues quickly
\2 In addition, the president was to protect from tyranny of the majority, keeping Congress in check, protecting rights and private property
\2 II.2-3 gives the main powers of commander-in-chief and head of the executive branch with Senate consent, as well as treaties and ambassador appointment with Senate consent
\2 It also gives the power to appoint federal court judges, give pardons except in cases of impeachment, and reduce sentences
\2 Finally, the executive must execute laws, deliver a State of the Union to Congress, propose laws, and can call a special session of Congress
\1 Most powers of the president are based on the personalities and needs of the nation, as well as the mandate, or expressed will of the people
\2 Jefferson’s Louisiana Purchase enlarged the power, creating the idea of inherent powers in the office, expanded further by Teddy Roosevelt taking action whenever the need arose, unless forbidden
\2 During the Civil War, Lincoln suspended habeas corpus, trials, and blockaded Southern ports, claiming all was legal to protect the Union
\2 Roosevelt’s New Deal created the expectation of controlling economic life, and Congress has often given special powers in crises such as war
\2 Bush was given powers for the war against terrorism, and his approval rating drastically rose, while Johnson in 1964 got the Tonkin Resolution passed after two destroyers were attacked, allowing all necessary steps, to protect Americans in Southeast Asia, leading to Vietnam
\2 Mandate of popular support gives a majority of power, such that media like TV or newspapers as forums/mediums for discussion, with full time reporters, are used to create the president and policies’ image
\2 In addition, the bully pulpit was used by Teddy Roosevelt to speak on specific topics, to persuade the public in support of policies
\1 Congress limits the president by congressional veto overrides, such as the War Powers Act in 1973 to prevent troops from being put in combat for 60 days without Congressional approval, to fight Nixon’s undeclared war in Vietnam
\2 The power of purse, impeachment, and confirmation are used, starting impeachment on Johnson in the House (losing in the Senate by 1 vote), Nixon, and Clinton (innocent on both charges by Senate)
\1 Marbury v. Madison in 1803 gave the idea of judicial review, while Youngstown Sheet and Tube Comp. v. Sawyer in 1952 declared that the president could not take action if Congress did nothing
\2 This was the result of Truman threatening to nationalize steel mills in 1952 if there was a strike, to protect national security
\1 Bureaucrats also can hinder, accidentally or intentionally, presidential policy, while the people act as a check of the moral character of the president
\end{outline*}
\subsection{Presidential Roles}
\begin{outline*}
\1 The head of state’s duty is to represent the nation for ceremonial roles, and take part in public ceremonial duties, such as giving awards or meeting musicians
\2 The president thus acts as a symbol for the collective US culture
\1 The chief executive controls the branch of over 2M people, trying to influence how laws are used through executive orders, which have the power of laws to provide details to legislation, appointing, and removing the 2.2k top officials
\2 It can be difficult to remove officials, such as J. Edgar Hoover, who many wanted to fire, but had too much public support, remaining until death
\2 Impoundment is the power to refuse to spend appropriated money, generally used for minor issues as circumstances change
\2 They can appoint Supreme Court justices, grant reprieves, or postponement of punishments, pardons, and amnesty, or group pardons for a specific crime such as draft dodging, for federal crimes
\2 Nixon impounded billions to remove social programs, until the courts ruled it was illegitimate, leading to anti-wholesale impounding laws, while Clinton pardoned several campaign supporters for financial crimes
\2 Obama didn’t enforce laws such as Controlled Substances or DOMA, believing they were unconstitutional, making the DOJ not enforce it in lawsuits to push it to the Supreme Court
\1 The chief legislator is expected to propose/draft desired bills, announcing a legislative program in the State of the Union about key national problems
\2 The president also has a staff for writing the yearly budget and laws and for working directly with Congress
\2 They also use political favors and federal projects to gain support, as well as the veto power to force changes to laws, but cannot Line Item Veto after the 1996 Act was declared unconstitutional in Clinton v. NYC in 1998
\1 The economic planner was created by the federal budget and the New Deal, with the Employment Act of 1946 requiring an annual economic report to Congress, creating the CEA to study the economy, and gave responsibility to the federal government to promote the economy
\2 In 1970, Congress gave Nixon the power to control prices and wages, freezing all for 90 days, though not renewed after the law expired
\1 The party leader selects the national chairperson, gives speeches, raises money, supports the party platform, and uses patronage to reward party supporters
\1 The chief diplomat uses the CIA, State, Defense, and NSC to get constant foreign policy information, often classified, and as a single person, allows them to take power from Congress for fast, informed action
\2 The president has the power to negotiate and sign treaties, though they require a 2/3 Senate confirmation, such as Treaty of Versailles after WWI
\2 Executive agreements have the power of treaties as an agreement with foreign heads of state, used by FDR to loan ships to the British in WWII to bypass the executive agreement
\2 Presidents are required to publicly release all agreements since 1972, though many do not, claiming classified, such as SE Asian protection
\2 The president is also allowed to determine if the US recognizes the legitimacy of a foreign government, such as not for Communist Cuba
\1 The commander-in-chief can make war, though requires congressional approval to declare war, though Latin American pro-regime in 1900s, HW Bush’s invasion of Panama and the War on Terror sent troops without declaration
\2 Generals make minor decisions, while many presidents such as Eisenhower came from the military, though others such as LBJ didn’t
\2 Special operations such as Carter’s Hostage Crisis or 9/11 are run by the president, and they have control over nuclear weapons
\2 They can take action to aid the war effort through special powers, and can use the military for domestic crises, like natural disasters or civil unrest
\end{outline*}
\subsection{Leadership Styles}
\begin{outline*}
\1 While the Founders expected Congress to lead, most people expect the president to act as a strong leader, such as big policy changes or event reactions
\2 This ranges from Johnson’s Great Society to Nixon opening diplomatic relations with China as USSR/PRC tensions rose
\1 Presidents must have a good understanding of the public to gain support for policies, leading to the shift in LBJ’s support, and Hoover’s losing the election to FDR after believing fear of federal intervention overruled unemployment
\2 Communicating policies effectively and often with the public is important in getting support, done by FDR and Reagan (Great Communicator)
\2 Flexibility of ideas is important, due to the constantly changing world, having debates among advisors, done by Kennedy/FDR, but not Reagan
\2 Political courage against public disagreement is important, done by Lincoln when many wanted peace with the South, due to the lack of progress and high casualties early in the war, risking reelection in 1864
\2 Timing of new policies/decisions is important, testing controversial ideas with assistants, the Cabinet, or press leaks, reacting quickly to clear situations like Soviet republic independence, slowly to giving money to the failing Union to promote democracy when the situation was unclear
\2 Compromising is important to be able to get policies passed, like Wilson refusing on the Treaty of Versailles and League of Nations, instead going on a speaking tour, ending in a stroke and the treaty rejected
\2 Johnson was famous for his treatment of Congress, getting over 95\% of bills suggested passed, acting very firm with Congress
\1 The president is isolated, with a large staff about him making him feel above the opinions of others, as well as causing all to be in awe, fearing disagreeing
\2 Access to the president is important political power, such that certain presidents who dislike dissent, like LBJ or Nixon, make it more risky for the staff to disagree or give bad news
\2 Since the Chief of Staff controls access, they can influence the president, in Reagan or Nixon’s case acting as a de facto president, leading to the lack of knowledge about the NSC covert Iran contra deal for the former
\2 Bush Sr. later tried to shift power away from the staff to protect against this, but it largely failed, due to the bias of the Cabinet and the difficulty in communicating directly with the public
\2 Lack of organization of staff would later slow down Clinton’s domestic platform, hurting discussion, forcing reorganization first
\1 Executive privilege is the right of executive officers with the president’s consent to refuse information to Congress or a court, based on separation of powers, used since Washington, though disputed, feeling it hinders oversight powers
\2 The move from the Cabinet to the Office questioned communications of the president, felt to need to be confidential for open sharing
\2 In US v. Nixon, it was ruled he had to give the secret tapes of talking with aides about Watergate, such that correspondences were public
\end{outline*}
\section{Chapter 10 - Federal Bureaucracy}
\subsection{Bureaucratic Organization}
\begin{outline*}
\1 The bureaucracy aid the executive branch to execute the laws, organized into boards, agencies, departments, corporations, and committees, created by Congress, most of which report to the President, though a few to Congress
\2 Article II allows the president to appoint department heads, though by 1801, only a couple thousand were employed, currently 3M
\1 The Cabinet departments, the 1st four were made in 1789, run by the secretary, deputy/under secretary, and assistant secretary, appointed by the president
\2 Under these are the directors of units of the branches and their assistants, following secretary department policies based on the advice of career specialist officials in the department, doing research on issues
\2 The State Department controls foreign policy, protecting rights of citizens abroad, staff embassies and the UN, and determine US interests
\2 The Treasury Department manage the money through the Bureau of the Mint (coins), Engraving and Printing (bills), the IRS (tax collection), and Public Debt (borrowing money for government operation)
\2 The Defense/War/Military Establishment Department protects security through the Joint Chiefs, cut after the Cold War, expanded after 9/11
\2 The Justice Department, made in 1870 from the Attorney General Office, controls the FBI, US Marshals, DEA, and the antitrust and civil rights divisions to oversee the legal affairs
\2 The Interior Department, made in 1849, manages natural resources, and Native relations, with the Bureau of Mines and contains the National Parks Service
\2 The Agricultural Department, made in 1862, has conservation and subsidy programs, to protect the national food supply
\2 The Commerce Department, made in 1903, protects industry by the National Institute of Standards and Tech (for weights and measures) and Patent and Trademark Office, and included the Census Bureau
\2 The Labor Department, made in 1913, protects workers with pension rights, safe working conditions, minimum wage, and has the Bureau of Labor Statistics for data analysis and the American Workplace Office for union-management cooperation
\2 The Housing and Urban Development Department, made in 1965, gives equal housing opportunities, including the Gov. National Mortgage Ass. to help provide mortgage money to people attempting to buy homes
\2 The Transportation Department, made in 1966, includes the Federal Aviation, Railroad, Highway. and Transit (for mass transit) Admin.
\2 The Energy Department, made in 1977, handles energy policy, research, and energy technology development, such as nuclear research
\2 The Health and Human Services Department, made in 1979, manages entitlements, the Public Health Service (for research/policy), and the FDA
\2 The Education Department, made in 1979, coordinates federal school funding, ESL programs, and disability programs
\2 The Veterans Affairs Department, made in 1989, controls hospitals and family benefits for veterans
\2 The Homeland Security Department, made in 2002, controls the Coast Guard, Border Patrol, Immigration and Naturalization Service, Customs Service, and FEMA to protect the US, analyzing FBI and CIA data
\1 Federal agencies, the heads chosen by the president exist, often to aid the executive, such as the General Services Admin. (for construction and maintaining of government buildings), CIA, or National Archives and Records Admin. (maintaining and publishing government rules and records), or NASA
\2 These also include government corporations, or businesses such as the Tennessee Valley Authority (for dams in the region) or the FDIC
\2 The USPS, originally a department, became a corporation in 1970 to prevent constant deficits, the only group allowed to carry first-class mail
\2 Corporations are allowed to take more risks, and keep surplus funds within, but are funded by Congress rather than private citizens
\1 Regulatory commissions have 5-11, appointed by the president with Senate consent, remaining for up to 14 years, not reporting to any branch
\2 These regulate major industries and act as courts to those businesses, but as a result, are pressured by lobbyists and subject to revolving door, gaining high paying jobs after they leave the commission at the company
\2 Since Carter, Congress has been deregulating to reduce the power of agencies and the burden on business, under Clinton increasing private property rights, marginal analysis, and simplification after GOP pushing
\2 Clinton also cut the bureaucracy by 252k, giving cash incentives for workers to resign, immediately cutting 7.5k from Agricultural, and eliminating the Interstate Commerce Comm. by 1995, the oldest agency
\2 One of the main goals of deregulation was to increase competition by making procurement of materials by businesses easier
\end{outline*}
\subsection{Bureaucratic Work}
\begin{outline*}
\1 Theoretically, bureaucrats only carry out public policy, but realistically, they make policy through administering programs, forcing the writing of regulations and standards to implement laws, which were not specified in the law
\2 The Social Security Act of 1935 didn’t specify how to define disabled workers, such that the DoHHS determines who gets benefits
\2 Rules often act similarly as laws, such as regulations on businesses for hiring to prevent discrimination, or for safety regulations
\2 Agencies, especially regulatory, also act as courts over law implementation, able to shape policy by these courts
\2 Interagency task-forces are needed to settle debates about overlapping regulations, as agencies try to influence the policies of other agencies
\1 As regulations grew, the amount of required paperwork by businesses grew, costing huge amounts, so a bill in 1995 attempted to reduce it by 5\% by 2001
\1 Bureaucrats testify in Congress as experts and draft suggested bills, getting support to let them pass, such as DoHHS writing Medicare, or the DoJ writing the Safe Streets Act of 1968 which made the Law Enforcement Assistance Admin.
\2 As a result, they often act as a research service for potentially needed policies as experts in the field
\1 Bureaucracy is needed to make policy due to mirroring the growth of the nation and technology, as more issues become existent
\2 Bureaucracy also grows as a result of rarely being removed, but being created whenever needed, such as the Federal Metal and Non-Metallic Safety Board of Review, never gaining work, yet staying for 4 years
\1 They also work to quickly react to international crises, combining the War and Navy departments in 1947, the Veterans Admin. after Vietnam, NASA after Sputnik in 1957, growing as situations become more dangerous
\2 Several agencies such as the CIA, Peace Corps. US Info. Agency, and the Arms Control and Disarmament Agency grew out of the Cold War also
\2 It also expands to respond to economic issues, doubling in size during the Great Depression due to the New Deal
\1 It also exists to respond to specific client groups, such as Agriculture, Commerce, or Labor, or groups which their decisions most impact, lobbying Congress and agencies for more programs, creating competitions between opposing agencies
\2 Iron triangles are close cooperation between congressional committees, client groups, and the agencies, such as the military-industrial complex, DoD, or the DoVA and the American Legion group
\2 The client groups then aid the congressmen in being reelected, who give money to the agency, and the group is given services by the agency
\2 There are fears of the revolving door allowing too much influence in iron triangles, creating the need for regulation
\1 Bureaucracy is influenced by the courts, when citizens directly affected challenge agencies in federal courts, and they may issue an injunction to stop the action
\2 Courts often uphold the decisions of the agency, losing >75\% of cases
\1 Bureaucracy also has liaison officers who keep relationships with Congress, monitoring bills and requesting information from legislators, due to their ability to use the budget and laws (either oversight or regulatory) to influence agencies
\2 The Government Performance and Results Act required them to write plans, set performance goals, and give data measures of effectiveness starting in 2000 to provide accountability
\2 Appropriation power is limited by entitlement expenditures, which cannot be cut, and agencies can specially cut the budget of their districts to force them to restore it, such as Amtrak in 1975
\end{outline*}
\section{Chapter 11 - Federal Court System}
\subsection{Federal Court Powers}
\begin{outline*}
\1 While the judicial originally had a far less active role in the government, John Marshall expanded the power during his 35 years from 1801 as Chief Justice
\1 The US has parallel court systems, with each state having their own system based on the state constitution and laws, and the federal and Supreme courts
\1 Jurisdiction, or the authority to hear cases, is based on the laws regarding, and federal courts also control treaties, bankruptcy, and maritime law
\2 Cases involving ambassadors, multiple state governments, the US government, the Constitution, citizens of different states, and citizens of the same state who claim land under different states are federal also
\2 In concurrent jurisdiction, the defendant can be sued in either, but can protest and get it changed to federal
\2 The original court, or trial court has original jurisdiction, held by district courts federally, while appellate jurisdiction is held by appeals courts, and then can be taken to the Supreme Court, which has both jurisdictions
\2 The dual system has state/federal district courts, appeals, state supreme courts, and then finally the federal supreme court, based on jurisdiction
\1 Federal courts are required to wait for litigants, or people engaged in a lawsuit, rather than taking their own action, determined when Secretary of State Jefferson wrote Chief Jay in 1793 asking questions about American neutrality
\1 Marbury v. Madison in 1801 allowed Marshall to create judicial review on acts of Congress for all federal courts when Jefferson prevented Marbury from getting his commission from Adams as Justice of the Peace, implied by III.iii
\2 Marbury had the Supreme Court order a writ for delivery by the Judiciary Act of 1789, which Marshall ruled unconstitutional for allowing that
\1 Marshall later expanded power in Fletcher v. Peck in 1810 allowed judicial review of state laws, and Dartmouth v. Woodward in 1819 gave protection of contracts to corporate charters
\2 He expanded federal power in McCulloch v. Maryland in 1819, stating states could not work against execution of laws in national interests, Gibbons v. Ogden extending interstate commerce to services in 1824
\1 In 1835, Jackson put Taney as Chief, working to give rights to citizens and states over federal, but in Dred Scott v. Sandford in 1857, he ruled that blacks could not be citizens and the Missouri Compromise was illegal, putting state over citizen
\1 The 14th amendment’s due process clause was generally not applied when individuals went against business or state rights after the Civil War
\2 In the 1873 Slaughterhouse Cases, it was ruled that state-sanctioned monopolies were allowed, due to rights only being required by federals
\2 In 1897 Plessy v. Ferguson, it was ruled separate but equal was allowed, with separate rail cars used to preserve peace, overruled in Brown v. Topeka Board of Ed. in 1954 under the Warren Court for civil liberties, preserving rights of accused and fair apportionment of representatives
\1 On the other hand, it gave state powers to protect against big business, in the Granger Cases in the 1870s, protecting regulatory laws and the ability of a state to regulate private property with invested public interest
\2 Later, they began to side with businesses, in the US vs. EC Knight and Co upholding business trust monopolies in the 1890s, in Debs v. US upholding the conviction of a labor leader who refused to end a strike
\2 It shifted again and back around the Progressive Era, until in the 1930s, in Schechter Poultry Corp. v. US, FDR tried to add justices to stack the court, but failed, though it shifted toward regulation again at that point
\end{outline*}
\subsection{Lower Federal Courts}
\begin{outline*}
\1 The constitutional courts are those established by Article III, including the federal district courts, federal court of appeals, and US Court of International Trade, established effectively by the Judiciary Act of 1789
\2 District courts were created based on state boundaries to serve as trial courts, currently with 94 district courts, with 1-4 courts per state and 1 for DC and Puerto Rico, using a grand jury and petit jury for criminal suits
\2 Grand juries are 16-23 people to hear evidence from the accuser, and decide whether there is evidence to issue an indictment, bringing to trial
\2 Petit juries are 6-12 people is the trial jury, though if both sides agree, a judge or panel of 3 judges can be used
\1 District courts have a US attorney to represent the US in all civil suits and act as prosecutor in criminal cases, a magistrate to issue warrants and decide if a grand jury hearing is needed, and a marshall for order, arresting, and getting jurors
\2 Deputy clerks, bailiffs, and stenographers are needed to keep records for appellate courts, created in 1891 to prevent all appeals from the SCOTUS
\1 There are 13 US court of appeals, in 12 circuits with 1 court as a national special appeals courts, all generally having a panel of 3 judges, occasionally calling all judges from the court to a panel for important cases
\2 These can uphold, reverse, or send the case back to the lower court, used for Tax Court, district/territorial court, and regulatory agency appeals
\2 The US Circuit Court of Appeals for the Federal Circuit is the special appeals court made in 1982 for the Federal Claims Court, Court of International Trade, and Patent Office
\2 Opposite circuit decisions force the Supreme Court to take the case, such that special interests groups use the system to get major issues from circuits in their favor up to the Supreme Court
\2 The DC Circuit Appeals Court hears cases involving executive agencies, such that many Supreme justices are from that court
\1 The US Customs/International Trade Court has jurisdiction on all tariff cases, appealing to the Circuit Court
\1 The legislative courts allow Congress to exercise its power under Article I
\1 The US Federal Claims Court, made in 1982, has original over claims against the US for money damages or lack of payment
\1 The US Tax Court, made in 1969, relates to tax disputes
\1 The US Court of Appeals for the Armed Forces, made in 1950, has appellate over military law, with only the Supreme Court having higher jurisdiction over it
\1 Territorial Courts and the Courts of DC are similar to district courts, to govern the 4 non-US territories and DC
\1 The Court of Veterans’ Appeals, made in 1988, has appellate over the Board of Veterans’ Appeals in the Veterans Affairs Department for unsettled claims
\1 The Foreign Intelligence Surveillance Court, made in 1978 by the Foreign Intelligence and Security Act, was allowed to wiretap on potential spies
\2 The Patriot Act gave it the ability for secret wiretaps and searches of anyone suspected of terrorism and spying without probable cause
\1 Article II, Section 2 allows the president to appoint federal judges from top legal minds, serving for life to avoid political pressures
\2 Party position acts as a factor, often increasing the number of judges when Congress and the president control the same party
\2 Judicial/political philosophy also plays a role, carrying the President’s opinions on even after the term
\2 Senatorial courtesy is the practice of submitting the name to the candidate’s state senators, before Senate confirmation, withdrawing if they disagree, but does not apply to the appeals or Supreme court
\2 Judge background is a factor, typically serving as state judges before federal, often white men, but Carter added hispanics, Johnson had Marshall as the first black Justice, Reagan, O’Connor as the first woman
\end{outline*}
\subsection{Supreme Court}
\begin{outline*}
\1 The Court has the ability to choose which cases it will hear, originally not an honor to be chosen until 1891, due to spending most time riding the circuit, holding court in regions around the country, now held purely in DC
\1 It has final appellate jurisdiction, and original over cases involving states and representatives of foreign governments, generally two states, or a state v. the US
\2 Generally, only 5 cases are original, most appellate from federal appeals courts or state supreme courts if it is related to federal law, but can rule only on the federal law portion of the case
\1 There are 9 justices, which Congress can change, varying from 5 to 10, 9 since 1869, though FDR attempted to raise it in 1937 to stack the Court
\2 Congress sets pay of associate and chief justices, and can only raise it
\2 Justices can be impeached for treason, bribery, or high crimes and misdemeanors, though Chase in 1804 was due to partisan political activity, but was found not guilty by the Senate
\1 The justices have no Constitutional duties, rather created by laws, need, and tradition, mainly hearing and ruling on cases, writing opinions for decisions, and deciding which case to take, presided over by the Chief Justice
\2 The Chief Justice also administers the federal court system overall and presides over presidential impeachment trials in the Senate
\2 Each justice also has 1-2 federal district circuits assigned, reading requests for special legal actions from them, such as temporarily nullifying a decision until a further legal proceeding
\2 In extreme circumstances, they have additional duties, such as Warren running the commission to investigate the Kennedy assassination and Jackson serving as chief prosecutor at the Nuremberg trials
\1 Law clerks assist justices, reading appeals, writing memos summarizing, preparing research and drafts of case opinions, generally top graduates from top law schools, staying for several years typically
\1 Justices are generally lawyers with legal experience, typically a former judge or court official, in Chief Justice Taft’s case, a former president, 50 years old at youngest, upper class, 6 born outside the US
\1 They are confirmed by the Senate, rejecting 25\% in the 1800s, rejecting fewer in modern day, often electing safe choices to avoid rejection
\2 They are chosen for political beliefs, but some, such as Warren or Bader Ginsburg act unexpectedly, the latter acting with large judicial restraint, the former elected by Eisenhower to be conservative
\2 The attorney general and upper members of Congress, as well as special interest groups also influence it, such as the National Organization for Women who fought Stevens, Souter, and Thomas’s nominations fearing pro-life stances, unions, NAACP, and lobbyists influence the Senate
\2 The American Bar Association’s Committee on the Federal Judiciary evaluates professional qualifications of all federal judges to advise, though Bush attempted to reduce it, fearing liberal bias
\2 Justices, especially the chief and those who work with new justices, can write recommendations or lobby for candidates
\end{outline*}
\end{document}
