\documentclass[11pt, titlepage]{article}
\usepackage{amsmath,amsthm,amssymb}
\usepackage{hyperref, pgf, tikz}
\usepackage{fancyhdr}
\usetikzlibrary{arrows}
\usepackage[margin=1.25in]{geometry}
\usepackage{graphicx}                     
\pagestyle{fancy}
\usepackage{array}
%\usepackage{wrapfig}

\lhead{Lab \#4}
\rhead{\thepage}
\cfoot{}

\title{Torques, Equilibrium, and Center of Gravity\\ \ \\ \large Lab \#4}
\author{Name: Avery Karlin \\ Partner: Nicholas Yang}
\date{}
\begin{document}

\maketitle

\begin{center}
\LARGE Torques, Equilibrium, and Center of Gravity
\end{center}

\section*{Objective}
The objective of the lab is to measure the spring constant of springs, and the spring constant when they are in series, parallel, and on opposite sides.

\section*{Introduction}
The period of an oscillation is defined as the time for one full cycle of oscillation, while the frequency is defined as the number of cycles per second, such that $f = \frac{1}{T}$. Angular frequency is further defined as the number of radians per second, such that $\omega = 2\pi f$, defined on a spring oscillation as $\omega = \sqrt{\frac{k}{m}}$. As a result, we can use that to derive the formula $$k = \frac{4\pi^2m}{T^2}.$$

\section*{Procedures and Results}
% Explain procedure here

\begin{figure}[p]
\centering
\hspace*{-10.5cm}
\includegraphics[scale=0.15, angle=270]{lab4.jpg}
\vspace*{19cm}
\end{figure}

\begin{center}
\begin{tabular}
{|m{5em}|m{5em}|m{5em}|m{5em}|m{5em}|m{5em}|}
Spring Format & Trial 1 (s) & 2 & 3 & 4 & 5 \\
Spring \#1 & 3.06 & 2.9 & 3.02 & 2.9 & 2.85 \\
Spring \#2 & 5.8 & 6.08 & 5.8 & 6.05 & 6.08 \\
Series & 6.75 & 6.73 & 6.53 & 6.46 & 6.77 \\
Parallel & 2.56 & 2.88 & 2.68 & 2.6 & 2.88 \\
Opposite & 2.7 & 2.78 & 2.68 & 2.83 & 2.65 \\
\end{tabular}
\end{center}

\section*{Discussion}
Sample calculations for the non-measured data are as shown:

$$k_t = k_1 + k_2 = 9.61 + 2.377 = 11.987 N*m$$
$$k_t = \frac{1}{\frac{1}{k_1} + \frac{1}{k_2}} = \frac{1}{\frac{1}{9.61} + \frac{1}{2.377}} = 1.906 N*m$$

$$\text{Series Percent Error} = \frac{|k_{act} - k_{exp}|}{k_{act}} * 100\% = \frac{|1.906 - 1.887|}{1.906} * 100\% = 0.996\% $$ 
$$\text{Parallel Percent Error} = 5.98\% $$
$$\text{Opposite Percent Error} = 6.53\% $$

\begin{center}
\begin{tabular}
{|m{8em}|m{8em}|m{8em}|m{8em}|}
Spring Format & Average (s) & Period (s) & k (N/m) \\\
Spring \#1 & 2.946 & 1.473 & 9.61 \\
Spring \#2 & 5.922 & 2.961 & 2.377 \\
Series & 6.648 & 3.324 & 1.887 \\
Parallel & 2.72 & 1.36 & 11.27 \\
Opposite & 2.728 & 1.364 & 11.204 \\
\end{tabular}
\end{center}

It is clear by the data that for springs in series, $\frac{1}{k_T} = \frac{1}{k_1} + \frac{1}{k_2}$, and for springs in parallel, $k_T = k_1 + k_2$. In addition, springs can be placed on opposite sides, where the overall $k_T = k_1 + k_2$, such that it acts as if it was in series.

The percent errors, especially that of the in series springs, are clearly %Finish

\section*{Conclusion}
It was found that in series springs have a spring constant of the reciprocal of the sum of the reciprocals of the individual spring constants, for a measured value of 1.887 N*m to the actual value of 1.906 N*m, for a percent error of 0.996\%.

Parallel springs have a spring constant of the sum of the individual spring constants, for a measured value of 11.27 N*m to the actual value of 11.987 N*m, for a percent error of 5.98\%.

Springs on opposite sides act as parallel springs for a spring constant of the sum of the individual spring constants, such that the measured value was 11.204 N*m to the actual value of 11.987 N*m for a percent error of 6.53\%.

\end{document}
